\section{Combinatorial $\M$-bialgebras}

\anibal{speak about presheaf categories, Yoneda and Kan extensions}

\subsection{Simplicial sets}

For any non-negative integer $n$ denote by $[n]$ the category generated by the poset (or finite ordinal) $\{0 \to 1 \to \dots \to n\}$.
The \textit{simplex category} $\simplex$ is the category with objects $[0], [1], \dots$ and morphisms given by all functors $f \colon [n] \to [m]$.
The morphisms in $\simplex$ are generated by the usual
\textit{simplicial co-face} and \textit{co-degeneracy maps}
\begin{align*}
& \delta_j \colon [n-1] \to [n], \\
& \sigma_j \colon [n+1] \to [n]
\end{align*}
for $j \in \{0, \dots, n\}$.

%A \textit{simplicial set} is a functor $S \colon \simplex^{op} \to \Set.$ Given a simplicial set $S$, we will write $S_n= S( [n] )$, $\partial_j = S( \partial^j) \colon S_n \to S_{n-1}$, and $s_j= S(s^j) \colon S_n \to S_{n+1}$, and call $\partial_j$ and $s_j$ the face and degeneracy maps of $S$, respectively.
%Simplicial sets form a category, denoted by $\sSet$, with morphisms being natural transformations.  We denote by $\Delta^n \in \sSet$ the simplicial set corepresented by $[n]$, i.e. $\Delta^n:= \Hom_{\simplex}(\text{ \_ ,}  [n])$. 
%
%The category $\sSet$ of simplicial sets is a symmetric monoidal category when equipped with the usual cartesian product $\times$ given by $(S \times S')_n =S_n \times S'_n$ together with face and degeneracy maps defined by the cartesian product of maps.

\anibal{define chains, extend to $\sSet$}

We review from \cite{Medina20prop1} a natural $\mathcal M$-bialgebra structure on the normalized chains of standard simplices $\chains(\triangle^n)$.
The $\M$-bialgebra is specified by three linear maps, the images of the generators
\begin{equation*}
\counit, \quad \coproduct, \quad \product,
\end{equation*}
satisfying the relations in the presentation of $\mathcal M$. For $n \in \mathbb{N}$, define: \vspace*{5pt} \\
(1) The counit $\epsilon \in \Hom(\chains(\triangle^n), \Z)$ by
\begin{equation*}
\epsilon \big( [v_0, \dots, v_q] \big) = \begin{cases} 1 & \text{ if } q = 0, \\ 0 & \text{ if } q>0. \end{cases}
\end{equation*}
(2) The coproduct $\Delta \in \Hom(\chains(\triangle^n), \chains(\triangle^n)^{\otimes2})$ by
\begin{equation*}
\Delta \big( [v_0, \dots, v_q] \big) = \sum_{i=0}^q [v_0, \dots, v_i] \otimes [v_i, \dots, v_q].
\end{equation*}
(3) The product $\ast \in \Hom(\chains(\triangle^n)^{\otimes 2}, \chains(\triangle^n))$ by
\begin{equation*}
\left[v_0, \dots, v_p \right] \ast \left[v_{p+1}, \dots, v_q\right] = \begin{cases} (-1)^{p+|\pi|} \left[v_{\pi(0)}, \dots, v_{\pi(q)}\right] & \text{ if } v_i \neq v_j \text{ for } i \neq j, \\
0 & \text{ if not}, \end{cases}
\end{equation*}
where $\pi$ is the permutation that orders the totally ordered set of vertices, and $(-1)^{|\pi|}$ its sign.

\begin{proposition}[\cite{Medina20prop1}] \label{p:simplicial chain bialgebra}
	For every $n \in \mathbb{N}$, the assignment
	\begin{equation*}
	\counit \mapsto \epsilon, \quad \coproduct \mapsto \Delta, \quad \product \mapsto \ast,
	\end{equation*}
	defines a natural $\mathcal M$-bialgebra structure  on $\chains(\triangle^n)$.
\end{proposition}

\begin{proof}
	See Theorem 4.2 in \cite{Medina20prop1}.
\end{proof}

\anibal{Extend to $\sSet$ \\
The associated $U(\M)$-coalgebra structures define a natural $U(\M)$-coalgebra structure on the normalized chains of any simplicial set via a Kan extension argument.}

\footnote{We remark that there is no $\M$-bialgebra structure on an arbitrary simplicial set.
For example, consider the simplicial set whose normalized chains are generated by two degree 0-elements.}

\begin{remark}
	The dashed arrow in
	\begin{equation*}
	\begin{tikzcd}
	\sSet \arrow[r] \arrow[dr, dashed] & \coAlg_{U(\M)} \arrow[d] \\
	& \coAlg
	\end{tikzcd}
	\end{equation*}
	agrees with the structure defined by the Alexander-Whitney diagonal \cite{bibid}.
	
	More generally, as explained in \cite{Medina20prop1}, our $E_\infty$-structure generalizes that defined by McClure-Smith \cite{mcclure03cochain} and Berger-Fresse \cite{berger04combinatorial}.
\end{remark}

\subsection{Chains of cubical sets}

The \textit{cube category} $\square$ is the free strict monoidal category with a \textit{bipointed object}
\begin{equation*}
\begin{tikzcd}
1 \arrow[r, bend left, "\delta^0"] \arrow[r, bend right, "\delta^1"'] & 2 \arrow[r, "\sigma"] & 1
\end{tikzcd}
\end{equation*}
such that $\sigma \circ \delta^0 = \sigma \circ \delta^1 = \mathrm{id}$. Explicitly, it contains an object $2^n$ for each non-negative integer $n$ and its morphisms are generated by the \textit{coface} and \textit{codegeneracy maps} defined by
\begin{align*}
\delta_i^\varepsilon & = \mathrm{id}_{2^{i-1}} \times \delta^\varepsilon \times \mathrm{id}_{2^{n-1-i}} \colon 2^{n-1} \to 2^n, \\
\sigma_i & = \mathrm{id}_{2^{i-1}} \times \, \sigma \times \mathrm{id}_{2^{n-i}} \colon 2^{n} \to 2^{n-1}.
\end{align*}

%A \textit{cubical set} $X$ is a contravariant functor from the cube category to the category of sets and a cubical map is a natural transformation between two cubical sets. As is customary, we use the notation
%\begin{equation*}
%X\big( 2^n \big) = X_n \qquad X(\delta^\varepsilon_i) = d^\varepsilon_i \qquad X(\sigma_i) = s_i,
%\end{equation*}
%and refer to elements in the image of any $s_i$ as \textit{degenerate}.
%
%For each $n \in \mathbb{N}$, the cubical set $\square^n$ is defined by
%\begin{equation*}
%\square^n_k  = \Hom_{\square} \big( 2^k, 2^n \big), \qquad 
%d^\varepsilon_i(x) = x \circ \delta^\varepsilon_i, \qquad 
%s_i(x) = x \circ \sigma_i,
%\end{equation*}
%notice that iteratively
%\begin{equation*}
%\square^n = \overbrace{\square^1 \times \cdots \times \square^1}^{n \text{ times }}.
%\end{equation*}
%We represent the non-degenerate elements of $\square^n$ as sequences $x_1 \cdots\, x_n$ with each $x_i \in \{[0], [1], [0,1]\}$. Any cubical set can be expressed as a colimit of these
%\begin{equation*}
%X \cong \colim_{\square^n \to X} \square^n.
%\end{equation*}

\anibal{define chains on representables and extend}

We recall a natural $\mathcal M$-bialgebra structure on $\chains(\square^n)$ for every $n \geq 0$.
These are determined by three linear maps satisfying the relations in the presentation of $\mathcal M$.
For $n \in \mathbb{N}$, define: \vspace*{5pt} \\
(1) The counit $\epsilon \in \Hom(\chains(\square^n), \Z)$ by
\begin{equation*}
\epsilon \left( x_1 \otimes \cdots \otimes x_d \right) = \epsilon(x_1) \cdots \, \epsilon(x_n),
\end{equation*}
where
\begin{equation*}
\epsilon([0]) = \epsilon([1]) = 1, \qquad \epsilon([0, 1]) = 0.
\end{equation*} \vspace*{-6pt} \\
(2) The coproduct $\Delta \in \Hom \left( \chains(\square^n), \chains(\square^n)^{\otimes 2} \right)$ by
\begin{equation*}	
\Delta (x_1 \otimes \cdots \otimes x_n) = 	
\sum \pm \left( x_1^{(1)} \otimes \cdots \otimes x_n^{(1)} \right) \otimes 	
\left( x_1^{(2)} \otimes \cdots \otimes x_n^{(2)} \right),	
\end{equation*}	
where the sign is determined using the Koszul convention, and we are using Sweedler's notation
\begin{equation*}	
\Delta(x_i) = \sum x_i^{(1)} \otimes x_i^{(2)}
\end{equation*}
for the chain map $\Delta \colon \chains(\square^1) \to \chains(\square^1)^{\otimes 2}$ defined by
\begin{equation*}
\Delta([0]) = [0] \otimes [0], \quad \Delta([1]) = [1] \otimes [1], \quad \Delta([0, 1]) = [0] \otimes [0, 1] + [0, 1] \otimes [1].
\end{equation*}
Using that $\chains(\square^n) = \chains(\square^1)^{\otimes n}$, $\Delta$ is the composition
\begin{equation*}
\begin{tikzcd}
\chains(\square^1)^{\otimes n} \arrow[r, "\Delta^{\otimes n}"] & \left( \chains(\square^1)^{\otimes 2}  \right)^{\otimes n} \arrow[r, "sh"] & \left( \chains(\square^1)^{\otimes n} \right)^{\otimes 2}
\end{tikzcd}
\end{equation*}
where $sh$ is the shuffle map that places tensor factors in odd position first. \vspace*{5pt} \\
(3) The product $\ast \in \Hom(\chains(\square^n)^{\otimes 2}, \chains(\square^n))$ by
\begin{align*}
(x_1 \otimes \cdots \otimes x_n) \ast (y_1 \otimes \cdots \otimes y_n) =
(-1)^{|x|} \sum_{i=1}^n x_{<i} \epsilon(y_{<i}) \otimes x_i \ast y_i \otimes \epsilon(x_{>i})y_{>i},
\end{align*}
where
\begin{align*}
x_{<i} & = x_1 \otimes \cdots \otimes x_{i-1}, &
y_{<i} & = y_1 \otimes \cdots \otimes y_{i-1}, \\
x_{>i} & = x_{i+1} \otimes \cdots \otimes x_n, & 
y_{>i} & = y_{i+1} \otimes \cdots \otimes y_n,
\end{align*}
with the convention
\begin{equation*}
x_{<1} = y_{<1} = x_{>n} = y_{>n} = 1 \in \Z,
\end{equation*}
and the only non-zero values of $x_i \ast y_i$ are
\begin{equation*}
\ast([0] \otimes [1]) = [0, 1], \qquad  \ast([1] \otimes [0]) = -[0, 1].
\end{equation*}

\begin{proposition}[\cite{bibid}] \label{thm: cubical chain bialgebra}
	For every $n \in \mathbb{N}$, the assignment
	\begin{equation*}
	\counit \mapsto \epsilon, \quad \coproduct \mapsto \Delta, \quad \product \mapsto \ast,
	\end{equation*}
	defines a natural $\mathcal M$-bialgebra structure on $\chains(\square^n)$.
\end{proposition}

\begin{proof}
	We present a proof of this statement in \cite{??}.
\end{proof}

\anibal{extension to $\cSet$}

\begin{remark}
	The dashed arrow in
	\begin{equation*}
	\begin{tikzcd}
	\cSet \arrow[r] \arrow[dr, dashed] & \coAlg_{U(\M)} \arrow[d] \\
	& \coAlg
	\end{tikzcd}
	\end{equation*}
	agrees with the structure  defined by the Serre diagonal \cite{bibid}.
\end{remark}