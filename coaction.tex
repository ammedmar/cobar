\section{Combinatorial $\M$-bialgebras}

Recall that for any pair of functors $\mathsf{B} \stackrel{f}{\leftarrow} \mathsf{A} \stackrel{g}{\to} \mathsf{C}$ where $\mathsf{A}$ is small and $\mathsf{C}$ is (small) cocomplete, the \textit{(left) Kan extension} of $g$ along $f$ exists, this is, the universal functor and natural transformation depicted below
\begin{equation*}
\begin{tikzcd}
\mathsf{A} \arrow[d, "f"'] \arrow[r, "g"] & \mathsf{C} \\
\mathsf{B} \arrow[dashed, ur] & \quad .
%\arrow[Rightarrow, dl, shorten <= 2.8em, shorten >= .1em]
\end{tikzcd}
\end{equation*}

\subsection{Simplicial sets}

For any non-negative integer $n$ denote by $[n]$ the category generated by the poset (or finite ordinal) $\{0 \to 1 \to \dots \to n\}$.
The \textit{simplex category} $\simplex$ is the category with objects $[0], [1], \dots$ and morphisms given by all functors $f \colon [n] \to [m]$.
The morphisms in $\simplex$ are generated by the usual
\textit{simplicial co-face} and \textit{co-degeneracy maps}
\begin{align*}
& \delta_j \colon [n-1] \to [n], \\
& \sigma_j \colon [n+1] \to [n]
\end{align*}
for $j \in \{0, \dots, n\}$.

\anibal{define chains, extend to $\sSet$}

We review from \cite{Medina20prop1} a natural $\mathcal M$-bialgebra structure on the normalized chains of standard simplices $\chains(\triangle^n)$.
The $\M$-bialgebra is specified by three linear maps, the images of the generators
\begin{equation*}
\counit, \quad \coproduct, \quad \product,
\end{equation*}
satisfying the relations in the presentation of $\mathcal M$. For $n \in \mathbb{N}$, define: \vspace*{5pt} \\
(1) The counit $\epsilon \in \Hom(\chains(\triangle^n), \Z)$ by
\begin{equation*}
\epsilon \big( [v_0, \dots, v_q] \big) = \begin{cases} 1 & \text{ if } q = 0, \\ 0 & \text{ if } q>0. \end{cases}
\end{equation*}
(2) The coproduct $\Delta \in \Hom(\chains(\triangle^n), \chains(\triangle^n)^{\otimes2})$ by
\begin{equation*}
\Delta \big( [v_0, \dots, v_q] \big) = \sum_{i=0}^q [v_0, \dots, v_i] \otimes [v_i, \dots, v_q].
\end{equation*}
(3) The product $\ast \in \Hom(\chains(\triangle^n)^{\otimes 2}, \chains(\triangle^n))$ by
\begin{equation*}
\left[v_0, \dots, v_p \right] \ast \left[v_{p+1}, \dots, v_q\right] = \begin{cases} (-1)^{p+|\pi|} \left[v_{\pi(0)}, \dots, v_{\pi(q)}\right] & \text{ if } v_i \neq v_j \text{ for } i \neq j, \\
0 & \text{ if not}, \end{cases}
\end{equation*}
where $\pi$ is the permutation that orders the totally ordered set of vertices, and $(-1)^{|\pi|}$ its sign.

\begin{proposition}[\cite{Medina20prop1}] \label{p:simplicial chain bialgebra}
	For every $n \in \mathbb{N}$, the assignment
	\begin{equation*}
	\counit \mapsto \epsilon, \quad \coproduct \mapsto \Delta, \quad \product \mapsto \ast,
	\end{equation*}
	defines a natural $\mathcal M$-bialgebra structure  on $\chains(\triangle^n)$.
\end{proposition}

\begin{proof}
	See Theorem 4.2 in \cite{Medina20prop1}.
\end{proof}

\anibal{Extend to $\sSet$ \\
The associated $U(\M)$-coalgebra structures define a natural $U(\M)$-coalgebra structure on the normalized chains of any simplicial set via a Kan extension argument.}

\footnote{We remark that there is no $\M$-bialgebra structure on an arbitrary simplicial set.
For example, consider the simplicial set whose normalized chains are generated by two degree 0-elements.}

\begin{remark}
	The dashed arrow in
	\begin{equation*}
	\begin{tikzcd}
	\sSet \arrow[r] \arrow[dr, dashed] & \coAlg_{U(\M)} \arrow[d] \\
	& \coAlg
	\end{tikzcd}
	\end{equation*}
	agrees with the structure defined by the Alexander-Whitney diagonal \cite{bibid}.
	
	More generally, as explained in \cite{Medina20prop1}, our $E_\infty$-structure generalizes that defined by McClure-Smith \cite{mcclure03cochain} and Berger-Fresse \cite{berger04combinatorial}.
\end{remark}

\subsection{Chains of cubical sets}

The \textit{cube category} $\square$ is the free strict monoidal category with a \textit{bipointed object}
\begin{equation*}
\begin{tikzcd}
1 \arrow[r, bend left, "\delta^0"] \arrow[r, bend right, "\delta^1"'] & 2 \arrow[r, "\sigma"] & 1
\end{tikzcd}
\end{equation*}
such that $\sigma \circ \delta^0 = \sigma \circ \delta^1 = \mathrm{id}$. Explicitly, it contains an object $2^n$ for each non-negative integer $n$ and its morphisms are generated by the \textit{coface} and \textit{codegeneracy maps} defined by
\begin{align*}
\delta_i^\varepsilon & = \mathrm{id}_{2^{i-1}} \times \delta^\varepsilon \times \mathrm{id}_{2^{n-1-i}} \colon 2^{n-1} \to 2^n, \\
\sigma_i & = \mathrm{id}_{2^{i-1}} \times \, \sigma \times \mathrm{id}_{2^{n-i}} \colon 2^{n} \to 2^{n-1}.
\end{align*}

We denote by $\cube_{\deg}$ the subcategory with the same objects as $\cube$ and morphisms of the form $\sigma_i \circ \tau$ for some morphisms $\tau$ of $\cube$.

The category of \textit{cubical sets} is the functor category $\cSet = \Fun(\cube^\op, \Set)$.
The \textit{standard $n$-cube} is the cubical set $\cube^n = \cube(-, 2^n)$, and the \textit{Yoneda embedding} $\Y \colon \cube \to \cSet$ is the functor induced by $2^n \mapsto \cube^n$.
For any cubical set $X$ we have
\begin{equation*}
X_n \cong \colim_{\cube^n \to X} \cube^n.
\end{equation*}
Let $\cchains \colon \cube \to \Ch$ be the symmetric monoidal functor defined by
\begin{equation*}
\cchains(2)_m =
\frac{R\{\cube(2^m, 2)\}}{R\{\cube_{\deg}(2^m, 2)\}}\,,
\qquad \qquad
\partial (\tau) =
\begin{cases}
\delta^1 - \delta^0 & \tau  = (\id \colon 2 \to 2), \\
0 & \text{ otherwise }.
\end{cases}
\end{equation*}
We remark that $\cchains(2)$ is isomorphic to the cellular chains $C(\mathbb I)$ on the topological interval with its standard CW structure, and that $\cchains(2^n)$ is therefore isomorphic to $C(\mathbb I)^{\otimes n}$. 

Since the category $\Ch$ is cocomplete, we can use a Kan extension of this functor along the Yoneda embedding to define the functor of \textit{(normalized) chains} $\cchains \colon \cSet \to \Ch$. This functor satisfies $\cchains(\cube^n) \cong C(\mathbb I)^{\otimes n}$.

We recall a natural $\mathcal M$-bialgebra structure on $\cchains(\square^n)$ for every $n \geq 0$.
These are determined by three linear maps satisfying the relations in the presentation of $\mathcal M$.
For $n \in \mathbb{N}$, define: \vspace*{5pt} \\
(1) The counit $\epsilon \in \Hom(\cchains(\square^n), R)$ by
\begin{equation*}
\epsilon \left( x_1 \otimes \cdots \otimes x_d \right) = \epsilon(x_1) \cdots \, \epsilon(x_n),
\end{equation*}
where
\begin{equation*}
\epsilon([0]) = \epsilon([1]) = 1, \qquad \epsilon([0, 1]) = 0.
\end{equation*} \vspace*{-6pt} \\
(2) The coproduct $\Delta \in \Hom \left( \cchains(\square^n), \cchains(\square^n)^{\otimes 2} \right)$ by
\begin{equation*}	
\Delta (x_1 \otimes \cdots \otimes x_n) = 	
\sum \pm \left( x_1^{(1)} \otimes \cdots \otimes x_n^{(1)} \right) \otimes 	
\left( x_1^{(2)} \otimes \cdots \otimes x_n^{(2)} \right),	
\end{equation*}	
where the sign is determined using the Koszul convention, and we are using Sweedler's notation
\begin{equation*}	
\Delta(x_i) = \sum x_i^{(1)} \otimes x_i^{(2)}
\end{equation*}
for the chain map $\Delta \colon \cchains(\square^1) \to \cchains(\square^1)^{\otimes 2}$ defined by
\begin{equation*}
\Delta([0]) = [0] \otimes [0], \quad \Delta([1]) = [1] \otimes [1], \quad \Delta([0, 1]) = [0] \otimes [0, 1] + [0, 1] \otimes [1].
\end{equation*}
Using that $\cchains(\square^n) = \cchains(\square^1)^{\otimes n}$, $\Delta$ is the composition
\begin{equation*}
\begin{tikzcd}
\cchains(\square^1)^{\otimes n} \arrow[r, "\Delta^{\otimes n}"] & \left( \cchains(\square^1)^{\otimes 2}  \right)^{\otimes n} \arrow[r, "sh"] & \left( \cchains(\square^1)^{\otimes n} \right)^{\otimes 2}
\end{tikzcd}
\end{equation*}
where $sh$ is the shuffle map that places tensor factors in odd position first. \vspace*{5pt} \\
(3) The product $\ast \in \Hom(\cchains(\square^n)^{\otimes 2}, \cchains(\square^n))$ by
\begin{align*}
(x_1 \otimes \cdots \otimes x_n) \ast (y_1 \otimes \cdots \otimes y_n) =
(-1)^{|x|} \sum_{i=1}^n x_{<i} \epsilon(y_{<i}) \otimes x_i \ast y_i \otimes \epsilon(x_{>i})y_{>i},
\end{align*}
where
\begin{align*}
x_{<i} & = x_1 \otimes \cdots \otimes x_{i-1}, &
y_{<i} & = y_1 \otimes \cdots \otimes y_{i-1}, \\
x_{>i} & = x_{i+1} \otimes \cdots \otimes x_n, & 
y_{>i} & = y_{i+1} \otimes \cdots \otimes y_n,
\end{align*}
with the convention
\begin{equation*}
x_{<1} = y_{<1} = x_{>n} = y_{>n} = 1 \in \Z,
\end{equation*}
and the only non-zero values of $x_i \ast y_i$ are
\begin{equation*}
\ast([0] \otimes [1]) = [0, 1], \qquad  \ast([1] \otimes [0]) = -[0, 1].
\end{equation*}

\begin{proposition}[\cite{bibid}] \label{thm: cubical chain bialgebra}
	For every $n \in \mathbb{N}$, the assignment
	\begin{equation*}
	\counit \mapsto \epsilon, \quad \coproduct \mapsto \Delta, \quad \product \mapsto \ast,
	\end{equation*}
	defines a natural $\mathcal M$-bialgebra structure on $\cchains(\square^n)$.
\end{proposition}

\begin{proof}
	We present a proof of this statement in \cite{??}.
\end{proof}

\anibal{extension to $\cSet$}

\begin{remark}
	The dashed arrow in
	\begin{equation*}
	\begin{tikzcd}
	\cSet \arrow[r] \arrow[dr, dashed] & \coAlg_{U(\M)} \arrow[d] \\
	& \coAlg
	\end{tikzcd}
	\end{equation*}
	agrees with the structure  defined by the Serre diagonal \cite{bibid}.
\end{remark}