\documentclass{amsart}
\usepackage{amssymb}
\usepackage{tikz-cd}
\usepackage{mathbbol}

% bibliography
\usepackage[backend=biber, style=alphabetic,
backref=true, url=false]{biblatex}
\addbibresource{aux/bibliography.bib}
\addbibresource{aux/einfinity.bib}
\addbibresource{aux/mypapers.bib}

% references
\usepackage[bookmarks=true, linktocpage=true,
bookmarksnumbered=true, breaklinks=true,
pdfstartview=FitH, hyperfigures=false,
plainpages=false, naturalnames=true,
colorlinks=true, pagebackref=false,
pdfpagelabels]{hyperref}
\hypersetup{
	colorlinks,
	citecolor=blue,
	filecolor=blue,
	linkcolor=blue,
	urlcolor=blue
}
\usepackage[capitalize, noabbrev]{cleveref}
\crefname{subsection}{\subsect\!}{subsections}
\let\subsect\S % copying the subsection symbol before overwriting it

% layout
\setlength{\textwidth}{\paperwidth}
\addtolength{\textwidth}{-2in}
\setlength{\textheight}{\paperheight}
\addtolength{\textheight}{-2in}
\calclayout

% update to MSC2020
\makeatletter
\@namedef{subjclassname@2020}{%
	\textup{2020} Mathematics Subject Classification}
\makeatother

% table of contents
\setcounter{tocdepth}{1}
\input{../aux/usualcmds}
\addbibresource{../aux/usualpapers.bib}

%%%%%%%%%%%%%%%%%%%%%%
% environments

% elements
\newcommand{\As}{{\mathcal{A}\mathsf{s}}}

% sets and spaces

% categories

% functions and functors
\DeclareMathOperator{\A}{\cA}
\DeclareMathOperator{\Ahat}{\mathcal{\widehat{A}}}
\DeclareMathOperator{\schainsAs}{N^{\simplex}_{\As}}
\DeclareMathOperator{\cchainsAs}{N^{\cube}_{\As}}
\DeclareMathOperator{\schainsUM}{N^{\simplex}_{\UM}}
\DeclareMathOperator{\cchainsUM}{N^{\cube}_{\UM}}
\DeclareMathOperator{\schainsUSL}{N^{\simplex}_{\USL}}
\DeclareMathOperator{\cchainsUSL}{N^{\cube}_{\USL}}
\DeclareMathOperator{\ccobar}{\mathbb{\Omega}}
\DeclareMathOperator{\ccobarE}{\widehat{\mathbb{\Omega}}}
\DeclareMathOperator{\ncobar}{\mathbb{\Omega}^{\mathrm{nec}}}
\DeclareMathOperator{\ncobarE}{\widehat{\mathbb{\Omega}}^{\mathrm{nec}}}
\DeclareMathOperator{\sS}{S^{\simplex}}
\DeclareMathOperator{\cS}{S^{\cube}}
\DeclareMathOperator{\CS}{\zeta}
\DeclareMathOperator{\rigid}{\mathfrak{C}}
\DeclareMathOperator{\nerve}{\mathfrak{N}}
\DeclareMathOperator{\classifying}{\overline{W}}
\DeclareMathOperator{\localization}{\mathcal{L}}
\DeclareMathOperator{\kan}{G}

% other

% comments
\newcommand{\manuel}[1]{\textcolor{red}{\underline{Manuel}: #1}}
 % add commands here
\addbibresource{../aux/bibliography.bib} % add references here
\usepackage{enumitem}
\setlist{label=\arabic{enumi}.,itemsep=\medskipamount, left=0pt}

%%%%%%%%%%%%%%%%%%%%%%
\title[Referee reply]{Response to the referee report of \\ Adams' cobar construction as a monoidal $E_{\infty}$-coalgebra model of the based loop space}

\newcommand{\ar}{\medskip\noindent\textit{Reply}:\ }
\renewcommand{\thesection}{\arabic{section}}
\newcommand{\rp}{\medskip\noindent}

\begin{document}
	\noindent\today
	\maketitle

	We would like to thank the reviewer for a careful and insightful analysis of our paper, and for the many suggestions improving its presentation.

	\section{Individual items}

	\noindent We now respond point by point to the referee report.

	\rp (1) Last line of page 4: The “no” should be a “not”?

	\ar: Changed as suggested, thank you.

	\rp (2) Section 2.3: I find it confusing the way the functors of chains are presented
	and given that it is quite important for the paper which types of chains
	precisely are being considered, one should be careful here.

	\rp (a) In the sentence “The functor of (normalized) simplicial chains [...]”, it seems that $N^\Delta$ is given as the composition of two functors factoring by Top (since the authors say realisation lands in Top), but cellular chains are defined on the category of cellular spaces (and of course, any
	given topological space has either zero or multiple cellular structures).
	I see what is meant, but not only this is strictly speaking wrong, it seems confusing and I don’t see why not give the more direct explicit construction.

	\rp (b) I have a similar remark concerning the singular chains, where the authors write explicitly a composition of functors.
	I do fear I might be missing something though.

	\rp (c) Also, I find it weird to write “(simplicial) singular chains”. Of course, later one realises simplicial is to distinguish from cubical, but in the context it is presented, it is more confusing than clarifying. I would	argue that “singular chains” is unambiguous and more clear than writing simplicial. Maybe with the rest of the paragraph more clear, this criticism goes away.

	\ar We have revised the introduction of the normalized chains and the singular chains functors based on the suggestions presented in (a), (b).
	It now reads:

	\begin{quote}
		The functor of (normalized) \textit{simplicial chains} $\schains \colon \sSet \to \Ch$ is defined on any simplicial set $X$ by first considering the chain complex $(\k[X], \partial)$, given on degree $n$ by the free $\k$-module generated by the $n$-simplices of $X$ with differential given by the alternating sum of face maps, and then modding out by the sub-chain complex of degenerate elements.
		We remark that this functor is naturally equivalent to the composition of the geometric realization and cellular chains functors with respect to the standard CW structure.
	\end{quote}

	Regarding (c), we prefer to keep the simplicial in parenthesis to emphasize the distinction between simplicial and cubical singular chains, which will play an important role in the paper.

	\rp (3) When defining the coproduct on $\chains(\triangle^n)$ the notation $[v_0,...,v_q]$ has not really been introduced.
	Maybe it would clarify a bit better what the constructions exactly are, if something else is said here.

	\ar This notation is introduced in the second paragraph of section 2.3.

	\rp (4) Section 2.4: I am not extremely familiar with cubical theory, and generally	speaking this is less known than simplicial theory.
	A reference for this section would be appreciated.

	\ar We have added the following references to publications covering the topic more thoroughly: \cite{grandis2003cubical,medina2022cube_einfty,medina2023flowing}.

	\rp (5) In the second diagram of this section, the authors talk about the obvious maps $\delta_i$, $\sigma$ and $\gamma \colon \mathbb{I}^1 \to \mathbb{I}^2$.
	While the others seem obvious, $\gamma$ is not at all to me. There are multiple topological maps sending the vertices 00 to 0 and 01, 10, 11 to 1.
	I suspect the authors intend to use the expression $\max(p, q)$, which does extend to the full square.

	\ar This has been appropriately revised by defining $\gamma$ explicitly. It now reads:
    \begin{quote}
     ... to the continuous maps
    \[
    \begin{tikzcd}
    	\gcube^0 \arrow[r, out=45, in=135, "\delta_0"] \arrow[r, out=-45, in=-135, "\delta_1"'] & \gcube^1 \arrow[l, "\sigma"'] &[-10pt] \arrow[l, "\gamma"'] \gcube^2,
    \end{tikzcd}
    \]
    where $\delta_i$ and $\sigma$ are the canonical cubical co-face and co-degeneracy maps, respectively, and $\gamma(s,t)=\text{max}(s,t)$.
    \end{quote}

	\rp (6) Assuming the lack of a reference, a little bit more should be said about the adjunction presented, or at least give the simple formula for Sing on objects.

	\ar References have been added to the beginning of this subsection.

	\rp (7) Same comment as 2 concerning the last paragraph of page 6.

	\ar This has been revised and an explicit formula for the differential is now given. It now reads:

	\begin{quote}
		The functor of (normalized) \textit{cubical chains} $\cchains \colon \cSet \to \Ch$ is the unique monoidal functor defined by assigning to $\cube^1$ the usual cellular chains of the interval $\gcube^1$.
		Explicitly, it is defined on any cubical set $Y$ by first considering the chain complex $(\k[Y], \delta)$, given on degree $n$ by the free $\k$-module generated by the $n$-cubes of $Y$ with differential $\delta$ given on any $n$-cube $y \in Y_n$ by
		\[
		\delta(y) = \sum_{i=1}^n (-1)^i
		\big(
		Y(\text{id}_2^{i-1} \times \delta_1 \times \text{id}_2^{n-i})(y) -
		Y(\text{id}_2^{i-1} \times \delta_0 \times \text{id}_2^{n-i})(y)
		\big),
		\]
		and then modding out by the sub-complex of degenerate cubes.
	\end{quote}

	\rp (8) Section 3.1: The based loop space are Moore loops, right?
	If so, the word ``Moore'' should appear somewhere for clarity, no?

	\ar This terminology has been added to the third sentence of section 3.1.

	\rp (9) Beginning of section 3.3: Strictly speaking, $\sSchainsA$ was not introduced, only $\schainsA$.

	\ar We have added the following sentence at the end of the first paragraph of section 3.3:
    \begin{quote}
         Here $\sSchainsA(\fX,x)$ denotes the dg coassociative coalgebra $\schainsA(\sSing(\fX,x))$ (using the notation introduced in 2.3) of (pointed) normalized singular chains.
    \end{quote}

	\rp (10) 5th line of page 9: “[t]he constant loop” has not been defined.
	There are multiple constant loops, we want the one of length zero.

	\ar This has now been clarified. We have indicated this in the first paragraph of page 8.
    It now reads:
    \begin{quote}
        There is a composition structure
        \begin{equation}
        \begin{split}
            P(\fX;x,x') \times P(\fX;x',x'') &\to P(\fX;x,x'') \\
            (\alpha, \beta) &\mapsto \alpha \cdot \beta
        \end{split}
        \end{equation}
        given by concatenation of paths with addition of parameters, whose identities are constant paths with $r=0$.
        We may assemble this data into a topologically enriched category $\mathcal{P}(\fX)$ that has the points of $\fX$ as objects, the spaces $P(\fX;x,x')$ as morphisms from $x$ to $x'$, concatenation of paths as composition, and constant paths with $r=0$ as identity morphisms.
    \end{quote}

	\rp (11) Section 3.5: I find in the description of the differential of $\Omega S^\Delta_{\mathcal A s}(X, x)$ “Generators that, in a certain sense, fit inside T ” is not clear at all.
	Also, the reader is expecting the differential to have two pieces and it seems that only one is mentioned, which is perhaps confusing.

	\ar We have rewritten this part to emphasize clarity. Note that this is only meant to be an informal introduction to a precise construction and statement (Theorem 3).

	\rp (12) Section 3.6: The first centered equation reads weirdly due to the way it is
	formatted.

	\ar We have reformatted it and it now display as follows:
	\[
	\set[\Big]{J \subseteq \set{0,\dots,n_1+\cdots+n_k-k} \mid 0,n_1+\cdots+n_k-k \in J}
	\]

	\rp (13) End of 3.6: In the last equation, the Yoneda $\yoneda$ was not introduced.
	Notice that the cubical set right before being called $Y$ makes it more confusing.
	($\yoneda$ is also being used later in the text.)

	\ar We have now mentioned that $\yoneda \colon \Nec \to \nSet$ stands for the Yoneda embedding.

	\rp (14) Last sentence of 3.6: “Since P is a monoidal,”

	\ar Thank you, we have fixed this typo.

	\rp (15) Last sentence of 3.6: Doesn’t the monoidality of P imply only the oplax monoidality of P!? In that case we wouldn’t be able to infer the sentence right before Remark 2.

	\ar The induced functor is clearly monoidal from its explicit description.

	\rp (16) End of 4.4: Exactly how does this $E_\infty$ structure generalizes those previously introduced in [MS03; BF04]?

	\ar The expanded sentence now reads:

	\begin{quote}
		We remark that this $E_\infty$-structure generalizes those previously introduced in \cite{mcclure2003multivariable,berger2004combinatorial} in the sense that any cooperation $\chains \to \chains^{\ot r}$ arising from the action of the Barratt-Eccles or surjection operad can be expressed as a cooperation arising from the action of $\UM$.
	\end{quote}

	\rp (17) Equation (7) would benefit from parenthesis. Formula is in principle ambiguous as it is written.

	\ar Changed as suggested.

	\rp (18) General remark for Section 4.4: When working with operads, the existence of a natural $\cO$-algebra structure on the tensor product of two $\cO$-algebras is equivalent to a morphism $\cO \to \cO \ot_H \cO$ where $\ot_H$ is the Hadamard (arity-wise) tensor product.
	It feels that something similar is being done here.
	Is there a prop structure on $\M \ot_H \M$? Are then the authors just defining a prop map $\Delta_\M \colon \M \to \M \ot_H \M$?

	\ar Yes, this is indeed what we are doing.

	\rp (19) Beginning of page 19: This is the first appearance of the phrase “Serre diagonal” besides the introduction.
	Previously Serre coalgebra was mentioned.
	Please clarify.

	\ar This has been revised for consistency. We have now clarified what we mean by ``Serre diagonal" in page 7 by adding the sentence:
    \begin{quote}
        We refer to this lift as the \textit{Serre coalgebra} structure on cubical chains, and to the coproduct $\Delta$ as the Serre diagonal.
    \end{quote}

	(20) Theorem 5 could be stated more clearly, either in the statement of the theorem or before.

	\ar It now reads:
	\begin{quote}
		We show that the monoidal structure on $\UM$-coalgebras recover the $E_\infty$-structure on cubical chains defined in 4.5.
		In fact, we have the following stronger statement at the level of $\M$-bialgebras.

		\medskip\noindent\textbf{Theorem 5}.
		\textit{For any $n \in \N$, the $\M$-bialgebra structure on $\chains(\cube^n)$ agrees with the monoidal extension of the $\M$-bialgebra structure on $\chains(\cube^1)$.}
	\end{quote}

	\rp (21) There is a loose x at the end of section 4.8.

	\ar Thank you, it is now fixed.

	\rp (22) As a general comment, I think it might be confusing to the non-linear reader to figure out which chains notation means what, between $N$, $S$, $\Sing$, $\As$ in subscript and $\triangle$, $\square$ in superscript.

    \ar We prefer to keep the current notations in order to:
    (a) emphasize the distinction between normalized chains on a simplicial/cubical set and singular simplicial/cubical chains on a topological space, and (b) emphasize how much structure is being taking into account at the level of chains, e.g. the notation $\As$ means we take into account the dg coassociative coalgebra structure, while $\UM$ means we take into account the $E_{\infty}$-coalgebra structure.
    These distinctions are crucial throughout the paper.

	\section{Other changes}

	\noindent (1) AMM's affiliation has been updated.

	\sloppy
	\printbibliography
\end{document}