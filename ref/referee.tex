\documentclass{amsart}
\usepackage{amssymb}
\usepackage{tikz-cd}
\usepackage{mathbbol}

% bibliography
\usepackage[backend=biber, style=alphabetic,
backref=true, url=false]{biblatex}
\addbibresource{aux/bibliography.bib}
\addbibresource{aux/einfinity.bib}
\addbibresource{aux/mypapers.bib}

% references
\usepackage[bookmarks=true, linktocpage=true,
bookmarksnumbered=true, breaklinks=true,
pdfstartview=FitH, hyperfigures=false,
plainpages=false, naturalnames=true,
colorlinks=true, pagebackref=false,
pdfpagelabels]{hyperref}
\hypersetup{
	colorlinks,
	citecolor=blue,
	filecolor=blue,
	linkcolor=blue,
	urlcolor=blue
}
\usepackage[capitalize, noabbrev]{cleveref}
\crefname{subsection}{\subsect\!}{subsections}
\let\subsect\S % copying the subsection symbol before overwriting it

% layout
\setlength{\textwidth}{\paperwidth}
\addtolength{\textwidth}{-2in}
\setlength{\textheight}{\paperheight}
\addtolength{\textheight}{-2in}
\calclayout

% update to MSC2020
\makeatletter
\@namedef{subjclassname@2020}{%
	\textup{2020} Mathematics Subject Classification}
\makeatother

% table of contents
\setcounter{tocdepth}{1}
\input{../aux/usualcmds}
\addbibresource{../aux/usualpapers.bib}

%%%%%%%%%%%%%%%%%%%%%%
% environments

% elements
\newcommand{\As}{{\mathcal{A}\mathsf{s}}}

% sets and spaces

% categories

% functions and functors
\DeclareMathOperator{\A}{\cA}
\DeclareMathOperator{\Ahat}{\mathcal{\widehat{A}}}
\DeclareMathOperator{\schainsAs}{N^{\simplex}_{\As}}
\DeclareMathOperator{\cchainsAs}{N^{\cube}_{\As}}
\DeclareMathOperator{\schainsUM}{N^{\simplex}_{\UM}}
\DeclareMathOperator{\cchainsUM}{N^{\cube}_{\UM}}
\DeclareMathOperator{\schainsUSL}{N^{\simplex}_{\USL}}
\DeclareMathOperator{\cchainsUSL}{N^{\cube}_{\USL}}
\DeclareMathOperator{\ccobar}{\mathbb{\Omega}}
\DeclareMathOperator{\ccobarE}{\widehat{\mathbb{\Omega}}}
\DeclareMathOperator{\ncobar}{\mathbb{\Omega}^{\mathrm{nec}}}
\DeclareMathOperator{\ncobarE}{\widehat{\mathbb{\Omega}}^{\mathrm{nec}}}
\DeclareMathOperator{\sS}{S^{\simplex}}
\DeclareMathOperator{\cS}{S^{\cube}}
\DeclareMathOperator{\CS}{\zeta}
\DeclareMathOperator{\rigid}{\mathfrak{C}}
\DeclareMathOperator{\nerve}{\mathfrak{N}}
\DeclareMathOperator{\classifying}{\overline{W}}
\DeclareMathOperator{\localization}{\mathcal{L}}
\DeclareMathOperator{\kan}{G}

% other

% comments
\newcommand{\manuel}[1]{\textcolor{red}{\underline{Manuel}: #1}}
 % add commands here
\addbibresource{../aux/bibliography.bib} % add references here
\usepackage{enumitem}
\setlist{label=\arabic{enumi}.,itemsep=\medskipamount, left=0pt}

%%%%%%%%%%%%%%%%%%%%%%
\title[Referee reply]{Response to referee report \\ Adams' cobar construction as a monoidal $E_{\infty}$-coalgebra model of the based loop space}

\newcommand{\ar}{\medskip\noindent\textit{Reply}:\ }
\renewcommand{\thesection}{\arabic{section}}

\begin{document}
	\noindent\today
	\maketitle

	We would like to thank the reviewer for a careful and insightful analysis of our paper, and for the many suggestions improving its presentation. We respond to each of the items in the reviewer's report. 
	%We copy their report for completeness.

	\section{Reviewer's summary}
(Should we address the incorrect initial comment about the homology of the based loop space being determined by the homology of the underlying space, or should we just not comment on this to accelerate the process?)

	\section{Reviewer's individual items}



	\begin{enumerate}
	\item fixed
 \item  We have revised the introduction of the normalized chains and the singular chains functors based on these suggestions in (a), (b), and (c).
  We prefer to keep the simplicial in parenthesis to emphasize the distinction between simplicial and cubical singular chains, which will play an important role in the paper. 
 \item This notation is introduced in the second paragraph of section 2.3. 
 \item (insert an approriate reference, perhaps Jardine's ``Categorical homotopy theory" or one of Anibal's papers on cubical sets)
 \item This has been appropriately revised.
 \item (maybe add Jardine's ``Categorical homotopy theory" section 3)
 \item  This has been revised and an explicit formula for the differential is now given. 
 \item This terminology has been added to the third sentence of section 3.1.
 \item This notation has been defined.
 \item This has been clarified. 
 \item We have rewritten this part to emphasize clarity. Note that this is only meant to be an informal introduction to a precise construction and statement (Theorem 3). 
 \item (not sure I followed this comment)
 \item (Should we define the Yoneda embedding?)
 \item fixed
 \item (Not sure I followed this comment, but the induced functor is clearly monoidal from its direct description). 
 \item (Can you add a sentence about this? Thank you)
\item fixed
\item (Is this what we are doing?)
 \item This has been revised for consistency. 
 \item (This statement should be revised, as it currently stands it sounds funny)
 \item fixed
 \item (how should we address this comment?)
 
 
 
	\end{enumerate}

	\section{Other changes}

\end{document}