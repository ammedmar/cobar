
\section{Adams's map as an $E_{\infty}$-bialgebra quasi-isomorphism}

The goal of this section is to prove \cref{t:1st main thm in the intro}. We do so by reformulating Baues construction of a cubical model for the based loop space in categorical terms. Then we use the natural $\UM$-coalgebra structure of the chains on a cubical set and its monoidal properties.


\subsection{Adams' map}

In \cite{adams1956cobar}, Adams constructed a natural map of algebras
\begin{equation} \label{e:adams map 2}
\theta_Z \colon \cobar(\sS(Z, z)) \to S^{\cube}(\Omega_z Z),
\end{equation}
for any pointed topological space $(Z, z)$.
The construction is based on a collection of continuous maps $$\theta_n \colon \gcube^{n-1} \to P(\gsimplex^n;0,n),$$
where $P(\gsimplex^n;0,n)$ denotes the topological space of Moore paths in $\gsimplex^n$ from the $0$-th vertex $v_0$ to the $n$-th vertex $v_n$.
These maps are constructed inductively so that following equations are satisfied:
\begin{enumerate}
	\item $\theta_1(0)\colon \gcube^1 \to \gsimplex^1$ is the path $\theta_1(0)(s) = sv_1 +(1-s)v_0$.
	\item $\theta_n \circ e_j^0 = P(d_j) \circ \theta_{n-1}$.
	\item $\theta_n \circ e_j^1 = (P(f_j) \circ \theta_j) \cdot (P(l_{n-j}) \circ \theta_{n-j})$.
\end{enumerate}
In the above equations $f_j\colon \gsimplex^j \rightarrow \gsimplex^n$ and $l_{n-j}\colon \gsimplex^{n-j} \rightarrow \gsimplex^n$ denote the first $j$-th face map and last $(n-j)$-th face map of $\gsimplex^n$, respectively, and
$e_j^0,e_j^1\colon \gcube^{n-1} \rightarrow \gcube^{n}$ denote the $j$-th bottom and top face inclusion maps, respectively.
For any continuous map of spaces $f \colon Z \to Z'$, we denote by $P(f) \colon P(Z) \to P(Z')$ the induced map at the level of spaces of paths. Given any two composable paths $\alpha$ and $\beta$, the dot symbol in $\alpha \cdot \beta$ denotes the composition (or concatenation) of paths.

Adams' map $\theta_Z$ is now given as follows.
For any singular $1$-simplex $\sigma \in S^{\Delta}_1(Z, z)$ let
$$\theta_Z([\sigma])= P(\sigma) \circ \theta_1 - c_z,$$
where $c_z \in S^{\cube}_0(\Omega_z Z)$ is the singular $0$-cube determined by the constant loop at $z \in Z.$ For any singular $n$-simplex $\sigma \in S^{\Delta}_n(Z, z)$ with $n>1$, let
$$\theta_Z([\sigma])= P(\sigma) \circ \theta_n.$$ Since the underlying graded algebra of $\cobar(\sS(Z, z))$ is free, we may extend the above to an algebra map $\theta_Z \colon \cobar(\sS(Z, z)) \to S^{\cube}(\Omega_z Z)$.
The compatibility equations for the maps $\theta_n$ imply that $\theta_Z$ is a chain map.

We will construct an $E_{\infty}$-bialgebra structure on $\cobar(\sS(Z, z))$ and show that $\theta_Z$ preserves this structure.
In order to do this, we shall describe a categorical version of Adams' constructions.

\subsection{Necklace category}

We construct a cubical model for the based loop space using necklical sets, a notion related to both simplicial sets and cubical sets with connections.
Different approaches to this framework can be found in \cite{baues1998hopf}, \cite{galvez2020hopf}, \cite{dugger2011rigidification}, \cite{rivera2018cubical}, \cite{rivera2019path}, among other works.

Let $\simplex_{*,*}$ be the subcategory of the simplex category $\simplex$ with the same objects and morphisms being functors $f \colon [n] \to [m]$ satisfying $f(0) = 0$ and $f(n) = m$.
The category $\simplex_{*,*}$ is a strict monoidal category when equipped with the monoidal structure $[n] \otimes [m] = [n+m]$, thought of as identifying the elements $n \in [n]$ and $0 \in [m]$, and unit given by $[0]$.

Denote by
\begin{equation*}
\barConst \colon \Mon_{\Cat} \to \Fun[\simplex^{op}, \Mon_{\Cat}]
\end{equation*}
the bar construction from the category of monoidal categories, i.e., monoids in $\Cat$, to simplicial objects in $\Mon_{\Cat}$ and by
\begin{equation*}
\bars{\ \cdot \ } \colon \Fun[\simplex^{op}, \Mon_{\Cat}] \to \Mon_{\Cat}
\end{equation*}
the realization functor.

We define the \textit{necklace category} to be the monoidal category
\begin{equation*}
\Nec = \bars{\barConst(\simplex_{*,*})}.
\end{equation*}
More explicitly, the set of objects of $\Nec$, called \textit{necklaces}, is given by
\begin{equation*}
\big\{ [n_1] \vee \cdots \vee[n_k] \mid n_i, k \in \N_{>0} \big\}
\end{equation*}
together with $[0]$, serving as the unit of the monoidal structure.

The morphism of $\Nec$ are generated by the following four types of morphisms:
\begin{enumerate}
	\item $\partial^j \colon [n-1] \to [n]$ for $j = 1, \dots, n-1$
	\item $\Delta_{[j], [n-j]} \colon  [j] \vee [n-j] \to [n]$ for $j = 1, \dots, n-1$
	\item $s^j \colon [n+1] \to [n]$ for $j = 0, \dots, n$ and $n>0$, and 
	\item $s^0 \colon [1] \to [0]$.
\end{enumerate}
The monoidal structure of $\Nec$ is given by concatenation of ordered sequences, denoted by $T \vee T'$ for any two necklaces $T, T' \in \Nec$.
Given any necklace $T = [n_1] \vee \dots \vee[n_k] \in \Nec$ define the \textit{dimension} of $T$ by $\text{dim}(T)=n_1 + \dots + n_k-k$.

\subsection{Necklical sets}

The category of \textit{necklical sets} is the functor category $\nSet=\Fun[\Nec^{\op}, \Set].$

We denote the Yoneda embedding by $$\mathcal{Y} \colon \Nec \to \nSet,$$ i.e. the functor induced by $T \mapsto \Nec(-,T).$

We denote by $$\mathcal{S} \colon \Nec \to \sSet$$ the functor induced by sending any necklace $T=[n_1] \vee \dots \vee[n_k] \in \Nec$ to the simplicial set $$\mathcal{S}(T)= \simplex^{n_1} \vee\dots \vee \simplex^{n_k} \in \sSet,$$ where the wedge symbol now means we identify the last vertex of $\simplex^{n_i}$ with the first vertex of $\simplex^{n_{i+1}}$ for $i=1,\dots,k-1$.

The category of necklical sets $\nSet$ is a (non-symmetric) monoidal category when equipped with the Day convolution monoidal structure $\otimes\colon \nSet \times \nSet \to \nSet$
$$K \otimes K' = \colim_{\substack{\mathcal{Y}(T) \to K, \\ \mathcal{Y}(T')\to K'}} \mathcal{Y}(T \vee T'),$$ as described in
 \cref{ss:day convolution}.

\subsection{From necklaces to cubes}

We now describe a categorical version of Adams' maps
\begin{equation*}
\theta_n \colon \gcube^{n-1} \to P(\gsimplex^{n};0,n)
\end{equation*}
by constructing a monoidal functor
\begin{equation*}
\mathcal{P} \colon \Nec \to \cube.
\end{equation*}

We first set $\mathcal{P}([0])=2^0$.On any other necklace $T \in \Nec$ we define $\mathcal{P}( T )= 2^{\text{dim}(T)}$.
In order to define $\mathcal{P}$ on morphisms, it is sufficient to consider the following cases.
\begin{enumerate}
	\item For any $\partial^j \colon [n] \to [n+1]$ such that $0< j<{n+1}$, define $\mathcal{P}(\partial^j) \colon 2^{n-1}\to 2^{n}$ to be the cubical coface functor $\mathcal{P}(f)= \delta_0^{j}.$ 
	
	\item For any $\Delta_{[j], [n+1-j]} \colon [j] \vee [n+1-j] \to [n+1]$ such that $0<j<n+1$, define $\mathcal{P}(\Delta_{[j], [n+1-j]}) \colon 2^{n-1}\to 2^{n}$ to be the cubical coface functor $\mathcal{P}(f)=\delta_1^{j}$.
	
	\item We now consider morphisms of the form $s^j \colon [n+1] \to [n]$ for $n>0$.
	If $j=0$ or $j=n$, then $\mathcal{P}(f) \colon 2^n \to 2^{n-1}$ is defined to be the cubical codegeneracy functor $\mathcal{P}(s^j)= \varepsilon^{j}.$ If $0<j<n$, then we define $\mathcal{P}(s^j) \colon 2^n \to 2^{n-1}$ to be the cubical coconnection functor $\gamma^{j}.$
	
	\item For $s^0 \colon [1] \to [0]$ we define $\mathcal{P}(s^0) \colon 2^0 \to 2^0$ to be the identity functor.
\end{enumerate}

\begin{remark}
	The functor $\mathcal{P}$ is neither faithful or full.
	However, for any necklace $T' \in \Nec$ with $\text{dim}(T')=n+1$ and any cubical coface functor $\delta_{\epsilon}^j \colon 2^n \to 2^{n+1}$ for $0 \leq j \leq n+1$, there exists a unique pair $(T, f \colon T \hookrightarrow T')$, where $T \in \Nec$ with $\text{dim}(T)=n$ and $f \colon T \hookrightarrow T'$ is an injective morphism in $\Nec$, such that $\mathcal{P}(f) = \delta_{\epsilon}^j $.
\end{remark}

The functor $\mathcal{P} \colon \Nec \to \cube$ induces an adjunction between $\cSet$ and $\nSet$ with right and left adjoint functors given respectively by
\begin{equation*}
\mathcal{P}^* \colon \cSet \to \nSet,
\qquad \text{and} \qquad
\mathcal{P}_{!}  \colon \nSet \to \cSet.
\end{equation*}
Explicitly, for a cubical set $Y \colon \cube^\op \to \Set$,
\begin{equation*}
\mathcal{P}^*(Y)= Y \circ \mathcal{P}^\op,
\end{equation*}
and for a necklical set $K \colon \Nec^\op \to \Set$,
\begin{equation*}
\mathcal{P}_{!}(K) \ =
\colim_{\mathcal{Y}(T) \to K} \mathcal{P}(T) \ \cong 
\colim_{\mathcal{Y}(T) \to K} \cube^{\text{dim}(T)}.
\end{equation*}
Since $\mathcal{P}$ is a monoidal functor, the functor $\mathcal{P}_{!} \colon \nSet \to \cSet$ is a monoidal as well.

\subsection{Cubical cobar construction}

Using the framework of necklical sets, we may reinterpret a construction of Baues as a functor
\begin{equation*}
\ncobar \colon \sSet^0 \to \Mon_{\nSet},
\end{equation*}
which we now define.

For any reduced simplicial set $X$, the underlying necklical set $\ncobar(X) \colon \Nec^\op \to \Set$ is defined by
\begin{equation*}
\ncobar(X) = \colim_{\mathcal{S}(T) \to X} \mathcal{Y}(T).
\end{equation*}
The monoidal structure
$$\ncobar(X) \otimes \ncobar(X) \to \ncobar(X)$$
is induced by the monoidal structure $\vee \colon \Nec \times \Nec \to \Nec$ together with the monoidal structure $\otimes \colon \nSet \times \nSet \to \nSet$ on necklicals sets.
More precisely, for any $S, S' \in \Nec$, the product $$\ncobar(X)(S) \otimes \ncobar(X)(S') \to \ncobar(X)(S \vee S')$$ is given by $$[f\colon \mathcal{S}(T) \to X, a] \otimes [f'\colon \mathcal{S}(T') \to X, a'] \mapsto [f \vee g\colon \mathcal{S}(T\vee T') \to X, a \otimes  a'],$$
where $(f\colon \mathcal{S}(T) \to X), (f'\colon \mathcal{S}(T') \to X) \in \mathcal{S} \downarrow X$, $a\in \mathcal{Y}(T)(S)$, and $a'\in \mathcal{Y}(T')(S')$.

We may now define the \textit{cubical cobar construction}
$$\ccobar \colon \sSet^0 \to \Mon_{\cSet}$$ as the composition $$\ccobar= \mathcal{P}_! \circ \ncobar.$$ This reinterpretation of Baues construction in terms of cubical sets with connections was also studied in \cite{rivera2018cubical}.

\subsection{Relation to the cobar construction}

The cubical cobar functor $\ccobar\colon \sSet^0 \to \Mon_{\cSet}$ is related to the cobar construction $\mathbf{\Omega}\colon \coAlg \to \Mon_{\Ch}$ via normalized chains as follows.

\begin{proposition} \label{p:ccobar and cobar}
	There is a natural isomorphisms of functors 
	\begin{equation*}
	\cchains \circ \ccobar \cong \mathbf{\Omega} \circ \schains \colon \sSet^0 \to \Mon_{\Ch}.
	\end{equation*}
\end{proposition}

\begin{proof} 
	Denote by $\iota_n \in (\cube^n)_n$ the top dimensional non-degenerate element of the standard $n$-cube with connections $\cube^n$.
	Note that for any $X \in \sSet^0$, we may represent any non-degenerate $n$-cube $\alpha \in (\mathcal{P}_!(\ncobar(X)))_n$ as a pair $\alpha=[\sigma \colon \mathcal{Y}(T) \to X, \iota_n]$ for some $T=[n_1] \vee \dots \vee [n_k] \in \Nec$ with $\text{dim}(T)=n_1+ \dots+n_k-k=n.$
	
	To define an algebra map
	\begin{equation*}
	\varphi_X \colon \cchains(\mathcal{P}_!(\ncobar(X))) \xra{\cong} \mathbf{\Omega}(\schains(X))
	\end{equation*}
	it suffices to define it on any generator of the form $\alpha=[\sigma \colon \simplex^{n+1} \to X, \iota_{n}]$, i.e., when $T$ is of the form $T=[n+1]$, for some $n\geq0$.
	If $n=0$ let $\varphi_X(\alpha)= [\overline{\sigma}]+ 1_R$, where $[\overline{\sigma}] \in s^{-1} \overline{ \schains(X)} \subset \mathbf{\Omega}(\schains(X))$ denotes the (length $1$) generator in the cobar construction of $\schains(X)$ determined by $\sigma \in X_{n+1}$ and $1_R$ denotes the unit of the underlying ring $R$.
	If $n>0$, we let $\varphi_X(\alpha)=[\overline{\sigma}]$.
	A straightforward computation yields that this gives rise to a well defined isomorphism of algebras, which is compatible with the differentials, and natural with respect to maps of simplicial sets.
\end{proof}

\subsection{An $E_{\infty}$-bialgebra structure on the cobar construction} \label{ss:e-infty on cobar}

Using \cref{p:ccobar and cobar} together with \cref{t:cubical e-infty chains are monoidal} we obtain a natural $E_{\infty}$-bialgebra structure on $\cobar(\schains(X))$ extending Baues' coproduct for any cubical set with connections $X$.
Namely, using the observation that $\cchainsUM \colon \cSet \to \coAlg_{\UM}$ is monoidal, the desired $E_{\infty}$-bialgebra structure is provided by the composition 
$$\cchainsUM \circ \ccobar \colon \sSet^0 \to \Mon_{\U(M)}.$$

\subsection{Factorization of Adams' map} \label{factorization}

Adams' map can now be factored as 
$$\theta_Z \colon \cobar(\sS(Z, z)) \xrightarrow{\cong} 
\cchains(\ccobar \sSing(Z, z)) \xrightarrow{\cchains(\Theta)} 
S^{\cube}(\Omega_z Z).$$
The first map is given by the natural isomorphism of algebras $$\cobar \schains (\sSing(Z, z)) \cong  \cchains \ccobar (\sSing(Z, z))$$ of \cref{p:ccobar and cobar}.
The second map is given by applying the cubical chains functor $\cchains \colon \Mon_{\cSet} \to \Mon_{\Ch}$ to the map of monoidal cubical sets
$$\Theta \colon \ccobar (\sSing(Z, z) )\to \cSing( \Omega_z Z)$$
determined by sending any singular $n$-simplex $$(\sigma\colon \gsimplex^n \to Z) \in \sSing(Z, z)$$ to the singular $(n-1)$-cube $$(P(\sigma) \circ \theta_n \colon \gcube^{n-1} \to \Omega_z Z) \in \cSing( \Omega_z Z).$$ 

\subsection{Proof of Theorem 1}

We now put together the constructions and results discussed above to deduce our first main theorem, stated in the introduction as \cref{t:1st main thm in the intro} and restated here for convenience.

\begin{nntheorem}
	The functor $\cchainsUM$ is monoidal and its associated functor on monoids fits in the following diagram commuting up to a natural isomorphism:
	\begin{equation*}
	\begin{tikzcd}
	& \Mon_{\coAlg_\UM} \arrow[d] \\
	\Mon_{\cSet} \arrow[ru, "\cchainsUM", out=70, in=180] \arrow[r, "\cchainsAS"]
	& \Mon_{\coAlg} \arrow[d] \\
	\sSet^0 \arrow[r, "\cobar \circ \schains"] \arrow[u, "\ccobar"]
	& \Mon_{\Ch}.
	\end{tikzcd}
	\end{equation*}
	Furthermore, when $(Z, z)$ is a pointed space, Adams' comparison map $\theta_Z \colon \cobar(\sS(Z, z)) \to S^{\cube}(\Omega_z Z)$ is a quasi-isomorphism of $E_{\infty}$-bialgebras or, more specifically, of monoids in $\coAlg_\UM$.
\end{nntheorem} 

\begin{proof}
	The fact that $\cchainsUM$ is monoidal and that it fits in the desired commutative diagram follows directly from Section 4.
	    
	The natural $\UM$-coalgebra structure on $\cobar(\sS(Z, z))$ is induced by $\cchainsUM ( \ccobar (\sSing(Z, z)) )$, as discussed in \cref{ss:e-infty on cobar}.
	The natural $\UM$-coalgebra structure on $S^{\cube}(\Omega_z Z)$ is precisely given by $\cchainsUM( \cSing(\Omega_z Z) ).$ These two $\UM$-coalgebra structures are compatible with the algebra structures, giving rise to monoinds in $\coAlg_{\UM}.$
	     
	By naturality, we obtain a map
	\begin{equation*}
	\cobar(\schains(\sSing(Z, z))) \cong
	\cchainsUM \big( \ccobar(\sSing(Z, z)) \big) \xra{\cchainsUM(\Theta)}
	\cchainsUM \big( \cSing(\Omega_z Z) \big)
	\end{equation*}
	of monoids in $\coAlg_{\UM}$, lifting Adams' classical map by \cref{factorization}, as desired.
	Since $\sSing(Z, z)$ is a fibrant reduced simplicial set, Adams' map $\theta_Z$ is a quasi-isomorphism, as proven in \cite{rivera2018cubical} and \cite{rivera2019path}. 
\end{proof}

%In this section we revisit two combinatorial models for the based loop space of a reduced simplicial set $X \in \sSet^0$.
%First we discuss a cubical model which has the feature that after taking cubical chains we obtain a algebra which is naturally \textit{isomorphic} to the cobar construction on the connected coalgebra of simplicial normalized chains $\schains(X)$.
%This is a reinterpretation of a construction of Baues \cite{baues1998hopf}.
%We also describe a version of the model which is localized at the $1$-simplices of $X$ and recovers the extended cobar construction defined in \cite{hess2010cobar} after applying cubical chains.
%This localization step is required in order to obtain the correct homotopy type of the based loop space in the non-simply connected, non-fibrant setting.
%The construction of the cubical model for the based loop space is best described using necklical sets, a notion related to both simplicial sets and cubical sets with connections.
%Different approaches to this framework can be found in \cite{baues1998hopf}, \cite{galvez2020hopf}, \cite{dugger2011rigidification}, \cite{rivera2018cubical}, \cite{rivera2018cubical}, among other works.
%We start the section by reviewing this framework.
%We finish the section by reviewing the classical Kan loop group functor and relating it to the cubical model for the based loop space.