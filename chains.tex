\subsection{Normalized chains} Let $R$ be a commutative ring and denote by $\mathsf{dgCoalg}_R$ the category of differential graded coassociative $R$-coalgebras. We will write $\otimes$ for $\otimes_R$.

We recall the definitions of the simplicial and cubical versions of the differential graded (dg) coalgebra of normalized chains. The \textit{simplicial normalized chains functor} $$N^{\Delta}_\bullet: \sSet \to \mathsf{dgCoalg}_R$$ is defined as follows. For any simplicial set $S$, the underlying graded $R$-module $N^{\Delta}_\bullet(S)$ is given by $N^{\Delta}_n(S)= R[S_n] / D_{\Delta} (S_n)$, the quotient of the free $R$-module generated by the set $S_n$ modulo the submodule $D_{\Delta} (S_n)$ generated by the degenerate elements in $S_n$. The differential $\partial: N^{\Delta}_n(S) \to N^{\Delta}_{n-1}(S)$ is induced by the usual simplicial chains boundary map given by $\partial(\sigma)= \sum_{j=0}^n (-1)^j \partial_j(\sigma)$ for any $\sigma \in S_n.$ The coproduct $\Delta: N^{\Delta}_\bullet(S) \to N^{\Delta}_\bullet(S) \otimes N^{\Delta}_\bullet(S)$ is induced by the Alexander-Whitney diagonal approximation
$\Delta(\sigma)= \sum_{j=0}^n \sigma(0,...,j) \otimes \sigma(j,...,n),$
where we denote by $\sigma(v_1,...,v_k) \in S_k$ the $k$-simplex obtained by restricting $\sigma \in S_n$ to the vertices $v_1,...,v_k$ of $\sigma$.
The triple $(N^{\Delta}_\bullet(S), \partial, \Delta)$ is a dg coassociative $R$-coalgebra which is natural with respect to maps of simplicial sets. 


The \textit{cubical normalized chains functor} $$N^{\square}_\bullet: \cSet \to \mathsf{dgCoalg}_R$$ is defined as follows. For any $C \in \cSet$, the underlying graded $R$-module $N^{\square}_\bullet(C)$ is given by $N^{\square}_n(C)= R[C_n] / D_{\square}(C_n)$, the quotient of the free $R$-module generated by the set $C_n$ modulo the submodule $D_{\square}(C_n)$ generated by those elements in $C_n$ that are images of degeneracies and connections. The differential $d: N^{\square}_n(C) \to N^{\square}_{n-1}(C)$ is induced by the usual cubical chains boundary map given by $d(\alpha)= \sum_{j=0}^n (-1)^j (\delta^1_j(\alpha)- \delta^0_j(\alpha))$ for any $\alpha \in C _n.$ The coproduct $\nabla: N^{\square}_\bullet(C) \to N^{\square}_\bullet(C) \otimes N^{\square}_\bullet(C)$ is induced by the Serre diagonal map, which is given as follows. For any $\alpha \in C_0$ define $\nabla(\alpha)= \alpha \otimes \alpha.$ For any $\alpha \in C_1$ define $\nabla(\alpha)= \delta^1_0(\alpha)\otimes \alpha + \alpha \otimes \delta^0_0(\alpha)$.
This determines a coassociative coproduct $\nabla: N^{\square}_\bullet(C) \to N^{\square}_\bullet(C) \otimes N^{\square}_\bullet(C)$ compatible with the monoidal structure on $\cSet$ through the identification $$N^{\square}_\bullet(C) \cong \colim_{\square^n_c \to C} N^{\square}_\bullet (\square^n_c) \cong  \colim_{\square^n_c \to C} N^{\square}_\bullet ((\square^1_c)^{\otimes n})\cong  \colim_{\square^n_c \to C}  N^{\square}_\bullet(\square^1_c)^{\otimes n}.$$ The triple $(N^{\square}_\bullet(C), d, \nabla)$ is a dg coassociative $R$-coalgebra which is natural with respect to maps of cubical sets with connections. 

%The functor $N_\bullet$ of \textit{normalized chains} is defined as follows. The chain complex $N_\bullet(\square^1)$ is isomorphic to the cellular chain complex of the interval
%\begin{equation*}
%\begin{tikzcd} [column sep = small, row sep = 0.1pt]
%\Z\{[0], [1]\}  & \arrow[l] \Z\{[0,1]\} \\
%\end{tikzcd}
%\end{equation*}
%Set
%\begin{equation} \label{eq: chains on I^n}
%N_\bullet(\square^n) = N_\bullet(\square^1)^{\otimes n}
%\end{equation}
%and define
%\begin{equation*}
%N_\bullet(X) = \colim_{\square^n \to X} N_\bullet(\square^n).
%\end{equation*}

%The functor of \textit{normalized cochains} $N^\bullet$ is defined by composing $N_\bullet$ with the linear duality functor $\Hom(-, \Z)$.

%The \textit{Serre diagonal} $\Delta \colon N_\bullet(X) \to N_\bullet(X) \otimes N_\bullet(X)$ is defined as follows. Let $\Delta \colon \chains(\square^1) \to \chains(\square^1)^{\otimes 2}$ be defined on basis elements by	
%\begin{gather*}	
%\Delta([0]) = [0] \otimes [0], \qquad 
%\Delta([1]) = [1] \otimes [1], \qquad
%\Delta([0, 1]) = [0] \otimes [0, 1] + [0, 1] \otimes [1].	
%\end{gather*}	
%Then, let
%$\Delta \colon \chains(\square^n) \to \chains(\square^n)^{\otimes 2}$
%be the composite
%\begin{equation*}
%%\chains(\square^1)^{\otimes n} \arrow[r, "\Delta^{\otimes n}"] & \left( \chains(\square^1)^{\otimes 2} \right)^{\otimes n} \arrow[r, "sh"] & \left( \chains(\square^1)^{\otimes n} \right)^{\otimes 2},
%\end{tikzcd}
%\end{equation*}
%where $sh$ is the shuffle map that reorders tensor factors so that those in odd positions occur first. More explicitly, using Sweedler's notation, we have that if $x_i^{(1)}$ and $x_i^{(2)}$ are defined through the identity
%\begin{equation*}	
%\Delta(x_i) = \sum x_i^{(1)} \otimes x_i^{(2)},
%\end{equation*}	
%then,
%\begin{equation} \label{E: Delta}	
%\Delta (x_1 \otimes \cdots \otimes x_n) = 	
%\sum \pm \left( x_1^{(1)} \otimes \cdots \otimes x_n^{(1)} \right) \otimes 	
%\left( x_1^{(2)} \otimes \cdots \otimes x_n^{(2)} \right),
%\end{equation}	
%where the sign is determined by the Koszul convention.

There is an action of $\S_n$ on $N^{\square}_\bullet(\square^n_c)$ given by permuting the factors of $\chains(\square^1_c)^{\otimes n}$.
The Serre diagonal is equivariant with respect to this action in the following sense.

\begin{proposition} \label{p:serre diagonal invariant}
	Let $T$ be the braiding on the symmetric monoidal category $(\Ch, \otimes, \Z)$. The following diagram commutes
	\begin{equation*}
	\begin{tikzcd}
	\chains(\square^1_c)^{\otimes 2} \arrow[r, "\Delta"] \arrow[d, "T"] &
	\chains(\square^1_c)^{\otimes 2} \otimes \chains(\square^1_c)^{\otimes 2} \arrow[d, "T \otimes T"] \\
	\chains(\square^1_c)^{\otimes 2} \arrow[r, "\Delta"] &
\chains(\square^1_c)^{\otimes 2} \otimes \chains(\square^1_c)^{\otimes 2}.
	\end{tikzcd}
	\end{equation*}
\end{proposition}

\begin{proof}
	This can be established by a straightforward verification. More conceptually, the reason behind this proposition is that in $\S_4$ one has $(12)(34)(23) = (23)(13)(24)$.
\end{proof}