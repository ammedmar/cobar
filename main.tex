\documentclass{amsart}
\usepackage{amsmath}

% hyperref
\usepackage[bookmarks=true, linktocpage=true,
bookmarksnumbered=true, breaklinks=true,
pdfstartview=FitH, hyperfigures=false,
plainpages=false, naturalnames=true,
colorlinks=true, pagebackref=true,
pdfpagelabels]{hyperref}
\hypersetup{
	colorlinks,
	citecolor=blue,
	filecolor=blue,
	linkcolor=blue,
	urlcolor=blue
}

% layout
\setlength{\textwidth}{\paperwidth}
\addtolength{\textwidth}{-1.85in}
\setlength{\textheight}{\paperheight}
\addtolength{\textheight}{-2in}
\calclayout

% Updating to MSC2020
\makeatletter
\@namedef{subjclassname@2020}{%
	\textup{2020} Mathematics Subject Classification}
\makeatother

% Commands
\newcommand{\Z}{\mathbb Z}

% Theorem environments

% Revision
\newcommand{\rev}[1]{\textcolor{red}{#1}} % red/black

\begin{document}
\title{An $E_\infty$-Hopf structure on the cobar construction}
\author{Anibal M. Medina-Mardones}
\address{Max Plank Institute for Mathematics, Bonn, Germany}
\email{ammedmar@mpim-bonn.mpg.de}
\address{Department of Mathematics, University of Notre Dame, Notre Dame, IN, USA}
\email{amedinam@nd.edu}
\author{Your name}
\address{Your address}
\email{Your email}

\keywords{.}
\subjclass[2020]{.}

\begin{abstract}
	
\end{abstract} 

\vspace*{-1cm}

\maketitle

\section{Introduction}
The philosophy is to construct an explicit $E_{\infty}$-bialgebra structure on the cobar model for the based loop space of a space such that
\\
1) it is given by the action of a ``small" cofibrant model for the $E_{\infty}$-operad, recently introduced by A. Medina
\\
2) the topological applications include general spaces with arbitrary fundamental group and no finite type assumptions using the recent observations of M. Rivera and M. Zeinalian 

\section{Simplices, cubes, and necklaces}
Discuss simplicial sets, cubical sets, and necklical sets. Define their normalized chain complexes and the coassociative coalgebra structures.

\section{A cofibrant model for the $E_{\infty}$-operad}

Recall the definition of the prop $\mathcal{M}$, the operad $U(\mathcal{M})$ and mention relation to the surjection operad. Explain what it means to be cofibrant. Discuss the Hopf operad structure of $U(\mathcal{M})$.  

Give examples of the $\mathcal{M}$-bialgebra structures on the normalized chains on a standard simplex, on a standard cube, and on a necklace. Describe the $U(\mathcal{M})$-coalgebra structure on the normalized chains of simplicial, cubical, and necklical sets. 

\section{The cobar construction}

Define the cobar construction of a connected dg coassociative coalgebra. 

Recall the construction of a coassociative associative bialgebra on the cobar construction of an $E_2$-coalgebra following Kadeishvhili/Matthias Schwartz. In fact, I think we can do this in the context of $U(M)$-coalgebras? Namely, if start with a $U(M)$-coalgebra then we may construct a coassociative coproduct on the cobar construction on the underlying $A_{\infty}$-coalgera by looking at its $E_2$ part.

Discuss the localized version of the cobar construction.

\section{A necklical model for the based loop space}

Construct a functor from simplicial sets to monoidal necklical sets modelling the based loop space functor. 

Discuss relationship with the cobar construction on the chains on a simplicial set. 

\section{Main theorem and applications}

Construct a functorial $E_{\infty}$-bialgebra structure on the cobar construction of the normalized chains on a simplicial set. This $E_{\infty}$-bialgebra on the cobar construction satisfies the following Hopf condition: it has a underlying dg associative/coassociative Hopf algebra structure. 

Applications: if the natural chain map between the normalized chains on a cubical set and its triangulation preserves the $U(M)$-coalgebra structure (check?) we may deduce that our model is equivalent as a $E_{\infty}$-bialgebra to the normalized chains on the classical Kan loop group functor (this would use recent work of Rivera, Zeinalian, and Minichello). 

\bibliographystyle{ieeetr} % ieeetr
\bibliography{bibliography}

\end{document}