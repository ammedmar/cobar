\documentclass{amsart}
\usepackage{amsmath, amssymb}
\usepackage{tikz-cd}
\usepackage{mathbbol} % mathbb on greek letters
\usepackage{verbatim} % just for comments, remove afterwards

% (hyper-)references
\usepackage[bookmarks=true, linktocpage=true,
bookmarksnumbered=true, breaklinks=true,
pdfstartview=FitH, hyperfigures=false,
plainpages=false, naturalnames=true,
colorlinks=true, pagebackref=true,
pdfpagelabels]{hyperref}
\hypersetup{
	colorlinks,
	citecolor=blue,
	filecolor=blue,
	linkcolor=blue,
	urlcolor=blue
}
\usepackage[capitalize, noabbrev]{cleveref}
\crefname{subsection}{\subsect\!}{subsections}
\let\subsect\S % copying the subsection symbol before overwriting it

% layout
\setlength{\textwidth}{\paperwidth}
\addtolength{\textwidth}{-1.85in}
\setlength{\textheight}{\paperheight}
\addtolength{\textheight}{-2in}
\calclayout

% Updating to MSC2020
\makeatletter
\@namedef{subjclassname@2020}{%
	\textup{2020} Mathematics Subject Classification}
\makeatother

% elements
\renewcommand{\P}{\mathcal{P}}
\renewcommand{\O}{\mathcal{O}}
\renewcommand{\S}{\mathbb{S}}
\newcommand{\As}{{\mathcal{A}\mathsf{s}}}
\newcommand{\M}{\mathcal{M}}
\newcommand{\UM}{{\U(\mathcal{M})}}
\newcommand{\SL}{{\mathcal{M}_{sl}}}
\newcommand{\USL}{{\U(\mathcal{M}_{sl})}}

% sets and spaces
\newcommand{\N}{\mathbb{N}}
\renewcommand{\k}{\Bbbk}
\newcommand{\id}{\mathrm{id}}
\newcommand{\End}{\mathrm{End}}
\newcommand{\Hom}{\mathrm{Hom}}
\newcommand{\Bij}{\mathfrak{Bij}}
\newcommand{\G}{\mathfrak{G}}
\newcommand{\gsimplex}{\mathbb{\Delta}}
\newcommand{\gcube}{\mathbb{I}}
\DeclareMathOperator{\coker}{coker}

% categories
\newcommand{\C}{\mathsf{C}}
\newcommand{\Cat}{\mathsf{Cat}}
\newcommand{\Top}{\mathsf{Top}}
\newcommand{\Fun}{\mathsf{Fun}}
\newcommand{\Ch}{\mathsf{Ch}}
\newcommand{\Alg}{\mathsf{Alg}}
\newcommand{\coAlg}{\mathsf{coAlg}}
\newcommand{\biAlg}{\mathsf{biAlg}}
\newcommand{\simplex}{\triangle}
\newcommand{\cube}{\square}
\newcommand{\Set}{\mathsf{Set}}
\newcommand{\sSet}{\mathsf{sSet}}
\newcommand{\cSet}{\mathsf{cSet}}
\newcommand{\sGrp}{\mathsf{sGrp}}
\newcommand{\Nec}{\mathsf{Nec}}
\newcommand{\nSet}{\mathsf{nSet}}
\newcommand{\Mon}{\mathsf{Mon}}
\newcommand{\smod}{\mathsf{Mod}_{\S}}
\newcommand{\sbimod}{\mathsf{biMod}_{\S}}
\newcommand{\operads}{\mathsf{Oper}}
\newcommand{\props}{\mathsf{Prop}}

% functions and functors
\newcommand{\op}{\mathrm{op}}
\DeclareMathOperator{\Y}{\mathcal{Y}}
\DeclareMathOperator{\A}{\mathcal{A}}
\DeclareMathOperator{\Ahat}{\mathcal{\widehat{A}}}
\DeclareMathOperator{\U}{U}
\DeclareMathOperator{\F}{F}
\DeclareMathOperator{\chains}{N}
\DeclareMathOperator{\schains}{N^{\simplex}}
\DeclareMathOperator{\cchains}{N^{\cube}}
\DeclareMathOperator{\schainsAS}{N^{\simplex}_{\As}}
\DeclareMathOperator{\cchainsAS}{N^{\cube}_{\As}}
\DeclareMathOperator{\schainsUM}{N^{\simplex}_{\UM}}
\DeclareMathOperator{\cchainsUM}{N^{\cube}_{\UM}}
\DeclareMathOperator{\schainsUSL}{N^{\simplex}_{\USL}}
\DeclareMathOperator{\cchainsUSL}{N^{\cube}_{\USL}}
\DeclareMathOperator*{\colim}{colim}
\DeclareMathOperator{\loops}{\Omega}
\DeclareMathOperator{\cobar}{\mathbf{\Omega}}
\DeclareMathOperator{\ccobar}{\mathbb{\Omega}}
\DeclareMathOperator{\ccobarE}{\widehat{\mathbb{\Omega}}}
\DeclareMathOperator{\ncobar}{\mathbb{\Omega}^{\mathrm{nec}}}
\DeclareMathOperator{\ncobarE}{\widehat{\mathbb{\Omega}}^{\mathrm{nec}}}
\DeclareMathOperator{\sSing}{Sing^{\simplex}}
\DeclareMathOperator{\cSing}{Sing^{\cube}}
\DeclareMathOperator{\sS}{S^{\simplex}}
\DeclareMathOperator{\cS}{S^{\cube}}
\DeclareMathOperator{\barConst}{Bar}
\DeclareMathOperator{\CS}{\zeta}
\DeclareMathOperator{\rigid}{\mathfrak{C}}
\DeclareMathOperator{\nerve}{\mathfrak{N}}
\DeclareMathOperator{\classifying}{\overline{W}}
\DeclareMathOperator{\localization}{\mathcal{L}}

% environments
\newtheorem{theorem}{Theorem}
\newtheorem*{nntheorem}{Theorem}
\newtheorem{proposition}[theorem]{Proposition}
\newtheorem{lemma}[theorem]{Lemma}
\newtheorem{corollary}[theorem]{Corollary}
\theoremstyle{definition}
\newtheorem{definition}[theorem]{Definition}
\newtheorem{example}[theorem]{Example}
\newtheorem{remark}[theorem]{Remark}

% other
\newcommand{\anibal}[1]{\textcolor{blue}{\underline{Anibal}: #1}}
\newcommand{\manuel}[1]{\textcolor{red}{\underline{Manuel}: #1}}
\renewcommand{\th}{^\mathrm{th}}
\newcommand{\bars}[1]{\vert{#1}\vert}
\newcommand{\xra}[1]{\xrightarrow{#1}}

% drawings
\include{aux/figures}

\begin{document}
\title{The cobar construction as an $E_{\infty}$-bialgebra model of the based loop space}
\author{Anibal M. Medina-Mardones}
\address{Max Plank Institute for Mathematics, Bonn, Germany}
\email{ammedmar@mpim-bonn.mpg.de}
\address{Department of Mathematics, University of Notre Dame, Notre Dame, IN, USA}
\email{amedinam@nd.edu}
\thanks{A.M-M. acknowledges financial support from Innosuisse grant \mbox{32875.1 IP-ICT - 1} and the hospitality of the Laboratory for Topology and Neuroscience at EPFL}

\author{Manuel Rivera}
\address{Department of Mathematics, Purdue University, West Lafayette, IN, USA}
\email{manuelr@purdue.edu}
\thanks{M.R. acknowledges support from the \textit{Karen EDGE Fellowship} and the excellent working conditions of the \textit{Max Planck Institute for Mathematics} in Bonn, Germany.}

\keywords{.}
\subjclass[2020]{.}

\begin{abstract}
    In the fifties, Adams introduced a comparison map $\theta_Z \colon \cobar(S^\simplex(Z,z)) \to S^\cube(\loops_z Z)$ from his cobar construction on the (simplicial) singular chains of a pointed space $(Z, z)$ to the cubical singular chains on its based loops space $\loops_z Z$.
	This comparison map is a quasi-isomorphism of algebras, which was shown by Baues to be one of bialgebras by considering Serre's cubical coproduct.
	In this work we generalize Baues result by constructing an $E_{\infty}$-bialgebra structure on the cobar construction of the coalgebra of singular chains, and proving that Adams' comparison map is a quasi-isomorphism of $E_{\infty}$-bialgebras.
	We also show that, for a general reduced simplicial set $X$, Hess and Tonks' extended cobar construction on $\schains(X)$ is quasi-isomorphic to $\schains(G(X))$, the chains on the Kan loop group of $X$, as $E_{\infty}$-coalgebras. 
\end{abstract}

\vspace*{-1cm}

\maketitle
\setcounter{tocdepth}{1}
\tableofcontents
% !TEX root = ../cobar1.tex

\section{Introduction}

For any topological space $\fX$, its complex of simplicial or cubical singular chains $\Schains(\fX)$ -- regarded as a differential graded (dg) abelian group -- encodes the homology of $\fX$ in its quasi-isomorphism type.
More homotopical information can be stored in the quasi-isomorphism type of this chain complex if considered as a (dg) coassociative coalgebra, which we will denote $\SchainsA(\fX)$, where the coproduct comes from a canonical choice of chain approximation to the diagonal $\fX \to \fX \times \fX$.
%, attributed respectively to Alexander--Whitney and Serre.
For instance, the cohomology ring of $\fX$ is retained, but the action of the Steenrod algebra on its mod $p$ cohomology is not.

In Mandell's seminal work \cite{mandell2006homotopy_type} it is shown that, when $\fX$ is nilpotent and finite type, the entire homotopy type of $\fX$ can be encoded in the quasi-isomorphism type of this complex if considered as an $E_\infty$-coalgebra, a structure providing $\SchainsA(\fX)$ with coherent homotopies witnessing the derived cocommutativity of the coproduct coming from the strict symmetry of the diagonal map.

%In this work we are interested in modeling algebraically the homotopy type of the based loop space $\loops_x \fX$ of a pointed space $(\fX, x)$.
%Via concatenation of loops this space is a topological monoid, whose induced structure on the set of path-connected components is a group; the fundamental group of the underlying space.

The first contribution of this paper is to explicitly endow the cubical singular chains of the based loop space $\loops_x \fX$, with the structure of a monoidal $E_\infty$-coalgebra extending the Serre diagonal.
More specifically, we verify that the monoid structure induced on $\cSchains(\loops_x \fX)$ by the concatenation of loops is compatible with a natural $E_\infty$-coalgebra structure on cubical singular chains, similar to the one defined in \cite{medina2022cube_einfty}.
%We denote this functor by $\cchainsUM \colon \cSet \to \coAlg_\UM$

%We will also concern ourselves with another algebraic model approximating the homotopy type of $\loops_x \fX$.
%which was introduced by Adams \cite{adams1956cobar}.
%This model \cite{adams1956cobar} is obtained by applying Adams' celebrated \textit{cobar construction} functor $\cobar$ to a pointed version of the coalgebra of simplicial singular chains of $(\fX, x)$.
%The cobar functor may be thought of as an algebraic counterpart to the based loop space.
%Just like the based loop space functor is part of a duality between pointed topological spaces and group-like topological monoids, the cobar functor is part of one between a pointed version of dg coalgebras and monoids in chain complexes.
% (i.e. dg algebras).
%Adams' cobar construction provides a model for the quasi-isomorphism type of the monoid of chains on the based loop space.

Applying Adams' cobar construction to the coalgebra of simplicial singular chains of $(\fX, x)$, he obtained another monoidal algebraic model $\cobar \sSchainsA(\fX, x)$ of $\loops_x \fX$ \cite{adams1956cobar}.
More precisely, he constructed a natural monoidal chain map $\theta$ from $\cobar \sSchainsA(\fX, x)$ to $\cSchains(\loops_x \fX)$ and proved it to be a quasi-isomorphism if $\fX$ is simply-connected, a statement that also holds true for path-connected spaces after \cite{rivera2018cubical}.
The model $\cobar \sSchainsA(\fX, x)$ is smaller than $\cSchains(\loops_x \fX)$ and unlocks effective analysis of quantitative and qualitative properties of $\loops_x \fX$, as illustrated for instance in \cite{chainalgebraloops} and \cite{adamscobarequivalence}.

The second main contribution of this paper is to make Adams model into a monoidal $E_\infty$-coalgebra and to prove that
\[
\theta \colon \cobar \sSchainsA(\fX, x) \to \cSchains(\loops_x \fX)
\]
respects this higher structure.

Our starting point is groundbreaking work by H.~Baues, which imply statements similar to those in this work but in the category of coalgebras.
He reinterpreted Adams' algebraic construction at a deeper geometric level \cite{baues1998hopf}, which allowed him to endow $\cobar \sSchainsA(\fX, x)$ with the structure of a monoidal coalgebra, and to show that $\theta$ preserves this structure.
To prove our statement we interpret Adams' construction at an even deeper categorical level.
%The simplicial singular complex $\sSing(\fX, x)$ is an example of an object in the category $\sSet^0$ of $0$-reduced simplicial sets.
We interpret Baues' geometric cobar construction, originally defined for $1$-reduced simplicial sets, as a functor
\begin{equation*}
	\ccobar \colon \sSet^0 \to \Mon_{\cSet},
\end{equation*}
from the category of $0$-reduced simplicial sets to that of monoidal cubical sets.
The key difference with Baues' original work is the use of connections to obtain a natural construction before geometric realization.

Additionally, we need a suitable model of the $E_\infty$-operad endowing cubical chains with a natural $E_\infty$-coalgebra extending the Serre diagonal.
For this we take the operad $\UM$ introduced in \cite{medina2020prop1}.
After proving that its coalgebras form a monoidal category, we show that the functor $\cchainsUM \colon \cSet \to \coAlg_\UM$ -- defined in \cite{medina2022cube_einfty} with a different sign convention -- is monoidal.
This allows us to prove the following generalization of Adams and Baues structures.
%This functors is equal to that introduced in \cite{medina2022cube_einfty} up to signs.

%that $\cchainsUM$ from cubical sets to $\UM$-coalgebras is monoidal, which implies that $\cchainsUM(\ccobar X)$ is a monoidal $\UM$-coalgebra for any 0-reduced simplicial set.
%More precisely, we prove the following.

\begin{theorem*}
	The following diagram commutes up to natural isomorphisms:
	\[
	\begin{tikzcd} [row sep=small]
		& \Mon_{\coAlg_\UM} \arrow[d] \\
		\Mon_{\cSet} \arrow[ru, "\cchainsUM", out=70, in=180, near start] \arrow[r, "\cchainsA"]
		& \Mon_{\coAlg} \arrow[d] \\
		\sSet^0 \arrow[r, "\cobar \schainsA"] \arrow[u, "\ccobar"]
		& \Mon_{\Ch},
	\end{tikzcd}
	\]
	where the unlabeled arrows are forgetful functors.
\end{theorem*}

In the above diagram, the arrow from $\sSet^0$ to $\Mon_{\Ch}$ is Adams' cobar construction, the one from $\sSet^0$ to $\Mon_{\coAlg}$ is Baues' enhancement, and the one from $\sSet^0$ to $\Mon_{\coAlg_{\UM}}$ is our lift.
Additionally we have the following statement about Adams's map.

\begin{theorem*}
	For any pointed space $(\fX, x)$,
	\[
	\theta \colon \cobar \sSchainsA(\fX, x) \to \cSchains(\loops_x \fX)
	\]
	is a quasi-isomorphism of monoidal $\UM$-coalgebras.
\end{theorem*}

The fact that $\theta$ respects the monoid structure in $\Ch$ was proven by Adams, whereas the compatibility of the monoid structure with the Serre coalgebra structure was established by Baues.
Our contribution is the compatibility of the monoid structure with a full $E_\infty$-coalgebra extension of Serre's coalgebra.
% !TEX root = ../cobar1.tex

\section{Conventions and preliminaries}\label{s:preliminaries}

\subsection{Presheaves and monoids}

Recall that a category is said to be \textit{small} if its objects and morphisms form sets.
We denote the category of small categories by $\Cat$.
Given categories $\sB$ and $\sC$ with $\sB$ small we denote their associated \textit{functor category} by $\Fun(\sB, \sC)$.
A category is said to be \textit{cocomplete} if any functor to it from a small category has a colimit.
If $\sA$ is small and $\sC$ cocomplete, then the (left) \textit{Kan extension} of $g$ along $f$ exists for any pair of functors $f$ and $g$ in the diagram below, and it is the initial object in $\Fun(\sB, \sC)$ making
\begin{equation*}
	\begin{tikzcd}[column sep=normal, row sep=normal]
		\sA \arrow[d, "f"'] \arrow[r, "g"] & \sC \\
		\sB \arrow[dashed, ur, bend right] & \quad
	\end{tikzcd}
\end{equation*}
commute.
A Kan extension along the \textit{Yoneda embedding}, i.e., the functor
\[
\yoneda \colon \sA \to \Fun(\sA^\op, \Set)
\]
induced by the assignment
\[
a \mapsto \big( a^\prime \mapsto \sA(a^\prime, a) \big),
\]
is referred to as a \textit{Yoneda extension}.
Objects in the image of the Yoneda embedding are said to be \textit{representable}.
Any presheaf $P$ is isomorphic to a colimit of representables presheaves $P \, \cong \colim_{\yoneda(a) \downarrow P} \yoneda(a)$.

A \textit{monoid} $(M, \mu, \eta)$ in a monoidal category $(\sC, \ot, \mathbb{1})$ is an object $M$ together with two morphisms $\mu \colon M \ot M \to M$ and $\eta \colon \mathbb{1} \to M$, called \textit{multiplication} and \textit{unit} respectively, satisfying the usual associativity and unital relations.
We denote the category of monoids in $\sC$ by $\Mon_{\sC}$ and remark that a monoidal functor $\sC \to \sC^\prime$ induces a functor between their categories of monoids $\Mon_\sC \to \Mon_{\sC^\prime}$.

\subsection{Coalgebras}\label{ss:coalgebras}

Throughout this article $\k$ denotes a commutative and unital ring and we work over its associated closed symmetric monoidal category of differential (homologically) graded $\k$-modules $(\Ch, \ot, \k)$.
We refer to the objects and morphisms of this category as \textit{chain complexes} and \textit{chain maps} respectively.
We denote by $\Hom(C, C^\prime)$ the chain complex of $\k$-linear maps between chain complexes $C$ and $C^\prime$, and refer to the functor $\Hom(-, \k)$ as \textit{linear duality}.
The $i^\th$ \textit{suspension} functor $s^i \colon \Ch \to \Ch$ is defined at the level of graded modules by $(s^{i}M)_n = M_{n-i}$.

A \textit{coalgebra} consists of a chain complex $C$ and chain maps $\Delta \colon C \to C \ot C$ and $\varepsilon \colon C \to \k$ satisfying the usual coassociativity and counitality relations.
This notion is equivalent to that of a comonoid in $\Ch$.
Denote by $\coAlg$ the category of coalgebras with morphisms being structure preserving chain maps.
The category $\coAlg$ is symmetric monoidal, with braiding induced from $\Ch$ and structure maps of a product $C \ot C^\prime$ given by
\begin{gather*}
	C \ot C^\prime \xra{\Delta \ot \Delta^\prime}
	(C \ot C) \ot (C^\prime \ot C^\prime) \xra{(23)}
	(C \ot C^\prime) \ot (C \ot C^\prime), \\
	C \ot C^\prime \xra{\varepsilon \ot \varepsilon^\prime}
	\k \ot \k \xra{\cong} \k.
\end{gather*}

A \textit{coaugmentation} on a coalgebra $C$ is a coalgebra map $\nu \colon \k \to C$.
A coalgebra is said to be \textit{coaugmented} if it is equipped with a coaugmentation.
We denote by $\coAlg^\ast$ the category of coaugmented coalgebras with morphisms being coaugmentation preserving coalgebra maps.
A coaugmented coalgebra is a \textit{connected coalgebra} if it is $0$ in negative degrees and the coaugmentation induces an isomorphism $\k \cong C_0$ of $\k$-modules.
We denote by $\coAlg^0$ the full subcategory of $\coAlg^\ast$ defined by these.

\subsection{Simplicial algebraic topology}\label{ss:simplicial}

The \textit{simplex category} is denoted by $\simplex$ and its objects by $[n]$.
%Its morphisms are generated by the usual \textit{coface} $\delta_i \colon [n-1] \to [n]$ and \textit{codegeneracy} $\sigma_i \colon [n+1] \to [n]$ maps.
The category of \textit{simplicial sets} $\Fun(\simplex^\op, \Set)$ is denoted by $\sSet$ and the \textit{standard $n$-simplex} $\yoneda [n] = \simplex(-, [n])$ by $\simplex^n$.
For any simplicial set $X$ we write, as usual, $X_n$ instead of $X [n]$, and identify the elements of $\simplex^n_m$ with increasing tuples $[v_0, \dots, v_m]$ where $v_i \in \{0, \dots, n\}$.

If $X$ is such that $X_0$ is a singleton set we say that $X$ is \textit{reduced}.
We denote the full subcategory of reduced simplicial sets by $\sSet^0$.

We denote the topological $n$-simplex by $\gsimplex^{n}$ and consider the usual adjunction pair defined by the \textit{geometric realization} and \textit{singular complex}
\[
\begin{tikzcd}
	\bars{\ } \colon \sSet  \arrow[r, shift left=.5ex] &
	\Top :\! \sSing. \arrow[l, shift left=.5ex]
\end{tikzcd}
\]

The functor of (normalized) \textit{simplicial chains} $\schains \colon \sSet \to \Ch$ is obtained from the geometric realization and the functor of cellular chains.
When no confusion arises from doing so we write $\chains$ instead of $\schains$ and refer to it simply as the functor of \textit{chains}.
We will denote the functor of (simplicial) \textit{singular chains} $\schains \circ \sSing \colon \Top \to \Ch$ by $\sSchains$.

We modify this construction for a pointed topological space $(\fZ,z)$ by only considering maps $\gsimplex^n \to \fZ$ sending all vertices to $z$.
This produces a reduced simplicial set $\sSing(\fZ,z)$ whose chains we denote by $\sSchains(\fZ,z)$.

There is a classical lift of the functor of chains to coalgebras:
\[
\begin{tikzcd}
	& \coAlg \arrow[d] \\
	\sSet \arrow[r, "\schains"] \arrow[ur, "\schainsA", out=70, in=180] & \Ch
\end{tikzcd}
\]
defined, via a Yoneda extension, by the following structure on the chains of standard simplices.
For any $n \in \N$, define $\epsilon \colon \chains(\simplex^n) \to \k$ by
\[
\epsilon \big( [v_0, \dots, v_q] \big) = \begin{cases} 1 & \text{ if } q = 0, \\ 0 & \text{ if } q>0, \end{cases}
\]
and $\Delta \colon \chains(\simplex^n) \to \chains(\simplex^n)^{\ot2}$ by
\[
\Delta \big( [v_0, \dots, v_q] \big) = \sum_{i=0}^q \ [v_0, \dots, v_i] \ot [v_i, \dots, v_q].
\]

If $X$ is a reduced simplicial set then $\schainsA(X)$ becomes a connected coalgebra with the coaugmentation $\nu \colon \k \to \schains(X)$ induced by sending $1$ to the basis element represented by the unique $0$-simplex of $X$.
Hence, the functor $\schainsA$ restricts to a functor
\[
\schainsA \colon \sSet^0 \to \coAlg^0.
\]

\subsection{Cubical algebraic topology}

The objects of the \textit{cube category} $\cube$ are the sets $2^n = \{0, 1\}^n$ with $2^0 = \{0\}$ for $n \in \N$, and its morphisms are generated by the \textit{coface, codegeneracy} and \textit{coconnection} maps defined by
\begin{align*}
	\delta_i^\varepsilon & =
	\mathrm{id}_{2^{i-1}} \times \delta^\varepsilon \times \mathrm{id}_{2^{n-1-i}} \colon 2^{n-1} \to 2^n, \\
	\sigma_i & =
	\mathrm{id}_{2^{i-1}} \times \, \sigma \times \mathrm{id}_{2^{n-i}} \quad \colon 2^{n} \to 2^{n-1}, \\
	\gamma_i & =
	\mathrm{id}_{2^{i-1}} \times \, \gamma \times \mathrm{id}_{2^{n-i}} \quad \colon 2^{n+1} \to 2^{n},
\end{align*}
where $\varepsilon \in \{0,1\}$ and
\[
\begin{tikzcd}
	1 \arrow[r, out=45, in=135, "\delta^0"] \arrow[r, out=-45, in=-135, "\delta^1"'] & 2 \arrow[l, "\sigma"'] &[-10pt] \arrow[l, "\gamma"'] 2 \times 2
\end{tikzcd}
\]
are defined by
\begin{gather*}
	\delta^0(0) = 0, \qquad
	\delta^1(0) = 1, \qquad
	\sigma(0) = \sigma(1) = 0, \qquad \\
	\gamma^1(0,0) = 0, \qquad
	\gamma^1(0,1) = 1, \qquad
	\gamma^1(1,0) = 1, \qquad
	\gamma^1(1,1) = 1.
\end{gather*}
We remark that the cube category without connections will not be considered in this work.

The category of \textit{cubical sets} $\Fun(\cube^\op, \Set)$ is denoted by $\cSet$ and
the \textit{standard $n$-cube} $\yoneda 2^n = \cube(-, 2^n)$ by $\cube^n$.
For any cubical set $Y$ we write, as usual, $Y_n$ instead of $Y(2^n)$.

We denote the topological $n$-cube by $\gcube^{n}$ and consider the usual adjunction pair defined by the \textit{geometric realization} and \textit{singular complex}
\[
\begin{tikzcd}
	\bars{\ } \colon \cSet  \arrow[r, shift left=.5ex] &
	\Top :\! \cSing. \arrow[l, shift left=.5ex]
\end{tikzcd}
\]

The functor of (normalized) \textit{cubical chains} $\cchains \colon \cSet \to \Ch$ is obtained from the geometric realization and the functor of cellular chains.
When no confusion arises from doing so we write $\chains$ instead of $\cchains$ and refer to it simply as the functor of \textit{chains}.
We remark that $\chains(\cube^n)$ is canonically isomorphic to $\chains(\cube^1)^{\ot n}$ and that $\chains(\cube^1)$ is canonically isomorphic to the cellular chains on the topological interval with its usual CW structure.
We use these isomorphisms to denote the elements of $\chains(\cube^n)$ as integral linear combinations monoids of the form $x_1 \ot\dotsb\ot x_n$ with $x_i \in \big\{[0], [0,1], [1] \big\}$.
We will denote the functor of (cubical) \textit{singular chains} $\cchains \circ \cSing \colon \Top \to \Ch$ by $\cSchains$.

There is a classical lift of the functor of cubical chains to the category of coalgebras
\[
\begin{tikzcd}
	& \coAlg \arrow[d] \\
	\cSet \arrow[r, "\cchains"] \arrow[ur, "\cchainsA", out=70, in=180] & \Ch
\end{tikzcd}
\]
defined, using a Yoneda extension, by the tensor product structure (\cref{ss:coalgebras}) defined on $\chains(\cube^n) \cong \chains(\cube^1)^{\ot n}$ via the identification of $\chains(\cube^1)$ and $\chains(\simplex^1)$.

%\subsubsection{Cubical chains}
%
%For non-negative integers $m$ and $n$, let $\cube_{\deg}(2^m, 2^n)$ be the subset of \textit{degenerate morphisms} in $\cube(2^m, 2^n)$, i.e., those of the form $\sigma_i \circ \tau$ or $\gamma_i \circ \tau$ with $\tau$ any morphism in $\cube(2^m, 2^{n+1})$.
%The functor of (cubical) \textit{chains} $\cchains \colon \cSet \to \Ch$ is the Yoneda extension of the functor $\cube \to \Ch$ defined next.
%It assigns to an object $2^n$ the chain complex having in degree $m$ the module
%\[
%\frac{\k\{\cube(2^m, 2^n)\}}{\k\{\cube_{\deg}(2^m, 2^n)\}}
%\]
%and differential induced by
%\[
%\partial (\id_{2^n}) = \sum_{i=1}^{n} \ (-1)^i \
%\big(\delta_i^1 - \delta_i^0 \big).
%\]
%To a morphism $\tau \colon 2^n \to 2^{n^\prime}$ it assigns the chain map
%\[
%\begin{tikzcd}[row sep=-3pt, column sep=normal,
%	/tikz/column 1/.append style={anchor=base east},
%	/tikz/column 2/.append style={anchor=base west}]
%	\cchains(\cube^n)_m \arrow[r] & \cchains(\cube^{n^\prime})_m \\
%	\big( 2^m \to 2^n \big) \arrow[r, mapsto] & \big( 2^m \to 2^n \stackrel{\tau}{\to} 2^{n^\prime} \big).
%\end{tikzcd}
%\]
%
%When no confusion arises we write $\chains$ instead of $\cchains$.


%\subsubsection{Cubical singular complex}
%
%Consider the topological $n$-cube
%\[
%\gcube^{n} = \{(x_1, \dots, x_n) \mid x_i \in [0,1]\}.
%\]
%The assignment $2^n \to \gcube^n$ defines a functor $\cube \to \Top$ whose Yoneda extension is known as \textit{geometric realization}.
%It has a right adjoint $\cSing \colon \Top \to \cSet$ given by
%\[
%Z \to \Big(2^n \mapsto \Top(\gcube^n, Z)\Big)
%\]
%and referred to as the \textit{cubical singular complex} of the topological space $Z$.
%We will refer to $\chains(\cSing Z)$ as the \textit{cubical singular chains} of $Z$ and for simplicity denote it by $\cSchains(Z)$.

%\subsubsection{Serre coalgebra}\label{ss:serre coalgebra}
%
%We review a classical lift of the functor of cubical chains to the category of coalgebras
%\[
%\begin{tikzcd}
%	& \coAlg \arrow[d] \\
%	\cSet \arrow[r, "\cchains"] \arrow[ur, "\cchainsA", out=70, in=180] & \Ch.
%\end{tikzcd}
%\]
%Using a Yoneda extension, it suffices to equip the chains on standard cubes with a natural coalgebra structure.
%For any $n \in \N$, define $\epsilon \colon \chains(\cube^n) \to \k$ by
%\[
%\epsilon \left( x_1 \ot \dots \ot x_d \right) = \epsilon(x_1) \dotsm \epsilon(x_n),
%\]
%where
%\[
%\epsilon([0]) = \epsilon([1]) = 1, \qquad \epsilon([0, 1]) = 0,
%\]
%and $\Delta \colon \chains(\cube^n) \to \chains(\cube^n)^{\ot 2}$ by
%\[
%\Delta (x_1 \ot \dots \ot x_n) =
%\sum \pm \left( x_1^{(1)} \ot \dots \ot x_n^{(1)} \right) \ot
%\left( x_1^{(2)} \ot \dots \ot x_n^{(2)} \right),
%\]
%where the sign is determined using the Koszul convention, and we are using Sweedler's notation
%\[
%\Delta(x_i) = \sum x_i^{(1)} \ot x_i^{(2)}
%\]
%for the chain map $\Delta \colon \chains(\cube^1) \to \chains(\cube^1)^{\ot 2}$ defined by
%\[
%\Delta([0]) = [0] \ot [0], \quad \Delta([1]) = [1] \ot [1], \quad \Delta([0,1]) = [0] \ot [0,1] + [0,1] \ot [1].
%\]
%
%Using the canonical isomorphism $\chains(\cube^n) \cong \chains(\cube^1)^{\ot n}$, the coproduct $\Delta$ can be described as the composition
%\[
%\begin{tikzcd}
%	\chains(\cube^1)^{\ot n} \arrow[r, "\Delta^{\ot n}"] & \left( \chains(\cube^1)^{\ot 2} \right)^{\ot n} \arrow[r, "sh"] & \left( \chains(\cube^1)^{\ot n} \right)^{\ot 2},
%\end{tikzcd}
%\]
%where $sh$ is the shuffle map that places tensor factors in odd position first.
\input{operads}

\section{Monoidal properties}

The first goal of this section is to construct a monoidal structure on $\coAlg_\UM$.
We will do so by providing $\M$ with the structure of a Hopf prop.
Then, we will show that the functor $\cchainsUM \colon \cSet \to \coAlg_{\UM}$ defined in \eqref{e:lift of cubical chains to UM-coalgs} is monoidal.
Finally, we will use this fact and the cubical cobar construction, a monoid in $\cSet$, to provide $\cobar(\schains(X))$ with the structure of an $E_\infty$ bialgebra, specifically, of a monoid in $\coAlg_{\UM}$

\subsection{Cartesian product and Day convolution} \label{ss:day convolution}

\anibal{Cartesian for simplicial sets}

\anibal{Day for cubical sets} 

\anibal{singular complex and realization realization in both cases: monoidal comments}

\subsection{Triangulation and its right adjoint} \label{ss:triangulation and its adjoint}

Simplicial and cubical sets can be related using the the simplicial cubes $(\simplex^1)^{\times n}$.

The \textit{triangulation} functor $\mathcal{T} \colon \cSet \to \sSet$ is defined by
\begin{equation*}
\mathcal{T}(Y) = \colim_{\cube^n \to Y} (\simplex^1)^{\times n}.
\end{equation*}
It has a right adjoint $\mathcal{U} \colon \sSet \to \cSet$ which is given by the formula
\begin{equation*}
\mathcal{U}(X)_n = \sSet((\simplex^1)^{\times n}, X).
\end{equation*}

With the monoidal structure defined in \cref{ss:day convolution}, the triangulation functor is monoidal.
Thus, it induces a functor
\begin{equation*}
\mathcal{T} \colon \Mon_{\cSet} \to \Mon_{\sSet}
\end{equation*}
between the associated categories of monoids.

\subsection{Serre coalgebra} \label{ss:serre coalgebra sym monoidal}

Recall that both $\Ch$ and $\coAlg$ are symmetric monoidal categories.
If $\cSet$ to be equipped with the Day convolution symmetric monoidal structure \cref{ss:day convolution}, then the functors $\cchains$ and its lift $\cchainsAS$ are symmetric monoidal.
Together with straightforward computations, these statements follow from the fact that the chain complexes $\chains(\cube^p) \otimes \chains(\cube^q)$ and $\chains(\cube^{p+q})$ are canonically isomorphic for any $p, q \geq 0$.

For later reference we record the following consequence of the previous claims and the fact that, for any coalgebra, the braiding $x \otimes y \mapsto (-1)^{\bars{x} \bars{y}} y \otimes x$ is a morphism of coalgebras.

\begin{lemma} \label{l:serre diagonal invariant}
	Let $\sigma \in \S_d$ acting on $\chains(\cube^1)^{\otimes d}$ by permuting tensor factors, then
	\begin{equation*}
	\begin{tikzcd}
	\chains(\cube^1)^{\otimes d} \arrow[r, "\Delta"] \arrow[d, "\sigma"] &
	\chains(\cube^1)^{\otimes d} \otimes \chains(\cube^1)^{\otimes d} \arrow[d, "\sigma \otimes \sigma"] \\
	\chains(\cube^1)^{\otimes d} \arrow[r, "\Delta"] &
	\chains(\cube^1)^{\otimes d} \otimes \chains(\cube^1)^{\otimes d}.
	\end{tikzcd}
	\end{equation*}
\end{lemma}

\subsection{Hopf operads and props}

So far we have consider operad and props over the category $\Ch$ but, since $\coAlg$ is also a symmetric monoidal category, we can consider them over $\coAlg$.
That is to say, demand that all defining chain complexes be coalgebras, and composition maps be morphisms of coalgebras.
We refer to operads and props over $\coAlg$ as \textit{Hopf operads and props} respectively.

If $\O$ is a Hopf operad, the category $\coAlg_\O$ is monoidal.
The structure maps of a product of $\O$-coalgebras $C \otimes C^\prime$ are given by
\begin{equation*}
\begin{tikzcd} [column sep = normal, row sep=large]
\O(r) \otimes (C \otimes C^\prime) \arrow[r, "\Delta_\O \otimes \id"] &[10 pt] \O(r) \otimes \O(r) \otimes (C \otimes C^\prime) \arrow[d, "Sh"'] & & \\ &
\big(\O(r) \otimes C \big) \otimes \big( \O(r) \otimes C \big) \arrow[r] & 
C^{\otimes r} \otimes C^{\prime\, \otimes r} \arrow[r, "Sh"] &
(C \otimes C^\prime)^{\otimes r}.
\end{tikzcd}
\end{equation*}

A similar statement holds for the category of representations of a Hopf prop $\P$.

\subsection{Cubical structure on $\M$}

We will now revisit from \cref{ss:definition of M} the definition of $\M$ and provide this prop with a cubical structure.
We will use this to make it into a Hopf prop in the next subsection.

We start with a general observation.
As explained in \cref{ss:free constructions}, the $(m,n)$-part of the free prop generated by an $\S$-module $N$ is given by
\begin{equation*}
\F(N)(m,n) \ = \bigoplus_{\Gamma \in \G(m,n)} \bigotimes_{v \in Vert(\Gamma)} out(v) \otimes_{\S_q} N(p, q) \otimes_{\S_p} in(v),
\end{equation*}
which is well defined only up to a choice of total order on the set of vertices of the $(m,n)$-graphs involved.

In the case we are concern with, the free prop whose quotient defines $\M$ is generated by an $\S$-bimodule $M$ with
\begin{equation*}
M(1,0) = \k \left\{ \; \counit \ \right\}, \qquad
M(1,2) = \k[\S_2] \left\{ \; \coproduct \ \right\}, \qquad
M(2,1) = \k[\S_2^\op] \left\{ \partial_0\! \product \, \right\} \xleftarrow{1 - T} \k[\S_2^\op] \left\{ \, \product \; \right\},
\end{equation*}
and $M(m,n) = 0$ otherwise.
We notice that the elements $\rightboundary\,$ and $\,\leftboundary\,$ in $\F(M)(2,1)$ are interchanged by the action of $T \in \S_2^\op$.
Let $\widetilde{\M}$ be the quotient of $\F(M)$ by the ideal generated by the identification of $\,\partial_0\! \product$ and $\,\rightboundary\,$, so that the boundary of $\product$ is $\,\boundary\,$.
Consider a basis element of $\widetilde{\M}(m,n)_d$ represented by an immersed $(m,n)$-graph $\Gamma$ with $d$ occurrences of $\product.$
A choice of total order of the vertices of $\Gamma$ defines a chain map
\begin{equation} \label{e:order chain map}
\begin{tikzcd}[row sep=tiny, column sep=small]
\iota_\Gamma \colon \chains(\cube^d) \arrow[r] & \widetilde{\M}(m,n) \\
\qquad {[0,1]}^{\otimes d} \arrow[r, |->] & \Gamma,
\end{tikzcd}
\end{equation}
referred to as the \textit{characteristic map of $\Gamma$}.
We obtain characteristic maps on $\M$ simply by composing these with the projection $\widetilde{\M} \to \M$ induced from the quotient by the prop ideal generated by the relations \eqref{e:relations of M}.

\subsection{Hopf structure on $\M$}

We now use the existence of characteristic maps on $\M$ to show it is a prop over $\coAlg$.
We will show the following is well defined in \cref{t:cubical structure on M}.

\begin{definition}
	The \textit{coproduct} $\Delta_\M \colon \M \to \M^{\otimes 2}$ is defined on basis elements $\Gamma$ by
	\begin{equation*}
	\Delta_{\M}(\Gamma) = \iota_\Gamma^{\otimes 2} \circ \Delta \left( [0,1]^{\otimes d} \right),
	\end{equation*}
	where $d = \bars{\Gamma}$, and the \textit{counit} $\epsilon_\M \colon \M \to \k$ by requiring $\epsilon_M(\Gamma) = 1$ if the degree of $\Gamma$ is $0$, where $\k$ is though of as a Hopf prop.
\end{definition}

\begin{theorem} \label{t:cubical structure on M}
	The maps $\Delta_\M$ and $\varepsilon_\M$ are well defined and make $\M$ into a Hopf prop.
\end{theorem}

\begin{proof}
	By \cref{l:serre diagonal invariant} these maps are independent of the total order on vertices used to define characteristic maps.
	
	The linear map $\Delta_\M$ is compatible with the relations defining $\M$ since
	\begin{center}
		\begin{tikzcd}
		\leftcounitcoproduct \arrow[r, "\Delta_\M"] \arrow[d, <->] &[-0pt] \leftcounitcoproduct \otimes \leftcounitcoproduct \arrow[d, <->] \\
		\identity \arrow[r, "\Delta_\M"] & \ \, \identity \otimes \identity \, ,
		\end{tikzcd}
		\qquad
		\begin{tikzcd}
		\rightcounitcoproduct \arrow[r, "\Delta_\M"] \arrow[d, <->] &[-0pt] \rightcounitcoproduct \otimes \rightcounitcoproduct \arrow[d, <->] \\
		\identity \arrow[r, "\Delta_\M"] & \ \, \identity \otimes \identity \, ,
		\end{tikzcd}
		\qquad
		\begin{tikzcd}
		\leftcomb \arrow[r, "\Delta_\M"] \arrow[d, <->] &[-0pt] \leftcomb \, \otimes \leftcomb \arrow[d, <->] \\
		\rightcomb \arrow[r, "\Delta_\M"] & \, \rightcomb \, \otimes \rightcomb,
		\end{tikzcd}
		\qquad
		\begin{tikzcd}
		\productcounit \arrow[r, "\Delta_\M"] \arrow[d, <->] &[-0pt] \counit\ \counit \, \otimes \productcounit + \productcounit \otimes \, \counit\ \counit \arrow[d, <->] \\
		0 \arrow[r, "\Delta_\M"] & \ 0,
		\end{tikzcd}
	\end{center}
	and it is a chain map since
	\begin{center}
		\begin{tikzcd}
		\counit \arrow[r, "\Delta_\M"] \arrow[d, "\partial"'] &[-0pt] \counit \otimes \counit \arrow[d, "\partial"'] \\
		0 \arrow[r, "\Delta_\M"] & \ \, 0 \, ,
		\end{tikzcd}
		\qquad
		\begin{tikzcd}
		\coproduct \arrow[r, "\Delta_\M"] \arrow[d, "\partial"'] &[-0pt] \coproduct \otimes \coproduct \arrow[d, "\partial"'] \\
		0 \arrow[r, "\Delta_\M"] & \ \, 0 \, ,
		\end{tikzcd}
		\qquad
		\begin{tikzcd}
		\product \arrow[r, "\Delta_\M"] \arrow[d, "\partial"'] &[-0pt] \leftboundary \ \otimes \product + \product \otimes\ \rightboundary \arrow[d, "\partial"'] \\
		\rightboundary \,-\, \leftboundary \arrow[r, "\Delta_\M"] &
		\, \rightboundary \otimes \rightboundary \,-\, \leftboundary \otimes \leftboundary\,.
		\end{tikzcd}
	\end{center}	
	One can check similarly that $\varepsilon_\M$ is a well defined morphism of $\S$ bimodules.
	
	The fact that for any biarity $(m,n)$ the triple $(\M(m,n), \Delta_\M(m,n), \epsilon_\M(m,n))$ is a coalgebra follows from the Serre structure \cref{ss:serre coalgebra} being a coalgebra.
	
	Furthermore, $\Delta_\M$ is compatible with the composition structure since for any pair of characteristic maps $\iota_\Gamma$ and $\iota_{\Gamma^\prime}$, the chain map $\iota_\Gamma \otimes \iota_{\Gamma^\prime}$ is a characteristic map for any composition of $\Gamma$ and $\Gamma^\prime$ with the induce order on vertices.
	The compatibility of $\epsilon_\M$ is immediate.
\end{proof}

\subsection{Monoidal $E_\infty$ structures}

We now prove the main technical result of this paper.

\begin{theorem} \label{t:cubical e-infty chains are monoidal}
	The functor $\cchainsUM \colon \cSet \to \coAlg_\UM$ is monoidal.
\end{theorem}

\begin{proof}
	It suffices to prove this result for chains on the standard cubes.
	These are equipped with a natural $\M$-structure, and we will prove the stronger statement that the functor $\mathrm{N}^\cube_\M \colon \cube \to \biAlg_{\M}$ is monoidal.
	
	From \cref{ss:serre coalgebra sym monoidal} we know that the Serre diagonal and the augmentation map are compatible with the monoidal structure on the site $\cube$.
	We need to establish this for the map $\ast$, the image of the generator $\product$.
	Diagrammatically, this is equivalent to the commutativity of
	\begin{equation} \label{e:hopf diagonal and the product}
	\begin{tikzcd}
	\product \otimes \chains(\cube^{p+q})^{\otimes 2} \arrow[d] \arrow[r, "\Delta_\M \otimes Sh"] &[20pt]
	\left(\, \ \leftboundary \ \otimes \product + \product \otimes \ \rightboundary \ \, \right) \otimes \chains(\cube^{p})^{\otimes 2} \otimes \chains(\cube^{q})^{\otimes 2} \arrow[d]\\
	\chains(\cube^{p+q}) \arrow[r, "\cong"] &
	\, \chains(\cube^{p}) \otimes \chains(\cube^{q}).
	\end{tikzcd}
	\end{equation}
	
	Since this commutativity is immediate for $p = 0$ or $q = 0$ and $\chains(\cube^n) \cong \chains(\cube^1)^{\otimes n}$, we only need to verify the commutativity of this diagram for $p = q = 1$.
	
	Consider $(x_1 \otimes y_1) \otimes (x_2 \otimes y_2) \in \chains(\cube^1)^{\otimes 2} \otimes \chains(\cube^1)^{\otimes 2}$.
	We have
	\begin{equation*}
	\begin{split}
	\big((\id \otimes \varepsilon) \otimes \ast \ + \ \ast \otimes (\varepsilon \otimes \Delta)\big) \ (-1)^{\bars{y_1} \bars{x_2}} \ (x_1 \otimes x_2) \otimes (y_1 \otimes y_2) \ & = \\
	(-1)^{\bars{y_1} \bars{x_2} + \bars{x_1} + \bars{x_2} + \bars{y_1}} \ x_1 \cdot \varepsilon(x_2) \otimes y_1 \ast y_2 \ & + \ 
	(-1)^{\bars{y_1} \bars{x_2} + \bars{x_1}} \ x_1 \ast x_2 \otimes \varepsilon(y_1) \ast y_2.
	\end{split}
	\end{equation*}
	The first summand on the left hand side is non-zero only if $\bars{x_2} = \bars{y_1} = \bars{y_2} = 0$, whereas the second is non-zero only if $\bars{x_1} = \bars{x_2} = \bars{y_1} = 0$, so the above is also equal to
	\begin{equation} \label{e:join is monoidal 1}
	(-1)^{\bars{x_1}} \ x_1 \cdot \varepsilon(x_2) \otimes y_1 \ast y_2 \ + \ 
	x_1 \ast x_2 \otimes \varepsilon(y_1) \ast y_2.
	\end{equation}
	On the other hand,
	\begin{align*}
	(x_1 \otimes y_1) \ast (x_2 \otimes y_2) \ & =\ 
	(-1)^{\bars{x_1} + \bars{y_1}} \ (x_1 \ast x_2) \otimes \varepsilon(y_1) \cdot y_2 \ +\
	(-1)^{\bars{x_1} + \bars{y_1}} \ x_1 \cdot \varepsilon(x_2) \otimes (y_1 \ast y_2) \\ \ & =\ 
	(x_1 \ast x_2) \otimes \varepsilon(y_1) \cdot y_2 \ +\
	(-1)^{\bars{x_1}} \ x_1 \cdot \varepsilon(x_2) \otimes (y_1 \ast y_2),
	\end{align*}
	which is equal to \eqref{e:join is monoidal 1} as claimed.
\end{proof}

%	In diagrammatic terms,
%	\begin{equation*}
%	\begin{tikzcd}
%	\coproduct \otimes \chains(\cube^{p+q}) \arrow[r, "\Delta_\M \otimes Sh"] \arrow[d] &[20pt]
%	\left( \ \coproduct \otimes \coproduct \ \right) \otimes
%	\chains(\cube^{p}) \otimes \chains(\cube^{q}) \arrow[d] \\
%	\chains(\cube^{p+q})^{\otimes 2} \arrow[r, "Sh"] &
%	\chains(\cube^{p})^{\otimes 2} \otimes \chains(\cube^{q})^{\otimes 2}
%	\end{tikzcd}
%	\end{equation*}
%	and
%	\begin{equation*}
%	\begin{tikzcd}
%	\counit \ \otimes \chains(\cube^{p+q}) \arrow[r, "\Delta_\M \otimes Sh"] \arrow[d] &[20pt]
%	\left( \ \counit \ \otimes \ \counit \ \right) \otimes
%	\chains(\cube^{p}) \otimes \chains(\cube^{q}) \arrow[d] \\
%	R \arrow[r, "\cong"] &
%	R \otimes R
%	\end{tikzcd}
%	\end{equation*}
%	commute.

\section{Theorem 1}

\subsection{Adams' map}

In \cite{adams1956cobar}, Adams constructed a natural map of differential graded algebras \begin{equation} \label{e:adams map}
\theta_Y \colon \cobar(\sS(Y, y)) \to S^{\square}(\Omega_y Y),
\end{equation}
for any pointed topological space $(Y,y)$. The construction is based on a collection of continuous maps $$\theta_n: \gcube^{n-1} \to P(\gsimplex^n;0,n),$$
where $P(\gsimplex^n;0,n)$ denotes the topological space of Moore paths in $\gsimplex^n$ from the $0$-th vertex $v_0$ to the $n$-th vertex $v_n$. These maps are constructed inductively so that following equations are satisfied:
\begin{itemize}
\item $\theta_1(0)\colon \gcube^1 \to \gsimplex^1$ is the path $\theta_1(0)(s)=sv_1 +(1-s)v_0$.
\item
 $\theta_n \circ e_j^0=P(d_j) \circ \theta_{n-1}$
\item
 $\theta_n \circ e_j^1=(P(f_j) \circ \theta_j) \cdot (P(l_{n-j}) \circ \theta_{n-j})$.
\end{itemize}
In the above equations $f_j\colon \gsimplex^j \rightarrow \gsimplex^n$ and $l_{n-j}\colon \gsimplex^{n-j} \rightarrow \gsimplex^n$ denote the first $j$-th face map and last $(n-j)$-th face map of $\gsimplex^n$, respectively, and
$e_j^0,e_j^1\colon \gcube^{n-1} \rightarrow \gcube^{n}$ denote the $j$-th bottom and top face inclusion maps, respectively. For any continuous map of spaces $f\colon X \to Y$, we denote by $P(f)\colon P(X) \to P(Y)$ the induced map at the level of spaces of paths. Given any two composable paths $\alpha$ and $\beta$, the dot symbol in $\alpha \cdot \beta$ denotes the composition (or concatenation) of paths.

Adams' map $\theta_Y$ is now given as follows. For any singular $1$-simplex $\sigma \in S^{\Delta}_1(Y, y)$ let
$$\theta_Y([\sigma])= P(\sigma) \circ \theta_1 - c_y,$$
where $c_y \in S^{\square}_0(\Omega_y Y)$ is the singular $0$-cube determined by the constant loop at $y \in Y.$ For any singular $n$-simplex $\sigma \in S^{\Delta}_n(Y, y)$ with $n>1$, let
$$\theta_Y([\sigma])= P(\sigma) \circ \theta_n.$$ Since the underlying graded algebra of $\cobar(\sS(Y, y))$ is free, we may extend the above to an algebra map $\theta_Y \colon \cobar(\sS(Y, y)) \to S^{\square}(\Omega_y Y)$. The compatibility equations for the maps $\theta_n$ imply that $\theta_Y$ is a chain map. 

We will construct an $E_{\infty}$-bialgebra structure on $\cobar(\sS(Y, y))$ and show that $\theta_Y$ preserves this structure. In order to do this, we shall describe a categorical version of Adams' constructions.




%In this section we revisit two combinatorial models for the based loop space of a reduced simplicial set $X \in \sSet^0$. First we discuss a cubical model which has the feature that after taking cubical chains we obtain a dg algebra which is naturally \textit{isomorphic} to the cobar construction on the connected dg coalgebra of simplicial normalized chains $\schains(X)$. This is a reinterpretation of a construction of Baues \cite{baues1998hopf}. We also describe a version of the model which is localized at the $1$-simplices of $X$ and recovers the extended cobar construction defined in \cite{hess2010cobar} after applying cubical chains. This localization step is required in order to obtain the correct homotopy type of the based loop space in the non-simply connected, non-fibrant setting. The construction of the cubical model for the based loop space is best described using necklical sets, a notion related to both simplicial sets and cubical sets with connections. Different approaches to this framework can be found in \cite{baues1998hopf}, \cite{galvez2020hopf}, \cite{dugger2011rigidification}, \cite{rivera2018cubical}, \cite{rivera2018cubical}, among other works. We start the section by reviewing this framework. 
%We finish the section by reviewing the classical Kan loop group functor and relating it to the cubical model for the based loop space. 

\subsection{Necklaces}

We construct a cubical model for the based loop space using necklical sets, a notion related to both simplicial sets and cubical sets with connections. Different approaches to this framework can be found in \cite{baues1998hopf}, \cite{galvez2020hopf}, \cite{dugger2011rigidification}, \cite{rivera2018cubical}, \cite{rivera2018cubical}, among other works. 

Let $\mathbf{\Delta}_{*,*}$ be the subcategory of $\mathbf{\Delta}$ consisting of objects $\{ [1], [2],\dots\}$ and morphisms being functors $f \colon [n] \to [m]$ satisfying $f(0)=0$ and $f(n)=m$. The category $\mathbf{\Delta}_{*,*}$ is a strict monoidal category when equipped with the monoidal structure $[n] \otimes [m]= [n+m]$ given by identifying objects $n \in [n]$ and $0 \in [m]$.

Denote by $$\mathsf{Bar} \colon \Mon_{\mathsf{Cat}} \to \mathsf{Fun}[\mathbf{\Delta}^{op}, \Mon_{\mathsf{Cat}}]$$ the bar construction from the category of monoidal categories to simplicial objects in $\Mon_{\mathsf{Cat}}$ and by 
$$| \text{ }\cdot\text{ }  | \colon \mathsf{Fun}[\mathbf{\Delta}^{op}, \Mon_{\mathsf{Cat}}] \to \Mon_{\mathsf{Cat}}$$ the realization functor. 

We define the \textit{category of necklaces} to be the monoidal category $$\Nec= |\mathsf{Bar}(\mathbf{\Delta}_{*,*}, \otimes)|.$$

 More explicitly, the objects of $\Nec$ are the elements of the set of ordered sequences
 $$ ob(\Nec) = \{ [n_1] \vee ... \vee[n_k] | n_i, k \in \mathbb{Z}_{>0} \} \bigcup \{ [0] \}$$ together with the following four types of generating morphisms:
\begin{itemize}
\item $\partial^j \colon [n-1] \to  [n] $
for $j=1,\dots,n-1$
\item $\Delta_{[j],[n-j]} \colon  [j]\vee [n-j] \to [n]$ for $j=1,\dots,n-1$
\item $ s^j \colon [n+1] \to [n]$ for $j=0,\dots,n$ and $n>0$, and 
\item $s^0 \colon [1] \to [0]$
\end{itemize}

 The objects of $\Nec$ are called \textit{necklaces}.
 The monoidal structure of $\Nec$ is given by concatenation of ordered sequences, denoted by $T \vee T'$ for any two necklaces $T, T' \in \Nec$. Given any necklace $T=[n_1] \vee ... \vee[n_k] \in \Nec$ define the \textit{dimension} of $T$ by $\text{dim}(T)=n_1+ ...+n_k-k$. A \textit{necklical set} is a functor $K \colon \Nec^{op} \to \Set$. Necklical sets form a category with natural transformations as morphisms, which we denote by $\nSet.$ Any necklace $T=[n_1] \vee ... \vee[n_k] \in \Nec$ gives rise to a simplicial set $\mathcal{S}(T)= \Delta^{n_1} \vee... \vee \Delta^{n_k} \in \sSet$, where the wedge symbol now means we identify the last vertex of $\Delta^{n_i}$ with the first vertex of $\Delta^{n_{i+1}}$ for $i=1,\dots,k-1$. Any necklace $T \in \Nec$ also gives rise to a necklical set $\mathcal{Y}(T)=\Hom_{\Nec}(\text{ \_ }, T) \colon \Nec^{op} \to \Set$. These constructions give rise to functors 
$$\mathcal{S} \colon \Nec \to \sSet$$
and
$$\mathcal{Y} \colon \Nec \to \nSet,$$ respectively. 

The category of necklical sets $\nSet$ is a (non-symmetric) monoidal category when equipped with monoidal structure $\otimes\colon \nSet \times \nSet \to \nSet$ given by $$K \otimes K' = \colim_{\mathcal{Y}(T) \to K, \mathcal{Y}(T')\to K'} \mathcal{Y}(T \vee T').$$

\subsection{Associating a cube to a necklace.}
We now describe a categorical version of Adams' collection of maps $\{\theta_n: \gcube^{n-1} \to P(\gsimplex^{n};0,n)\}$ by constructing a monoidal functor
$$\mathcal{P} \colon \Nec \to \cube$$
that associates a cube to any necklace. 

If $T= \Delta^0$ we set $\mathcal{P}(\Delta^0)=2^0$ and on any other $T \in \Nec$ we define $\mathcal{P}( T )= 2^{\text{dim}(T)}$. In order to define $\mathcal{P}$ on morphisms, it is sufficient to consider the following cases.
\begin{itemize}
\item For any $\partial^j \colon [n] \to [n+1]$ such that $0< j<{n+1}$, define $\mathcal{P}(\partial^j) \colon 2^{n-1}\to 2^{n}$ to be the cubical coface functor $\mathcal{P}(f)= \delta_0^{j}.$ 

\item For any $\Delta_{[j], [n+1-j]} \colon [j] \vee [n+1-j] \to [n+1]$ such that $0<j<n+1$, define $\mathcal{P}(\Delta_{[j], [n+1-j]}) \colon 2^{n-1}\to 2^{n}$ to be the cubical coface functor $\mathcal{P}(f)=\delta_1^{j}$.

\item We now consider morphisms of the form $s^j \colon [n+1] \to [n]$ for $n>0$. If $j=0$ or $j=n$, then $\mathcal{P}(f) \colon 2^n \to 2^{n-1}$ is defined to be the cubical codegeneracy functor $\mathcal{P}(s^j)= \varepsilon^{j}.$ If $0<j<n$, then we define $\mathcal{P}(s^j) \colon 2^n \to 2^{n-1}$ to be the cubical coconnection functor $\gamma^{j}.$

\item For $s^0 \colon [1] \to [0]$ we define $\mathcal{P}(s^0) \colon 2^0 \to 2^0$ to be the identity functor.

\end{itemize}
\begin{remark}
The functor $\mathcal{P}$ is neither faithful or full. However, for any necklace $T' \in \Nec$ with $\text{dim}(T')=n+1$ and any cubical coface functor $\delta_{\epsilon}^j \colon 2^n \to 2^{n+1}$ for $0 \leq j \leq n+1$, there exists a unique pair $(T, f \colon T \hookrightarrow T')$, where $T \in \Nec$ with $\text{dim}(T)=n$ and $f \colon T \hookrightarrow T'$ is an injective morphism in $\Nec$, such that $\mathcal{P}(f)=\delta_{\epsilon}^j $.
\end{remark}

The functor $\mathcal{P} \colon \Nec \to \cube$ induces an adjunction between $\cSet$ and $\nSet$
with right adjoint
$$\mathcal{P}^* \colon \cSet \to \nSet$$
and left adjoint
$$\mathcal{P}_{!}  \colon \nSet \to \cSet_c.$$
Given a cubical set with connections $C \colon \square_c^{op} \to \Set$, we have $$\mathcal{P}^*(C)= C \circ \mathcal{P}^{op}.$$ Given a necklical set $K \colon \Nec^{op} \to \Set$, we have $$\mathcal{P}_{!}(K)= \colim_{(\mathcal{Y}(T) \to K) \in \mathcal{Y} \downarrow K} \mathcal{P}(T) \cong \colim_{(\mathcal{Y}(T) \to K) \in \mathcal{Y} \downarrow K} \cube^{\text{dim}(T)}.$$ 
The functor $\mathcal{P}_{!}: \nSet \to \cSet$ is clearly a monoidal functor. 

\subsection{Geometric cobar construction}

Using the framework of necklical sets, we may reinterpret a construction of Baues as a functor
$$\gcobar^{\text{nec}} \colon \sSet^0 \to \Mon_{\nSet},$$ which we now define. 

For any $0$-reduced simplicial set $X$, the underlying necklical set $\gcobar^{\text{nec}}(X) \colon \Nec^{op} \to \Set$ is defined by
$$\gcobar^{\text{nec}}(X) = \colim_{(f \colon \mathcal{S}(T) \to X) \in  \mathcal{S} \downarrow X} \mathcal{Y}(T).$$
The monoidal structure $$\gcobar^{\text{nec}}(X) \otimes \gcobar^{\text{nec}}(X) \to \gcobar^{\text{nec}}(X)$$
is induced by the monoidal structure $\vee: \Nec \times \Nec \to \Nec$ together with the monoidal structure $\otimes: \nSet \times \nSet \to \nSet$ on necklicals sets. More precisely, for any $S, S' \in \Nec$, the product $$\gcobar^{\text{nec}}(X)(S) \otimes \gcobar^{\text{nec}}(X)(S') \to \gcobar^{\text{nec}}(X)(S \vee S')$$ is given by $$[f\colon \mathcal{S}(T) \to X, a] \otimes [f'\colon \mathcal{S}(T') \to X, a'] \mapsto [f \vee g\colon \mathcal{S}(T\vee T') \to X, a \otimes  a'],$$
where $(f\colon \mathcal{S}(T) \to X), (f'\colon \mathcal{S}(T') \to X) \in \mathcal{S} \downarrow X$, $a\in \mathcal{Y}(T)(S)$, and $a'\in \mathcal{Y}(T')(S')$.

We may now define the \textit{geometric cobar construction} $$\gcobar: \sSet^0 \to \Mon_{\cSet}$$ as the composition $$\gcobar= \mathcal{P}_! \circ \gcobar^{\text{nec}}.$$ This is a further reinterpretation of Baues construction in terms of cubical sets with connections, which was also studied in \cite{rivera2018cubical}. 

\subsection{Relation to the cobar construction}

The geometric cobar functor $\gcobar\colon \sSet^0 \to \Mon_{\cSet}$ is related to the cobar construction $\mathbf{\Omega}\colon \coAlg \to \Mon_{\Ch}$ via normalized chains as follows.

\begin{proposition} \label{gcobarandcobar}
There is a natural isomorphisms of functors 
$$\cchains \circ \gcobar \cong \mathbf{\Omega} \circ \schains: \sSet^0 \to \Mon_{\Ch}.$$
\end{proposition}

\begin{proof} 
Denote by $\iota_n \in (\cube^n)_n$ the top dimensional non-degenerate element of the standard $n$-cube with connections $\cube^n$. Note that for any $X \in \sSet^0$, we may represent any non-degenerate $n$-cube $\alpha \in (\mathcal{P}_!(\gcobar^{\text{nec}}(X)))_n$ as a pair $\alpha=[\sigma: \mathcal{Y}(T) \to X, \iota_n]$ for some $T=[n_1] \vee ... \vee [n_k] \in \Nec$ with $\text{dim}(T)=n_1+ ...+n_k-k=n.$

To define an algebra map
$$\varphi_X: \cchains(\mathcal{P}_!(\gcobar^{\text{nec}}(X))) \xra{\cong} \mathbf{\Omega}(\schains(X))$$
it suffices to define it on any generator of the form $\alpha=[\sigma \colon \Delta^{n+1} \to X, \iota_{n}]$, i.e. when $T$ is of the form $T=[n+1]$, for some $n\geq0$. If $n=0$ let $\varphi_X(\alpha)= [\overline{\sigma}]+ 1_R$, where $[\overline{\sigma}] \in s^{-1} \overline{ \schains(X)} \subset \mathbf{\Omega}(\schains(X))$ denotes the (length $1$) generator in the cobar construction of $\schains(X)$ determined by $\sigma \in X_{n+1}$ and $1_R$ denotes the unit of the underlying ring $R$. If $n>0$, we let $\varphi_X(\alpha)=[\overline{\sigma}]$. A straightforward computation yields that this gives rise to a well defined isomorphism of algebras, which is compatible with the differentials, and natural with respect to maps of simplicial sets.  
\end{proof}

\subsection{$E_{\infty}$-bialgebra structure on the cobar construction}\label{UMoncobar}

Using Proposition \ref{gcobarandcobar} together with Theorem \ref{t:lift chains on cSet to UM coAlg is monoidal} we may obtain a natural $E_{\infty}$-bialgebra structure on $\cobar(\schains(X))$ of any $X \in \sSet^0$ extending Baues' coproduct. Namely, using the observation that $\cchainsUM \colon \cSet \to \coAlg_{\UM}$ is monoidal, the desired $E_{\infty}$-bialgebra structure is provided by the composition 
$$\cchainsUM \circ \gcobar \colon \sSet^0 \to \Mon_{\U(M)}.$$

\subsection{Factorization of Adams' map}\label{factorization}

Adams' map can now be factored as 
$$\theta_Y \colon \cobar(\sS(Y, y)) \xrightarrow{\cong} 
\cchains(\gcobar \sSing(Y, y)) \xrightarrow{\cchains(\Theta)} 
S^{\square}(\Omega_y Y).$$
The first map is given by the natural isomorphism of algebras $$\cobar \schains (\sSing(Y, y)) \cong  \cchains \gcobar (\sSing(Y, y))$$ of Proposition \ref{gcobarandcobar}. The second map is given by applying the cubical chains functor $\cchains \colon \Mon_{\cSet} \to \Mon_{\Ch}$ to the map of monoidal cubical sets
$$\Theta \colon \gcobar (\sSing(Y, y) \to \cSing( \Omega_yY)$$
determined by sending any singular $n$-simplex $$(\sigma\colon \gsimplex^n \to Y) \in \sSing(Y, y)$$ to the singular $(n-1)$-cube $$(P(\sigma) \circ \theta_n \colon \gcube^{n-1} \to \Omega_yY) \in \cSing( \Omega_yY).$$ 

\subsection{Proof of Theorem 1.}
We now put together the constructions and results discussed above to deduce our first main theorem. 
\begin{theorem}
	The functor $\cchainsUM$ is monoidal and its associated functor on monoids fits in the following diagram commuting up to a natural isomorphism:
	\begin{equation*}
	\begin{tikzcd}
	& \Mon_{\coAlg_\UM} \arrow[d] \\
	\Mon_{\cSet} \arrow[ru, "\cchainsUM", out=70, in=180] \arrow[r, "\cchainsAS"]
	& \Mon_{\coAlg} \arrow[d] \\
	\sSet^0 \arrow[r, "\cobar \circ \schains"] \arrow[u, "\gcobar"]
	& \Mon_{\Ch}.
	\end{tikzcd}
	\end{equation*}
	Furthermore, when $(Y, y)$ is a pointed space, Adams' comparison map $\theta_Y \colon \cobar(\sS(Y, y)) \to S^{\square}(\Omega_y Y)$ is a quasi-isomorphism of $E_\infty$ bialgebras or, more specifically, of monoids in $\coAlg_\UM$.
\end{theorem} 

\begin{proof}
The fact that $\cchainsUM$ is monoidal and that it fits in the desired commutative diagram follows directly from Section 4. 
    
The natural $\UM$-coalgebra structure on $\cobar(\sS(Y, y))$ is induced by $\cchainsUM ( \gcobar (\sSing(Y,y)) )$, as discussed in \ref{UMoncobar}. The natural $\UM$-coalgebra structure on $S^{\square}(\Omega_yY)$ is precisely given by $\cchainsUM( \cSing(\Omega_yY) ).$ These two $\UM$-coalgebra structures are compatible with the algebra structures, giving rise to monoinds in $\coAlg_{\UM}.$
     
By naturality, we obtain a map
     $$ \cobar(\schains(\sSing(Y,y))) \cong \cchainsUM ( \gcobar (\sSing(Y,y)) ) \xrightarrow{ \cchainsUM(\Theta)} \cchainsUM( \cSing(\Omega_yY) )$$
of monoids in $\coAlg_{\UM}$,  lifting Adams' classical map by \ref{factorization}, as desired. Since $\sSing(Y,y)$ is a fibrant reduced simplicial set, Adams' map $\theta_Y$ is a quasi-isomorphism, as proven in \cite{rivera2018cubical} and \cite{rivera2019path}.  
\end{proof}


\section{Theorem 2}

The goal of this section is to prove \cref{t:2nd main thm in the intro}. This will require some basic constructions and results from homotopy theory, which we will review for completeness. 

\subsection{Kan's loop group construction}

Recall that simplicial sets form a combinatorial model for the homotopy theory of spaces.
This may be expressed using the language of model categories by saying that the adjoint functors
\begin{equation*}
\bars{\ \cdot\ } \colon \sSet \leftrightarrows \mathsf{Top} \colon \sSing
\end{equation*}
defines a Quillen equivalence between model category structures.
%in which the weak equivalences are the weak homotopy equivalences.
We call this model category structure on $\sSet$ the \textit{Kan-Quillen model structure}.
There is also a induced model structure on $\sSet^0$ which we call by the same name.

The based loop space of a reduced simplicial set may be modeled in purely combinatorial terms through the \textit{Kan loop group functor} 
\begin{equation*}
G \colon \sSet^0 \to \mathsf{sGrp}
\end{equation*}
defined as follows.
For any $X \in \sSet^0$ each $G(X)_n$ is the group with one generator $\overline{x}$ for every simplex $x \in X_{n+1}$ modulo the relation $\overline{s_0(x)} = e$.
The face and degeneracy maps are defined by the following equations
\begin{enumerate}
    \item $\delta_0(\overline{x}) = \overline{\delta_1(x)} \cdot (\overline{\delta_0(x)})^{-1}$,
    \item $\delta_n(\overline{x})= \overline{\delta_{n+1}(x)}$ for $n >0$,
    \item $s_n(\overline{x})= \overline{s_{n+1}(x)}$.
\end{enumerate}

Berger showed that this construction models the based loop space \textit{as a topological monoid,} more precisely, he showed that for any $X \in \sSet^0$, the geometric realization of the Kan loop group $|G(X)|$ is weak homotopy equivalent, as a topological monoid, to the based loop space $\Omega|X|$ on the geometric realization of $X$ \cite{berger1995loops}.

One may construct a model category structure on the category $\mathsf{sGrp}$ of simplicial groups such that 
both weak equivalences and fibrations are maps of simplicial groups whose underlying maps of simplicia sets are weak homotopy equivalences and Kan fibrations. We call this model category structure the \textit{Kan model structure on $\sGrp.$}

The Kan loop group functor has a right adjoint usually denoted by 
$$\overline{W} \colon \mathsf{sGrp} \to \sSet^0,$$ which is a model for the classifying space of a simplicial group. This adjunction induces an equivalence of homotopy theories in the following sense.


\begin{theorem} \label{kan}
The adjunction
		$$G \colon \sSet^0 \leftrightarrows \mathsf{sGrp}\colon \overline{W}$$
		becomes a Quillen equivalence when $\sSet^0$ is equipped with the Kan-Quillen model category structure and $\sGrp$ with the Kan model structure.
\end{theorem}

For a proof of the above statement we refer to \cite[Chapter V]{goerss2009simplicial}.

\subsection{Localization of the cubical cobar construction}

The proof that Adams' map in \cref{t:1st main thm in the intro} is a quasi-isomorphism relies on the fibrancy condition of the underlying Kan complex. In general, for an arbitrary reduced simplicial set $X$, the cobar construction $\cobar \schains(X)$ is not naturally quasi-isomorphic to $S^{\cube}(\Omega|X|),$ the singular cubical chains on the based loop space of the geometric realization of $X$. This is true, however, if $X$ is $1$-connected.

In \cite{hess2010cobar}, it is described how to formally invert the elements of $X_1$ inside $\cobar \schains(X)$ to obtain the correct quasi-isomorphism type. The resulting model, called the extended cobar construction, is then compared as an algebra to the chains on the classical the Kan loop group construction $G(X)$. We lift these constructions to the ``space-level" with the goal of comparing the $E_{\infty}$-coalgebra structures after taking chains. 

Denote by $$\mathcal{L} \colon \Mon_{\sSet} \to \sGrp$$ the functor from simplicial monoids to simplicial groups defined by formally inverting all morphisms (degree by degree) subject to the usual relations.
If $M \in \Mon_{\sSet}$ and $A \in M_0$ is a set of $0$-morphisms, we denote by $\mathcal{L}_AM$ the simplicial monoid obtained by formally inverting the elements of $A$, i.e., the pushout
$$\mathcal{L}_AM = M \coprod_{A} \mathcal{L}F(A),$$
where $F(A)$ is the monoid freely generated by the set $A$.

Define 
$$\ncobarE \colon \sSet^0 \to \Mon_{\nSet}$$
as the localization
$$\ncobarE(X)= \mathcal{L}_{X_1} \ncobar(X),$$
namely $\ncobarE(X)$ is the monoidal necklical set obtained by adding to $\ncobar(X)$  formal inverses for all $f\colon \simplex^1 \to X$ (together with the corresponding degenerate elements generated by the new formal inverses) subject to the usual relations.

Finally denote by $$\widehat{\ccobar} \colon \sSet^0 \to \Mon_{\cSet}$$ the composition $$\widehat{\ccobar}= \mathcal{P}_{!} \circ \ncobarE.$$ 

\subsection{Relation to the extended cobar construction.}

Denote $$\A= \cobar \circ \schains \colon \sSet^0 \to \Mon_{\Ch}.$$ The extended cobar construction of \cite{hess2010cobar} is not quite functorial on $\coAlg$ since it depends on a choice of basis of the degree one $R$-module of the underlying coalgebra.
It is, however, functorial with respect to maps of reduced simplicial sets.
Hence, we reinterpret the extended cobar construction as the functor
\begin{equation*}
\Ahat \colon \sSet^0 \to \Mon_{\Ch}
\end{equation*}

given on any $X \in \sSet^0$ by formally inverting the set of $0$-cycles
\begin{equation*}
A_X = \{ [\overline{\sigma}]+1_R \mid \sigma \in X_1 \} \subset \A(X)_0
\end{equation*}
in the associative algebra $\A(X)$.
This construction has the property that, for any $X \in \sSet^0$ there is a natural isomorphism of algebras
\begin{equation*}
H_0(\Ahat(X)) \cong R[\pi_1(X)].
\end{equation*}

As an immediate consequence of \cref{p:ccobar and cobar}, we obtain the following isomorphism after localizing.

\begin{corollary}
	There is a natural isomorphism of functors
	$$\cchains \circ\ \widehat{\ccobar} \cong \Ahat :\sSet^0 \to \Mon_{\Ch}.$$ 
\end{corollary}

\subsection{Rigidification and homotopy coherent nerve}

In order to relate the functors $\widehat{\ccobar} \colon \sSet^0 \to \Mon_{\cSet}$ and $G \colon \sSet^0\to \sGrp$ we introduce a third player, the \textit{rigidification functor} $$\mathfrak{C} \colon \sSet^0 \to \Mon_{\sSet}.$$

This functor is obtained by restricting to $\sSet^0$ a more general construction
$$\mathfrak{C}: \sSet \to \Cat_{\simplex},$$
where $\Cat_{\simplex}$ denotes the category of small categories enriched over the monoidal category $(\sSet, \times)$. 

Given integers $0 \leq  i \leq j$ denote by $P_{i,j}$ the category whose objects are all the subsets of $\{i, i+1, \dots, j\}$ containing both $i$ and $j$ and morphisms are inclusions.
For each integer $n \geq 0$ define $\mathfrak{C}(\simplex^n) \in \Cat_{\simplex}$ to have the set $\{0, \dots, n\}$ as objects and if $i \leq j$, define $\mathfrak{C}(\simplex^n)(i,j)= N(P_{i,j})$, where $N$ is the ordinary nerve functor.
If $j < i$, $\mathfrak{C}(\simplex^n)(i,j) = \emptyset.$ The composition law in $\mathfrak{C}(\simplex^n)$ is induced by the functor $P_{j,k} \times P_{i,k} \to P_{i,k}$ defined as the union of sets.
The assignment $[n] \mapsto \mathfrak{C}(\simplex^n)$ defines a cosimplicial object in $\Cat_{\simplex}$.
For any $S \in \sSet$ we can now define
 $$\mathfrak{C}(S) = \underset{{\simplex^n \to S} }{\text{colim }} \mathfrak{C}(\simplex^n).$$
The functor $\mathfrak{C}$ has a right adjoint, called the \textit{homotopy coherent nerve functor} and denoted by
$$\mathfrak{N} \colon \Cat_{\simplex} \to \sSet,$$ whose $n$-simplices are given by 
$$\mathfrak{N}(\mathcal{C})_n=\text{Hom}_{\Cat_{\simplex}}(\mathfrak{C}(\simplex^n), \mathcal{C}).$$

The homotopy coherent nerve functor was originally introduced by Cordier and further studied by Joyal and Lurie in the theory of $\infty$-categories.
Dugger and Spivak described the construction more explicitly in terms of necklaces in \cite{dugger2011rigidification}.
The adjunction between rigidification and homotopy coherent nerve yield an equivalence of two homotopy theories that model the theory of $\infty$-categories.
The following statement is attributed to both Lurie and Joyal.

\begin{theorem} \label{joyalbergner}
	The adjunction $$ \mathfrak{C} \colon \sSet \leftrightarrows \Cat_{\simplex} :\!\mathfrak{N}$$ induces a Quillen equivalence when $\sSet$ is equipped with the Joyal model structure and $\Cat_{\simplex}$ with the Bergner model structure.
\end{theorem}


\begin{remark}
	The Bergner model structure on $\Cat_{\simplex}$ has as weak equivalences those maps of simplicially enriched categories that induce a weak homotopy equivalence between the simplicial sets of morphisms and are essentially surjective after passing to the homotopy categories.
	Fibrant objects in the Bergner model structure are precisely categories enriched over Kan complexes.
	On the other hand, fibrant objects in the Joyal model structure on $\sSet$ are precisely quasi-categories and all objects are cofibrant.
	The weak equivalences are known as \textit{Joyal equivalences} and, based on the above result, we can take them to be maps of simplicial sets that become weak equivalences of simplicially enriched categories after applying $\mathfrak{C}$.
	Furthermore, the Kan-Quillen model structure is a left Bousfield localization of the Joyal model structure obtained by localizing the single morphism $\simplex^1 \to \simplex^0$.
\end{remark}

The next result relates the cubical cobar construction and the rigidification functor via the triangulation functor as defined in  \cref{ss:triangulation and its adjoint}.

\begin{proposition}\label{Candgcobar}
	The composition 
	$$\mathcal{T} \circ \ccobar \colon \sSet^0 \to \Mon_{\sSet}$$ is naturally isomorphic to
	$$\mathfrak{C} \colon \sSet^0 \to \Mon_{\sSet}.$$
\end{proposition}

The above statement is Proposition 5.3 in \cite{rivera2018cubical}, where $\widehat{\ccobar}$ is denoted by $\mathfrak{C}_{\cube_c}.$

\cref{Candgcobar} also holds in the many-object setting, but we do not need this level of generality for the purposes of this article.




\subsection{The rigidification functor and Kan's loop group construction}

The starting point of the comparison between the rigidification functor and Kan's loop group functor is the following statement about their adjoints, which is Proposition 2.6.2 in \cite{hinich2007deformation}.

\begin{proposition} \label{hinich}
	For any simplicial group $\mathcal{G} \in \mathsf{sGrp}$, we have a natural weak homotopy equivalence
	$$\psi_{\mathcal{G}} \colon \overline{W}(\mathcal{G}) \xrightarrow{\simeq} \mathfrak{N}(\mathcal{G}).$$
\end{proposition} 

We now deduce the following relationship between the rigidification functor $\mathfrak{C}$ and the Kan loop group functor $G$.

\begin{proposition} \label{CandG}
	Let $X \in \sSet^0$ be a reduced simplicial set.
	There are natural weak equivalences of simplicial monoids
	$$\mathcal{L}_{X_1} \mathfrak{C}(X) \xra{\simeq} \mathcal{L}\mathfrak{C}(X) \xra{\simeq} G(X).$$
\end{proposition}

\begin{proof}
	Since $\mathfrak{C}$ is a left Quillen functor and every $X \in \sSet$ is cofibrant in the Joyal model structure, it follows from \cref{joyalbergner} that the simplicially enriched category $\mathfrak{C}(X)$ is a cofibrant.
	Proposition 9.5 of \cite{dwyer1980simplicial} implies that the natural inclusion $\mathcal{L}_{X_1} \mathfrak{C}(X) \to \mathcal{L}\mathfrak{C}(X)$ is a weak equivalence of simplicially enriched categories.
	
	By \cref{hinich} we have that for any $X \in \sSet^0$, $\psi_{G(X)} \colon \overline{W}(G(X)) \xrightarrow{\simeq} \mathfrak{N}(G(X))$ is a weak homotopy equivalence.
	By \cref{kan}, we have a weak homotopy equivalence $X \xrightarrow{\simeq} \overline{W}(G(X))$ given by the unit of the adjunction.
	Composing these two maps we obtain a weak homotopy equivalence
	$$X \xrightarrow{\simeq} \mathfrak{N}(G(X)).$$ 
	The Quillen equivalence of \cref{joyalbergner} localizes to a Quillen equivalence
	$$\mathcal{L} \circ \mathfrak{C} \colon \sSet^0 \leftrightarrows \mathsf{sGrp} \colon \mathfrak{N} \circ \iota$$
	when $\sSet^0$ is equipped with the Kan-Quillen model structure and $\mathsf{sGrp}$ with the model structure of \cref{kan}.
	Here $\iota \colon \mathsf{sGrp} \to \Cat_{\simplex}$ denotes the natural inclusion functor.
	It follows that the adjoint of the weak homotopy equivalence $X \xra{\simeq} \mathfrak{N}(G(X))$ is a weak equivalence of simplicial groups
	$$\mathcal{L}\mathfrak{C}(X) \xra{\simeq} G(X).$$ 
\end{proof}

\subsection{Localized cubical cobar and the Kan loop group}

We have the following comparison between the triangulation of the localized geometric cobar construction and the Kan loop group.

\begin{corollary} \label{widehatgcobarandG}
	Let $X \in \sSet^0$.
	There is a natural weak equivalence of simplicial monoids
	$$\mathcal{T} \widehat{\ccobar}(X) \xrightarrow{\simeq}G(X).$$
\end{corollary}

\begin{proof}
	After localizing the isomorphism of \cref{Candgcobar} at the $1$-simplices, we obtain a natural isomorphism of simplicial monoids
	$$\mathcal{T} \widehat{\ccobar}(X) \cong \mathcal{L}_{X_1}\mathfrak{C}(X).$$
	Hence, the result follows from \cref{CandG}.
\end{proof}

The above statement should be understood as a lift to the level of simplicial sets of Hess and Tonks' map between the extended cobar construction and the chains on the Kan loop group.

\subsection{The map $\Ahat$}

\anibal{The map should be defined outside the proof. It is OK to prove the first half of thm 2 here.}

\subsection{Serre--Cartan comparison map}

We review from \cite{medina2021cubical} the relationship between the $E_\infty$-structures defined on simplicial chains in \cref{ss:e-infty on simplicial} and on cubical chains in \cref{ss:e-infty on cubical}.

In \cite[sect]{serre1951homologie}, the \textit{Serre--Cartan collapse} $\gcube^n \to \gsimplex^n$ was introduced and used to define a natural quasi-isomorphism of coalgebras $\sS(Z) \to \cS(Z)$ for any topological space $Z$.
This map factors as
\begin{equation*}
\chains(\sSing(Z)) \xra{\CS_{\sSing(Z)}}
\cchains(\mathcal{U} \sSing(Z)) \to
\chains(\cSing(Z)),
\end{equation*}
where $\CS_X$, referred to as the \textit{Serre--Cartan comparison map}, is a quasi-isomorphism defined for any simplicial set $X$, whereas the other chain map is induced from a cubical map.

Although both $\schains(X)$ and $\cchains(\mathcal{U} X)$ have natural $\UM$-structures, the map $\CS_X$ is not a morphism of $\UM$-coalgebras for a generic simplicial set $X$.
But, after restriction of their $\UM$-structures to a sub-$E_\infty$-operad, the Cartan-Serre comparison map becomes a morphism of $E_\infty$-coalgebras.
This operad, denoted $\USL$, is generated as a suboperad of $\UM$ by and all immerse graphs of the form
\begin{center}
	\begin{tikzpicture}[scale=1]
	\draw (0,0)--(0,-.6) node[below, scale=.75]{$1$};
	\draw (0,0)--(.5,.5);
	\draw (-.3, .3)-- (-.2,.5) node[scale=.75] at (-.2,.7) {\qquad $1\, \ \ 2\ \ ...\ \ k_1$};
	\draw (-.5,.5)--(0,0);
	\node[scale=.75] at (.11,.4){$...$};
	
	\node[scale=.75] at (1,0){$\cdots$};
	\node[scale=.75] at (1,-.9){$\cdots$};
	
	\draw (2,0)--(2,-.68) node[scale=.75, below]{$r$};
	\draw (2,0)--(2.5,.5);
	\draw (1.7, .3)--(1.8,.5) node[scale=.75] at (1.78,.7) {\qquad $1\, \ \ 2\ \ ...\ \ k_r$};
	\draw (1.5,.5)--(2,0);
	\node[scale=.75] at (2.11,.4){$...$};
	
	\draw (1,2.5)--(1,3) node[scale=.75, above]{$1$};
	\draw (1,2.5)--(0,2) node[scale=.75, below]{$1$};
	\draw (.25,2.125)--(.5,2) node[scale=.75, below]{$2$};
	\draw (.5,2.25)--(1,2) node[scale=.75, below]{$3$};
	\draw (1,2.5)--(2,2) node[scale=.75, below]{\ \quad $n + r$};
	\node[scale=.75] at (1.5,1.75){$\cdots$};
	
	\node[scale=.75] at (1,1.3) {$\vdots$};
	
	\node at (2.85,0){};
	\end{tikzpicture}
\end{center}
where there are no hidden vertices and the strands are joined so that the associated maps $\{1, \dots, k_j\} \to \{1, \dots, n+k\}$ are order-preserving.
From \cite{medina2021cubical} we have the following statement.
\begin{lemma}
	The suboperad $\USL$ of $\UM$ is an $E_\infty$-operad and the Cartan-Serre comparison map
	\begin{equation*}
	\CS_X \colon \schains(X) \to \cchains(\mathcal U X)
	\end{equation*}
	is a natural quasi-isomorphism of $\USL$-coalgebras for any simplicial set $X$.
\end{lemma}

We will denote by $\chains_\USL$ the functor of (cubical or simplicial) chains with its induced $\USL$-structure.

We remark that the inclusion $\As \to \UM$ factors through $\USL$.

\subsection{A zig-zag of $E_\infty$-coalgebras}

\anibal{Please the zig-zag outside the proof}

\subsection{Proof of Theorem 2}

We now put together the constructions and results discussed above to deduce our first main theorem, stated in the introduction as \cref{t:2nd main thm in the intro} and restated here for convenience.

\begin{nntheorem}
	There exists a functor $\Ahat_\UM \colon \sSet^0 \to \Mon_{\coAlg_\UM}$ that fits into a commutative diagram
	\begin{equation*}
	\begin{tikzcd}[row sep=small]
	& \Mon_{\coAlg_\UM} \arrow[d] \\
	& \Mon_{\coAlg} \arrow[d] \\
	\sSet^0
	\arrow[r, "\Ahat"', ]
	\arrow[ru, "\Ahat_{\coAlg}"', out=45, in=180] 
	\arrow[ruu, "\Ahat_\UM", out=90, in=180]
	& \Mon_{\Ch}.
	\end{tikzcd}
	\end{equation*}
	Furthermore, for any $X \in \sSet^0$, there is a zig-zag of natural quasi-isomorphisms of $\USL$-coalgebras between $\Ahat_\USL(X)$ and $\schainsUSL(G(X))$.
\end{nntheorem}

\begin{proof}
	Define the functor
	$$\Ahat_{\UM} \colon \sSet^0 \to \Mon_{\coAlg_{\UM}}$$
	as the composition
	$$\Ahat_{\UM} = \cchainsUM \circ \widehat{\ccobar}.$$
	The commutativity of the desired diagram is straightforward by construction.
	
	We now construct a zig-zag of quasi-isomorphisms of $E_{\infty}$-coalgebras between $\cchainsUSL(\widehat{\ccobar}(X)) = \Ahat_{\USL}(X)$ and $\schainsUSL (G(X))$ using the $E_{\infty}$ suboperad $\USL$ of $\UM$, which is suitable for comparing simplicial and cubical chains.
	
	\manuel{the following should just obtained by referencing the corresponging statement in the previous section about serre comparison map.}
	
	For any $C \in \cSet$ there are quasi-isomorphisms of $\USL$-coalgebras
	\begin{equation} \label{e:zig-zag of quasi-isos}
	\schainsUSL(\mathcal{T}C) \xra{\simeq}
	\cchainsUSL(\mathcal{U}\mathcal{T}C) \xleftarrow{\simeq}
	\cchainsUSL(C).
	\end{equation}	
	The first map $\schainsUSL(\mathcal{T}C) \to \cchainsUSL(\mathcal{U}\mathcal{T}C)$ is induced by the Serre collapse map.
	This is a natural map
	$$\schains(S) \to \cchains(\mathcal{U}S)$$
	defined for any simplicial set $S \in \sSet.$
	The fact that this is a quasi-isomorphism follows from a standard acyclic models argument by taking the set of simplicial cubes $\{ \simplex^0, \simplex^1, (\simplex^1)^{\times 2}, \dots\}$ as the collection of acyclic models inside $\sSet$. This map preserves the $\USL$-coalgebra structure by \cite{medina2021cubical}.
	
	The second map in \eqref{e:zig-zag of quasi-isos} is induced by the morphism of cubical sets $C \to \mathcal{U} \mathcal{T} (C)$ given by the unit of the adjunction
	$$\mathcal{T} \colon \cSet \leftrightarrows \sSet \colon \mathcal{U}.$$
	The unit map is in fact a weak homotopy equivalence of cubical sets since the adjunction defines a Quillen equivalence between the Kan-Quillen model structure and an analogue model structure on $\cSet$ in which all objects are cofibrant \cite{cisinski2006presheaves}.
	
	Hence, by naturality, the unit map induces a quasi-isomorphism of $\USL$-coalgebras after applying cubical chains.
	
	Applying the above to $C= \widehat{\ccobar}(X),$ we obtain that $\schainsUSL (\mathcal{T} \widehat{\ccobar}(X))$ and $\cchainsUSL(\widehat{\ccobar}(X)) = \Ahat_{\USL}(X)$ are quasi-isomorphic as $\USL$-coalgebras.
	
	By \cref{widehatgcobarandG} we have a weak equivalence of simplicial monoids
	$$\mathcal{T}\widehat{\ccobar}(X) \xrightarrow{\simeq} G(X)$$ for any $X \in \sSet^0.$ By naturality, we have an induced quasi-isomorphism of $\UM$-coalgebras
	$$\schainsUSL(\mathcal{T}\widehat{\ccobar}(X)) \xra{\simeq} \schainsUSL(G(X)).$$
	Therefore, $\Ahat_{\USL}(X)$ and $\schainsUSL(G(X))$ are quasi-isomorphic as $\USL$-coalgebras, as desired.
\end{proof}

\bibliographystyle{alpha} % ieeetr
\bibliography{biblio}

\end{document}