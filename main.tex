\documentclass{amsart}
\usepackage{amsmath, amssymb}
\usepackage{tikz-cd}
\usepackage{mathbbol} % mathbb on greek letters

% hyperref
\usepackage[bookmarks=true, linktocpage=true,
bookmarksnumbered=true, breaklinks=true,
pdfstartview=FitH, hyperfigures=false,
plainpages=false, naturalnames=true,
colorlinks=true, pagebackref=true,
pdfpagelabels]{hyperref}
\hypersetup{
	colorlinks,
	citecolor=blue,
	filecolor=blue,
	linkcolor=blue,
	urlcolor=blue
}

% layout
\setlength{\textwidth}{\paperwidth}
\addtolength{\textwidth}{-1.85in}
\setlength{\textheight}{\paperheight}
\addtolength{\textheight}{-2in}
\calclayout

% Updating to MSC2020
\makeatletter
\@namedef{subjclassname@2020}{%
	\textup{2020} Mathematics Subject Classification}
\makeatother

% elements
\renewcommand{\P}{\mathcal{P}}
\renewcommand{\O}{\mathcal{O}}
\renewcommand{\S}{\mathbb{S}}
\newcommand{\A}{\mathcal{A}s}
\newcommand{\M}{\mathcal{M}}

% sets
\newcommand{\Z}{\mathbb{Z}}
\newcommand{\id}{\mathrm{id}}
\newcommand{\End}{\mathrm{End}}
\newcommand{\Hom}{\mathrm{Hom}}
\newcommand{\Bij}{\mathfrak{Bij}}
\newcommand{\G}{\mathfrak{G}}

% categories
\newcommand{\C}{\mathsf{C}}
\newcommand{\Fun}{\mathsf{Fun}}
\newcommand{\Ch}{\mathsf{Ch}}
\newcommand{\Alg}{\mathsf{Alg}}
\newcommand{\coAlg}{\mathsf{coAlg}}
\newcommand{\biAlg}{\mathsf{biAlg}}
\newcommand{\simplex}{\triangle}
\newcommand{\cube}{\square}
\newcommand{\Set}{\mathsf{Set}}
\newcommand{\sSet}{\mathsf{sSet}}
\newcommand{\cSet}{\mathsf{cSet}}
\newcommand{\Nec}{\mathsf{Nec}}
\newcommand{\nSet}{\mathsf{nSet}}
\newcommand{\Mon}{\mathsf{Mon}}
\newcommand{\smod}{\mathsf{Mod}_{\S}}
\newcommand{\sbimod}{\mathsf{biMod}_{\S}}
\newcommand{\operads}{\mathsf{Oper}}
\newcommand{\props}{\mathsf{Prop}}

% functors
\DeclareMathOperator*{\tensor}{\otimes}
\newcommand{\Y}{\mathcal{Y}}
\newcommand{\chains}{N^{\triangle}}
\newcommand{\cochains}{N}
\newcommand{\cchains}{N}
\newcommand{\ccochains}{N}
\newcommand{\op}{\mathrm{op}}
\DeclareMathOperator*{\colim}{colim}
\newcommand{\cobar}{\mathbf{\Omega}}
\newcommand{\gcobar}{\mathbb{\Omega}}

% environments
\newtheorem{theorem}{Theorem}
\newtheorem{proposition}[theorem]{Proposition}
\newtheorem{lemma}[theorem]{Lemma}
\newtheorem{corollary}[theorem]{Corollary}
\theoremstyle{definition}
\newtheorem{definition}[theorem]{Definition}
\newtheorem{example}[theorem]{Example}
\newtheorem{remark}[theorem]{Remark}

% other
\newcommand{\anibal}[1]{\textcolor{blue}{\underline{Anibal}: #1}}
\newcommand{\manuel}[1]{\textcolor{red}{\underline{Manuel}: #1}}
\renewcommand{\th}{^\mathrm{th}}
\newcommand{\bars}[1]{\vert{#1}\vert}

% drawings
\include{aux/figures}

\begin{document}
\title{The cobar construction as an $E_\infty$-Hopf algebra model for the based loop space}
\author{Anibal M. Medina-Mardones}
\address{Max Plank Institute for Mathematics, Bonn, Germany}
\email{ammedmar@mpim-bonn.mpg.de}
\address{Department of Mathematics, University of Notre Dame, Notre Dame, IN, USA}
\email{amedinam@nd.edu}
\author{Manuel Rivera}
\address{Purdue University}
\email{manuelr@purdue.edu}

\keywords{.}
\subjclass[2020]{.}

\begin{abstract}
    We extend a dg bialgebra structure on the cobar coconstruction of the dg coalgebra of chains on a reduced simplicial set, originally constructed by H. Baues in \cite{Baues}, to a natural $E_{\infty}$-coalgebra structure compatible with the associative product of the cobar construction. This is achieved by taking advantage of the cubical structure of the cobar construction of the coalgebra chains on a reduced simplicial set, as discussed in \cite{rivera-zeinalian-cubical}, and constructing a Hopf operad structure on a particular model of the $E_{\infty}$-operad suited for acting on cubical chains, as described in \cite{medina2021cubical}. After localizing the cobar construction appropriately, we prove the resulting $E_{\infty}$-coalgebra structure is naturally quasi-isomorphic to the $E_{\infty}$-coalgebra on the simplicial chains on the Kan loop group construction, extending a result of M. Franz for the underlying coassociative coalgebra structures \cite{franz}.
\end{abstract}

\begin{abstract}
    The classical cobar construction of Adams, producing a monoid in chain complexes from the coalgebra of chains on a reduced simplicial set, was shown by Baues to be a monoid in the category of coalgebras by identifying it with the chains of a cubical monoid, a chain complex naturally equipped with the Serre coproduct.
    In this paper we extend this result by showing the cobar construction to be a monoid in the category of $E_\infty$ coalgebras for a combinatorial model of the $E_\infty$ operad.
    Additionally, for any reduced simplicial set we show that, as monoids in $E_\infty$ coalgebras, the cobar construction and the chains on the Kan loop group are quasi-isomorphic.
\end{abstract}



\vspace*{-1cm}

\maketitle
\tableofcontents
% !TEX root = ../cobar1.tex

\section{Introduction}

For any topological space $\fX$, its complex of simplicial or cubical singular chains $\Schains(\fX)$ -- regarded as a differential graded (dg) abelian group -- encodes the homology of $\fX$ in its quasi-isomorphism type.
More homotopical information can be stored in the quasi-isomorphism type of this chain complex if considered as a (dg) coassociative coalgebra, which we will denote $\SchainsA(\fX)$, where the coproduct comes from a canonical choice of chain approximation to the diagonal $\fX \to \fX \times \fX$.
%, attributed respectively to Alexander--Whitney and Serre.
For instance, the cohomology ring of $\fX$ is retained, but the action of the Steenrod algebra on its mod $p$ cohomology is not.

In Mandell's seminal work \cite{mandell2006homotopy_type} it is shown that, when $\fX$ is nilpotent and finite type, the entire homotopy type of $\fX$ can be encoded in the quasi-isomorphism type of this complex if considered as an $E_\infty$-coalgebra, a structure providing $\SchainsA(\fX)$ with coherent homotopies witnessing the derived cocommutativity of the coproduct coming from the strict symmetry of the diagonal map.

%In this work we are interested in modeling algebraically the homotopy type of the based loop space $\loops_x \fX$ of a pointed space $(\fX, x)$.
%Via concatenation of loops this space is a topological monoid, whose induced structure on the set of path-connected components is a group; the fundamental group of the underlying space.

The first contribution of this paper is to explicitly endow the cubical singular chains of the based loop space $\loops_x \fX$, with the structure of a monoidal $E_\infty$-coalgebra extending the Serre diagonal.
More specifically, we verify that the monoid structure induced on $\cSchains(\loops_x \fX)$ by the concatenation of loops is compatible with a natural $E_\infty$-coalgebra structure on cubical singular chains, similar to the one defined in \cite{medina2022cube_einfty}.
%We denote this functor by $\cchainsUM \colon \cSet \to \coAlg_\UM$

%We will also concern ourselves with another algebraic model approximating the homotopy type of $\loops_x \fX$.
%which was introduced by Adams \cite{adams1956cobar}.
%This model \cite{adams1956cobar} is obtained by applying Adams' celebrated \textit{cobar construction} functor $\cobar$ to a pointed version of the coalgebra of simplicial singular chains of $(\fX, x)$.
%The cobar functor may be thought of as an algebraic counterpart to the based loop space.
%Just like the based loop space functor is part of a duality between pointed topological spaces and group-like topological monoids, the cobar functor is part of one between a pointed version of dg coalgebras and monoids in chain complexes.
% (i.e. dg algebras).
%Adams' cobar construction provides a model for the quasi-isomorphism type of the monoid of chains on the based loop space.

Applying Adams' cobar construction to the coalgebra of simplicial singular chains of $(\fX, x)$, he obtained another monoidal algebraic model $\cobar \sSchainsA(\fX, x)$ of $\loops_x \fX$ \cite{adams1956cobar}.
More precisely, he constructed a natural monoidal chain map $\theta$ from $\cobar \sSchainsA(\fX, x)$ to $\cSchains(\loops_x \fX)$ and proved it to be a quasi-isomorphism if $\fX$ is simply-connected, a statement that also holds true for path-connected spaces after \cite{rivera2018cubical}.
The model $\cobar \sSchainsA(\fX, x)$ is smaller than $\cSchains(\loops_x \fX)$ and unlocks effective analysis of quantitative and qualitative properties of $\loops_x \fX$, as illustrated for instance in \cite{chainalgebraloops} and \cite{adamscobarequivalence}.

The second main contribution of this paper is to make Adams model into a monoidal $E_\infty$-coalgebra and to prove that
\[
\theta \colon \cobar \sSchainsA(\fX, x) \to \cSchains(\loops_x \fX)
\]
respects this higher structure.

Our starting point is groundbreaking work by H.~Baues, which imply statements similar to those in this work but in the category of coalgebras.
He reinterpreted Adams' algebraic construction at a deeper geometric level \cite{baues1998hopf}, which allowed him to endow $\cobar \sSchainsA(\fX, x)$ with the structure of a monoidal coalgebra, and to show that $\theta$ preserves this structure.
To prove our statement we interpret Adams' construction at an even deeper categorical level.
%The simplicial singular complex $\sSing(\fX, x)$ is an example of an object in the category $\sSet^0$ of $0$-reduced simplicial sets.
We interpret Baues' geometric cobar construction, originally defined for $1$-reduced simplicial sets, as a functor
\begin{equation*}
	\ccobar \colon \sSet^0 \to \Mon_{\cSet},
\end{equation*}
from the category of $0$-reduced simplicial sets to that of monoidal cubical sets.
The key difference with Baues' original work is the use of connections to obtain a natural construction before geometric realization.

Additionally, we need a suitable model of the $E_\infty$-operad endowing cubical chains with a natural $E_\infty$-coalgebra extending the Serre diagonal.
For this we take the operad $\UM$ introduced in \cite{medina2020prop1}.
After proving that its coalgebras form a monoidal category, we show that the functor $\cchainsUM \colon \cSet \to \coAlg_\UM$ -- defined in \cite{medina2022cube_einfty} with a different sign convention -- is monoidal.
This allows us to prove the following generalization of Adams and Baues structures.
%This functors is equal to that introduced in \cite{medina2022cube_einfty} up to signs.

%that $\cchainsUM$ from cubical sets to $\UM$-coalgebras is monoidal, which implies that $\cchainsUM(\ccobar X)$ is a monoidal $\UM$-coalgebra for any 0-reduced simplicial set.
%More precisely, we prove the following.

\begin{theorem*}
	The following diagram commutes up to natural isomorphisms:
	\[
	\begin{tikzcd} [row sep=small]
		& \Mon_{\coAlg_\UM} \arrow[d] \\
		\Mon_{\cSet} \arrow[ru, "\cchainsUM", out=70, in=180, near start] \arrow[r, "\cchainsA"]
		& \Mon_{\coAlg} \arrow[d] \\
		\sSet^0 \arrow[r, "\cobar \schainsA"] \arrow[u, "\ccobar"]
		& \Mon_{\Ch},
	\end{tikzcd}
	\]
	where the unlabeled arrows are forgetful functors.
\end{theorem*}

In the above diagram, the arrow from $\sSet^0$ to $\Mon_{\Ch}$ is Adams' cobar construction, the one from $\sSet^0$ to $\Mon_{\coAlg}$ is Baues' enhancement, and the one from $\sSet^0$ to $\Mon_{\coAlg_{\UM}}$ is our lift.
Additionally we have the following statement about Adams's map.

\begin{theorem*}
	For any pointed space $(\fX, x)$,
	\[
	\theta \colon \cobar \sSchainsA(\fX, x) \to \cSchains(\loops_x \fX)
	\]
	is a quasi-isomorphism of monoidal $\UM$-coalgebras.
\end{theorem*}

The fact that $\theta$ respects the monoid structure in $\Ch$ was proven by Adams, whereas the compatibility of the monoid structure with the Serre coalgebra structure was established by Baues.
Our contribution is the compatibility of the monoid structure with a full $E_\infty$-coalgebra extension of Serre's coalgebra.
% !TEX root = ../cobar1.tex

\section{Conventions and preliminaries}\label{s:preliminaries}

\subsection{Presheaves and monoids}

Recall that a category is said to be \textit{small} if its objects and morphisms form sets.
We denote the category of small categories by $\Cat$.
Given categories $\sB$ and $\sC$ with $\sB$ small we denote their associated \textit{functor category} by $\Fun(\sB, \sC)$.
A category is said to be \textit{cocomplete} if any functor to it from a small category has a colimit.
If $\sA$ is small and $\sC$ cocomplete, then the (left) \textit{Kan extension} of $g$ along $f$ exists for any pair of functors $f$ and $g$ in the diagram below, and it is the initial object in $\Fun(\sB, \sC)$ making
\begin{equation*}
	\begin{tikzcd}[column sep=normal, row sep=normal]
		\sA \arrow[d, "f"'] \arrow[r, "g"] & \sC \\
		\sB \arrow[dashed, ur, bend right] & \quad
	\end{tikzcd}
\end{equation*}
commute.
A Kan extension along the \textit{Yoneda embedding}, i.e., the functor
\[
\yoneda \colon \sA \to \Fun(\sA^\op, \Set)
\]
induced by the assignment
\[
a \mapsto \big( a^\prime \mapsto \sA(a^\prime, a) \big),
\]
is referred to as a \textit{Yoneda extension}.
Objects in the image of the Yoneda embedding are said to be \textit{representable}.
Any presheaf $P$ is isomorphic to a colimit of representables presheaves $P \, \cong \colim_{\yoneda(a) \downarrow P} \yoneda(a)$.

A \textit{monoid} $(M, \mu, \eta)$ in a monoidal category $(\sC, \ot, \mathbb{1})$ is an object $M$ together with two morphisms $\mu \colon M \ot M \to M$ and $\eta \colon \mathbb{1} \to M$, called \textit{multiplication} and \textit{unit} respectively, satisfying the usual associativity and unital relations.
We denote the category of monoids in $\sC$ by $\Mon_{\sC}$ and remark that a monoidal functor $\sC \to \sC^\prime$ induces a functor between their categories of monoids $\Mon_\sC \to \Mon_{\sC^\prime}$.

\subsection{Coalgebras}\label{ss:coalgebras}

Throughout this article $\k$ denotes a commutative and unital ring and we work over its associated closed symmetric monoidal category of differential (homologically) graded $\k$-modules $(\Ch, \ot, \k)$.
We refer to the objects and morphisms of this category as \textit{chain complexes} and \textit{chain maps} respectively.
We denote by $\Hom(C, C^\prime)$ the chain complex of $\k$-linear maps between chain complexes $C$ and $C^\prime$, and refer to the functor $\Hom(-, \k)$ as \textit{linear duality}.
The $i^\th$ \textit{suspension} functor $s^i \colon \Ch \to \Ch$ is defined at the level of graded modules by $(s^{i}M)_n = M_{n-i}$.

A \textit{coalgebra} consists of a chain complex $C$ and chain maps $\Delta \colon C \to C \ot C$ and $\varepsilon \colon C \to \k$ satisfying the usual coassociativity and counitality relations.
This notion is equivalent to that of a comonoid in $\Ch$.
Denote by $\coAlg$ the category of coalgebras with morphisms being structure preserving chain maps.
The category $\coAlg$ is symmetric monoidal, with braiding induced from $\Ch$ and structure maps of a product $C \ot C^\prime$ given by
\begin{gather*}
	C \ot C^\prime \xra{\Delta \ot \Delta^\prime}
	(C \ot C) \ot (C^\prime \ot C^\prime) \xra{(23)}
	(C \ot C^\prime) \ot (C \ot C^\prime), \\
	C \ot C^\prime \xra{\varepsilon \ot \varepsilon^\prime}
	\k \ot \k \xra{\cong} \k.
\end{gather*}

A \textit{coaugmentation} on a coalgebra $C$ is a coalgebra map $\nu \colon \k \to C$.
A coalgebra is said to be \textit{coaugmented} if it is equipped with a coaugmentation.
We denote by $\coAlg^\ast$ the category of coaugmented coalgebras with morphisms being coaugmentation preserving coalgebra maps.
A coaugmented coalgebra is a \textit{connected coalgebra} if it is $0$ in negative degrees and the coaugmentation induces an isomorphism $\k \cong C_0$ of $\k$-modules.
We denote by $\coAlg^0$ the full subcategory of $\coAlg^\ast$ defined by these.

\subsection{Simplicial algebraic topology}\label{ss:simplicial}

The \textit{simplex category} is denoted by $\simplex$ and its objects by $[n]$.
%Its morphisms are generated by the usual \textit{coface} $\delta_i \colon [n-1] \to [n]$ and \textit{codegeneracy} $\sigma_i \colon [n+1] \to [n]$ maps.
The category of \textit{simplicial sets} $\Fun(\simplex^\op, \Set)$ is denoted by $\sSet$ and the \textit{standard $n$-simplex} $\yoneda [n] = \simplex(-, [n])$ by $\simplex^n$.
For any simplicial set $X$ we write, as usual, $X_n$ instead of $X [n]$, and identify the elements of $\simplex^n_m$ with increasing tuples $[v_0, \dots, v_m]$ where $v_i \in \{0, \dots, n\}$.

If $X$ is such that $X_0$ is a singleton set we say that $X$ is \textit{reduced}.
We denote the full subcategory of reduced simplicial sets by $\sSet^0$.

We denote the topological $n$-simplex by $\gsimplex^{n}$ and consider the usual adjunction pair defined by the \textit{geometric realization} and \textit{singular complex}
\[
\begin{tikzcd}
	\bars{\ } \colon \sSet  \arrow[r, shift left=.5ex] &
	\Top :\! \sSing. \arrow[l, shift left=.5ex]
\end{tikzcd}
\]

The functor of (normalized) \textit{simplicial chains} $\schains \colon \sSet \to \Ch$ is obtained from the geometric realization and the functor of cellular chains.
When no confusion arises from doing so we write $\chains$ instead of $\schains$ and refer to it simply as the functor of \textit{chains}.
We will denote the functor of (simplicial) \textit{singular chains} $\schains \circ \sSing \colon \Top \to \Ch$ by $\sSchains$.

We modify this construction for a pointed topological space $(\fZ,z)$ by only considering maps $\gsimplex^n \to \fZ$ sending all vertices to $z$.
This produces a reduced simplicial set $\sSing(\fZ,z)$ whose chains we denote by $\sSchains(\fZ,z)$.

There is a classical lift of the functor of chains to coalgebras:
\[
\begin{tikzcd}
	& \coAlg \arrow[d] \\
	\sSet \arrow[r, "\schains"] \arrow[ur, "\schainsA", out=70, in=180] & \Ch
\end{tikzcd}
\]
defined, via a Yoneda extension, by the following structure on the chains of standard simplices.
For any $n \in \N$, define $\epsilon \colon \chains(\simplex^n) \to \k$ by
\[
\epsilon \big( [v_0, \dots, v_q] \big) = \begin{cases} 1 & \text{ if } q = 0, \\ 0 & \text{ if } q>0, \end{cases}
\]
and $\Delta \colon \chains(\simplex^n) \to \chains(\simplex^n)^{\ot2}$ by
\[
\Delta \big( [v_0, \dots, v_q] \big) = \sum_{i=0}^q \ [v_0, \dots, v_i] \ot [v_i, \dots, v_q].
\]

If $X$ is a reduced simplicial set then $\schainsA(X)$ becomes a connected coalgebra with the coaugmentation $\nu \colon \k \to \schains(X)$ induced by sending $1$ to the basis element represented by the unique $0$-simplex of $X$.
Hence, the functor $\schainsA$ restricts to a functor
\[
\schainsA \colon \sSet^0 \to \coAlg^0.
\]

\subsection{Cubical algebraic topology}

The objects of the \textit{cube category} $\cube$ are the sets $2^n = \{0, 1\}^n$ with $2^0 = \{0\}$ for $n \in \N$, and its morphisms are generated by the \textit{coface, codegeneracy} and \textit{coconnection} maps defined by
\begin{align*}
	\delta_i^\varepsilon & =
	\mathrm{id}_{2^{i-1}} \times \delta^\varepsilon \times \mathrm{id}_{2^{n-1-i}} \colon 2^{n-1} \to 2^n, \\
	\sigma_i & =
	\mathrm{id}_{2^{i-1}} \times \, \sigma \times \mathrm{id}_{2^{n-i}} \quad \colon 2^{n} \to 2^{n-1}, \\
	\gamma_i & =
	\mathrm{id}_{2^{i-1}} \times \, \gamma \times \mathrm{id}_{2^{n-i}} \quad \colon 2^{n+1} \to 2^{n},
\end{align*}
where $\varepsilon \in \{0,1\}$ and
\[
\begin{tikzcd}
	1 \arrow[r, out=45, in=135, "\delta^0"] \arrow[r, out=-45, in=-135, "\delta^1"'] & 2 \arrow[l, "\sigma"'] &[-10pt] \arrow[l, "\gamma"'] 2 \times 2
\end{tikzcd}
\]
are defined by
\begin{gather*}
	\delta^0(0) = 0, \qquad
	\delta^1(0) = 1, \qquad
	\sigma(0) = \sigma(1) = 0, \qquad \\
	\gamma^1(0,0) = 0, \qquad
	\gamma^1(0,1) = 1, \qquad
	\gamma^1(1,0) = 1, \qquad
	\gamma^1(1,1) = 1.
\end{gather*}
We remark that the cube category without connections will not be considered in this work.

The category of \textit{cubical sets} $\Fun(\cube^\op, \Set)$ is denoted by $\cSet$ and
the \textit{standard $n$-cube} $\yoneda 2^n = \cube(-, 2^n)$ by $\cube^n$.
For any cubical set $Y$ we write, as usual, $Y_n$ instead of $Y(2^n)$.

We denote the topological $n$-cube by $\gcube^{n}$ and consider the usual adjunction pair defined by the \textit{geometric realization} and \textit{singular complex}
\[
\begin{tikzcd}
	\bars{\ } \colon \cSet  \arrow[r, shift left=.5ex] &
	\Top :\! \cSing. \arrow[l, shift left=.5ex]
\end{tikzcd}
\]

The functor of (normalized) \textit{cubical chains} $\cchains \colon \cSet \to \Ch$ is obtained from the geometric realization and the functor of cellular chains.
When no confusion arises from doing so we write $\chains$ instead of $\cchains$ and refer to it simply as the functor of \textit{chains}.
We remark that $\chains(\cube^n)$ is canonically isomorphic to $\chains(\cube^1)^{\ot n}$ and that $\chains(\cube^1)$ is canonically isomorphic to the cellular chains on the topological interval with its usual CW structure.
We use these isomorphisms to denote the elements of $\chains(\cube^n)$ as integral linear combinations monoids of the form $x_1 \ot\dotsb\ot x_n$ with $x_i \in \big\{[0], [0,1], [1] \big\}$.
We will denote the functor of (cubical) \textit{singular chains} $\cchains \circ \cSing \colon \Top \to \Ch$ by $\cSchains$.

There is a classical lift of the functor of cubical chains to the category of coalgebras
\[
\begin{tikzcd}
	& \coAlg \arrow[d] \\
	\cSet \arrow[r, "\cchains"] \arrow[ur, "\cchainsA", out=70, in=180] & \Ch
\end{tikzcd}
\]
defined, using a Yoneda extension, by the tensor product structure (\cref{ss:coalgebras}) defined on $\chains(\cube^n) \cong \chains(\cube^1)^{\ot n}$ via the identification of $\chains(\cube^1)$ and $\chains(\simplex^1)$.

%\subsubsection{Cubical chains}
%
%For non-negative integers $m$ and $n$, let $\cube_{\deg}(2^m, 2^n)$ be the subset of \textit{degenerate morphisms} in $\cube(2^m, 2^n)$, i.e., those of the form $\sigma_i \circ \tau$ or $\gamma_i \circ \tau$ with $\tau$ any morphism in $\cube(2^m, 2^{n+1})$.
%The functor of (cubical) \textit{chains} $\cchains \colon \cSet \to \Ch$ is the Yoneda extension of the functor $\cube \to \Ch$ defined next.
%It assigns to an object $2^n$ the chain complex having in degree $m$ the module
%\[
%\frac{\k\{\cube(2^m, 2^n)\}}{\k\{\cube_{\deg}(2^m, 2^n)\}}
%\]
%and differential induced by
%\[
%\partial (\id_{2^n}) = \sum_{i=1}^{n} \ (-1)^i \
%\big(\delta_i^1 - \delta_i^0 \big).
%\]
%To a morphism $\tau \colon 2^n \to 2^{n^\prime}$ it assigns the chain map
%\[
%\begin{tikzcd}[row sep=-3pt, column sep=normal,
%	/tikz/column 1/.append style={anchor=base east},
%	/tikz/column 2/.append style={anchor=base west}]
%	\cchains(\cube^n)_m \arrow[r] & \cchains(\cube^{n^\prime})_m \\
%	\big( 2^m \to 2^n \big) \arrow[r, mapsto] & \big( 2^m \to 2^n \stackrel{\tau}{\to} 2^{n^\prime} \big).
%\end{tikzcd}
%\]
%
%When no confusion arises we write $\chains$ instead of $\cchains$.


%\subsubsection{Cubical singular complex}
%
%Consider the topological $n$-cube
%\[
%\gcube^{n} = \{(x_1, \dots, x_n) \mid x_i \in [0,1]\}.
%\]
%The assignment $2^n \to \gcube^n$ defines a functor $\cube \to \Top$ whose Yoneda extension is known as \textit{geometric realization}.
%It has a right adjoint $\cSing \colon \Top \to \cSet$ given by
%\[
%Z \to \Big(2^n \mapsto \Top(\gcube^n, Z)\Big)
%\]
%and referred to as the \textit{cubical singular complex} of the topological space $Z$.
%We will refer to $\chains(\cSing Z)$ as the \textit{cubical singular chains} of $Z$ and for simplicity denote it by $\cSchains(Z)$.

%\subsubsection{Serre coalgebra}\label{ss:serre coalgebra}
%
%We review a classical lift of the functor of cubical chains to the category of coalgebras
%\[
%\begin{tikzcd}
%	& \coAlg \arrow[d] \\
%	\cSet \arrow[r, "\cchains"] \arrow[ur, "\cchainsA", out=70, in=180] & \Ch.
%\end{tikzcd}
%\]
%Using a Yoneda extension, it suffices to equip the chains on standard cubes with a natural coalgebra structure.
%For any $n \in \N$, define $\epsilon \colon \chains(\cube^n) \to \k$ by
%\[
%\epsilon \left( x_1 \ot \dots \ot x_d \right) = \epsilon(x_1) \dotsm \epsilon(x_n),
%\]
%where
%\[
%\epsilon([0]) = \epsilon([1]) = 1, \qquad \epsilon([0, 1]) = 0,
%\]
%and $\Delta \colon \chains(\cube^n) \to \chains(\cube^n)^{\ot 2}$ by
%\[
%\Delta (x_1 \ot \dots \ot x_n) =
%\sum \pm \left( x_1^{(1)} \ot \dots \ot x_n^{(1)} \right) \ot
%\left( x_1^{(2)} \ot \dots \ot x_n^{(2)} \right),
%\]
%where the sign is determined using the Koszul convention, and we are using Sweedler's notation
%\[
%\Delta(x_i) = \sum x_i^{(1)} \ot x_i^{(2)}
%\]
%for the chain map $\Delta \colon \chains(\cube^1) \to \chains(\cube^1)^{\ot 2}$ defined by
%\[
%\Delta([0]) = [0] \ot [0], \quad \Delta([1]) = [1] \ot [1], \quad \Delta([0,1]) = [0] \ot [0,1] + [0,1] \ot [1].
%\]
%
%Using the canonical isomorphism $\chains(\cube^n) \cong \chains(\cube^1)^{\ot n}$, the coproduct $\Delta$ can be described as the composition
%\[
%\begin{tikzcd}
%	\chains(\cube^1)^{\ot n} \arrow[r, "\Delta^{\ot n}"] & \left( \chains(\cube^1)^{\ot 2} \right)^{\ot n} \arrow[r, "sh"] & \left( \chains(\cube^1)^{\ot n} \right)^{\ot 2},
%\end{tikzcd}
%\]
%where $sh$ is the shuffle map that places tensor factors in odd position first.
\section{Combinatorial $\M$-bialgebras}

Recall that a category is said to be \textit{small} if its objects and morphisms form sets. A category is said to be \textit{cocomplete} if any functor to it from a small category has a colimit.
If $\mathsf{A}$ is small and $\mathsf{C}$ cocomplete, then the (left) \textit{Kan extension of $g$ along $f$} exists for any pair of functors
\begin{equation*}
\begin{tikzcd}[column sep=normal, row sep=normal]
\mathsf{A} \arrow[d, "f"'] \arrow[r, "g"] & \mathsf{C} \\ 
\mathsf{B} \arrow[dashed, ur, bend right] & \quad .
\end{tikzcd}
\end{equation*}
Given categories $\mathsf{B}$ and $\C$, we denote the \textit{functor category} between them as $\Fun(\mathsf{B}, \C)$.


\subsection{On simplices}

For any non-negative integer $n$ denote by $[n]$ the category generated by the poset (or finite ordinal) $\{0 \to 1 \to \dots \to n\}$.
The \textit{simplex category} $\simplex$ is the category with objects $[0], [1], \dots$ and morphisms given by all functors $f \colon [n] \to [m]$.
The morphisms in $\simplex$ are generated by the usual
\textit{simplicial co-face} and \textit{co-degeneracy maps}
\begin{equation*}
\delta_i \colon [n-1] \to [n], \qquad \sigma_i \colon [n+1] \to [n]
\end{equation*}
for $j \in \{0, \dots, n\}$.
We denote by $\simplex_{\deg}$ the subcategory of $\simplex$ with the same objects and morphisms of the form $\sigma_i \circ \tau$ for any morphism $\tau$ of $\simplex$.

The category of \textit{simplicial sets} is the functor category $\sSet = \Fun(\simplex^\op, \Set)$.
The \textit{standard $n$-simplex} is the simplicial set $\simplex^n = \simplex(-, [n])$, and the \textit{Yoneda embedding} $\Y \colon \simplex \to \sSet$ is the functor induced by $[n] \mapsto \cube^n$.
For any simplicial set $X$ we have
\begin{equation*}
X_n \cong \colim_{\simplex^n \to X} \simplex^n.
\end{equation*}

Consider the functor $\simplex \to \Ch$ defined by
\begin{equation*}
\schains([n])_m = \frac{R\{\simplex([m], [n])\}}{R\{\simplex_{\deg}([m], [n])\}},
\qquad \qquad
\partial(\tau) = \sum (-1)^i \tau \circ \delta_i.
\end{equation*}

We review from \cite{Medina20prop1} a natural $\mathcal M$-bialgebra structure on the normalized chains of standard simplices $\schains(\triangle^n)$.
The $\M$-bialgebra is specified by three linear maps, the images of the generators
\begin{equation*}
\counit, \quad \coproduct, \quad \product,
\end{equation*}
satisfying the relations in the presentation of $\mathcal M$. For $n \in \mathbb{N}$, define: \vspace*{5pt} \\
(1) The counit $\epsilon \in \Hom(\schains(\triangle^n), \Z)$ by
\begin{equation*}
\epsilon \big( [v_0, \dots, v_q] \big) = \begin{cases} 1 & \text{ if } q = 0, \\ 0 & \text{ if } q>0. \end{cases}
\end{equation*}
(2) The coproduct $\Delta \in \Hom(\schains(\triangle^n), \schains(\triangle^n)^{\otimes2})$ by
\begin{equation*}
\Delta \big( [v_0, \dots, v_q] \big) = \sum_{i=0}^q [v_0, \dots, v_i] \otimes [v_i, \dots, v_q].
\end{equation*}
(3) The product $\ast \in \Hom(\schains(\triangle^n)^{\otimes 2}, \schains(\triangle^n))$ by
\begin{equation*}
\left[v_0, \dots, v_p \right] \ast \left[v_{p+1}, \dots, v_q\right] = \begin{cases} (-1)^{p+|\pi|} \left[v_{\pi(0)}, \dots, v_{\pi(q)}\right] & \text{ if } v_i \neq v_j \text{ for } i \neq j, \\
0 & \text{ if not}, \end{cases}
\end{equation*}
where $\pi$ is the permutation that orders the totally ordered set of vertices, and $(-1)^{|\pi|}$ its sign. The coproduct $\Delta$ is known as the \textit{Alexander-Whitney diagonal}.

\begin{proposition}[\cite{Medina20prop1}] \label{p:simplicial chain bialgebra}
	For every $n \in \mathbb{N}$, the assignment
	\begin{equation*}
	\counit \mapsto \epsilon, \quad \coproduct \mapsto \Delta, \quad \product \mapsto \ast,
	\end{equation*}
	defines a natural $\mathcal M$-bialgebra structure  on $\schains(\triangle^n)$.
\end{proposition}

\begin{proof}
	See Theorem 4.2 in \cite{Medina20prop1}.
\end{proof}

 By applying the forgetful functor $U$ from the category of props to the category of operads, the construction of Proposition \ref{p:simplicial chain bialgebra} gives rise to a functor $\simplex \to \coAlg_{U(\M)}$. Since the category of coalgebras over an operad is cocomplete, we may use Kan extension along the Yoneda embedding to define a lift of the functor of chains regarded as a coalgebra with the Alexander-Whitney diagonal:
\begin{equation*}
\begin{tikzcd}[column sep=small, row sep=small]
& \coAlg_{U(\M)} \arrow[d] \\
& \coAlg \arrow[d] \\
\sSet \arrow[r, "N"] \arrow[dashed, uur, bend left]& \Ch.
\end{tikzcd}
\end{equation*}

As explained in \cite{Medina20prop1}, this $E_\infty$-coalgebra on simplicial chains generalizes the constructions of McClure-Smith \cite{McClure-Smith} and Berger-Fresse \cite{Berger-Fresse}.

\subsection{On cubes}

The \textit{cube category with connections} $\cube$ is the subcategory of $\mathsf{Cat}$ such that
\begin{itemize}
    \item the objects of $\cube$ are the categories $1, 2^1, 2^2, ...$, where $1=2^0$ is the category with one object and one morphism, $2$ is the category with two objects $\mathfrak{0}$ and $\mathfrak{1}$ and only one non-identity morphism $\mathfrak{0}\to \mathfrak{1}$, and $2^n$ denotes the Cartesian product of $n$ copies of $2$.
    
    \item the morphisms of $\cube$ are generated by  \textit{cubical co-faces} $\delta^{\epsilon}_i: 2^{n-1} \to 2^{n}$ $(i=0,...,n, \epsilon=0,1$), \textit{cubical co-degeneracies} $\sigma_i: 2^n \to 2^{n-1}$, $(i=1,...,n)$ and \textit{cubical co-connections} $\gamma_i: 2^n \to 2^{n-1}$ ($i=1,...,n-1, n\geq 2$), see \cite{Brown-Higgins}, \cite{rivera-zeinalian-cubical}. 
\end{itemize}
We denote by $\cube_{\deg}$ the subcategory of $\cube$ with the same objects and morphisms of the form $\sigma_i \circ \tau$ and $\gamma_i \circ \tau$ for any morphism $\tau$ in $\cube$.  The category $\square$ becomes a monoidal category when equipped with $2^n \otimes 2^m := 2^{n+m}$. 

%is the free strict monoidal category with a \textit{bipointed object}
%\begin{equation*}
%\begin{tikzcd}
%1 \arrow[r, bend left, "\delta^0"] \arrow[r, bend right, "\delta^1"'] & 2 \arrow[r, "\sigma"] & 1
%\end{tikzcd}
%\end{equation*}
%such that $\sigma \circ \delta^0 = \sigma \circ \delta^1 = \mathrm{id}$. Explicitly, it contains an object $2^n$ for each non-negative integer $n$ and its morphisms are generated by the \textit{coface} and \textit{codegeneracy maps} defined by
%\begin{align*}
%\delta_i^\varepsilon & = \mathrm{id}_{2^{i-1}} \times \delta^\varepsilon \times \mathrm{id}_{2^{n-1-i}} \colon 2^{n-1} \to 2^n, \\
%\sigma_i & = \mathrm{id}_{2^{i-1}} \times \, \sigma \times \mathrm{id}_{2^{n-i}} \colon 2^{n} \to 2^{n-1}.
%\end{align*}
%We denote by $\cube_{\deg}$ the subcategory with the same objects as $\cube$ and morphisms of the form $\sigma_i \circ \tau$ for any morphisms $\tau$ of $\cube$.

The category of \textit{cubical sets with connections} is the functor category $\cSet = \Fun(\cube^\op, \Set)$.
The \textit{standard $n$-cube} is the cubical set $\cube^n = \cube(-, 2^n)$, and the \textit{Yoneda embedding} $\Y \colon \cube \to \cSet$ is the functor induced by $2^n \mapsto \cube^n$.
For any cubical set $X$ we have
\begin{equation*}
X_n \cong \colim_{\cube^n \to X} \cube^n.
\end{equation*}
Let $\cchains \colon \cube \to \Ch$ be the monoidal functor defined by
\begin{equation*}
N(2^n)_m =
\frac{R\{\cube(2^m, 2^n)\}}{R\{\cube_{\deg}(2^m, 2^n)\}}\,,
\qquad \qquad
\partial (\tau) = \sum_{i=1}^m (-1)^i (\tau \circ \delta^1_i - \tau \circ \delta^0_i)
\end{equation*}
We remark that $\cchains(2)$ is isomorphic to the cellular chains $C(\mathbb I)$ on the interval with its standard CW structure, and that $\cchains(2^n)$ is therefore isomorphic to $C(\mathbb I)^{\otimes n}$. 

Since the category $\Ch$ is cocomplete, we may extend $\cchains \colon \cube \to \Ch$ by Kan extension along the Yoneda embedding to define the functor of \textit{(normalized) chains} $$\cchains \colon \cSet \to \Ch.$$
By construction this functor satisfies $\cchains(\cube^n) \cong C(\mathbb I)^{\otimes n}$.

We recall a natural $\mathcal M$-bialgebra structure on $\cchains(\square^n)$ for every $n \geq 0$, where $\mathcal M$ is the $E_{\infty}$-prop defined in \ref{propM}. This structure is determined by three linear maps satisfying the relations in the presentation of $\mathcal M$.
For $n \in \mathbb{N}$, define: \vspace*{5pt} \\
(1) The counit $\epsilon \in \Hom(\cchains(\square^n), R)$ by
\begin{equation*}
\epsilon \left( x_1 \otimes \cdots \otimes x_d \right) = \epsilon(x_1) \cdots \, \epsilon(x_n),
\end{equation*}
where
\begin{equation*}
\epsilon([0]) = \epsilon([1]) = 1, \qquad \epsilon([0, 1]) = 0.
\end{equation*} \vspace*{-6pt} \\
(2) The coproduct $\Delta \in \Hom \left( \cchains(\square^n), \cchains(\square^n)^{\otimes 2} \right)$ by
\begin{equation*}	
\Delta (x_1 \otimes \cdots \otimes x_n) = 	
\sum \pm \left( x_1^{(1)} \otimes \cdots \otimes x_n^{(1)} \right) \otimes 	
\left( x_1^{(2)} \otimes \cdots \otimes x_n^{(2)} \right),	
\end{equation*}	
where the sign is determined using the Koszul convention, and we are using Sweedler's notation
\begin{equation*}	
\Delta(x_i) = \sum x_i^{(1)} \otimes x_i^{(2)}
\end{equation*}
for the chain map $\Delta \colon \cchains(\square^1) \to \cchains(\square^1)^{\otimes 2}$ defined by
\begin{equation*}
\Delta([0]) = [0] \otimes [0], \quad \Delta([1]) = [1] \otimes [1], \quad \Delta([0, 1]) = [0] \otimes [0, 1] + [0, 1] \otimes [1].
\end{equation*}
Using that $\cchains(\square^n) = \cchains(\square^1)^{\otimes n}$, $\Delta$ is the composition
\begin{equation*}
\begin{tikzcd}
\cchains(\square^1)^{\otimes n} \arrow[r, "\Delta^{\otimes n}"] & \left( \cchains(\square^1)^{\otimes 2}  \right)^{\otimes n} \arrow[r, "sh"] & \left( \cchains(\square^1)^{\otimes n} \right)^{\otimes 2}
\end{tikzcd}
\end{equation*}
where $sh$ is the shuffle map that places tensor factors in odd position first. \vspace*{5pt} \\
(3) The product $\ast \in \Hom(\cchains(\square^n)^{\otimes 2}, \cchains(\square^n))$ by
\begin{align*}
(x_1 \otimes \cdots \otimes x_n) \ast (y_1 \otimes \cdots \otimes y_n) =
(-1)^{|x|} \sum_{i=1}^n x_{<i} \epsilon(y_{<i}) \otimes x_i \ast y_i \otimes \epsilon(x_{>i})y_{>i},
\end{align*}
where
\begin{align*}
x_{<i} & = x_1 \otimes \cdots \otimes x_{i-1}, &
y_{<i} & = y_1 \otimes \cdots \otimes y_{i-1}, \\
x_{>i} & = x_{i+1} \otimes \cdots \otimes x_n, & 
y_{>i} & = y_{i+1} \otimes \cdots \otimes y_n,
\end{align*}
with the convention
\begin{equation*}
x_{<1} = y_{<1} = x_{>n} = y_{>n} = 1 \in \Z,
\end{equation*}
and the only non-zero values of $x_i \ast y_i$ are
\begin{equation*}
\ast([0] \otimes [1]) = [0, 1], \qquad  \ast([1] \otimes [0]) = -[0, 1].
\end{equation*}

The counit $\varepsilon$ and coproduct $\Delta$ are known respectively as the \textit{augmentation} and \textit{Serre's diagonal}.
They define a natural counital coassociative coalgebra structure on each $\cchains(\cube^n)$, or equivalently a lift of the functor of chains
\begin{equation} \label{e:Serre lift}
\begin{tikzcd}[column sep=small, row sep=small]
& \coAlg \arrow[d] \\
\cube \arrow[r, "N"] \arrow[dashed, ur, bend left]& \Ch.
\end{tikzcd}
\end{equation}
The following statement was shown in \cite{medina2021cubical}.

\begin{proposition} \label{thm: cubical chain bialgebra}
	For every $n \in \mathbb{N}$, the assignment
	\begin{equation*}
	\counit \mapsto \epsilon, \quad \coproduct \mapsto \Delta, \quad \product \mapsto \ast,
	\end{equation*}
	defines a natural $\mathcal M$-bialgebra structure on $\cchains(\square^n)$ or, equivalently, a lift of the functor of chains
	\begin{equation} \label{e:cubical lift to bialgebras}
	\begin{tikzcd}[column sep=small, row sep=small]
	& \biAlg_{\M} \arrow[d] \\
	\cube \arrow[r, "N"] \arrow[dashed, ur, bend left]& \Ch.
	\end{tikzcd}
	\end{equation}
\end{proposition}

Using the forgetful functor $U$ on \eqref{e:cubical lift to bialgebras} we obtain a functor $\cube \to \coAlg_{U(\M)}$.
Kan extending this functor and \eqref{e:Serre lift} along the Yoneda embedding we obtain a functor $\cchains_{U(\M)} \colon \cSet \to \coAlg_{U(\M)}$. Hence, we obtain an $E_{\infty}$-coalgebra structure, described explicitly through the operad $U(\M)$, on the normalized chains of any cubical set extending the coassociative Serre diagonal.
\begin{theorem} \label{t:lift of chains of cSets to UM coalgebras}
    The following diagram commutes up to natural isomorphism
    \begin{equation}
    \begin{tikzcd}[column sep=small, row sep=small]
    & \coAlg_{U(\M)} \arrow[d] \\
    & \coAlg \arrow[d] \\
    \cSet \arrow[r, "N"] \arrow[dashed, uur, bend left, "\cchains_{U(\M)}"]& \Ch.
    \end{tikzcd}
    \end{equation}
\end{theorem}


\section{Hopf structures}

By definition, the functor of chains $\cchains \colon \cube \to \Ch$ is monoidal, i.e., the diagram
\begin{equation*}
\begin{tikzcd}
2^p \times 2^q \arrow[r, "\cong"] \arrow[d, "\cchains \otimes \cchains"']& 2^{p+q} \arrow[d, "\cchains"] \\
\cchains(\cube^p) \otimes \cchains(\cube^q) \arrow[r, "\cong"] & \cchains(\cube^{p+q})
\end{tikzcd}
\end{equation*}
commutes and $\cchains(2^0) = R$.
Its lift to the symmetric monoidal category $\coAlg$ is also monoidal since the following diagrams associated to the Serre diagonal and augmentation map commute:
\begin{equation*}
\begin{tikzcd}
\cchains(\cube^d) \otimes \cchains(\cube^{d^\prime}) \arrow[r, "\cong"] \arrow[d, "(23) \circ (\Delta \otimes \Delta)"'] & \cchains(\cube^{d+d^\prime}) \arrow[d, "\Delta"]\\
\cchains(\cube^d)^{\otimes 2} \otimes \cchains(\cube^{d^\prime})^{\otimes 2} \arrow[r, "\cong"]& \cchains(\cube^{d+d^\prime})^{\otimes 2}
\end{tikzcd}
\qquad \qquad
\begin{tikzcd}
\cchains(\cube^d) \otimes \cchains(\cube^{d^\prime}) \arrow[r, "\cong"] \arrow[d, "(\varepsilon \otimes \varepsilon)"'] & \cchains(\cube^{d+d^\prime}) \arrow[d, "\varepsilon"] \\
R \otimes R \arrow[r, "\cong"] & R,
\end{tikzcd}
\end{equation*}
please consult \eqref{e:coalgebras are symmetric monoidal1} and \eqref{e:coalgebras are symmetric monoidal2} for the structure maps of the monoidal product of counital coassociative coalgebras.

In this section we will show that the lift \eqref{e:cubical lift to bialgebras} is also monoidal.
In order to make sense of this statement we first need to describe a monoidal structure on $\M$-bialgebras.

To do so we now make $\M$ into a Hopf prop, that is, a prop over the category $\coAlg$.
Explicitly, we will construct chain maps $\Delta_\M(m,n) \colon \M(m,n) \to \M(m,n) \otimes \M(m,n)$ and $\varepsilon_\M(m,n) \colon \M(m,n) \to R$ for every biarity $(m,n)$, compatible with the prop structure of $\M$ and making each $\M(m,n)$ into a counital coassociative coalgebra.
Intuitively, this diagonal is defined on basis elements in $\M(m,n)_d$ as the sum of the $2^d$ summands obtained from the assignments 

FIGURE

Then, the structure map of the monoidal product of $\M$-bialgebras is given in biarity $(m,n)$ by 

\begin{equation*}
\begin{tikzcd} [column sep = small, row sep=large]
\M(m,n) \otimes (C \otimes C^\prime)^{\otimes m} \arrow[r, "\Delta_\M"] & \M(m,n) \otimes \M(m,n) \otimes (C \otimes C^\prime)^{\otimes m} \arrow[dl, out=-150, in=30, "Sh"'] &[-25pt] \\
\big(\M(m,n) \otimes C^{\otimes m}\big) \otimes \big(\M(m,n) \otimes C^{\prime\, \otimes m}\big) \arrow[r] & 
C^{\otimes n} \otimes C^{\prime\, \otimes n} \arrow[r, "Sh"] &[-25pt]
(C \otimes C^\prime)^{\otimes n}.
\end{tikzcd}
\end{equation*}

We will need the following observations about the Serre diagonal to define the Hopf structure on $\M$.
There is an action of $\S_n$ on $\cchains(\square^n)$ given by permuting the factors of $\cchains(\square^1)^{\otimes n}$.
The Serre diagonal is equivariant with respect to this action in the following sense.

\begin{lemma} \label{l:serre diagonal invariant}
	Let $T$ be the braiding of $\Ch$.
	The following diagram commutes
	\begin{equation*}
	\begin{tikzcd}
	\cchains(\square^1)^{\otimes 2} \arrow[r, "\Delta"] \arrow[d, "T"] &
	\cchains(\square^1)^{\otimes 2} \otimes \cchains(\square^1)^{\otimes 2} \arrow[d, "T \otimes T"] \\
	\cchains(\square^1)^{\otimes 2} \arrow[r, "\Delta"] &
	\cchains(\square^1)^{\otimes 2} \otimes \cchains(\square^1)^{\otimes 2}.
	\end{tikzcd}
	\end{equation*}
\end{lemma}

\begin{proof}
	This can be proven by a straightforward verification. Conceptually, it holds because $(12)(34)(23) = (23)(13)(24)$ in $\S_4$.
\end{proof}

As can be seen from \eqref{e:free prop}, the $(m,n)$-part of the free prop is defined only up to a choice of total order on the set of vertices of the $(m,n)$-graph involved.

\begin{definition}
	Let $\Gamma \in \G(m,n)$ be a representative of an element in $\M(m,n)$ of degree $d$. A \textit{characteristic map} for $\Gamma$ is the chain map \vspace*{-5pt}
	\begin{equation} \label{e:order chain map}
	\begin{tikzcd}[row sep=tiny, column sep=small]
	\iota_\Gamma \colon \cchains(\square^d) \arrow[r] & \M(m,n) \\
	\qquad {[0,1]}^{\otimes d} \arrow[r, |->] & \Gamma.
	\end{tikzcd}
	\end{equation}
	induced by a choice of total order of the vertices of $\Gamma$.
\end{definition}

Since by Lemma~\ref{l:serre diagonal invariant} the Serre diagonal is equivariant, the following is well-defined.
\begin{definition}
	Let $m,n$ be a pair of non-negative integers.
	The \textit{coproduct} $\Delta_\M(m,n) \colon \M(m,n) \to \M(m,n)^{\otimes 2}$ is defined on a basis element $\Gamma$ by
	\begin{equation} \label{e:diagonal of M}
	\Delta_{\M}(\Gamma) = \iota_\Gamma^{\otimes 2} \circ \Delta \left([0,1]^{\otimes d}\right)
	\end{equation}
	where $\iota_\Gamma$ is a characteristic map of $\Gamma$ and $\Delta$ is the Serre diagonal.
	The \textit{counit} $\varepsilon_\M(m,n) \colon \M(m,n) \to R$ is the chain map whose value on basis elements of degree $0$ is $1$. 
\end{definition}

\begin{theorem}
	The collections of maps $\Delta_\M = \{\Delta_\M(m,n)\}_{m,n \geq 0}$ and $\varepsilon_\M = \{\varepsilon_\M(m,n)\}_{m,n\geq0}$ make $\M$ into a Hopf prop, i.e., a prop over the category $\coAlg$.
\end{theorem}

\begin{proof}
	The collection of maps $\Delta_\M$ compatible with the composition structure on $\M$ since given characteristic maps $\iota_\Gamma$ and $\iota_{\Gamma^\prime}$, then $\iota_\Gamma \otimes \iota_{\Gamma^\prime}$ is a characteristic map of any (vertical or horizontal) composition of $\Gamma$ and $\Gamma^\prime$.
	It respects the relations defining $\M$ since
	\begin{center}
		\begin{tikzcd}
		\leftcounitcoproduct \arrow[r, "\Delta_\M"] \arrow[d, <->] &[-0pt] \leftcounitcoproduct \otimes \leftcounitcoproduct \arrow[d, <->] \\
		\identity \arrow[r, "\Delta_\M"] & \ \, \identity \otimes \identity \, ,
		\end{tikzcd}
		\qquad
		\begin{tikzcd}
		\rightcounitcoproduct \arrow[r, "\Delta_\M"] \arrow[d, <->] &[-0pt] \rightcounitcoproduct \otimes \rightcounitcoproduct \arrow[d, <->] \\
		\identity \arrow[r, "\Delta_\M"] & \ \, \identity \otimes \identity \, ,
		\end{tikzcd}
		\qquad
		\begin{tikzcd}
		\leftcomb \arrow[r, "\Delta_\M"] \arrow[d, <->] &[-0pt] \leftcomb \, \otimes \leftcomb \arrow[d, <->] \\
		\rightcomb \arrow[r, "\Delta_\M"] & \, \rightcomb \, \otimes \rightcomb,
		\end{tikzcd}
		\qquad
		\begin{tikzcd}
		\productcounit \arrow[r, "\Delta_\M"] \arrow[d, <->] &[-0pt] \counit\ \counit \, \otimes \productcounit + \productcounit \otimes \, \counit\ \counit \arrow[d, <->] \\
		0 \arrow[r, "\Delta_\M"] & \ 0.
		\end{tikzcd}
	\end{center}
	It is a chain map since
	\begin{center}
		\begin{tikzcd}
		\counit \arrow[r, "\Delta_\M"] \arrow[d, "\partial"'] &[-0pt] \counit \otimes \counit \arrow[d, "\partial"'] \\
		0 \arrow[r, "\Delta_\M"] & \ \, 0 \, ,
		\end{tikzcd}
		\qquad
		\begin{tikzcd}
		\coproduct \arrow[r, "\Delta_\M"] \arrow[d, "\partial"'] &[-0pt] \coproduct \otimes \coproduct \arrow[d, "\partial"'] \\
		0 \arrow[r, "\Delta_\M"] & \ \, 0 \, ,
		\end{tikzcd}
		\qquad
		\begin{tikzcd}
		\product \arrow[r, "\Delta_\M"] \arrow[d, "\partial"'] &[-0pt] \leftboundary \ \otimes \product + \product \otimes\ \rightboundary \arrow[d, "\partial"'] \\
		\rightboundary \,-\, \leftboundary \arrow[r, "\Delta_\M"] &
		\, \rightboundary \otimes \rightboundary \,-\, \leftboundary \otimes \leftboundary\,.
		\end{tikzcd}
	\end{center}
	The compatibility of the collection of maps $\varepsilon_\M$ is checked similarly and is left to the interested reader.
\end{proof}

\begin{theorem}
	The functor $\cube \to \coAlg_{U(\M)}$ of Theorem~\ref{thm: cubical chain bialgebra} is monoidal.
\end{theorem}

\begin{proof}
	content...
\end{proof}
\section{A necklical model for the based loop space}

We now describe a cubical model for the based loop space of a reduced simplicial set $X \in \sSet^0$. This particular model has the feature that after taking cubical chains we obtain a dg algebra which is naturally \textit{isomorphic} to the cobar construction on the connected dg coalgebra of simplicial normalized chains $\schains(X)$. We also describe a version of the model which is localized at the $1$-simplices of $X$ and recovers the extended cobar construction defined in \cite{Hess-Tonks}. This localization step is required in order to obtain the correct homotopy type of the based loop space in the non-simply connected setting.

The construction of the combinatorial model for the based loop space is best described using necklical sets, a notion related to both simplicial sets and cubical sets with connections. Different approaches to this framework can be found in \cite{Baues}, \cite{Galvez-Kaufmann-Tonks}, \cite{Dugger-Spivak}, \cite{rivera-zeinalian-cubical}, \cite{Rivera-Saneblidze}, among other works. Below we give a self contained exposition suitable for our purposes. 




\subsection{Necklaces} For any ordered sequence of positive integers $(n_1,...n_k)$ we denote by $\Delta^{n_1} \vee...\vee \Delta^{n_k} \in \sSet$ the simplicial set obtained by identifying the last vertex of $\Delta^{n_i}$ with the first vertex of $\Delta^{n_{i+1}}.$ A simplicial set of the form $T=\Delta^{n_1} \vee...\vee \Delta^{n_k}$, for $n_i>0$ if $k>0$ and $n_i \geq 0$ if $k=0$, is called a \textit{necklace}. For any necklace $T= \Delta^{n_1} \vee...\vee \Delta^{n_k}$ we call each $\Delta^{n_i}$ the \textit{$i$-th bead} of $T$. The set $V_T=T_0$ will be called the \textit{vertices} of the necklace $T$. We denote by $J_T$ to be the subset of $V_T$ consisting of those vertices which are the first or final vertex of some bead; the elements of $J_T$ are called \textit{joints}. For any necklace $T$, the ordering of the vertices of each bead together with the ordering of the beads of $T$ induces a total ordering on $T_0.$ We denote the first and last vertices of $T$ by $\alpha_T$ and $\omega_T$, respectively. 

Necklaces form a category $\Nec$ with morphisms being maps of simplicial sets $f: T \to T'$ that preserve the first and last vertices, i.e. satisfying $f(\alpha_T)=\alpha_{T'}$ and $f(\omega_T)=\omega_{T'}.$ A \textit{neckilcal set} is a functor $K: \Nec^{op} \to \Set.$ Necklical sets form a category, denoted by $\nSet$, with morphisms being natural transformations. For any necklace $T \in \Nec$ we denote by $Y(T)\in \nSet$ the necklical set corpresented by $T$, i.e. $Y(T):=\Hom_{\Nec}( \text{ \_ ,} T).$


Necklaces have been studied in \cite{Dugger and Spivak}, \cite{Rivera and Zeinalian}, and in a different form in \cite{Baues}. We recall Proposition 3.1 of \cite{Rivera and Zeinalian, cubical rigidification}, which provides a set of generators for the morphisms in $\Nec.$

\begin{proposition} Any non-identity morphism in $\Nec$ is a composition of morphisms of the following type
\begin{item}
\item (i)  $f: T \to T'$ is an injective morphism of necklaces and $ |V_{T'}-J_{T'}|-|V_T-J_T| =1$
\\
\item (ii) $f: \Delta^{n_1} \vee ... \vee \Delta^{n_k} \to \Delta^{m_1} \vee ... \vee \Delta^{m_k}$ is a morphism of necklaces of the form $f=f_1 \vee ... \vee f_k$ such that for exactly one $p$, $f_p: \Delta^{n_p} \to \Delta^{m_p}$ is a codegeneracy morphism (so $m_p=n_p-1$) and for all $i \neq p$, $f_i: \Delta^{n_i}  \to \Delta^{m_i}$ is the identity map of standard simplices (so $n_i=m_i$ for $i \neq p$)
\\
\item (iii) $f: \Delta^{n_1} \vee ...\vee \Delta^{n_{p-1}} \vee \Delta^1 \vee \Delta^{n_{p+1}} \vee... \vee  \Delta^{n_k} \to \Delta^{n_1} \vee ...\vee \Delta^{n_{p-1}} \vee \Delta^{n_{p+1}} \vee... \vee  \Delta^{n_k}$ is a morphism of necklaces such that $f$ collapses the $p$-th bead $\Delta^1$  in the domain to the last vertex of the $(p-1)$-th bead in the target and the restriction of $f$ to all the other beads is injective. 
\end{item}
\end{proposition}




\section{Main theorems}

We now have all the ingredients to prove the theorems stated in the introduction. 

\begin{proof}[Proof of Theorem~\ref{t:1st main thm in the intro}]
    Recall Theorem~\ref{t:lift of chains of cSets to UM coalgebras} stating that the normalized cubical chains functor lifts to a functor $\cchains_{U(\M)} \colon \cSet_c \to \coAlg_{U(\M)}$ extending the coassociative coalgebra given by the Serre diagonal construction.
    Theorem~\ref{t:lift chains on cSet to UM coAlg is monoidal} states that this lift is  monoidal with respect to the Day convolution product on $\cSet_c$ and the monoidal structure on $\coAlg_{U(\M)}$ induced by the Hopf structure of $\M$.
    Therefore, there is an induced functor $\Mon_{\cSet_c} \to \Mon_{\coAlg_{U(\M)}}$ between the associated categories of monoids.
    By Proposition \ref{gcobarandcobar}, the functors $\cchains \circ \gcobar \colon \sSet^0 \to \Mon_{\Ch}$ and $\mathcal{F} \colon \sSet^0 \to \Mon_{\Ch}$ are naturally isomorphic.
    Hence, the functor $\Mon_{\cSet_c} \to \Mon_{\coAlg_{U(\M)}}$ provides a lift of $\mathcal{F} \colon \sSet^0 \to \Mon_{\Ch}$ to a functor  $\sSet^0 \to \Mon_{\coAlg_{U(\M)}}$ which factors through $\gcobar,$ as desired.
\end{proof}

\begin{proof}[Proof of Theorem~\ref{t:2nd main thm in the intro}]
Using Corollary \ref{localizedcobar}, a similar argument as above gives the desired lift in the localized case. 
\end{proof}

Kan constructed a functor $$G: \sSet^0 \to \mathsf{sGrp}$$ known as the Kan loop group, which models the based loop space functor in terms of simplicial sets. More precisely, for any $X \in \sSet^0$, the geometric realization of the Kan loop group $|G(X)|$ is homotopy equivalent, as a topological monoid, to the based loop space $\Omega|X|$ on the geometric realization of $X$ \cite{Berger}. 

We finish by arguing that $\widehat{\mathcal{F}}(X)$ is naturally quasi-isomorphic, \textit{as an $E_{\infty}$-coalgebra}, to the chains on $G(X)$, the Kan loop group of $X$. The argument relies on the fundamental observation that the simplicial monoid obtained by triangulating $\widehat{\gcobar}(X)$ is naturally weak homotopy equivalent as a simplicial monoid to $G(X)$ a result outlined in \cite{Hinich} and also proven in \cite{minichello-rivera-zeinalian}. 

\begin{proof}[Proof of Theorem ~\ref{t:3rd main thm in the intro}]

We construct a zig-zag of quasi-isomorphisms of $U(\M)$-coalgebras between $\widehat{\gcobar}(X)$ and $\schains (G(X))$.
The triangulation functor $$\mathcal{T}\colon \cSet \to \sSet$$
has a right adjoint $$\mathcal{U}\colon \sSet \to \cSet$$
given by the formula
$$\mathcal{U}(X)_n= \sSet((\Delta^1)^{\times n}, X)$$
together with the obvious induced cubical faces, degeneracies, and connections. For any $C \in \cSet_c$ there are quasi-isomorphisms of $U(\M)$-coalgebras
$$\schains(TC) \xrightarrow{\simeq} \cchains(UTC) \xleftarrow{\simeq} \cchains(C).$$

[Insert proof of the above fact: The first is a weak-equivalence of $E_\infty$ algebras by \cite{medina2021cubical}.]

Applying the above to $C= \widehat{\gcobar}(X),$ we obtain that $\schains(T\widehat{\gcobar}(X))$ and $\schains(\widehat{\gcobar}(X)) \cong \widehat{\mathcal{F}}(X)$ are quasi-isomorphic as $U(\M)$-coalgebras. 

By Theorem ?? of \cite{minichello-rivera-zeinalian} it follows that $T\widehat{\gcobar}(X)$, which is naturally isomorphic to $\mathfrak{C}(X)$, is naturally weak homotopy equivalent to $G(X)$ via a map of simplicial monoids $Sz \colon \mathfrak{C}(X) \xrightarrow{\simeq} G(X)$ induced by the ``Szczarba operators". By naturality, we have an induced quasi-isomorphism of $U(\M)$-coalgebras
$$\schains(T\widehat{\gcobar}(X)) \xrightarrow{\simeq} \schains(G(X)).$$
Therefore, $\widehat{\mathcal{F}}(X) \cong \cchains(\widehat{\gcobar}(X))$ and $\schains(G(X))$ are quasi-isomorphic as $U(\M)$-coalgebras, as desired. 




\end{proof}

\bibliographystyle{alpha} % ieeetr
\bibliography{biblio}

\end{document}