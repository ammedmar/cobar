\subsection{A monoidal necklical set model for the based loop space}
We define a functor 
$$\gcobar^{\text{nec}} \colon \sSet^0 \to \Mon_{\nSet}.$$
For any $0$-reduced simplicial set $X$, the underlying necklical set $\gcobar^{\text{nec}}(X) \colon \Nec^{op} \to \Set$ is given by
$$\gcobar^{\text{nec}}(X) = \colim_{(f \colon \mathcal{S}(T) \to X) \in  \mathcal{S} \downarrow X} \mathcal{Y}(T).$$
The monoidal structure $$\gcobar^{\text{nec}}(X) \otimes \gcobar^{\text{nec}}(X) \to \gcobar^{\text{nec}}(X)$$
is induced by the monoidal structure $\vee: \Nec \times \Nec \to \Nec$ together with the monoidal structure $\otimes: \nSet \times \nSet \to \nSet$ on necklicals sets. More precisely, for any $S, S' \in \Nec$, the product $$\gcobar^{\text{nec}}(X)(S) \otimes \gcobar^{\text{nec}}(X)(S') \to \gcobar^{\text{nec}}(X)(S \vee S')$$ is given by $$[f\colon \mathcal{S}(T) \to X, a] \otimes [f'\colon \mathcal{S}(T') \to X, a'] \mapsto [f \vee g\colon \mathcal{S}(T\vee T') \to X, a \otimes  a'],$$
where $(f\colon \mathcal{S}(T) \to X), (f'\colon \mathcal{S}(T') \to X) \in \mathcal{S} \downarrow X$, $a\in \mathcal{Y}(T)(S)$, and $a'\in \mathcal{Y}(T')(S')$.

We may now define the \textit{cubical cobar functor} $$\gcobar: \sSet^0 \to \Mon_{\cSet}$$ as the composition $$\gcobar= \mathcal{P}_! \circ \gcobar^{\text{nec}}.$$ This is a reinterpretation of a classical construction of Baues \cite{Baues} also studied in \cite{rivera-zeinalian-cubical}. 


\begin{remark}

In the case of arbitrary (not necessarily $0$-reduced) simplicial sets, we may extend  $\gcobar$ to a functor
$$\Lambda: \sSet \to \mathsf{Cat}_{\cSet}$$ from the category of simpicial sets to the category of small categories enriched over the monoidal category of necklical sets. Define a functor $\mathcal{T} \colon \cSet \to \sSet$ by $$\mathcal{T}(C) = \colim_{\cube^n \to C} (\Delta^1)^{\times n}.$$ This is a monoidal functor known as \textit{triangulation} and it induces a functor $$\mathfrak{T}: \mathsf{Cat}_{\cSet} \to \mathsf{Cat}_{\sSet}$$ between enriched categories. The composition 
$$\mathfrak{T} \circ \Lambda \colon \sSet \to \mathsf{Cat}_{\sSet}$$ is naturally isomorphic to the \textit{rigidification functor}
$$\mathfrak{C} \colon \sSet \to \mathsf{Cat}_{\sSet}$$
introduced by Cordier and studied by Lurie, Dugger and Spivak in the theory of $\infty$-categories, see \cite{rivera-zeinalian-cubical} and the references therein. The rigidification functor is the left adjoint of the homotopy coherent nerve functor, which we denote by
$$\mathfrak{N} \colon \mathsf{Cat}_{\sSet} \to \sSet.$$
\end{remark}

The functor $\gcobar: \sSet^0 \to \Mon_{\cSet}$ is related to the cobar construction $\mathbf{\Omega}: \coAlg \to \Mon_{\Ch}$ via normalized chains as follows.

\begin{proposition} \label{gcobarandcobar}
There is a natural isomorphisms of functors 
$$\cchains \circ \gcobar \cong \mathbf{\Omega} \circ \chains: \sSet^0 \to \Mon_{\Ch}.$$
\end{proposition}

\begin{proof} 
Denote by $\iota_n \in (\cube^n)_n$ the top dimensional non-degenerate element of the standard $n$-cube with connections $\cube^n$. Note that for any $X \in \sSet^0$, we may represent any non-degenerate $n$-cube $\alpha \in (\mathcal{P}_!(\gcobar^{\text{nec}}(X)))_n$ as a pair $\alpha=[\sigma: \mathcal{Y}(T) \to X, \iota_n]$ for some $T=[n_1] \vee ... \vee [n_k] \in \Nec$ with $\text{dim}(T)=n_1+ ...+n_k-k=n.$

To define an algebra map
$$\varphi_X: \cchains(\mathcal{P}_!(\gcobar^{\text{nec}}(X))) \xrightarrow{\cong} \mathbf{\Omega}(\chains(X))$$
it suffices to define it on any generator of the form $\alpha=[\sigma \colon \Delta^{n+1} \to X, \iota_{n}]$, i.e. when $T$ is of the form $T=[n+1]$, for some $n\geq0$. If $n=0$ let $\varphi_X(\alpha)= [\overline{\sigma}]+ 1_R$, where $[\overline{\sigma}] \in s^{-1} \overline{ \chains(X)} \subset \mathbf{\Omega}(\chains(X))$ denotes the (length $1$) generator in the cobar construction of $\chains(X)$ determined by $\sigma \in X_{n+1}$ and $1_R$ denotes the unit of the underlying ring $R$. If $n>0$, we let $\varphi_X(\alpha)=[\overline{\sigma}]$. A straightforward computation yields that this gives rise to a well defined isomorphism of algebras, which is compatible with the differentials, and natural with respect to maps of simplicial sets.  
\end{proof}
Denote $$\mathcal A= \cobar \circ \chains \colon \sSet^0 \to \Mon_{\Ch}.$$ In \cite{Hess-Tonks}, a localized version of the cobar construction of the chains on a $0$-reduced simplicial set was introduced in order to relate to the Kan loop group construction. This localized construction is not quite functorial on $\coAlg$ since it depends on a choice of basis of the degree one $R$-module of the underlying coalgebra. However, one may define a functor $$\widehat{\mathcal A}: \sSet^0 \to \Mon_{\Ch}$$
by formally inverting the set of $0$-cycles $$A_X=\{ [\overline{\sigma}]+1_R | \sigma \in X_1 \} \subset \mathcal A(X)_0$$ in the associative algebra $\mathcal A(X)$ to obtain a new associative algebra
$$\widehat{\mathcal A}(X):= \mathcal A(X)[A_X^{-1}]$$
with the property that there is an isomorphism of algebras $$H_0(\widehat{\mathcal A}(X)) \cong R[\pi_1(X)].$$
K. Hess and A. Tonks constructed a natural chain homotopy equivalence of associative algebras $$\widehat{\mathcal A}(X)\simeq \chains(G(X)),$$ where $G(X)$ denotes the Kan loop group of $X$. 

We may formally invert all the $0$-dimensional elements in $\gcobar^{\text{nec}}(X)$ to obtain a localized version of $\gcobar^{\text{nec}}$ as follows. Define 
$$\widehat{\gcobar}^{\text{nec}} \colon \sSet^0 \to \Mon_{\nSet}$$
as the localization
$$\widehat{\gcobar}^{\text{nec}}(X)= \mathcal{L}_{X_1}\gcobar^{\text{nec}}(X),$$
namely $\widehat{\gcobar}^{\text{nec}}(X)$ is the monoidal necklical set obtained by adding to $\gcobar^{\text{nec}}(X)$  formal inverses for all $f\colon \Delta^1 \to X$ (together with the corresponding degenerate elements generated by the new formal inverses) subject to the usual relations. This results in group completing the monoid
$$\bigsqcup_{\{(f: \mathcal{S}(T)\to X )\in \mathcal{S} \downarrow X, \text{ dim}(T)=0\}} \gcobar^{\text{nec}}(X)(T).$$

Finally denote by $$\widehat{\gcobar}: \sSet^0 \to \Mon_{\cSet}$$ the composition $$\widehat{\gcobar}= \mathcal{P}_{!} \circ \widehat{\gcobar}^{\text{nec}}.$$ 

As an immediate consequence of Proposition \ref{gcobarandcobar}, we obtain the following isomorphism after localizing.

\begin{corollary}\label{localizedcobar}
There is a natural isomorphism of functors
$$\cchains \circ \widehat{\gcobar} \cong \widehat{\mathcal A} :\sSet^0 \to \Mon_{\Ch}.$$ 
\end{corollary}


