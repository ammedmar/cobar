\section{Combinatorial $E_\infty$ structures}

\subsection{Simplicial chains}

We review from \cite{medina2020prop1} a natural $\mathcal M$-structure on the chains of standard simplices leading to a natural $U(\M)$-structure on the chains of any simplicial set.
The $\M$-bialgebra is specified by three linear maps, the images of the generators
\begin{equation*}
\counit, \quad \coproduct, \quad \product,
\end{equation*}
satisfying the relations in the presentation of $\mathcal M$.

(3) The product $\ast \in \Hom(\schains(\triangle^n)^{\otimes 2}, \schains(\triangle^n))$ by
\begin{equation*}
\left[v_0, \dots, v_p \right] \ast \left[v_{p+1}, \dots, v_q\right] = \begin{cases} (-1)^{p+|\pi|} \left[v_{\pi(0)}, \dots, v_{\pi(q)}\right] & \text{ if } v_i \neq v_j \text{ for } i \neq j, \\
0 & \text{ if not}, \end{cases}
\end{equation*}
where $\pi$ is the permutation that orders the totally ordered set of vertices, and $(-1)^{|\pi|}$ its sign.
Notice that the coproduct $\Delta$ is the \textit{Alexander-Whitney coproduct} of .

\begin{proposition}[\cite{medina2020prop1}] \label{p:simplicial chain bialgebra}
	For every $n \in \mathbb{N}$, the assignment
	\begin{equation*}
	\counit \mapsto \epsilon, \quad \coproduct \mapsto \Delta, \quad \product \mapsto \ast,
	\end{equation*}
	defines a natural $\mathcal M$-bialgebra structure  on $\schains(\triangle^n)$.
\end{proposition}

\begin{proof}
	See Theorem 4.2 in \cite{medina2020prop1}.
\end{proof}

By applying the forgetful functor $U$ from the category of props to the category of operads, the construction of Proposition \ref{p:simplicial chain bialgebra} gives rise to a functor $\simplex \to \coAlg_{U(\M)}$. Since the category of coalgebras over an operad is cocomplete, we may use Kan extension along the Yoneda embedding to define a lift of the functor of chains regarded as a coalgebra with the Alexander-Whitney diagonal:
\begin{equation*}
\begin{tikzcd}[column sep=small, row sep=small]
& \coAlg_{U(\M)} \arrow[d] \\
& \coAlg \arrow[d] \\
\sSet \arrow[r, "N"] \arrow[dashed, uur, bend left]& \Ch.
\end{tikzcd}
\end{equation*}

As explained in \cite{medina2020prop1}, this $E_\infty$-coalgebra on simplicial chains generalizes the constructions of McClure--Smith \cite{mcclure2003multivariable} and Berger--Fresse \cite{berger2004combinatorial}.

\subsection{Cubical chains}

We recall a natural $\mathcal M$-bialgebra structure on $\cchains(\square^n)$ for every $n \geq 0$, where $\mathcal M$ is the $E_{\infty}$-prop defined in \ref{propM}. This structure is determined by three linear maps satisfying the relations in the presentation of $\mathcal M$.

\anibal{improve}

(3) The product $\ast \in \Hom(\cchains(\square^n)^{\otimes 2}, \cchains(\square^n))$ by
\begin{align*}
(x_1 \otimes \cdots \otimes x_n) \ast (y_1 \otimes \cdots \otimes y_n) =
(-1)^{|x|} \sum_{i=1}^n x_{<i} \epsilon(y_{<i}) \otimes x_i \ast y_i \otimes \epsilon(x_{>i})y_{>i},
\end{align*}
where
\begin{align*}
x_{<i} & = x_1 \otimes \cdots \otimes x_{i-1}, &
y_{<i} & = y_1 \otimes \cdots \otimes y_{i-1}, \\
x_{>i} & = x_{i+1} \otimes \cdots \otimes x_n, & 
y_{>i} & = y_{i+1} \otimes \cdots \otimes y_n,
\end{align*}
with the convention
\begin{equation*}
x_{<1} = y_{<1} = x_{>n} = y_{>n} = 1 \in \Z,
\end{equation*}
and the only non-zero values of $x_i \ast y_i$ are
\begin{equation*}
\ast([0] \otimes [1]) = [0, 1], \qquad  \ast([1] \otimes [0]) = -[0, 1].
\end{equation*}

The counit $\varepsilon$ and coproduct $\Delta$ are known respectively as the \textit{augmentation} and \textit{Serre's diagonal}.
They define a natural counital coassociative coalgebra structure on each $\cchains(\cube^n)$, or equivalently a lift of the functor of chains
\begin{equation} \label{e:Serre lift}
\begin{tikzcd}[column sep=small, row sep=small]
& \coAlg \arrow[d] \\
\cube \arrow[r, "N"] \arrow[dashed, ur, bend left]& \Ch.
\end{tikzcd}
\end{equation}
The following statement was shown in \cite{medina2021cubical}.

\begin{proposition} \label{thm: cubical chain bialgebra}
	For every $n \in \mathbb{N}$, the assignment
	\begin{equation*}
	\counit \mapsto \epsilon, \quad \coproduct \mapsto \Delta, \quad \product \mapsto \ast,
	\end{equation*}
	defines a natural $\mathcal M$-bialgebra structure on $\cchains(\square^n)$ or, equivalently, a lift of the functor of chains
	\begin{equation} \label{e:cubical lift to bialgebras}
	\begin{tikzcd}[column sep=small, row sep=small]
	& \biAlg_{\M} \arrow[d] \\
	\cube \arrow[r, "N"] \arrow[dashed, ur, bend left]& \Ch.
	\end{tikzcd}
	\end{equation}
\end{proposition}

Using the forgetful functor $U$ on \eqref{e:cubical lift to bialgebras} we obtain a functor $\cube \to \coAlg_{U(\M)}$.
Kan extending this functor and \eqref{e:Serre lift} along the Yoneda embedding we obtain a functor $\cchains_{U(\M)} \colon \cSet \to \coAlg_{U(\M)}$. Hence, we obtain an $E_{\infty}$-coalgebra structure, described explicitly through the operad $U(\M)$, on the normalized chains of any cubical set extending the coassociative Serre diagonal.
\begin{theorem} \label{t:lift of chains of cSets to UM coalgebras}
	The following diagram commutes up to natural isomorphism
	\begin{equation}
	\begin{tikzcd}[column sep=small, row sep=small]
	& \coAlg_{U(\M)} \arrow[d] \\
	& \coAlg \arrow[d] \\
	\cSet \arrow[r, "N"] \arrow[dashed, uur, bend left, "\cchains_{U(\M)}"]& \Ch.
	\end{tikzcd}
	\end{equation}
\end{theorem}

