\subsection{A cubical model}
We define a functor 
$$\gcobar^{\text{nec}} \colon \sSet^0 \to \Mon_{\nSet}.$$
For any $0$-reduced simplicial set $X$, the underlying necklical set $\gcobar^{\text{nec}}(X) \colon \Nec^{op} \to \Set$ is given by
$$\gcobar^{\text{nec}}(X) = \colim_{(f \colon \mathcal{S}(T) \to X) \in  \mathcal{S} \downarrow X} \mathcal{Y}(T).$$
The monoidal structure $$\gcobar^{\text{nec}}(X) \otimes \gcobar^{\text{nec}}(X) \to \gcobar^{\text{nec}}(X)$$
is induced by the monoidal structure $\vee: \Nec \times \Nec \to \Nec$ together with the monoidal structure $\otimes: \nSet \times \nSet \to \nSet$ on necklicals sets. More precisely, for any $S, S' \in \Nec$, the product $$\gcobar^{\text{nec}}(X)(S) \otimes \gcobar^{\text{nec}}(X)(S') \to \gcobar^{\text{nec}}(X)(S \vee S')$$ is given by $$[f\colon \mathcal{S}(T) \to X, a] \otimes [f'\colon \mathcal{S}(T') \to X, a'] \mapsto [f \vee g\colon \mathcal{S}(T\vee T') \to X, a \otimes  a'],$$
where $(f\colon \mathcal{S}(T) \to X), (f'\colon \mathcal{S}(T') \to X) \in \mathcal{S} \downarrow X$, $a\in \mathcal{Y}(T)(S)$, and $a'\in \mathcal{Y}(T')(S')$.

We may now define the \textit{cubical cobar functor} $$\gcobar: \sSet^0 \to \Mon_{\cSet}$$ as the composition $$\gcobar= \mathcal{P}_! \circ \gcobar^{\text{nec}}.$$ This is a reinterpretation of a classical construction of Baues \cite{baues1998hopf} also studied in \cite{rivera2018cubical}. 

\subsection{Relation to the cobar construction}

The functor $\gcobar\colon \sSet^0 \to \Mon_{\cSet}$ is related to the cobar construction $\mathbf{\Omega}\colon \coAlg \to \Mon_{\Ch}$ via normalized chains as follows.

\begin{proposition} \label{gcobarandcobar}
There is a natural isomorphisms of functors 
$$\cchains \circ \gcobar \cong \mathbf{\Omega} \circ \schains: \sSet^0 \to \Mon_{\Ch}.$$
\end{proposition}

\begin{proof} 
Denote by $\iota_n \in (\cube^n)_n$ the top dimensional non-degenerate element of the standard $n$-cube with connections $\cube^n$. Note that for any $X \in \sSet^0$, we may represent any non-degenerate $n$-cube $\alpha \in (\mathcal{P}_!(\gcobar^{\text{nec}}(X)))_n$ as a pair $\alpha=[\sigma: \mathcal{Y}(T) \to X, \iota_n]$ for some $T=[n_1] \vee ... \vee [n_k] \in \Nec$ with $\text{dim}(T)=n_1+ ...+n_k-k=n.$

To define an algebra map
$$\varphi_X: \cchains(\mathcal{P}_!(\gcobar^{\text{nec}}(X))) \xra{\cong} \mathbf{\Omega}(\schains(X))$$
it suffices to define it on any generator of the form $\alpha=[\sigma \colon \Delta^{n+1} \to X, \iota_{n}]$, i.e. when $T$ is of the form $T=[n+1]$, for some $n\geq0$. If $n=0$ let $\varphi_X(\alpha)= [\overline{\sigma}]+ 1_R$, where $[\overline{\sigma}] \in s^{-1} \overline{ \schains(X)} \subset \mathbf{\Omega}(\schains(X))$ denotes the (length $1$) generator in the cobar construction of $\schains(X)$ determined by $\sigma \in X_{n+1}$ and $1_R$ denotes the unit of the underlying ring $R$. If $n>0$, we let $\varphi_X(\alpha)=[\overline{\sigma}]$. A straightforward computation yields that this gives rise to a well defined isomorphism of algebras, which is compatible with the differentials, and natural with respect to maps of simplicial sets.  
\end{proof}

\subsection{Localizing the cubical model}

We denote by $$\mathcal{L} \colon \Mon_{\sSet} \to \sGrp$$ the functor from simplicial monoids to simplicial groups defined by formally inverting all morphisms (degree by degree) subject to the usual relations. If $M \in \Mon_{\sSet}$ and $A \in M_0$ is a set of $0$-morphisms, we denote by $\mathcal{L}_AM$ the simplicial monoid obtained by formally inverting the elements of $A$, i.e. the pushout
$$\mathcal{L}_AM = M \coprod_{A} \mathcal{L}F(A),$$
where $F(A)$ is the monoid freely generated by the set $A$. 

In order to obtain a cubical model for the based loop space of a possibly non-fibrant and non-simply connected reduced simplicial set we consider the following localization. 

Define 
$$\widehat{\gcobar}^{\text{nec}} \colon \sSet^0 \to \Mon_{\nSet}$$
as the localization
$$\widehat{\gcobar}^{\text{nec}}(X)= \mathcal{L}_{X_1}\gcobar^{\text{nec}}(X),$$
namely $\widehat{\gcobar}^{\text{nec}}(X)$ is the monoidal necklical set obtained by adding to $\gcobar^{\text{nec}}(X)$  formal inverses for all $f\colon \Delta^1 \to X$ (together with the corresponding degenerate elements generated by the new formal inverses) subject to the usual relations. 

Finally denote by $$\widehat{\gcobar}: \sSet^0 \to \Mon_{\cSet}$$ the composition $$\widehat{\gcobar}= \mathcal{P}_{!} \circ \widehat{\gcobar}^{\text{nec}}.$$ 

\subsection{Relation to the extended cobar construction.}

Denote $$\A= \cobar \circ \schains \colon \sSet^0 \to \Mon_{\Ch}.$$ In \cite{hess2010cobar}, a localized version of the cobar construction of the chains on a $0$-reduced simplicial set was introduced in order to treat the case of non-simply connected simplicial sets. This localized construction is not quite functorial on $\coAlg$ since it depends on a choice of basis of the degree one $R$-module of the underlying coalgebra. However, one may define a functor $$\widehat{\A}: \sSet^0 \to \Mon_{\Ch}$$
given on any $X \in \sSet^0$ by formally inverting the set of $0$-cycles $$A_X=\{ [\overline{\sigma}]+1_R | \sigma \in X_1 \} \subset \A(X)_0$$ in the associative algebra $\A(X)$. This construction has the property that, for any $X \in \sSet^0$ there is a natural isomorphism of algebras $$H_0(\widehat{\A}(X)) \cong R[\pi_1(X)].$$

%\subsection{Comparison with Kan loop group}

%K. Hess and A. Tonks constructed a natural quasi-isomorphism of associative algebras $$\phi_{X} \colon \widehat{\A}(X)\xrightarrow{\simeq} \schains(G(X)),$$ where $G(X)$ denotes the Kan loop group of $X$. 



As an immediate consequence of Proposition \ref{gcobarandcobar}, we obtain the following isomorphism after localizing.

\begin{corollary}\label{localizedcobar}
There is a natural isomorphism of functors
$$\cchains \circ \widehat{\gcobar} \cong \widehat{\A} :\sSet^0 \to \Mon_{\Ch}.$$ 
\end{corollary}

\subsection{Relationship to the rigidification functor}

In the case of arbitrary (not necessarily $0$-reduced) simplicial sets, we may extend  $\gcobar$ to a functor
$$\gcobar: \sSet \to \mathsf{Cat}_{\cSet}$$ from the category of simpicial sets to the category of small categories enriched over the monoidal category of necklical sets. This construction was studied in \cite{rivera2018cubical} and relates to the homotopy coherent nerve functor as follows. 

Define a functor $\mathcal{T} \colon \cSet \to \sSet$ by $$\mathcal{T}(C) = \colim_{\cube^n \to C} (\Delta^1)^{\times n}.$$
This is a monoidal functor known as \textit{triangulation} and it induces a functor $$\mathfrak{T}\colon \mathsf{Cat}_{\cSet} \to \mathsf{Cat}_{\sSet}$$ between enriched categories. The following is Proposition 5.3 in \cite{rivera2018cubical}, where $\gcobar$ is denoted by $\mathfrak{C}_{\cube}$.
\begin{proposition}\label{Candgcobar} The composition 
$$\mathfrak{T} \circ \gcobar \colon \sSet \to \mathsf{Cat}_{\sSet}$$ is naturally isomorphic to the \textit{rigidification functor}
$$\mathfrak{C} \colon \sSet \to \mathsf{Cat}_{\sSet}$$
constructed in \cite{dugger2011rigidification}. 
\end{proposition}
The rigidication functor was originally introduced by Cordier and studied by Lurie in the theory of $\infty$-categories. Dugger and Spivak described the construction more explicitly in terms of necklaces in \cite{dugger2011rigidification}. The rigidification functor is the left adjoint of the \textit{homotopy coherent nerve functor}, which we denote by
$$\mathfrak{N} \colon \mathsf{Cat}_{\sSet} \to \sSet.$$ The adjunction between rigidification and homotopy coherent nerve yield an equivalence of two homotopy theories that model the theory of $\infty$-categories. Below, we recall the precise statement, a proof of which may be found in \cite{dugger2011rigidification}.

\begin{theorem} \label{joyalbergner} The adjunction $$ \mathfrak{C} \colon \sSet \leftrightarrows \mathsf{Cat}_{\sSet} \colon \mathfrak{N}$$ induces a Quillen equivalence when $\sSet$ is equipped with the Joyal model structure and $\mathsf{Cat}_{\sSet}$ with the Bergner model structure.
\end{theorem}


\begin{remark} The Bergner model structure on $\mathsf{Cat}_{\sSet}$ has as weak equivalences those maps of simplicially enriched categories that induce a weak homotopy equivalence between the simplicial sets of morphisms and are essentially surjective after passing to the homotopy categories. Fibrant objects in the Bergner model structure are precisely categories enriched over Kan complexes. On the other hand, fibrant objects in the Joyal model structure on $\sSet$ are precisely quasi-categories and all objects are cofibrant. The weak equivalences are known as \textit{Joyal equivalences} and, based on the above result, we can take them to be maps of simplicial sets that become weak equivalences of simplicially enriched categories after applying $\mathfrak{C}$. Furthermore, the Kan-Quillen model structure is a left Bousfield localization of the Joyal model structure obtained by localizing the single morphism $\Delta^1 \to \Delta^0.$

\end{remark}


\subsection{The Kan loop group construction} 
Another combinatorial model for the based loop space is given by a classical construction of Kan, which we now recall. 

The \textit{Kan loop group functor}  $$G\colon \sSet^0 \to \mathsf{sGrp}$$ is defined as follows. For any $X \in \sSet^0$ each $G(X)_n$ is the group with one generator $\overline{x}$ for every simplex $x \in X_{n+1}$ modulo the relation $\overline{s_0(x)}=e$ The face and degeneracy maps are defined by the following equations
\begin{itemize}
    \item $\delta_0(\overline{x}) = \overline{\delta_1(x)} \cdot (\overline{\delta_0(x)})^{-1}$
    \item $\delta_n(\overline{x})= \overline{\delta_{n+1}(x)}$ for $n >0$
    \item $s_n(\overline{x})= \overline{s_{n+1}(x)}$
\end{itemize}


Berger showed that this construction models the based loop space as a topological monoid, more precisely, he showed that for any $X \in \sSet^0$, the geometric realization of the Kan loop group $|G(X)|$ is weak homotopy equivalent, as a topological monoid, to the based loop space $\Omega|X|$ on the geometric realization of $X$ \cite{berger1995loops}. 


The Kan loop group functor has a right adjoint usually denoted by 
$$\overline{W} \colon \mathsf{sGrp} \to \sSet^0,$$ which is a model for the classifying space of a simplicial group. This adjunction induces an equivalence of homotopy theories in the following sense.

\begin{theorem} \label{kan} There exists a model category structure on the category $\mathsf{sGrp}$ of simplicial groups such that 
\begin{itemize}
    \item weak equivalences are those maps simplicial groups whose underlying map of simplicial sets is a weak homotopy equivalence
    \item all objects are fibrant, and
\item the adjunction
$$G \colon \sSet^0 \leftrightarrows \mathsf{sGrp}\colon \overline{W}$$
becomes a Quillen equivalence when $\sSet^0$ is equipped with the Kan-Quillen model category structure.
\end{itemize}
\end{theorem}

For a proof of the above statement we refer to \cite{goerss2009simplicial} Chapter V. 

\subsection{Relation between the cubical cobar construction and the Kan loop group}

K. Hess and A. Tonks constructed a natural quasi-isomorphism of associative algebras $$\phi_{X} \colon \widehat{\A}(X)\xrightarrow{\simeq} \schains(G(X)),$$ where $G(X)$ denotes the Kan loop group of $X$. Our next goal is to lift this comparison to the ``space-level", namely, to compare $\widehat{\gcobar}(X)$ and the simplicial group $G(X)$ for any $X \in \sSet^0$. We will do this by triangulating $\widehat{\gcobar}(X)$. The starting point of this comparison is the following statement, which is Proposition 2.6.2 in \cite{hinich2007deformation}, relating $\overline{W}$ and $\mathfrak{N}$.

\begin{proposition}\label{hinich}
There is a natural transformation of functors $$\psi: \overline{W} \Longrightarrow \mathfrak{N}$$ such that for any simplicial group $C \in \mathsf{sGrp}$, we have a weak equivalence
$$\psi_C: \overline{W}(C) \xrightarrow{\simeq} \mathfrak{N}(C).$$
\end{proposition} 

We now deduce the following relationship between the rigidification functor $\mathfrak{C}$ and the Kan loop group functor $G$.

\begin{proposition}\label{CandG} Let $X \in \sSet^0$ be a reduced simplicial set. There are natural weak equivalences of simplicial monoids
$$\mathcal{L}_{X_1} \mathfrak{C}(X) \xra{\simeq} \mathcal{L}\mathfrak{C}(X) \xra{\simeq} G(X).$$
\end{proposition}

\begin{proof}
Since $\mathfrak{C}$ is a left Quillen functor and every $X \in \sSet$ is cofibrant in the Joyal model structure, it follows from Theorem \ref{joyalbergner} that the simplicially enriched category $\mathfrak{C}(X)$ is a cofibrant. Proposition 9.5 of \cite{dwyer1980simplicial} implies that the natural inclusion $\mathcal{L}_{X_1} \mathfrak{C}(X) \to \mathcal{L}\mathfrak{C}(X)$ is a weak equivalence of simplicially enriched categories. 

By Proposition \ref{hinich} we have that for any $X \in \sSet^0$, $\psi_{G(X)}: \overline{W}(G(X)) \xrightarrow{\simeq} \mathfrak{N}(G(X))$ is a weak homotopy equivalence. By Theorem \ref{kan}, we have a weak homotopy equivalence $X \xrightarrow{\simeq} \overline{W}(G(X))$ given by the unit of the adjunction. Composing these two maps we obtain a weak homotopy equivalence
$$X \xrightarrow{\simeq} \mathfrak{N}(G(X)).$$ 
The Quillen equivalence of Theorem \ref{joyalbergner} localizes to a Quillen equivalence
$$\mathcal{L} \mathfrak{C} \colon \sSet^0 \leftrightarrows \mathsf{sGrp} \colon \mathfrak{N}\iota$$
when $\sSet^0$ is equipped with the Kan-Quillen model structure and $\mathsf{sGrp}$ with the model structure of Theorem \ref{kan}. Here $\iota: \mathsf{sGrp} \to \mathsf{Cat}_{\sSet}$ denotes the natural inclusion functor. It follows that the adjoint of the weak homotopy equivalence $X \xra{\simeq} \mathfrak{N}(G(X))$ is a weak equivalence of simplicial groups
$$\mathcal{L}\mathfrak{C}(X) \xra{\simeq} G(X).$$ 
\end{proof}
The following is now immediate.
\begin{corollary}\label{widehatgcobarandG} Let $X \in \sSet^0$. There is a natural weak equivalence of simplicial monoids
$$\mathcal{T} \widehat{\gcobar}(X) \xrightarrow{\simeq}G(X).$$
\end{corollary}
\begin{proof} After localizing the isomorphism of Proposition \ref{Candgcobar} at the $1$-simplices, we obtain a natural isomorphism of simplicial monoids $\mathcal{T} \widehat{\gcobar}(X) \cong \mathcal{L}_{X_1}\mathfrak{C}(X).$ Hence, the result follows from Proposition \ref{CandG}. 
\end{proof}