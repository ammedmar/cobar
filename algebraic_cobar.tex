

\subsection{The cobar construction}

Let $C$ be a coassociative counital coalgebra.
A \textit{coaugmentation} is a morphism $\nu \colon R \to C$ in $\coAlg$.
Denote by $\coAlg^*$ the category of \textit{coaugmented coassociative counital coalgebras} with morphisms being structure preserving chain maps.

We recall the definition of the \textit{cobar} functor 
\begin{equation*}
\cobar \colon \coAlg^* \to \Mon_{\Ch}.
\end{equation*}

Let  $(C, \partial, \Delta, \varepsilon, \nu)  \in \coAlg^*$ where $\nu: R \to C$ is the coaugmentation. Define
$$\mathbf{\Omega}(C, \partial, \Delta, \varepsilon, \nu) := ( T(s^{-1}  \overline{C} ), D, \mu) \in \Mon_{\Ch}$$ where 
\begin{itemize}
\item $\overline{C}=\text{coker}(\nu: R \to C)$
\item $T(s^{-1} \overline{C})= R \oplus \overline{C} \oplus (\overline{C}  \otimes \overline{C} ) \oplus ( \overline{C} \otimes \overline{C} \otimes \overline{C} ) \oplus\cdots $
\item $\mu: T(s^{-1}  \overline{C} )^{\otimes 2} \to T(s^{-1}  \overline{C} ) $ is the free associative unital product given by concatenation of monomials
\item $D: T(s^{-1}  \overline{C} ) \to T(s^{-1}  \overline{C} )$ is the derivation of degree $-1$ determined by extending the linear map $$- s^{-1} \circ \partial \circ s^{+1} + (s^{-1} \otimes s^{-1}) \circ \Delta \circ s^{+1}: s^{-1}\overline{C} \to s^{-1}\overline{C} \oplus (s^{-1}\overline{C} \otimes s^{-1}\overline{C}) \hookrightarrow T(s^{-1}C)$$ as a derivation.
\end{itemize}

The coassociativity of $\Delta$, the compatibility of $\partial$ and $\Delta$, and the fact that $\partial^2 =0$ together imply that $D^2=0$. This construction is clearly functorial with respect to maps in $\coAlg^*$. We will denote $\mathbf{\Omega} (C, \partial, \Delta, \varepsilon, \nu)$ simply by $\mathbf{\Omega}(C)$. 

In this article we are mostly concerned with \textit{connected} coalgebras, namely $C \in \coAlg^*$ which are non-negatively graded and the coaugmentation $\nu: R \to C$ induces an isomorphism of coalgebras $R \cong C_0$. We denote by $\coAlg^0$ the full subcategory of $\coAlg^*$ consisting of connected coalgebras. If $C \in \coAlg^0$, then $\mathbf{\Omega}(C)$ is concentrated on non-negative degrees. 

The cobar construction was originally applied by Adams' to the coalgebra of normalized chains on a $1$-reduced simplicial set to obtain a model for the chains on the based loop space \cite{Adams}. The cobar construction as a model for the based loop space was furthered studied in \cite{Baues} and more recently, in the non-simply connected setting in \cite{Hess-Tonks}, \cite{Rivera-Zeinalian}.


---
Discuss the localized version of the cobar construction. ---