\section{Simplices, cubes, and necklaces}

Discuss simplicial sets, cubical sets, and necklical sets. Define their normalized chain complexes and the coassociative coalgebra structures.

%\subsection{Linear and power posets}
%
%Let us consider the posets $\overline{n} = \{1 < \cdots < n\} \subset [n] = \{0 < \cdots < n\}$, and $\P(n) = \left\{U \subset \overline{n}\right\}$ ordered by inclusion. We identify $\P(n-1)$ with the set $\Omega(0, n)$ of ascending chains $(0 < \cdots < n)$ in $[n]$.
%
%Given two elements $x, x^\prime$ in a poset, their interval subposet is defined by
%\begin{equation*}
%[x, x^\prime] = \{x \leq x^{\prime\prime} \leq x^\prime\}.
%\end{equation*}
%
%Any subposet of $[n]$ is canonically isomorphic to $[m]$ for some $m$, and any interval subposet of $\P(n)$ is canonically isomorphic to $\P(m)$ for some $m$.
%
%The coface maps
%
%\begin{equation*}
%\delta_i^\varepsilon \colon \P(n) \to \P(n-1)
%\end{equation*}
%and
%\begin{equation*}
%\delta_i \colon [n] \to [n-1]
%\end{equation*}
%are ...
%
%The simplex and cube categories are ...
%
%\subsection{Splittings}
%Given a poset with a minimal and maximal elements $\minel$ and $\maxel$ we define the splitting of $P$ as the 