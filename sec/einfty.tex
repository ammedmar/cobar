% !TEX root = ../cobar1.tex

\section{Monoidal \pdfEinfty-structures}

\todo{@anibal: Write mini-intro for s:monoidal e-infty}
talking points

that is a cofibrant resolution of the terminal operad -- which controls cocommutative and coassociative coalgebras.

\subsection{$\cM$-bialgebras}

\begin{definition*}
	An $\cM$-bialgebra is a coalgebra $(C, \Delta, \aug)$ together with a degree~1 map $\ast \in \Hom(C \ot C, C)$ whose boundary in this complex is $\aug \ot \, \id - \id \ot \aug$ and such that the composition $(\aug \circ \, \ast) \in \Hom(C \ot C, \k)$ is the $0$ map.
\end{definition*}

\begin{proposition*}[\cite{medina2020prop1}]
	Let $(C, \Delta, \aug, \ast)$ be an $\cM$-bialgebra.
	The collection of all maps $\set{C \to C^{\ot r}}_{r \in \N}$ generated by $\Delta$, $\aug$ and $\ast$ make $C$ into an $E_\infty$-coalgebra.
\end{proposition*}

The model of the $E_\infty$-operad used in the proposition above is denoted by $\UM$.

\begin{example}[\cite{medina2020prop1}]\label{ex:simplicial e-infty}
	The Alexander--Whitney coalgebra $\chains(\simplex^n)$ can be made into a natural $\cM$-bialgebra considering an algebraic version of the \textit{join product} $\ast \colon \chains(\simplex^n)^{\ot 2} \to \chains(\simplex^n)$ defined by
	\begin{equation*}
		\left[v_0, \dots, v_p \right] \ast \left[v_{p+1}, \dots, v_q\right] =
		\begin{cases} (-1)^{p} \sign(\pi) \left[v_{\pi(0)}, \dots, v_{\pi(q)}\right] &
			\text{ if } v_i \neq v_j \text{ for } i \neq j, \\
			\hfil 0 & \text{ if not},
		\end{cases}
	\end{equation*}
	where $\pi$ is the permutation that orders the vertices.
	A Kan extension of the associated $\UM$-coalgebra defines a lift of the Alexander--Whitney coalgebra structure into a category of $E_\infty$-coalgebras
	\begin{equation*}
		\begin{tikzcd}[column sep=normal, row sep=small]
			& \coAlg_{\UM} \arrow[d] \\
			\sSet \arrow[r]
			\arrow[ur,out=60, in=180, dashed, "\schainsUM"]
			\arrow[r, "\schainsA"]
			& \coAlg.
		\end{tikzcd}
	\end{equation*}
\end{example}

\begin{example}[\cite{medina2022cube_einfty}]\label{ex:cubical e-infty}
	The Serre coalgebra coalgebra $\chains(\cube^n)$ can be made into a natural $\cM$-bialgebra considering the product $\ast \colon \chains(\cube^n)^{\ot 2} \to \chains(\cube^n)$ defined using the following notation.
	For a basis element $x = x_1 \ot \dotsb \ot x_n$ of $\chains(\cube^n)$ and an integer $\ell \in \set{1,\dots,n}$ we write
	\begin{align*}
		x_{<\ell} & = x_1 \ot \dotsb \ot x_{\ell-1}, \\
		x_{>\ell} & = x_{\ell+1} \ot \dotsb \ot x_n,
	\end{align*}
	with the convention $x_{<1} = x_{>n} = 1 \in \Z$.
	Then, for two such basis elements $x$ and $x'$ we have
	\begin{multline*}
		(x_1 \ot \dotsb \ot x_n) \ast (x'_1 \ot \dotsb \ot x'_n)
		=
		(-1)^{|x|} \sum_{i=1}^n x_{<i}\, \epsilon(x'_{<i}) \ot x_i \ast x'_i \ot \epsilon(x_{>i}) \, x'_{>i},
	\end{multline*}
	where the only non-zero values of $x_i \ast x'_i$ are
	\[
	[0] \ast [1] = [0, 1], \qquad [1] \ast [0] = -[0, 1].
	\]
	A Kan extension of the associated $\UM$-coalgebra defines a lift of the Serre coalgebra structure into a category of $E_\infty$-coalgebras
	\begin{equation}\label{eq:lift to e-infty cubical}
		\begin{tikzcd}[column sep=normal, row sep=small]
			& \coAlg_{\UM} \arrow[d] \\
			\cSet \arrow[r]
			\arrow[ur,out=60, in=180, dashed, "\cchainsUM"]
			\arrow[r, "\cchainsA"]
			& \coAlg.
		\end{tikzcd}
	\end{equation}
\end{example}

\subsection{Tensor product of $\cM$-bialgebras}

In this section we show that the category of $\cM$-bialgebras is monoidal.
More specifically, that the monoidal category structure on $\coAlg$ induces one on $\biAlg_{\M}$.

\begin{lemma}\label{l:monoidal M-bialg}
	Let $C$ and $C'$ be $\M$-bialgebras.
	The coalgebra $C \ot C'$ is a natural $\M$-bialgebra with
	\begin{equation}\label{eq:monoidal_product}
		(a \ot b) \ast (c \ot d) =
		a \aug(c) \ot (b \ast d) + (a \ast c) \ot \aug(b) d,
	\end{equation}
	where $a,c \in C$ and $b,d \in C'$ are arbitrary elements.
\end{lemma}

The proof of this lemma is a straightforward computation which for completeness is presented in \cref{s:appendix}.

\begin{remark}
	The prop $\cM$ controlling $\cM$-bialgebras is obtained by applying the functor of chains to a cellular prop whose cells are cubical \cite{medina2021prop2}.
	The $\ast$ map is represented by a cell isomorphic to the interval $[01]$, with $[1]$ corresponding to $\aug \ot \, \id$ and $[0]$ to $\id \ot \aug$.
	\cref{eq:monoidal_product} comes from the diagonal $[01] \mapsto [0] \ot [01] + [01] \ot [1]$.
	In fact, if so inclined, one can observe that $\cM$ is Hopf, that is to say, that for each biarity $(s,r)$ the the chain complex $\cM(s,r)$ is a coalgebra in a compatible way.
\end{remark}

It can be inspected that using this theorem on $\chains(\cube^n) \cong \chains(\cube^1)^{\ot n}$ one recovers the $\cM$-bialgebra described in \cref{ex:cubical e-infty}.
Since for any two cubical sets $X$ and $X'$ we have $\chains(X \times X') \cong \chains(X) \ot \chains(X')$ the above lemma implies the following statement, which constitutes the first part of our main result.

\begin{theorem}\label{t:cubical e-infty chains are monoidal}
	The functor $\cchainsUM$ extending the Serre coalgebra structure to an $E_\infty$-coalgebra structure is monoidal.
\end{theorem}

\subsection{A monoidal \pdfEinfty-coalgebra structure on the cobar construction}\label{ss:e-infty on cobar}

%The original cobar construction of Adams produces a monoid in $\Ch$ from a connected coalgebra. Baues enriched this construction to produce a monoid in $\coAlg$.
%He did this by using a version of the isomorphism $\cchains \ccobar \cong \cobar \schainsA$ of \cref{t:ccobar and cobar} and the Serre coalgebra on cubical chains.
%Since $\ccobar$ produces monoids in $\cSet$ and $\cchainsA$ is monoidal, the functor $\cchainsA \ccobar$ produces bialgebras, as claimed.

We now use the natural equivalence of functors $\cchains \ccobar \cong \cobar \schainsA$ proven in \cref{t:ccobar and cobar} to transfer structure.

After \cite{medina2022cube_einfty}, we know that cubical chains can be enriched with an $E_\infty$-coalgebra structure extending the Serre coalgebra (\cref{ss:e-infty on cubical}).
This $\UM$-structure can be transferred via $\cchains \ccobar \cong \cobar \schainsA$ to provide the cobar construction with and $E_\infty$-structure extending Baues coalgebra structure.
Furthermore, thanks to \cref{t:cubical e-infty chains are monoidal}, we know that $\cchainsUM$ is monoidal and therefore, $\cchains \ccobar$ produces monoids in $\coAlg_{\UM}$, i.e., $E_\infty$-bialgebras.
This discussion proves the first half of \cref{t:1st main thm in the intro}, which we collect in the following.

\begin{lemma}\label{l:lift of cobar to e-infty}
	The functor $\cchainsUM$ is monoidal and fits in the following diagram commuting up to a natural isomorphism:
	\[
	\begin{tikzcd}
		& \Mon_{\coAlg_\UM} \arrow[d] \\
		\Mon_{\cSet} \arrow[ru, "\cchainsUM", out=70, in=180] \arrow[r, "\cchainsA"]
		& \Mon_{\coAlg} \arrow[d] \\
		\sSet^0 \arrow[r, "\cobar \schainsA"] \arrow[u, "\ccobar"]
		& \Mon_{\Ch}.
	\end{tikzcd}
	\]
\end{lemma}

\subsection{Steenrod operations}

Throughout this subsection let us consider $\k$ to be the finite field $\Fp$ and $X$ a reduced simplicial set.

Since $\cobar \schainsA(X)$ is isomorphic to the chains on a cubical set (\cref{t:ccobar and cobar}), its mod $p$ homology is a priori equipped with a right action of the Steenrod algebra \cite{steenrod1962cohomology, milnor1958steenrod}.
As an application of the explicit construction of a natural $\UM$-coalgebra structure on $\cobar \schainsA(X)$, we can describe the chain level this action using the constructions of \cite{medina2021may_st}.
More explicitly, let $\Cp$ be the cyclic group of order $p$ and
\[
\begin{tikzcd} [column sep = .5cm]
	\cW(p) = \Fp[\Cp]\{e_0\} & \arrow[l, "\,T"'] \Fp[\Cp]\{e_1\} & \arrow[l, "\,N"'] \Fp[\Cp]\{e_2\} & \arrow[l, "\,T"'] \dotsb
\end{tikzcd}
\]
be the minimal free resolution of $\Fp$ as an $\Fp[\Cp]$-module.
Consider the $\Cp$-equivari-ant quasi-isomorphism $\psi_\UM \colon \cW(p) \to \UM(p)$ constructed in \cite{medina2021may_st}, and denote by $\psi^{\cobar}$ its composition with the morphism $\UM \to \End^{\cobar \schainsA(X)}$ defining our $E_\infty$-coalgebra structure on $\cobar \schainsA(X)$.

If $p$ is the even prime, the Steenrod square $P_s$ is represented by the map sending a representative $\mu$ of a class in the $k^\th$ homology of $\cobar \schainsA(X)$ to the following element in its double linear dual:
\[
\mu \mapsto
\Big( \alpha \mapsto (\alpha \ot \alpha) \big( \psi^{\cobar}(e_{k+2s})(\mu) \big) \Big).
\]
Similarly, if $p$ is an odd prime, the action of the Steenrod operations $\beta^\varepsilon P_{s}$ where $\varepsilon = 0,1$ are represented by
\[
\mu \mapsto
\Big( \alpha \mapsto \alpha^{\ot p} \big( (-1)^p \, \nu(q) \, \psi^{\cobar}(e_{(2s-q)(p-1)-\varepsilon})(\mu) \big) \Big)
\]
where
\[
q = -k -2s(p-1) + \varepsilon, \qquad
\nu(q) = (-1)^{q(q-1)m/2}(m!)^q, \qquad
m = (p-1)/2.
\]

The map $\psi^{\cobar}$, being a special case of the map $\psi^\cube$ defined in \cite{medina2021may_st}, can be described effectively, and it has been implemented in the specialized computer algebra system \texttt{ComCH} \cite{medina2021comch}.
In similar effective manner, explicit Cartan and Adem coboundaries \cite{medina2020cartan, medina2021adem} can be described, in the $p = 2$ case, for cochain level Steenrod squares of the cobar construction of $\schains(X)$.


\subsection{Adams's map as a monoidal \pdfEinfty-coalgebra quasi-isomorphism}

Adams' comparison map
\[
\theta_Z \colon \cobar \sSchainsA(Z,z) \to \cSchains(\loops_z Z)
\]
was originally identified to be a morphism of monoids in $\Ch$ which, if applied to simply-connected spaces, induces a homology isomorphism.
Baues showed that $\theta_Z$ is in fact a morphism of monoids in $\coAlg$, i.e., of bialgebras.
Whereas, using that $\sSing(Z,z)$ is a fibrant reduced simplicial set, the second named author and M. Zeinalian \cite{rivera2018cubical} showed that for a possibly non-simply connected pointed space $(Z,y)$ the algebras $\cobar \sSchainsA(Z,z)$ and $\cSchains(\loops_zZ)$ are quasi-isomorphic.
Later on, the second named author and S. Saneblidze
\cite{rivera2019path} showed that $\Theta \colon \ccobar(\sSing(Z,z)) \to \cSing(\loops_z Z)$ is a weak homotopy equivalence of monoids in $\cSet$.

We now show that $\theta_Z$ is in fact a quasi-isomorphism of $E_{\infty}$-bialgebras.
Considering the factorization \eqref{e:factorization of adams} of Adams' comparison map.
The isomorphism is by construction one of $\UM$-bialgebras, whereas the second map, induced from a weak-equivalence of monoids in $\cSet$, is also a morphism of $\UM$-bialgebras
since $\cchainsUM$ is monoidal (\cref{t:cubical e-infty chains are monoidal}).
We summarize this discussion, which constitutes the second half of \cref{t:1st main thm in the intro}, in the following.

\begin{lemma}\label{l:adams comparison is an e-infty bialgebra map}
	Adams' comparison map $\theta_Z \colon \cobar \sSchainsA(Z,z) \to \cSchains(\loops_z Z)$ is a quasi-isomorphism of $E_{\infty}$-bialgebras or, more specifically, of monoids in $\coAlg_\UM$ for any pointed space $(Z,z)$.
\end{lemma}