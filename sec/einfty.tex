% !TEX root = ../cobar1.tex

\section{Monoidal \pdfEinfty-structures}

\todo{@anibal: Write mini-intro for s:monoidal e-infty}
talking points

that is a cofibrant resolution of the terminal operad -- which controls cocommutative and coassociative coalgebras.

\subsection{$\cM$-bialgebras}

This section is an elaboration on the presentation given in \cite{medina2020prop1}, where proofs to the statement made here can be found.

\begin{definition*}
	An $\cM$-bialgebra is a coalgebra $(C, \Delta, \aug)$ together with a degree~1 product satisfying for any $a,b \in C$ that:
	\begin{align}
		\label{eq:M-bialg def 1}
		\aug(a \ast b) =\ & 0, \\
		\label{eq:M-bialg def 2}
		\bd (a \ast b) =\ & \bd a \ast b - (-1)^{a} a \ast \bd b + \aug(a) b - (-1)^{a} a \aug(b).
	\end{align}
\end{definition*}

\cref{eq:M-bialg def 2} states that the product, regarded as an element in the chain complex $\Hom(C \ot C, C)$, has boundary equal to $\aug \ot\, \id - \id \ot \aug$.

%\begin{proof}
	%\begin{align*}
	%	(\bd \ast)(x \ot y) =\ & (\aug \ot\, \id - \id \ot \aug)(x \ot y), \\
	%	(\bd \circ \ast + \ast \circ (\bd \ot\, \id + \id \ot \bd))(x \ot y) =\ & \aug(x) \ot\, y - x \ot \aug(y), \\
	%	(-1)^x \bd (x \ast y) + (-1)^{x-1} \bd x \ast y + x \ot \bd y =\ & \aug(x) \ot\, y - x \ot \aug(y), \\
	%	\bd (x \ast y) =\ & \bd x \ast y + (-1)^{x+1} x \ast \bd y + \aug(x) y + (-1)^{x+1} x \aug(y).
	%\end{align*}
%\end{proof}

\begin{proposition*}
	Let $(C, \Delta, \aug, \ast)$ be an $\cM$-bialgebra.
	The collection of all maps $\set{C \to C^{\ot r}}_{r \in \N}$ generated by $\Delta$, $\aug$ and $\ast$ makes $C$ into an $E_\infty$-coalgebra.
	More specifically, into a coalgebra over $\UM$, a model of the $E_\infty$-operad.
\end{proposition*}

The operad $\UM$ allude to above is obtained from a prop $\M$ by restricting its structure to biarities $(1,r)$ for $r \in \N$.

\subsection{\pdfEinfty-structure on simplicial chains}

We recall the construction of an $E_\infty$ extension of the Alexander--Whitney coalgebra structure on simplicial chains introduced in \cite{medina2020prop1}.
Let us start by considering the representable simplicial sets.
The coalgebra $\chains(\simplex^n)$ can be made into a natural $\cM$-bialgebra considering an algebraic version of the \textit{join product} defined by
\begin{equation*}
	\left[v_0, \dots, v_p \right] \ast \left[v_{p+1}, \dots, v_q\right] =
	\begin{cases}
		(-1)^{p} \sign(\pi) \left[v_{\pi(0)}, \dots, v_{\pi(q)}\right] &
		\text{ if } v_i \neq v_j \text{ for } i \neq j, \\
		\hfil 0 & \text{ if not},
	\end{cases}
\end{equation*}
where $\pi$ is the permutation that orders the vertices.
A Kan extension of the induced $\UM$-coalgebra structure on $\chains(\simplex^n)$ defines a lift of the Alexander--Whitney coalgebra structure on simplicial chains into the category of $E_\infty$-coalgebras defined by $\UM$.
Diagrammatically, we have
\begin{equation*}
	\begin{tikzcd}[column sep=normal, row sep=small]
		& \coAlg_{\UM} \arrow[d] \\
		\sSet \arrow[r]
		\arrow[ur,out=60, in=180, "\schainsUM"]
		\arrow[r, "\schainsA"]
		& \coAlg,
	\end{tikzcd}
\end{equation*}
where the vertical arrow is the obvious forgetful functor.

\subsection{\pdfEinfty-structure on cubical chains}\label{ss:cubical e-infty}

We recall the construction of an $E_\infty$ extension of the Serre coalgebra structure on cubical chains introduced in \cite{medina2022cube_einfty}.
Let us first consider representable cubical sets.
The coalgebra structure on $\chains(\cube^n)$ can be made into a natural $\cM$-bialgebra considering a product defined using the following notation.
For a basis element $x = x_1 \ot \dotsb \ot x_n$ of $\chains(\cube^n)$ and an integer $\ell \in \set{1,\dots,n}$ we write
\begin{align*}
	x_{<\ell} & = x_1 \ot \dotsb \ot x_{\ell-1}, \\
	x_{>\ell} & = x_{\ell+1} \ot \dotsb \ot x_n,
\end{align*}
with the convention $x_{<1} = x_{>n} = 1 \in \Z$.
Then, for two such basis elements $x$ and $y$ we set
\begin{equation}\label{eq:product on cubes}
	(x_1 \ot \dotsb \ot x_n) \ast (y_1 \ot \dotsb \ot y_n) =
	\sum_{i=1}^n x_{<i}\, \epsilon(y_{<i}) \ot x_i \ast y_i \ot \epsilon(x_{>i}) \, y_{>i},
\end{equation}
where the only non-zero values of $x_i \ast y_i$ are
\[
[0] \ast [1] = [0, 1], \qquad [1] \ast [0] = -[0, 1].
\]
%\begin{proof}
%	Just to be sure
%	\begin{align*}
%		\bd (x \ast y) &=
%		\bd (\sum_{i=1}^n x_{<i}\, \epsilon(y_{<i}) \ot x_i \ast y_i \ot \epsilon(x_{>i}) \, y_{>i}) \\ &=
%		\sum_{i=1}^n \bd x_{<i}\, \epsilon(y_{<i}) \ot x_i \ast y_i \ot \epsilon(x_{>i} \, y_{>i}) \\ &+		\sum_{i=1}^n (-1)^{x_{<i}} x_{<i}\, \epsilon(y_{<i}) \ot \bd (x_i \ast y_i) \ot \epsilon(x_{>i}) \, y_{>i} \\ &+
%		\sum_{i=1}^n (-1)^{x+1} x_{<i}\, \epsilon(y_{<i}) \ot x_i \ast y_i \ot \epsilon(x_{>i}) \, \bd y_{>i}
%	\end{align*}
%	\begin{align*}
%		\bd x \ast y &=
%		\sum_{i=1}^n \bd x_{<i} \, \epsilon(y_{<i}) \ot x_i \ast y_i \ot \epsilon(x_{>i}) \, y_{>i} \\ &+
%		\sum_{i=1}^n (-1)^{x_{<i}} x_{<i} \, \epsilon(y_{<i}) \ot \bd x_i \ast y_i \ot \epsilon(x_{>i}) \, y_{>i}.
%	\end{align*}
%	\begin{align*}
%		x \ast \bd y &=
%		\sum_{i=1}^n x_{<i} \, \epsilon(y_{<i}) \ot x_i \ast y_i \ot \epsilon(x_{>i}) \, \bd y_{>i} \\ &+
%		\sum_{i=1}^n x_{<i} \, \epsilon(y_{<i}) \ot x_i \ast \bd y_i \ot \epsilon(x_{>i}) \, y_{>i}.
%	\end{align*}
%	Therefore, $\bd(x \ast y) - \bd x \ast y + (-1)^x x \ast y$ is equal to
%	\begin{align*}&
%		\sum_{i=1}^n (-1)^{x_{<i}} x_{<i}\, \epsilon(y_{<i}) \ot \big(\bd (x_i \ast y_i) - \bd x_i \ast y_i + (-1)^{x-i} x_i \ast y_i\big) \ot \epsilon(x_{>i}) \, y_{>i} \\ &=
%		\sum_{i=1}^n (-1)^{x_{<i}} x_{<i}\, \epsilon(y_{<i}) \ot \aug(x_i) y_i - (-1)^{x_i} x_i \aug(y_i) \ot \epsilon(x_{>i}) \, y_{>i} \\ &=
%		\sum_{i=1}^n (-1)^{x_{<i}} x_{<i}\, \epsilon(y_{<i}) \ot \aug(x_i) y_i \ot \epsilon(x_{>i}) \, y_{>i} \\ &-
%		\sum_{i=1}^n (-1)^{x_{<i+1}} x_{<i}\, \epsilon(y_{<i}) \ot x_i \aug(y_i) \ot \epsilon(x_{>i}) \, y_{>i}
%	\end{align*}
%	And a telescopic sum finishes the proof.
%\end{proof}
A Kan extension of the induced $\UM$-coalgebra structure on $\chains(\cube^n)$ defines a lift of the Serre coalgebra structure into the category of $E_\infty$-coalgebras defined by $\UM$.
That is, a commutative diagram
\begin{equation}\label{eq:lift to e-infty cubical}
	\begin{tikzcd}[column sep=normal, row sep=small]
		& \coAlg_{\UM} \arrow[d] \\
		\cSet \arrow[r]
		\arrow[ur,out=60, in=180, "\cchainsUM"]
		\arrow[r, "\cchainsA"]
		& \coAlg.
	\end{tikzcd}
\end{equation}

\begin{remark*}
	The product defined above in \cref{eq:product on cubes} differs from the one defined in \cite{medina2022cube_einfty} by the sign $(-1)^x$.
	The convention used here is more natural as we will see in \cref{ss:cube_einfty revisited}.
\end{remark*}

\subsection{Tensor product of $\cM$-bialgebras}

In this subsection we describe an extension of the tensor product of coalgebras to $\cM$-bialgebras, which
naturally induces a monoidal structure on the category of $E_\infty$-coalgebras defined by the $\UM$.

\begin{lemma}\label{l:monoidal M-bialg}
	Let $C$ and $C'$ be $\M$-bialgebras.
	The coalgebra $C \ot C'$ is a natural $\M$-bialgebra with
	\begin{equation}\label{eq:monoidal_product}
		(a \ot b) \ast (c \ot d) =
		a \aug(c) \ot (b \ast d) + (a \ast c) \ot \aug(b) d,
	\end{equation}
	for any $a,c \in C$ and $b,d \in C'$.
\end{lemma}

\begin{proof}
	We verify Identity \eqref{eq:M-bialg def 1} using that $\aug(b \ast d) = \aug(a \ast c) = 0$,
	\begin{align*}
		\aug\big((a \ot b) \ast (c \ot d)\big) =&
		\aug\big(a \aug(c) \ot (b \ast d)\big) + \aug\big((a \ast c) \ot \aug(b) d\big) \\ =&
		\aug(a) \aug(c) \ot \aug(b \ast d) + \aug(a \ast c) \ot \aug(b) \aug(d) \\ =& \ 0.
	\end{align*}
	To verify Identity \eqref{eq:M-bialg def 2} we need to show that
	\begin{equation}\label{eq:monoidal appendix}
		\begin{split}
			\bd \big((a \ot b) \ast (c \ot d)\big) =\, &
			\bd (a \ot b) \ast (c \ot d) - (-1)^{a+b} (a \ot b) \ast \bd (c \ot d) \\ +\, &
			\aug(a \ot b) (c \ot b) - (-1)^{a+b} (a \ot b) \aug(c \ot d).
		\end{split}
	\end{equation}
	Let us start computing the left hand side of the above expression.
	\begin{align*}
		\bd\big((a \ot b) \ast (c \ot d)\big) =\ &
		\bd\big(a \aug(c) \ot (b \ast d) + (a \ast c) \ot \aug(b) d\big) \\ =\ &
		\bd a \aug(c) \ot (b \ast d) \,+\, (-1)^{a} a \aug(c) \ot \bd(b \ast d) \\ +\ &
		\bd (a \ast c) \ot \aug(b) d \,+\, (-1)^{a+c+1} (a \ast c) \ot \aug(b) \bd d.
	\end{align*}
	Using that $C$ and $C'$ satisfy Identity \eqref{eq:M-bialg def 2} we have that
	\begin{align*}
		\bd\big((a \ot b) \ast (c \ot d)\big) =\ &
		\bd a \aug(c) \ot (b \ast d) \\ +\ &
		(-1)^a a \aug(c) \ot \big(\bd b \ast d + (-1)^{b+1} b \ast \bd d + \aug(b) d + (-1)^{b+1} b \aug(d)\big) \\ +\ &
		\big(\bd a \ast c + (-1)^{a+1} a \ast \bd c + \aug(a) c + (-1)^{a+1} a \aug(c)\big) \ot \aug(b) d \\ +\ &
		(-1)^{a+c+1} (a \ast c) \ot \aug(b) \bd d.
	\end{align*}
	Inspecting this expression we label terms which sum to $\bd\big((a \ot b) \ast (c \ot d)\big)$:\vspace*{-10pt}
	\begin{minipage}[t]{0.5\textwidth}
		\begin{align}& \label{x4}
			(-1)^0 \bd a \aug(c) \ot (b \ast d) \\& \label{x5}
			(-1)^{a} a \aug(c) \ot (\bd b \ast d) \\& \label{x6}
			(-1)^{a+b+1} a \aug(c) \ot (b \ast \bd d) \\& \label{x7}
			(-1)^a a \aug(c) \ot \aug(b) d \\& \label{x8}
			(-1)^{a+b+1} a \aug(c) \ot b \aug(d)
		\end{align}
	\end{minipage}
	\begin{minipage}[t]{0.5\textwidth}
		\begin{align}& \label{x9}
			(-1)^0 (\bd a \ast c) \ot \aug(b) d \\& \label{x10}
			(-1)^{a+1} (a \ast \bd c) \ot \aug(b) d \\& \label{x11}
			(-1)^0 \aug(a) c \ot \aug(b) d \\& \label{x12}
			(-1)^{a+1} a \aug(c) \ot \aug(b) d \\& \label{x13}
			(-1)^{a+c+1} (a \ast c) \ot \aug(b) \bd d \quad ||
		\end{align}
		\vskip\parskip
	\end{minipage}
	Additionally, since
	\begin{equation*}
		\bd (a \ot b) \ast (c \ot d) \,=\,
		(\bd a \ot b) \ast (c \ot d) \,+\,
		(-1)^a (a \ot \bd b) \ast (c \ot d),
	\end{equation*}
	the following labeled terms sum to $\bd (a \ot b) \ast (c \ot d)$:
	\begin{align}& \label{x14}
		(-1)^0 \bd a \aug(c) \ot (b \ast d) \\& \label{x15}
		(-1)^0 (\bd a \ast c) \ot \aug(b) d \\& \label{x16}
		(-1)^{a} a \aug(c) \ot (\bd b \ast d) \quad ||
	\end{align}
	Similarly, since
	\begin{equation*}
		(a \ot b) \ast \bd (c \ot d) \,=\,
		(a \ot b) \ast (\bd c \ot d) \,+\,
		(-1)^{c} (a \ot b) \ast (c \ot \bd d),
	\end{equation*}
	the following labeled terms sum to $(-1)^{a+b+1}(a \ot b) \ast \bd (c \ot d)$:
	\begin{align}& \label{x17}
		(-1)^{a+1} (a \ast \bd c) \ot \aug(b) d \\& \label{x18}
		(-1)^{a+b+1} a \aug(c) \ot (b \ast \bd d) \\& \label{x19}
		(-1)^{a+c+1} (a \ast c) \ot \aug(b) \bd d \quad ||
	\end{align}
	We have the following matching pairs of labeled summands.
	\begin{center}
		\eqref{x4}-\eqref{x14} :
		\eqref{x5}-\eqref{x16} :
		\eqref{x6}-\eqref{x18} :
		\eqref{x7}-\eqref{x12} :
		\eqref{x9}-\eqref{x15} :
		\eqref{x10}-\eqref{x17} :
		\eqref{x13}-\eqref{x19}.
	\end{center}
	Additionally, the sum of the unmatched terms \eqref{x11} and \eqref{x8} correspond to
	\[
	\aug(a \ot b) (c \ot b) - (-1)^{a+b} (a \ot b) \aug(c \ot d),
	\]
	which concludes the verification of both Identity \eqref{eq:monoidal appendix} and the lemma.
\end{proof}

\begin{definition*}
	We consider $\coAlg_\UM$ as a monoidal category with the tensor product induced by \cref{l:monoidal M-bialg}.
\end{definition*}

\begin{remark*}
	The prop $\cM$ controlling $\cM$-bialgebras is obtained by applying the functor of cellular chains to a CW prop whose cells are cubical \cite{medina2021prop2}.
	As such, the Serre coalgebra construction induces a Hopf structure on $\cM$, that is to say, a coalgebra structure on each biarity compatible with the prop structure.
	The product $\ast$ of an $\cM$-bialgebra is represented by a cell isomorphic to the interval $[0,1]$, whose boundary points $[0]$ and $[1]$ represent respectively $\aug \ot \, \id$ and $\id \ot \aug$.
	Diagrammatically, the boundary of this cell is represented by
	\[
	\product \quad \raisebox{3pt}{$\xra{\bd}$} \quad \leftboundary \ \raisebox{3pt}{$-$} \ \rightboundary \,,
	\]
	whereas its diagonal by
	\[
	\product \quad \raisebox{3pt}{$\xra{\Delta}$} \quad
	\rightboundary \ \raisebox{3pt}{$\ot$} \product \ \raisebox{3pt}{$+$} \ \product \raisebox{3pt}{$\ot$} \ \leftboundary \,.
	\]
	\cref{eq:monoidal_product} defining the product on the tensor product of $\M$-bialgebras is determined by this diagonal, and the monoidal structure on $\coAlg_\UM$ is the one induced by this Hopf structure.
\end{remark*}

\subsection{Revisiting the \pdfEinfty-coalgebra structure on cubical chains}\label{ss:cube_einfty revisited}

In this subsection we use the monoidal structure on $\M$-bialgebras to recover the $E_\infty$-structure on cubical chains.
We prove the following generalization of \cref{p:serre coalgebra}.

\begin{theorem}\label{t:cubical e-infty chains are monoidal}
	The functor $\cchainsUM \colon \cSet \to \coAlg_\UM$ is the unique monoidal functor defined by the $\M$-bialgebra structure on $\chains(\cube^1)$.
\end{theorem}

\begin{proof}
	It suffices to show that the coalgebra isomorphism $\chains(\cube^n) \cong \chains(\cube^1)^{\ot n}$ is one of $\M$-bialgebra for each $n \in \N$.
	We will proceed by induction with the base case holding by definition.
	Let $x = x_1 \ot \dotsb \ot x_n$ and $y = y_1 \ot \dotsb \ot y_n$ in $\chains(\cube^n)$.
	The following straightforward computation verifies that $x \ast y$ corresponds to $(x_{<n} \ot x_n) \ast (y_{<n} \ot y_n)$ in $\chains(\cube^{n-1}) \ot \chains(\cube^1)$:
	\begin{align*}
		x \ast y &=
		\sum_{i=1}^n x_{<i}\, \epsilon(y_{<i}) \ot x_i \ast y_i \ot \epsilon(x_{>i}) \, y_{>i} \\ &=
		\sum_{i=1}^{n-1} x_{<i}\, \epsilon(y_{<i}) \ot x_i \ast y_i \ot \epsilon(x_{>i}) \, y_{>i} \ +\
		x_{<n}\, \epsilon(y_{<n}) \ot x_n \ast y_n \\ &=
		x_{<n} \ast y_{<n} \ot \aug(x_n)\, y_n \ +\ x_{<n}\, \epsilon(y_{<n}) \ot x_n \ast y_n \\ &=
		(x_{<n} \ot x_n) \ast (y_{<n} \ot y_n).
	\end{align*}
	Which concludes the proof.
\end{proof}

\subsection{A monoidal \pdfEinfty-coalgebra structure on the cobar construction}\label{ss:e-infty on cobar}

%The original cobar construction of Adams produces a monoid in $\Ch$ from a connected coalgebra. Baues enriched this construction to produce a monoid in $\coAlg$.
%He did this by using a version of the isomorphism $\cchains \ccobar \cong \cobar \schainsA$ of \cref{t:ccobar and cobar} and the Serre coalgebra on cubical chains.
%Since $\ccobar$ produces monoids in $\cSet$ and $\cchainsA$ is monoidal, the functor $\cchainsA \ccobar$ produces bialgebras, as claimed.

We now use the natural equivalence of functors $\cchains \ccobar \cong \cobar \schainsA$ proven in \cref{t:ccobar and cobar} to transfer structure.

After \cite{medina2022cube_einfty}, we know that cubical chains can be enriched with an $E_\infty$-coalgebra structure extending the Serre coalgebra (\cref{ss:e-infty on cubical}).
This $\UM$-structure can be transferred via $\cchains \ccobar \cong \cobar \schainsA$ to provide the cobar construction with and $E_\infty$-structure extending Baues coalgebra structure.
Furthermore, thanks to \cref{t:cubical e-infty chains are monoidal}, we know that $\cchainsUM$ is monoidal and therefore, $\cchains \ccobar$ produces monoids in $\coAlg_{\UM}$, i.e., $E_\infty$-bialgebras.
This discussion proves the first half of \cref{t:1st main thm in the intro}, which we collect in the following.

\begin{lemma}\label{l:lift of cobar to e-infty}
	The functor $\cchainsUM$ is monoidal and fits in the following diagram commuting up to a natural isomorphism:
	\[
	\begin{tikzcd}
		& \Mon_{\coAlg_\UM} \arrow[d] \\
		\Mon_{\cSet} \arrow[ru, "\cchainsUM", out=70, in=180] \arrow[r, "\cchainsA"]
		& \Mon_{\coAlg} \arrow[d] \\
		\sSet^0 \arrow[r, "\cobar \schainsA"] \arrow[u, "\ccobar"]
		& \Mon_{\Ch}.
	\end{tikzcd}
	\]
\end{lemma}

\subsection{Steenrod operations}

Throughout this subsection let us consider $\k$ to be the finite field $\Fp$ and $X$ a reduced simplicial set.

Since $\cobar \schainsA(X)$ is isomorphic to the chains on a cubical set (\cref{t:ccobar and cobar}), its mod $p$ homology is a priori equipped with a right action of the Steenrod algebra \cite{steenrod1962cohomology, milnor1958steenrod}.
As an application of the explicit construction of a natural $\UM$-coalgebra structure on $\cobar \schainsA(X)$, we can describe the chain level this action using the constructions of \cite{medina2021may_st}.
More explicitly, let $\Cp$ be the cyclic group of order $p$ and
\[
\begin{tikzcd} [column sep = .5cm]
	\cW(p) = \Fp[\Cp]\{e_0\} & \arrow[l, "\,T"'] \Fp[\Cp]\{e_1\} & \arrow[l, "\,N"'] \Fp[\Cp]\{e_2\} & \arrow[l, "\,T"'] \dotsb
\end{tikzcd}
\]
be the minimal free resolution of $\Fp$ as an $\Fp[\Cp]$-module.
Consider the $\Cp$-equivari-ant quasi-isomorphism $\psi_\UM \colon \cW(p) \to \UM(p)$ constructed in \cite{medina2021may_st}, and denote by $\psi^{\cobar}$ its composition with the morphism $\UM \to \End^{\cobar \schainsA(X)}$ defining our $E_\infty$-coalgebra structure on $\cobar \schainsA(X)$.

If $p$ is the even prime, the Steenrod square $P_s$ is represented by the map sending a representative $\mu$ of a class in the $k^\th$ homology of $\cobar \schainsA(X)$ to the following element in its double linear dual:
\[
\mu \mapsto
\Big( \alpha \mapsto (\alpha \ot \alpha) \big( \psi^{\cobar}(e_{k+2s})(\mu) \big) \Big).
\]
Similarly, if $p$ is an odd prime, the action of the Steenrod operations $\beta^\varepsilon P_{s}$ where $\varepsilon = 0,1$ are represented by
\[
\mu \mapsto
\Big( \alpha \mapsto \alpha^{\ot p} \big( (-1)^p \, \nu(q) \, \psi^{\cobar}(e_{(2s-q)(p-1)-\varepsilon})(\mu) \big) \Big)
\]
where
\[
q = -k -2s(p-1) + \varepsilon, \qquad
\nu(q) = (-1)^{q(q-1)m/2}(m!)^q, \qquad
m = (p-1)/2.
\]

The map $\psi^{\cobar}$, being a special case of the map $\psi^\cube$ defined in \cite{medina2021may_st}, can be described effectively, and it has been implemented in the specialized computer algebra system \texttt{ComCH} \cite{medina2021comch}.
In similar effective manner, explicit Cartan and Adem coboundaries \cite{medina2020cartan, medina2021adem} can be described, in the $p = 2$ case, for cochain level Steenrod squares of the cobar construction of $\schains(X)$.


\subsection{Adams's map as a monoidal \pdfEinfty-coalgebra quasi-isomorphism}

Adams' comparison map
\[
\theta_Z \colon \cobar \sSchainsA(Z,z) \to \cSchains(\loops_z Z)
\]
was originally identified to be a morphism of monoids in $\Ch$ which, if applied to simply-connected spaces, induces a homology isomorphism.
Baues showed that $\theta_Z$ is in fact a morphism of monoids in $\coAlg$, i.e., of bialgebras.
Whereas, using that $\sSing(Z,z)$ is a fibrant reduced simplicial set, the second named author and M. Zeinalian \cite{rivera2018cubical} showed that for a possibly non-simply connected pointed space $(Z,y)$ the algebras $\cobar \sSchainsA(Z,z)$ and $\cSchains(\loops_zZ)$ are quasi-isomorphic.
Later on, the second named author and S. Saneblidze
\cite{rivera2019path} showed that $\Theta \colon \ccobar(\sSing(Z,z)) \to \cSing(\loops_z Z)$ is a weak homotopy equivalence of monoids in $\cSet$.

We now show that $\theta_Z$ is in fact a quasi-isomorphism of $E_{\infty}$-bialgebras.
Considering the factorization \eqref{e:factorization of adams} of Adams' comparison map.
The isomorphism is by construction one of $\UM$-bialgebras, whereas the second map, induced from a weak-equivalence of monoids in $\cSet$, is also a morphism of $\UM$-bialgebras
since $\cchainsUM$ is monoidal (\cref{t:cubical e-infty chains are monoidal}).
We summarize this discussion, which constitutes the second half of \cref{t:1st main thm in the intro}, in the following.

\begin{lemma}\label{l:adams comparison is an e-infty bialgebra map}
	Adams' comparison map $\theta_Z \colon \cobar \sSchainsA(Z,z) \to \cSchains(\loops_z Z)$ is a quasi-isomorphism of $E_{\infty}$-bialgebras or, more specifically, of monoids in $\coAlg_\UM$ for any pointed space $(Z,z)$.
\end{lemma}