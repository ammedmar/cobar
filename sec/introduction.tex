% !TEX root = ../cobar1.tex

\section{Introduction}

For any topological space $\fX$, its complex of singular chains $\Schains(\fX)$ -- regarded as a differential graded module -- encodes the homology of $\fX$ in its quasi-isomorphism type.
More homotopical information can be stored in the q-i type of this chain complex if considered as a (dg) coalgebra, which we will denote $\SchainsA(\fX)$, where the coproduct comes from a chain approximation to the diagonal $\fX \to \fX \times \fX$.
For example, the cohomology ring of $\fX$ is retained, but additional invariants like the action of the Steenrod algebra on its mod $p$ cohomology is not.
In Mandell's seminal work \cite{mandell2006homotopy_type} it is shown that, when $\fX$ is nilpotent and finite type, the entire homotopy type of $\fX$ can be encoded in the q-i type of this complex if considered as an $E_\infty$-coalgebra, a structure providing $\SchainsA(\fX)$ with coherent homotopies witnessing the derived coassociativity and cocommutativity of the coproduct, which come from the corresponding strict relations satisfied by the diagonal map.

In this work we are interested in modeling algebraically the homotopy type of the based loop space $\loops_z \fZ$ of a pointed space $(\fZ,z)$, which is not just a space but a topological monoid with the concatenation of loops.
%Recall \textit{based loop space} $\loops_z \fZ$ construction for a pointed space $(\fZ,z)$ -- build from pointed maps from the circle and equipped with a monoid structure coming from concatenation.

The first contribution of this paper is to explicitly endow the cubical singular chains of $\loops_z \fZ$ with the structure of a monoid in a category of $E_\infty$-coalgebras extending the Serre diagonal.
More specifically, we verify that the monoid structure in $\cSchains(\loops_z \fZ)$ -- induced from concatenation of loops -- is compatible with the $E_\infty$-coalgebra structure on cubical singular chains introduced in \cite{medina2022cube_einfty}.

We will also concern ourselves with another algebraic approximation to the homotopy type of $\loops_z \fZ$, which was introduced by Adams \cite{adams1956cobar}.
It is obtained using his celebrated \textit{cobar construction} functor $\cobar$, which associates to a coaugmented coalgebra a monoid in chain complexes.
More specifically, by applying $\cobar$ to the simplicial singular chains of $(\fZ, z)$, a coalgebra with the Alexander--Whitney diagonal that is build using maps of simplices sending all vertices to the base point.
Adams constructed a natural monoid chain map $\theta$ from his model $\cobar \sSchainsA(\fZ, z)$ to $\cSchains(\loops_z \fZ)$ and proved it to be a q-i if $\fZ$ is simply-connected, a statement that also holds true for path-connected spaces after \cite{hess2010cobar} or \cite{rivera2018cubical}.
The second main contribution of this paper is to make Adams model $\cobar \sSchainsA(\fZ, z)$ into a monoid in $E_\infty$-coalgebras and to prove that
\[
\theta \colon \cobar \sSchainsA(\fZ, z) \to \cSchains(\loops_z \fZ)
\]
respects this higher structure.

Our starting point is groundbreaking work by H.~Baues, which imply statements similar to those in this work but in the category of coalgebras.
In \cite{baues1998hopf}, he reinterpreted Adams' algebraic construction at a deeper geometric level.
This allowed him to endow $\cobar \sSchainsA(\fZ, z)$ with the structure of a monoid in coalgebras and to show that $\theta$ preserves this structure.
To prove the statements of this work we interpret Adams' constructions at an even deeper categorical level.
The simplicial singular complex $\sSing(\fZ, z)$ is an example of an object in the category $\sSet^0$ of $0$-reduced simplicial sets.
We interpret Baues' geometric cobar construction, originally defined for $1$-reduced simplicial sets, as a functor
\begin{equation*}
	\ccobar \colon \sSet^0 \to \Mon_{\cSet},
\end{equation*}
where $\Mon_{\cSet}$ is the category of monoids in cubical sets.
The key difference with Baues' original work is the use of connections to obtain a natural construction before geometric realization.

Additionally, we need a suitable model of the $E_\infty$-operad, which we take to be the operad $\UM$ introduced in \cite{medina2020prop1}.
After proving that its coalgebras form a monoidal category, we show that the functor $\cchainsUM$ from cubical sets to $\UM$-coalgebras is monoidal, which implies that $\cchainsUM (\ccobar X)$ is a monoid in $\UM$-coalgebras for any 0-reduced simplicial set.
More precisely we have the following.

\begin{theorem*}
	The functor $\cchainsUM$ is monoidal and its associated functor on monoids fits in the following diagram commuting up to natural isomorphisms:
	\[
	\begin{tikzcd} [row sep=small]
		& \Mon_{\coAlg_\UM} \arrow[d] \\
		\Mon_{\cSet} \arrow[ru, "\cchainsUM", out=70, in=180, near start] \arrow[r, "\cchainsA"]
		& \Mon_{\coAlg} \arrow[d] \\
		\sSet^0 \arrow[r, "\cobar \schainsA"] \arrow[u, "\ccobar"]
		& \Mon_{\Ch},
	\end{tikzcd}
	\]
	where the unlabeled arrows are forgetful functors.
\end{theorem*}

In the above diagram, the arrow from $\sSet^0$ to $\Mon_{\Ch}$ is Adams' construction, the one from $\sSet^0$ to $\Mon_{\coAlg}$ is Baues' enhancement, and the one from $\sSet^0$ to $\Mon_{\coAlg_{\UM}}$ is our contribution.

Additionally we have the following statement when the 0-reduced simplicial set comes from a pointed space.

\begin{theorem*}
	For any pointed space $(\fZ, z)$, the map
	\[
	\theta \colon \cobar \sSchainsA(\fZ, z) \to \cSchains(\loops_z \fZ)
	\]
	is a q-i of monoids in $\UM$-coalgebras.
\end{theorem*}

As mentioned before, the fact that $\theta$ respects the monoid structure in $\Ch$ was proven by Adams, whereas the compatibility of the monoid structure with the Serre coalgebra structure was established by Baues.
Our contribution is the compatibility of the monoid structure with a full $E_\infty$-coalgebra extension of Serre's coalgebra.