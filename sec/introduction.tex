
\section{Introduction}

An embodiment of the \textit{homotopy hypothesis}, a cornerstone of modern algebraic topology, recasts the homotopy theory of pointed topological spaces in terms of topological groups.
More concretely, associated to any pointed topological space $(Z,z)$ one may naturally associate a topological monoid $\loops_z Z$ -- its \textit{based loop space} -- whose points are pairs $(\gamma, r)$ where $r \in \R_{\geq 0}$ and $\gamma \colon [0,r] \to Z$ is a continuous map with $\gamma(0) = z = \gamma(r)$.
We consider $\loops_z Z$ to be equipped with the compact-open topology and the continuous multiplication
\[
\star \colon \loops_z Z \times \loops_z Z \to \loops_z Z
\]
given by \textit{concatenation} of loops, with the identity element being the \textit{constant loop} $(c_r,0)$.
This construction gives rise to a functor
\[
\loops \colon \Top^{\ast} \to \Mon_{\Top}
\]
from the category of pointed topological spaces to the category of topological monoids.
For any pointed topological space $(Z,z)$, the topological monoid $\loops_z Z$ has the special property that its elements have inverses \textit{up to homotopy}.
In fact, for any loop $\gamma \colon [0,r] \to Z $ one may define a new loop $\overline{\gamma} \colon [0,r] \to Z$ which runs $\gamma$ in the opposite direction, namely $\overline{\gamma}(t) = \gamma(r-t)$.
The loops $\gamma \star \overline{\gamma}$ and $\overline{\gamma} \star \gamma$ are both homotopic to $c_z$.
Consequently, the set of path components $\pi_0(\loops_z Z)$ has an induced group structure, which is naturally isomorphic to the fundamental group $\pi_1(Z,z)$.
Furthermore, $\loops_z Z$ is weak homotopy equivalent to a topological group where the group identities hold strictly.
The pointed space $(Z,z)$ may then be recovered, up to weak homotopy equivalence, from $\loops_z Z$ through the \textit{classifying space}, or \textit{delooping}, construction.

A principal goal of algebraic topology, if not its \textit{raison d'\^{e}tre}, is to encode the theory of topological spaces up to some specified notion of equivalence in terms of combinatorial or algebraic objects, allowing for the effective analysis of topological properties quantitatively.
We will focus on spaces up to weak homotopy equivalence, where a first example of this algebraization process is provided by the functor of simplicial or cubical singular chains
\[
\Schains \colon \Top \to \Ch,
\]
which associates a chain complex over a fixed commutative ring to any topological space.

The singular chains construction may be thought of as a linearization of a topological space and, as expected, much structure is lost by it.
More homotopical information can be encoded if the resulting chain complex is equipped with non-linear structure, for example, a chain approximation to the diagonal map $Z \to Z \times Z$.
In this case, we regard the singular chains as a differential graded (dg) coassociative coalgebra, or better still, if a coherent family of homotopies enforcing its derived cocommutativity is considered, as an $E_\infty$-coalgebra.
Diagrammatically, this can be represented as lifts of the form
\begin{equation}
\begin{tikzcd}[column sep=normal, row sep=small]
& \coAlg_{E_\infty} \arrow[d] \\
& \coAlg \arrow[d] \\
\Top \arrow[uur, "\Schains_{E_\infty}", out=90, in=180] \arrow[r, "\Schains"]
\arrow[ur, "\SchainsAS", out=60, in=180, near end]
\arrow[r, "\Schains"]
& \Ch.
\end{tikzcd}
\end{equation}
The study of $E_\infty$-structures has a long history, where (co)homology operations \cite{steenrod1962cohomology, may1970general}, the recognition of infinite loop spaces \cite{boardman1973homotopy, may1972geometry}, and the complete algebraic representation of the $p$-adic homotopy category \cite{mandell2001padic} are key milestones.

For based loop spaces we will consider two algebraic models.
The first is given simply by the monoid of cubical singular chains $\cSchains(\loops_z Z)$.
The second $\cobar \sSchainsA(Z,z)$, introduced by F.~Adams in 1956, is defined as the \textit{cobar construction} on the coalgebra of simplicial singular chains, where
\[
\cobar \colon \coAlg^\ast \to \Mon_{\Ch}
\]
is a functor from the category of coaugemented dg coassociative coalgebras to the category of monoids in $\Ch$ \cite{adams1956cobar}.
In the same reference, Adams constructed a comparison map between these models, a natural chain map of monoids in $\Ch$
\begin{equation} \label{e:adams map}
\theta_Z \colon \cobar \sSchainsA(Z,z) \to \cSchains(\loops_z Z),
\end{equation}
which he proved to be a quasi-isomorphism if $Z$ is simply connected, a result extended to path-connected spaces in \cite{rivera2018cubical} by considering finer homotopical algebra, and in \cite{hess2010cobar} by a localization procedure.

In this article we enhance both of these models by explicitly constructing an $E_\infty$-bialgebra structure on them, i.e., an $E_\infty$-coalgebra structure compatible with its monoid structure, and show that Adams' comparison map is a quasi-isomorphism of $E_\infty$-bialgebras.

Our starting point is the groundbreaking work of H.~Baues that allowed him to accomplish analogous statements using bialgebras instead of $E_\infty$-bialgebras.
His insights are best explained in terms of \textit{cubical sets} (with connections), i.e., objects in the category $\cSet$ of presheaves over the strict monoidal category generated by the diagram
\[
\begin{tikzcd}
1 \arrow[r, out=45, in=135, "\delta^0"] \arrow[r, out=-45, in=-135, "\delta^1"'] & 2 \arrow[l, "\sigma"'] &[-10pt] \arrow[l, "\gamma"'] 2 \times 2
\end{tikzcd}
\]
restricted by certain relations \cite{brown1981cubes}.
With the Day convolution monoidal structure on $\cSet$, the singular complex functor $\cSing \colon \cSet \to \Top$ and the functor of (normalized) chains $\cchains \colon \cSet \to \Ch$ are both monoidal, which explains the monoid structure on
\[
\cSchains(\loops_z Z) \defeq \cchains \cSing(\loops_z Z).
\]
Furthermore, cubical chains are equipped with a natural coalgebra structure coming from Serre's chain approximation to the diagonal.
This structure lifts $\cchains$ to the category of coalgebras as a monoidal functor $\cchainsA$ which, in particular, makes $\cSchainsA(\loops_z Z)$ into a bialgebra, i.e., a monoid in $\coAlg$.

In \cite{baues1998hopf}, Baues reinterpreted Adams' algebraic construction at a deeper geometric level allowing him to
endow $\cobar \sSchainsA(Z,z)$ with the structure of a bialgebra, and to show that Adams' comparison map $\theta_Z$ is a quasi-isomorphism of bialgebras when $Z$ is simply-connected.

The simplicial singular complex of a topological space is a example of an object in $\sSet^0$, the category of ($0$-)reduced simplicial sets.
We interpret Baues' geometric cobar construction, originally defined for $1$-reduced simplicial sets, in more category-theoretic terms as a functor
\begin{equation} \label{e:cubical cobar construction}
\ccobar \colon \sSet^0 \to \Mon_{\cSet}
\end{equation}
whose key difference with Baues' original construction is the use of connections to obtain a natural construction before geometric realization.
Baues showed that Adams' comparison map factors as a composition
\begin{equation} \label{e:baues factorization of adams map}
\begin{tikzcd}[column sep = small]
\theta_Z \colon \cobar \sSchainsA(Z,z) \arrow[r] &
\cchains(\ccobar \sSing(Z,z)) \arrow[r] &
\cSchains(\loops_z Z),
\end{tikzcd}
\end{equation}
where the first chain map is an isomorphism and the second, a morphism of bialgebras, is induced from an inclusion of cubical sets $\ccobar \sSing(Z,z) \to \cSing(\loops_z Z)$ which is a weak equivalence following \cite{rivera2019path}.

The first map in \eqref{e:baues factorization of adams map} is defined for a general reduced simplicial set $X$ and it is always an isomorphism.
This permitted Baues to transfer the Serre coalgebra structure on $\cchainsA(\ccobar X)$ to $\cobar \schainsA(X)$ and show it compatible with the monoid structure.
In other words, Baues' lifted Adams' cobar construction $\cobar \schainsA$ so that it fits in the following diagram commuting up to a natural isomorphism:
\[
\begin{tikzcd}
\Mon_{\cSet} \arrow[r, "\cchainsA \ "] & \Mon_{\coAlg} \arrow[d] \\
\sSet^0 \arrow[r, "\cobar \schainsA"] \arrow[u, "\ccobar"] & \Mon_{\Ch}.
\end{tikzcd}
\]

Returning to our contribution, the first result presented here is the construction of a lift of $\cchainsA \ccobar$ to the category of monoids in $\coAlg_\UM$, where $\coAlg_\UM$ is the category of coalgebas over the operad $\UM$ introduced via a finitely presented prop in \cite{medina2020prop1}.
This is a model of the $E_\infty$-operad that explicitly acts on both simplicial and cubical chains.
For this lift to be meaningful, we first need to define a monoidal structure on $\coAlg_\UM$.
We do so by introducing a Hopf structure on $\UM$, that is to say, by making it into an operad over $\coAlg$, a structure that has other potential applications \cite{livernet2008hopf}.
Once we have shown that $\coAlg_\UM$ is a monoidal category, we prove the main technical result of this work, that the lift $\cchainsUM \colon \cSet \to \coAlg_\UM$ of $\cchainsA$ defined in \cite{medina2021cubical} is monoidal.

\anibal{continue here}

Generalizing Baues' first step in the construction of his lift of Adams' cobar construction, we have that $\cchainsUM(\ccobar X)$ is a monoid in $\coAlg_\UM$.
This is because $\ccobar X$ is a monoid in $\cSet$ and $\cchainsUM$ is shown to be a monoidal functor.
Then, for any reduced simplicial set $X$, we use the natural isomorphism $\cobar \schainsA(X) \to \cchains(\ccobar X)$ to transfer the constructed $E_{\infty}$-bialgebra structure on $\cchains(\ccobar X)$ to $\cobar \schainsA(X)$.
In particular, for any pointed space $(Z,z)$ Adams' comparison map $\theta_Z \colon \cobar \sSchainsA(Z,z) \to \cSchains(\loops_z Z)$ is a quasi-isomorphism of $E_{\infty}$-bialgebras.
We condense this discussion in our main result.

\begin{theorem} \label{t:1st main thm in the intro}
	The functor $\cchainsUM$ is monoidal and its associated functor on monoids fits in the following diagram commuting up to a natural isomorphism:
	\[
	\begin{tikzcd} [row sep=small]
	& \Mon_{\coAlg_\UM} \arrow[d] \\
	\Mon_{\cSet} \arrow[ru, "\cchainsUM", out=70, in=180, near start] \arrow[r, "\cchainsA"]
	& \Mon_{\coAlg} \arrow[d] \\
	\sSet^0 \arrow[r, "\cobar \schainsA"] \arrow[u, "\ccobar"]
	& \Mon_{\Ch}.
	\end{tikzcd}
	\]
Furthermore, for any pointed space $(Z,z)$, Adams' comparison map $\theta_Z \colon \cobar \sSchainsA(Z,z) \to \cSchains(\loops_z Z)$ is a quasi-isomorphism of $E_{\infty}$-bialgebras or, more specifically, of monoids in $\coAlg_\UM$.

\end{theorem}

We turn now to a second result in this work.
The homotopy theory of spaces can be modeled as the Kan-Quillen model category of simplicial sets.
We would also like to understand how the cobar construction and its algebraic structure fit in this combinatorial framework.
A model in this context for the based loop space of an arbitrary reduced simplicial set is given by the classical \textit{Kan loop group functor}.
This is a functor
\[
\kan \colon \sSet^0 \to \sGrp
\]
from reduced simplicial sets to the category of simplicial groups with the property that the geometric realization $|\kan(X)|$ is homotopy equivalent to the based loop space $\loops_b|X|$ as a topological monoid \cite{berger1995loops}.
Furthermore, the Kan loop group together with its right adjoint define a Quillen equivalence of model categories when $\sSet^0$ is equipped with the Kan-Quillen model category structure and $\sGrp$ with an appropriate model structure, having as weak equivalences maps of simplicial groups whose underlying maps of simplicial sets are weak homotopy equivalences.

In \cite{mcclure2003multivariable, berger2004combinatorial}, the functor of (normalized) chains $\schains$ was lifted to that of $E_{\infty}$-coalgebras, a result generalized in \cite{medina2020prop1} using a lift $\schainsUM \colon \sSet \to \coAlg_{\UM}$.
Hence, a natural problem is to compare the algebraic structures in $\cobar \schainsA(X)$ and $\schains_{\UM}(\kan(X))$.
In general, $\cobar \schainsA(X)$ and $\schains(\kan(X))$ are not quasi-isomorphic for an arbitrary reduced simplicial set $X$.
This can be seen from the fact that, if $X$ is not fibrant, $H_0(\cobar \schainsA(X))$ need not be a group ring, while $H_0(\kan(X))$ is always isomorphic to $ \k[\pi_1(|X|)]$.

K. Hess and A. Tonks proposed a way of modifying the cobar construction so that it could be compared to the chains on the Kan loop group \cite{hess2010cobar}.
They construct a localized version of the cobar construction, called \textit{extended cobar construction}, which we interpret as a functor
\[
\Ahat \colon \sSet^0 \to \Mon_\Ch
\]
extending the composition
\[
\A \defeq \cobar \schainsA \colon \sSet^0 \to \Mon_\Ch
\]
by formally inverting a basis of $0$-degree elements in $\mathcal{A}(X)$ determined by the $1$-simplices of $X$ for any reduced simplicial set $X$.
Hess and Tonks also describe a natural quasi-isomorphism of algebras
\[
\phi_X \colon \Ahat(X) \to \schains(\kan(X))
\]
in terms of a sequence of simplicial operators known as the \textit{Szczarba operators}.

Following Baues’ construction, Franz described in \cite{franz2020szczarba} a lift of $\Ahat$ to a functor
\[
\Ahat_{\coAlg} \colon \sSet^0 \to \Mon_{\coAlg}
\]
and showed that Hess--Tonks' comparison map preserves the coalgebra structure, namely, it induces a natural quasi-isomorphism of bialgebras
\[
\phi_X \colon \Ahat_{\coAlg}(X) \to \schains (\kan(X)).
\]

Using similar methods to those that lead to \cref{t:1st main thm in the intro}, we construct a further lift
\[
\Ahat_\UM \colon \sSet^0 \to \Mon_{\coAlg_\UM}.
\]
We may now try to compare the above functor with the Kan loop group construction via Hess and Tonk's comparison map.
However, there is a fundamental asymmetry between the interaction of the cubical chains and simplicial chains with the monoidal structures that we have to take into account.
For any two cubical sets $Y$ and $Y'$ the isomorphism
\[
\cchains(Y) \otimes \cchains(Y') \xrightarrow{\cong} \cchains(Y \otimes Y')
\]
preserves the $\UM$-coalgebra structures.
However, for any two simplicial sets $X$ and $X^\prime$, the Eilenberg-Zilber shuffle map
\[
\schains(X) \otimes \schains(X^\prime) \xrightarrow{\simeq} \schains(X \times X^\prime)
\]
is a quasi-isomorphism of coassociative coalgebras but not of $\UM$-coalgebras.
This last point has two important manifestations: 1) the $\UM$-coalgebra structure of $\schainsUM(\kan(X))$ is not (strictly) compatible with the algebra structure, i.e. it is not an $E_{\infty}$-bialgebra in our sense, and 2) the Hess and Tonks comparison map $\phi_X \colon \Ahat(X) \to \schains(\kan(X))$ does not preserve the $\UM$-coalgebra structures, namely, it is not compatible with the lifts $\Ahat_\UM \colon \sSet^0 \to \Mon_{\coAlg_\UM}$
and $\schainsUM \colon \sSet^0 \to \Mon_{\coAlg_\UM}$.
Nevertheless, we construct a zig-zag of natural quasi-isomorphisms of $E_{\infty}$-coalgebras connecting $\Ahat(X)$ and $\schains(\kan(X))$, for any reduced simplicial set $X$.
We summarize this discussion in the following statement.

\begin{theorem} \label{t:2nd main thm in the intro}
	There exists a functor $\Ahat_\UM \colon \sSet^0 \to \Mon_{\coAlg_\UM}$ that fits into a commutative diagram
	\[
	\begin{tikzcd}[row sep=small]
	& \Mon_{\coAlg_\UM} \arrow[d] \\
	& \Mon_{\coAlg} \arrow[d] \\
	\sSet^0
	\arrow[r, "\Ahat"', ]
	\arrow[ru, "\Ahat_{\!\As}"', out=45, in=180]
	\arrow[ruu, "\Ahat_\UM", out=90, in=180]
	& \Mon_{\Ch}.
	\end{tikzcd}
	\]
	Furthermore, for any reduced simplicial set $X$, there is a zig-zag of natural quasi-isomorphisms of $E_\infty$-coalgebras between $\Ahat_\USL(X)$ and $\schainsUSL(\kan(X))$, where $\USL \subseteq \UM$ is a sub-$E_\infty$-operad.
\end{theorem}

\subsection*{Relation to previous work}

Using more abstract methods, in \cite{hess2006adamshilton} Hess, Parent, Scott, and Tonks constructed a coproduct on $\cobar \schainsA(X)$ for any $1$-reduced simplicial set $X$, which coincides with Baues' coproduct.
A priori, their methods imply that this coproduct corresponds, up to homotopy, to the Alexander--Whitney coproduct on $\schains(\kan(X))$, a result that Franz showed to hold strictly in \cite{franz2020szczarba}.
Our approach allows us to go beyond the coalgebra structure and compare directly the cobar construction and the chains on the Kan loop group as $E_{\infty}$-coalgebras.

The works of V.A. Smirnov \cite{smirnov1990iterated}, J.R. Smith \cite{smith1994cobar, smith2000operads}, and Kadeishvili–Saneblidze \cite{kadeishvili1998iterating} predict the existence of an $E_\infty$-coalgebra structure on $\cobar \schainsA(X)$.
In \cite{fresse2010bar}, B. Fresse used a model category structure on reduced operads \cite{berger2003modelcategory, hinich1997homologicalalgebra} to iterate the bar construction on algebras over cofibrant $E_\infty$-operads.
We mention that the reduced version of the $E_\infty$-operad $\UM$ is cofibrant, although we do not use this observation here.
In the work of Fresse, finiteness assumptions and restrictions on the fundamental group are required to model the based loop space.

The results we have surveyed establish the existence of higher structures on the bar and cobar constructions associated to reduced simplicial sets.
We now review part of the history of the problem of describing such structures effectively, and our contribution to it.
Kadeishvili described explicitly Steenrod cup-$i$ coproducts on $\cobar \schainsA(X)$ compatible with its monoid structure \cite{kadeishvili1999coproducts, kadeishvili2003cupi}.
Kadeishvili, as Baues, used cubical methods to define these operations, and to compare them, in the $1$-connected setting, with cup-$i$ coproducts extending the Serre coalgebra structure on the cubical singular chains on the based loop space.
We can regard Kadeishvili's operations as part of the arity 2 action of our model $\UM$ for the $E_{\infty}$-operad.
Taking Kadeishvili's cup-$i$ construction as starting point, Fresse provided the bar construction of an algebra over the surjection operad with the structure of an algebra over the Barratt–Eccles operad \cite{fresse2003hopf}.
Fresse applies this construction to the surjection algebra structure of the cochains on a simplicial set $X$ and compares the resulting Baratt-Eccles algebra to the singular cochains on the based loop space, under restrictions on the fundamental group of $X$.
Our results in the first part of the present article are similar to Fresse's.
However, we make use of coalgebras instead of algebras, which allows us to relate our constructions to the based loop space in the general non-simply connected case.
Additionally, our approach is grounded on and further develops the cubical perspective at the heart of Adams' and Baues' seminal papers.

\subsection*{Outline}

In \cref{s:preliminaries}, we recall well known algebraic and categorical definitions and constructions that will be used throughout this article.
We review, in \cref{s:operads and props}, the foundations of the theory of operads and props with the goal of recalling the $E_{\infty}$-operad $\UM$ and its actions on simplicial and cubical chains.
We construct a Hopf operad structure on $\UM$ in \cref{s:monoidal}, and show that the cubical chains functor from cubical sets to $\UM$-coalgeras is monoidal.
We prove \cref{t:1st main thm in the intro} in \cref{s:theorem1} using Baues' cubical cobar construction \eqref{e:cubical cobar construction} and the $\UM$-structure on cubical chains.
Finally, in \cref{s:theorem2}, we review some facts and constructions from homotopy theory and use them to prove \cref{t:2nd main thm in the intro}.