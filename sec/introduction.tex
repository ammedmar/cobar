% !TEX root = ../cobar1.tex

\section{Introduction}

For any topological space $\fX$, its complex of simplicial or cubical singular chains $\Schains(\fX)$ -- regarded as a differential graded (dg) abelian group -- encodes the homology of $\fX$ in its quasi-isomorphism type.
More homotopical information can be stored in the quasi-isomorphism type of this chain complex if considered as a coassociative coalgebra, which we will denote $\SchainsA(\fX)$, where the coproduct comes from a natural choice of chain approximation to the diagonal $\fX \to \fX \times \fX$.
For instance, the cohomology ring of $\fX$ is retained, but the action of the Steenrod algebra on its mod~$p$ cohomology is not.

In Mandell's seminal work \cite{mandell2006homotopy_type} it is shown that, when $\fX$ is nilpotent and finite type, the entire homotopy type of $\fX$ can be encoded in the quasi-isomorphism type of this complex if considered as an $E_\infty$-coalgebra, a structure providing $\SchainsA(\fX)$ with coherent homotopies witnessing the derived cocommutativity of the coproduct coming from the strict symmetry of the diagonal map.

The first contribution of this paper is to explicitly endow the cubical singular chains of the based loop space $\loops_x \fX$, with the structure of a monoidal $E_\infty$-coalgebra extending the Serre diagonal.
More specifically, we verify that the monoid structure induced on $\cSchains(\loops_x \fX)$ by the concatenation of loops is compatible with a natural $E_\infty$-coalgebra structure on cubical singular chains, similar to the one defined in \cite{medina2022cube_einfty}.

Applying Adams' cobar construction to the coalgebra of simplicial singular chains of $(\fX, x)$, one obtains another monoidal algebraic model $\cobar \sSchainsA(\fX, x)$ of $\loops_x \fX$ \cite{adams1956cobar}.
More precisely, Adams constructed a natural monoidal chain map $\theta$ from $\cobar \sSchainsA(\fX, x)$ to $\cSchains(\loops_x \fX)$ and proved it to be a quasi-isomorphism if $\fX$ is simply-connected, a statement that also holds true for path-connected spaces after \cite{rivera2018cubical}.
The model $\cobar \sSchainsA(\fX, x)$ is smaller than $\cSchains(\loops_x \fX)$ and unlocks effective analysis of quantitative and qualitative properties of $\loops_x \fX$, as illustrated for instance in \cite{chainalgebraloops} and \cite{adamscobarequivalence}.

The second main contribution of this paper is to make Adams model into a monoidal $E_\infty$-coalgebra and to prove that
\[
\theta \colon \cobar \sSchainsA(\fX, x) \to \cSchains(\loops_x \fX)
\]
respects this higher structure.
Although not pursued in the present article, we remark that the explicit nature of our $E_\infty$-extension invites the study of primary and secondary operations for loops spaces using Adams' model and the tools developed in \cite{medina2021may_st}, \cite{medina2020cartan}, \cite{medina2021adem}, and \cite{medina2021comch}.

Our starting point is groundbreaking work by Baues, which imply statements similar to those in this work but in the category of (coassociative) coalgebras. Baues reinterpreted Adams' algebraic construction at a deeper geometric level \cite{baues1998hopf}, which allowed him to endow $\cobar \sSchainsA(\fX, x)$ with the structure of a monoidal coalgebra, and to show that $\theta$ preserves this structure.
To prove our statement we interpret Adams' construction at an even deeper categorical level.
%The simplicial singular complex $\sSing(\fX, x)$ is an example of an object in the category $\sSet^0$ of $0$-reduced simplicial sets.
We interpret Baues' geometric cobar construction, originally defined for $1$-reduced simplicial sets, as a functor
\begin{equation*}
	\ccobar \colon \sSet^0 \to \Mon_{\cSet},
\end{equation*}
from the category of $0$-reduced simplicial sets to that of monoidal cubical sets.
The key difference with Baues' original work is the use of connections to obtain a natural construction before geometric realization.

Additionally, we need a suitable model of the $E_\infty$-operad endowing cubical chains with a natural $E_\infty$-coalgebra extending the Serre diagonal.
For this we take the operad $\UM$ introduced in \cite{medina2020prop1}.
After proving that its coalgebras form a monoidal category, we show that the functor $\cchainsUM \colon \cSet \to \coAlg_\UM$ -- defined in \cite{medina2022cube_einfty} with a different sign convention -- is monoidal.
This allows us to construct the following extension of Adams and Baues' structures.
%This functors is equal to that introduced in \cite{medina2022cube_einfty} up to signs.

%that $\cchainsUM$ from cubical sets to $\UM$-coalgebras is monoidal, which implies that $\cchainsUM(\ccobar X)$ is a monoidal $\UM$-coalgebra for any 0-reduced simplicial set.
%More precisely, we prove the following.

\begin{theorem*}
	The following diagram commutes up to natural isomorphisms:
	\[
	\begin{tikzcd} [row sep=small]
		& \Mon_{\coAlg_\UM} \arrow[d] \\
		\Mon_{\cSet} \arrow[ru, "\cchainsUM", out=70, in=180, near start] \arrow[r, "\cchainsA"]
		& \Mon_{\coAlg} \arrow[d] \\
		\sSet^0 \arrow[r, "\cobar \schainsA"] \arrow[u, "\ccobar"]
		& \Mon_{\Ch},
	\end{tikzcd}
	\]
	where the unlabeled arrows are forgetful functors.
\end{theorem*}

In the diagram of the above theorem, the arrow from $\sSet^0$ to $\Mon_{\Ch}$ is Adams' cobar construction, the one from $\sSet^0$ to $\Mon_{\coAlg}$ is Baues' enhancement, and the one from $\sSet^0$ to $\Mon_{\coAlg_{\UM}}$ is our lift.
Additionally, we prove the following statement about Adams's map.


\begin{theorem*}
	For any pointed space $(\fX, x)$,
	\[
	\theta \colon \cobar \sSchainsA(\fX, x) \to \cSchains(\loops_x \fX)
	\]
	is a quasi-isomorphism of monoidal $\UM$-coalgebras.
\end{theorem*}
 
The fact that $\theta$ respects the monoid structure in $\Ch$ was proven by Adams, whereas the compatibility of the monoid structure with the Serre coalgebra structure was established by Baues. 
Our contribution is the compatibility of the monoid structure with a full $E_\infty$-coalgebra extension of Serre's coalgebra.
We also remark that, whereas both Adams and Baues worked in the setting where the underlying space is simply connected, the above theorem does not require any connectivity or finiteness hypotheses. %When $\fX$ is a $2$-connected space of finite type, Mandell's result implies that the homotopy type of the based loop space is completely encoded on the quasi-isomorphism type of the $E_{\infty}$-coalgebra structure on the cobar construction. A version of this result is expected to hold in  greater generality by considering $E_{\infty}$-coalgebras under a stronger notion of weak equivalence drawn from Koszul duality theory, see \cite{RWZ22}. 

% In \cite{fresse2010bar}, abstract homotopy theoretic tools are used to show that the bar construction of the underlying $E_1$-algebra of an arbitrary $E_{\infty}$-algebra carries an induced $E_{\infty}$-algebra structure.
% No compatibility between this structure and the comonid one is established there, and the relationship with the based loop space is accomplished only under certain connectivity, finiteness, and completeness assumptions.

\subsection*{Related work}

Kadeishvili \cite{kadeishvili1999coproducts, kadeishvili2003cupi} explicitly described monoidal cup-$i$ coproducts on $\cobar \schainsA(X)$ extending Baues coalgebra.
Kadeishvili, as Baues, used cubical methods to define these coproducts and to compare them, in the $1$-connected setting, to cup-$i$ coproducts extending the Serre coalgebra structure on the cubical singular chains of the based loop space.
Additionally, there are several papers \cite{smirnov1990iterated, smith1994cobar, smith2000operads, kadeishvili1998iterating} that predict the existence of, but do not construct, an $E_\infty$-structure on the cobar construction on the chains of simply connected simplicial sets.

On the dual side, Fresse \cite{fresse2003hopf} provided the bar construction of an algebra over the surjection operad with the structure of a comonoid in the category of algebras over the Barratt--Eccles operad.
Additionally, in \cite{fresse2010bar} he used a model category structure on reduced operads \cite{berger2003modelcategory, hinich1997homologicalalgebra} to iterate the bar construction on algebras over cofibrant $E_\infty$-operads.

The use of coalgebras instead of algebras allows us to relate the cobar construction to the based loop space directly --via the Adams map-- without imposing restrictions on the underlying homotopy type, as done by Fresse.
Furthermore, by grounding our approach on the cubical perspective at the heart of Adams' and Baues' seminal papers, we are able to preserve the natural monoidal structures when defining our $E_\infty$-enhancements.
