% !TEX root = ../cobar1.tex

\section{Adams's model of based loop spaces} \label{s:theorem1}

%The goal of this section is to prove \cref{t:1st main thm in the intro}.
%Explicitly, we construct a natural $E_{\infty}$-bialgebra structure on $\cobar \schainsA(X)$ for any reduced simplicial set $X$ (\cref{l:lift of cobar to e-infty}); and show that when $X$ is the singular complex of a topological space $(Z,z)$, Adams' comparison map $\theta_Z \colon \cobar \sSchainsA(Z,z) \to \cSchains(\loops_z Z)$ is a quasi-isomorphism of $E_\infty$-bialgebras (\cref{l:adams comparison is an e-infty bialgebra map}).
%
%The first goal is achieved using a categorical reformulation of Baues' geometric cobar construction (\cref{ss:cubical cobar}) together with the natural $\UM$-coalgebra structure defined on cubical chains (\cref{ss:e-infty on cubical}) and its monoidal properties (\cref{l:cubical e-infty chains are monoidal}).
%The second goal is obtained by factoring Adams' comparison map through this cubical model (\cref{ss:factorization of adams}).

\subsection{The cobar construction} \label{ss:cobar construction}

A \textit{coaugmentation} on a coalgebra $C$ is a morphism $\nu \colon \k \to C$ in $\coAlg$.
A coalgebra is said to be \textit{coaugmented} if it is equipped with a coaugmentation.

We denote by $\coAlg^\ast$ the category of coaugmented coalgebras with morphisms being coaugmentation preserving coalgebra morphisms.

The \textit{cobar construction} is the functor
\[
\cobar \colon \coAlg^\ast \to \Mon_{\Ch}
\]
defined on objects as follows.
Let $(C, \partial, \Delta, \varepsilon, \nu)$ be the data of a coaugmented coalgebra.
Denote by $\overline{C}$ the cokernel of $\nu \colon \k \to C$ and recall that $s^{-1}$ is the $(-1)^\th$ suspension.
The cobar construction $\cobar C$ of this coaugmented coalgebra is the graded module
\[
T(s^{-1} \overline{C}) = \k \oplus s^{-1}\overline{C} \oplus (s^{-1}\overline{C})^{\otimes 2} \oplus (s^{-1}\overline{C})^{\otimes 3} \oplus \dots
\]
regarded as a monoid in $\Ch$ with $\mu \colon T(s^{-1} \overline{C})^{\otimes 2} \to T(s^{-1} \overline{C})$ given by concatenation, $\eta \colon \k \to T(s^{-1} \overline{C})$ by the obvious inclusion, and differential constructed by extending the linear map
\[
- s^{-1} \circ \partial \circ s^{+1} \ + \ (s^{-1} \otimes s^{-1}) \circ \Delta \circ s^{+1} \colon
s^{-1} \overline{C} \to s^{-1}\overline{C} \oplus (s^{-1}\overline{C} \otimes s^{-1}\overline{C}) \hookrightarrow T(s^{-1}C)
\]
as a derivation.

In this article we are mostly concerned with \textit{connected} coalgebras, namely coaugmented coalgebras that are non-negatively graded and for which the coaugmentation induces an isomorphism $\k \cong C_0$ of $\k$-modules.
We denote by $\coAlg^0$ the full subcategory of $\coAlg^\ast$ consisting of connected coalgebras, and remark that if $C \in \coAlg^0$, then $ \cobar(C)$ is concentrated on non-negative degrees.

The object $\cobar(C) \in \Mon_{\Ch}$ may be equipped with a natural augmentation map $\cobar(C) \to \k$ in $\Mon_{\Ch}$ given by the projection map to the first summand term.
Therefore, the cobar construction may be regarded as a functor from coaugmented coalgebras to augmented algebras.
However, we will not require the data of augmentations in this article.

%We now review the central examples of coaugmented coalgebras to which Adams applied the cobar construction.

\subsection{Adams' map} \label{ss:adams maps}

In \cite{adams1956cobar}, Adams constructed a natural map of algebras
\begin{equation} \label{e:adams map 2}
	\theta_Z \colon \cobar \sSchainsA(Z,z) \to \cSchains(\loops_z Z),
\end{equation}
for any pointed topological space $(Z,z)$.
The construction is based on a collection of continuous maps
\[
\theta_n \colon \gcube^{n-1} \to P(\gsimplex^n;0,n),
\]
where $P(\gsimplex^n;0,n)$ denotes the topological space of Moore paths in $\gsimplex^n$ from the $0^\th$ vertex $v_0$ to the $n^\th$ vertex $v_n$.
These maps are constructed inductively so that following equations are satisfied:
\begin{enumerate}
	\item $\theta_1(0)\colon \gcube^1 \to \gsimplex^1$ is the path $\theta_1(0)(s) = sv_1 +(1-s)v_0$.
	\item $\theta_n \circ e_j^0 = P(d_j) \circ \theta_{n-1}$.
	\item $\theta_n \circ e_j^1 = (P(f_j) \circ \theta_j) \cdot (P(l_{n-j}) \circ \theta_{n-j})$.
\end{enumerate}
In the above equations $f_j\colon \gsimplex^j \rightarrow \gsimplex^n$ and $l_{n-j}\colon \gsimplex^{n-j} \rightarrow \gsimplex^n$ denote the first $j^\th$ face map and last $(n-j)^\th$ face map of $\gsimplex^n$, respectively, and
$e_j^0,e_j^1\colon \gcube^{n-1} \rightarrow \gcube^{n}$ denote the $j^\th$ bottom and top cubical face maps, respectively.
For any continuous map of spaces $f \colon Z \to Z'$, we denote by $P(f) \colon P(Z) \to P(Z')$ the induced map at the level of spaces of paths.
Given any two composable paths $\alpha$ and $\beta$, the dot symbol in $\alpha \cdot \beta$ denotes the composition (or concatenation) of paths.

Adams' map $\theta_Z$ is now given as follows.
For any singular $1$-simplex $\sigma \in S^{\Delta}_1(Z,z)$ let
\[
\theta_Z([\sigma]) = P(\sigma) \circ \theta_1 - c_z,
\]
where $c_z \in \cSchains_0(\loops_z Z)$ is the singular $0$-cube determined by the constant loop at $z \in Z$.
For any singular $n$-simplex $\sigma \in S^{\Delta}_n(Z,z)$ with $n>1$, let
\[
\theta_Z([\sigma]) = P(\sigma) \circ \theta_n.
\]
Since the underlying graded algebra of $\cobar \sSchainsA(Z,z)$ is free, we may extend the above to an algebra map $\theta_Z \colon \cobar \sSchainsA(Z,z) \to \cSchains(\loops_z Z)$.
The compatibility equations for the maps $\theta_n$ imply that $\theta_Z$ is a chain map.

We will construct an $E_{\infty}$-bialgebra structure on $\cobar \sSchainsA(Z,z)$ and show that, when $\cSchains(\loops_z Z)$ is equipped with the $E_{\infty}$-coalgebra structure described in \cref{ss:e-infty on cubical}, the map $\theta_Z$ is a map of $E_{\infty}$-coalgebras.
With this goal in mind, we describe a categorical version of Adams' constructions.

