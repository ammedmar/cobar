
\section{Adams's map as an \texorpdfstring{$E_{\infty}$}{}-bialgebra quasi-isomorphism} \label{s:theorem1}

The goal of this section is to prove \cref{t:1st main thm in the intro}.
Explicitly, we construct a natural $E_{\infty}$-bialgebra structure on $\cobar \schains(X)$ for any reduced simplicial set $X$ (\cref{l:lift of cobar to e-infty}); and show that when $X$ is the singular complex of a topological space $(Z, z)$, Adams' comparison map $\theta_Z \colon \cobar \sS(Z, z) \to S^{\cube}(\loops_z Z)$ is a quasi-isomorphism of $E_\infty$-bialgebras (\cref{l:adams comparison is an e-infty bialgebra map}).

The first goal is achieved using a categorical reformulation of Baues' geometric cobar construction (\cref{ss:cubical cobar}) together with the natural $\UM$-coalgebra structure defined on cubical chains (\cref{ss:e-infty on cubical}) and its monoidal properties (\cref{l:cubical e-infty chains are monoidal}).
The second goal is obtained by factoring Adams' comparison map through this cubical model (\cref{ss:factorization of adams}).

\subsection{Adams' map}\label{adamsmaps}

In \cite{adams1956cobar}, Adams constructed a natural map of algebras
\begin{equation} \label{e:adams map 2}
\theta_Z \colon \cobar \sS(Z, z) \to S^{\cube}(\loops_z Z),
\end{equation}
for any pointed topological space $(Z, z)$.
The construction is based on a collection of continuous maps 
\[
\theta_n \colon \gcube^{n-1} \to P(\gsimplex^n;0,n),
\]
where $P(\gsimplex^n;0,n)$ denotes the topological space of Moore paths in $\gsimplex^n$ from the $0$-th vertex $v_0$ to the $n$-th vertex $v_n$.
These maps are constructed inductively so that following equations are satisfied:
\begin{enumerate}
	\item $\theta_1(0)\colon \gcube^1 \to \gsimplex^1$ is the path $\theta_1(0)(s) = sv_1 +(1-s)v_0$.
	\item $\theta_n \circ e_j^0 = P(d_j) \circ \theta_{n-1}$.
	\item $\theta_n \circ e_j^1 = (P(f_j) \circ \theta_j) \cdot (P(l_{n-j}) \circ \theta_{n-j})$.
\end{enumerate}
In the above equations $f_j\colon \gsimplex^j \rightarrow \gsimplex^n$ and $l_{n-j}\colon \gsimplex^{n-j} \rightarrow \gsimplex^n$ denote the first $j$-th face map and last $(n-j)$-th face map of $\gsimplex^n$, respectively, and
$e_j^0,e_j^1\colon \gcube^{n-1} \rightarrow \gcube^{n}$ denote the $j$-th bottom and top cubical face maps, respectively.
For any continuous map of spaces $f \colon Z \to Z'$, we denote by $P(f) \colon P(Z) \to P(Z')$ the induced map at the level of spaces of paths.
Given any two composable paths $\alpha$ and $\beta$, the dot symbol in $\alpha \cdot \beta$ denotes the composition (or concatenation) of paths.

Adams' map $\theta_Z$ is now given as follows.
For any singular $1$-simplex $\sigma \in S^{\Delta}_1(Z, z)$ let
\[
\theta_Z([\sigma]) = P(\sigma) \circ \theta_1 - c_z,
\]
where $c_z \in S^{\cube}_0(\loops_z Z)$ is the singular $0$-cube determined by the constant loop at $z \in Z$.
For any singular $n$-simplex $\sigma \in S^{\Delta}_n(Z, z)$ with $n>1$, let
\[
\theta_Z([\sigma]) = P(\sigma) \circ \theta_n.
\]
Since the underlying graded algebra of $\cobar \sS(Z, z)$ is free, we may extend the above to an algebra map $\theta_Z \colon \cobar \sS(Z, z) \to S^{\cube}(\loops_z Z)$.
The compatibility equations for the maps $\theta_n$ imply that $\theta_Z$ is a chain map.

We will construct an $E_{\infty}$-bialgebra structure on $\cobar \sS(Z, z)$ and show that, when $S^{\cube}(\loops_z Z)$ is equipped with the $E_{\infty}$-coalgebra structure described in \cref{ss:e-infty on cubical}, the map $\theta_Z$ is a map of $E_{\infty}$-coalgebras. With this goal in mind, we  describe a categorical version of Adams' constructions.

\subsection{Necklical sets}

We construct a cubical model for the based loop space using necklical sets, a notion related to both simplicial sets and cubical sets.
Other approaches to this framework can be found in \cite{baues1998hopf}, \cite{galvez2020hopf}, \cite{dugger2011rigidification}, and \cite{rivera2018cubical, rivera2019path}.

Let us first consider the subcategory $\simplex_{*,*}$ of the simplex category $\simplex$ with the same objects and morphisms given by functors $f \colon [n] \to [m]$ satisfying $f(0) = 0$ and $f(n) = m$.
It is a strict monoidal category when equipped with the monoidal structure $[n] \otimes [m] = [n+m]$, thought of as identifying the elements $n \in [n]$ and $0 \in [m]$, and unit given by $[0]$.
In passing, we mention that this category is $\op$-dual to the augmented simplex category, a form of Joyal duality.

The \textit{necklace category} $\Nec$ is obtained from $\simplex_{*,*}$ as follows.
Thinking of $\simplex_{*,*}$ as a monoid in $\Cat$, we first apply the bar construction to it and produce a simplicial object in $\Mon_{\Cat}$ which, after realization, defines the strict monoidal category $\Nec$.
We denote the monoidal structure by 
\[
\vee \colon \Nec \times \Nec \to \Nec.
\]
We describe $(\Nec, \vee)$ in more explicit terms.
The objects of $\Nec$, called \textit{necklaces}, are freely generated by the objects of $\simplex_{*,*}$ through the monoidal structure $\vee$.
Namely, the set of objects of $\Nec$ is given by
\[
\big\{ [n_1] \vee \dots \vee[n_k] \mid n_i, k \in \N_{>0} \big\}
\]
together with $[0]$ serving as the monoidal unit.

The morphism of $\Nec$ are generated through the monoidal structure by the following four types of morphisms:
\begin{enumerate}
	\item $\partial^j \colon [n-1] \to [n]$ for $j = 1, \dots, n-1$
	\item $\Delta_{[j], [n-j]} \colon  [j] \vee [n-j] \to [n]$ for $j = 1, \dots, n-1$
	\item $s^j \colon [n+1] \to [n]$ for $j = 0, \dots, n$ and $n>0$, and 
	\item $s^0 \colon [1] \to [0]$.
\end{enumerate}
We may identify $\Nec$ with a full sub-category of the category of double pointed simplicial sets $\sSet_{*,*}$ as follows.
Consider the functor
\[
\mathcal{S} \colon \Nec \to \sSet_{*,*}
\]
 induced by sending any necklace $T = [n_1] \vee \dots \vee[n_k]$ to the simplicial set
\[
\mathcal{S}(T) = \simplex^{n_1} \vee \dots \vee \simplex^{n_k},
\]
where the wedge symbol now means we identify the last vertex of $\simplex^{n_i}$ with the first vertex of $\simplex^{n_{i+1}}$ for $i = 1, \dots, k-1$ and the two base points are given by the first vertex of $\simplex^{n_1}$ and the last vertex of $\simplex^{n_k}$.
Then $\mathcal{S}$ is a fully faithful functor, so $\Nec$ may be identified with the full sub-category of $\sSet_{*,*}$ having as objects those double pointed simplicial sets of the form $\simplex^{n_1} \vee \dots \vee \simplex^{n_k}$.
The \textit{dimension} of a necklace $T = [n_1] \vee \dots \vee[n_k]$ is defined by $\dim(T) = n_1 + \dots + n_k-k$.

The category of \textit{necklical sets} is the functor category $\nSet = \Fun[\Nec^{\op}, \Set]$.
The \textit{Yoneda embedding}
\[
\mathcal{Y} \colon \Nec \to \nSet,
\]
is the functor induced by by $T \mapsto \Nec(-,T)$.

The category of necklical sets $\nSet$ is a (non-symmetric) monoidal category when equipped with the Day convolution structure $\otimes \colon \nSet \times \nSet \to \nSet$ (\cref{ss:day convolution}) given by
\[
K \otimes K' \, = \colim_{\substack{\mathcal{Y}(T) \to K, \\ \mathcal{Y}(T')\to K'}} \mathcal{Y}(T \vee T').
\]

\subsection{From necklaces to cubes}

We now describe a categorical version of Adams' maps
\[
\theta_n \colon \gcube^{n-1} \to P(\gsimplex^{n};0,n)
\]
by constructing a monoidal functor
\[
\mathcal{P} \colon \Nec \to \cube.
\]

We first set $\mathcal{P}([0])=2^0$.
On any other necklace $T \in \Nec$ we define $\mathcal{P}( T )= 2^{\dim(T)}$.
In order to define $\mathcal{P}$ on morphisms, it is sufficient to consider the following cases.
\begin{enumerate}
	\item For any $\partial^j \colon [n] \to [n+1]$ such that $0< j<{n+1}$, define $\mathcal{P}(\partial^j) \colon 2^{n-1}\to 2^{n}$ to be the cubical coface functor $\mathcal{P}(f)= \delta_0^{j}$.
	
	\item For any $\Delta_{[j], [n+1-j]} \colon [j] \vee [n+1-j] \to [n+1]$ such that $0<j<n+1$, define $\mathcal{P}(\Delta_{[j], [n+1-j]}) \colon 2^{n-1}\to 2^{n}$ to be the cubical coface functor $\mathcal{P}(f)=\delta_1^{j}$.
	
	\item We now consider morphisms of the form $s^j \colon [n+1] \to [n]$ for $n>0$.
	If $j=0$ or $j=n$, then $\mathcal{P}(f) \colon 2^n \to 2^{n-1}$ is defined to be the cubical codegeneracy functor $\mathcal{P}(s^j)= \varepsilon^{j}$.
	If $0<j<n$, then we define $\mathcal{P}(s^j) \colon 2^n \to 2^{n-1}$ to be the cubical coconnection functor $\gamma^{j}$.
	
	\item For $s^0 \colon [1] \to [0]$ we define $\mathcal{P}(s^0) \colon 2^0 \to 2^0$ to be the identity functor.
\end{enumerate}

\begin{remark}
	The functor $\mathcal{P}$ is neither faithful or full.
	However, for any necklace $T' \in \Nec$ with $\dim(T')=n+1$ and any cubical coface functor $\delta_{\epsilon}^j \colon 2^n \to 2^{n+1}$ for $0 \leq j \leq n+1$, there exists a unique pair $(T, f \colon T \hookrightarrow T')$, where $T \in \Nec$ with $\dim(T)=n$ and $f \colon T \hookrightarrow T'$ is an injective morphism in $\Nec$, such that $\mathcal{P}(f) = \delta_{\epsilon}^j $.
\end{remark}

The functor $\mathcal{P} \colon \Nec \to \cube$ induces an adjunction between $\cSet$ and $\nSet$ with right and left adjoint functors given respectively by
\[
\mathcal{P}^\ast \colon \cSet \to \nSet,
\qquad \text{and} \qquad
\mathcal{P}_{!}  \colon \nSet \to \cSet.
\]
Explicitly, for a cubical set $Y \colon \cube^\op \to \Set$,
\[
\mathcal{P}^\ast(Y)= Y \circ \mathcal{P}^\op,
\]
and for a necklical set $K \colon \Nec^\op \to \Set$,
\[
\mathcal{P}_{!}(K) \ =
\colim_{\mathcal{Y}(T) \to K} \mathcal{P}(T) \ \cong 
\colim_{\mathcal{Y}(T) \to K} \cube^{\dim(T)}.
\]
Since $\mathcal{P}$ is a monoidal functor, the functor $\mathcal{P}_{!} \colon \nSet \to \cSet$ is a monoidal as well.

\subsection{Cubical cobar construction} \label{ss:cubical cobar}

Using the framework of necklical sets, we may reinterpret a Baues' geometric cobar construction from \cite{baues1998hopf} as a functor
\[
\ncobar \colon \sSet^0 \to \Mon_{\nSet},
\]
which we now define.

For any reduced simplicial set $X$, we define a  necklical set $\ncobar(X) \colon \Nec^\op \to \Set$ having as necklical cells all necklaces inside $X$; namely
\[
\ncobar(X) \, = \! \colim_{\mathcal{S}(T) \to X} \mathcal{Y}(T).
\]
The product 
\[
\ncobar(X) \otimes \ncobar(X) \to \ncobar(X)
\]
is induced by the monoidal structure $\vee \colon \Nec \times \Nec \to \Nec$ given by concatenation of necklaces.
More precisely, using that
\[
\mathcal{Y}(T) \cong \colim_{(R \to T) \in \Nec \downarrow T} \mathcal{Y}(R),
\] 
we may describe the product
\[
\ncobar(X) \otimes \ncobar(X) \to \ncobar(X)
\]
by
\[
[f\colon \mathcal{S}(T) \to X, R \to T] \otimes [f'\colon \mathcal{S}(T') \to X, R' \to T'] \mapsto [f \vee g\colon \mathcal{S}(T\vee T') \to X, R \vee  R'\to T \vee T'],
\]
where $(f\colon \mathcal{S}(T) \to X), (f'\colon \mathcal{S}(T') \to X) \in \mathcal{S} \downarrow X$, $ (R\to T) \in \Nec \downarrow T$, and $(R' \to T') \in \Nec \downarrow T'$. The bracket notation $[f\colon \mathcal{S}(T) \to X, R\to T]$ denotes the equivalence class of a pair $(f \colon \mathcal{S}(T) \to X, R \to T)$ in the colimit defining $\ncobar(X)$.

We may now define the \textit{cubical cobar construction}
\[
\ccobar \colon \sSet^0 \to \Mon_{\cSet}
\]
as the composition 
\[
\ccobar = \mathcal{P}_! \, \ncobar.
\]
This reinterpretation of Baues construction in terms of cubical sets was also studied in \cite{rivera2018cubical}.

\subsection{Relation to the cobar construction}

The cubical cobar functor $\ccobar \colon \sSet^0 \to \Mon_{\cSet}$ is related to the cobar construction $\cobar \colon \coAlg \to \Mon_{\Ch}$ (\cref{ss:cobar construction}) as follows.

\begin{lemma} \label{l:ccobar and cobar}
	There is a natural isomorphisms of functors 
	\[
	\cchains \ccobar \cong \cobar \schains \colon \sSet^0 \to \Mon_{\Ch}.
	\]
\end{lemma}

\begin{proof} 
	Denote by $\iota_n \in (\cube^n)_n$ the top dimensional non-degenerate element of the standard $n$-cube $\cube^n$.
	Note that for a reduced simplicial set $X$, we may represent any non-degenerate $n$-cube $\alpha \in (\mathcal{P}_!(\ncobar(X)))_n$ as a pair $\alpha = [\sigma \colon \mathcal{Y}(T) \to X, \iota_n]$ for some $T = [n_1] \vee \dots \vee [n_k] \in \Nec$ with $\dim(T) = n_1 + \dots + n_k - k = n$.
	
	To define an algebra map
	\[
	\varphi_X \colon \cchains(\mathcal{P}_!(\ncobar(X))) \xra{\cong} \cobar \schains(X)
	\]
	it suffices to define it on any generator of the form $\alpha=[\sigma \colon \simplex^{n+1} \to X, \iota_{n}]$, i.e., when $T$ is of the form $T = [n+1]$, for some $n \geq 0$.
	If $n = 0$ let $\varphi_X(\alpha)= [\overline{\sigma}] + 1_\k$, where $[\overline{\sigma}] \in s^{-1} \overline{ \schains(X)} \subset \cobar \schains(X)$ denotes the (length $1$) generator in the cobar construction of $\schains(X)$ determined by $\sigma \in X_{n+1}$ and $1_\k$ denotes the unit of the underlying ring $\k$.
	If $n > 0$, we let $\varphi_X(\alpha)=[\overline{\sigma}]$.
	A straightforward computation yields that this gives rise to a well defined isomorphism of algebras, which is compatible with the differentials, and natural with respect to maps of simplicial sets.
\end{proof}

\subsection{An $E_{\infty}$-bialgebra structure on the cobar construction} \label{ss:e-infty on cobar}

The original cobar construction of Adams produces monoids in $\Ch$.
Baues enriched this construction to produce monoids in $\coAlg$.
He did this by using a version of the isomorphism $\cchains \ccobar \cong \cobar \schains$ of \cref{l:ccobar and cobar} and the Serre coalgebra on cubical chains.
Since $\ccobar$ produces monoids in $\cSet$ and $\cchainsAs$ is monoidal, the functor $\cchainsAs \ccobar$ produces bialgebras, as claimed.

After \cite{medina2021cubical}, we know that cubical chains can be enriched with an $E_\infty$-coalgebra structure extending the Serre coalgebra (\cref{ss:e-infty on cubical}).
This $\UM$-structure can be transferred via $\cchains \ccobar \cong \cobar \schains$ to provide the cobar construction with and $E_\infty$-structure extending Baues coalgebra structure.
Furthermore, thanks to \cref{l:cubical e-infty chains are monoidal}, we know that $\cchainsUM$ is monoidal and therefore, $\cchains \ccobar$ produces monoids in $\coAlg_{\UM}$, i.e., $E_\infty$-bialgebras.
This discussion proves the first half of \cref{t:1st main thm in the intro}, which we collect in the following.

\begin{lemma} \label{l:lift of cobar to e-infty}
	The functor $\cchainsUM$ is monoidal and fits in the following diagram commuting up to a natural isomorphism:
	\[
	\begin{tikzcd}
	& \Mon_{\coAlg_\UM} \arrow[d] \\
	\Mon_{\cSet} \arrow[ru, "\cchainsUM", out=70, in=180] \arrow[r, "\cchainsAs"]
	& \Mon_{\coAlg} \arrow[d] \\
	\sSet^0 \arrow[r, "\cobar \schains"] \arrow[u, "\ccobar"]
	& \Mon_{\Ch}.
	\end{tikzcd}
	\]	
\end{lemma}

\subsection{Steenrod operations}

Throughout this subsection let us consider $\k$ to be the finite field $\Fp$ and $X$ a reduced simplicial set.

Since $\cobar \schains(X)$ is isomorphic to the chains on a cubical set by \cref{l:ccobar and cobar}, its mod $p$ homology is a priori equipped with a right action of the Steenrod algebra \cite{steenrod1962cohomology, milnor1958steenrod}.
As an application of the explicit construction of a natural $\UM$-coalgebra structure on $\cobar \schains(X)$, we can describe at the chain level this action using the constructions of \cite{medina2020maysteenrod}.
More explicitly, let $\Cp$ be the cyclic group of order $p$ and
\[
\begin{tikzcd} [column sep = .5cm]
\mathcal W(p) = \Fp[\Cp]\{e_0\} & \arrow[l, "\,T"'] \Fp[\Cp]\{e_1\} & \arrow[l, "\,N"'] \Fp[\Cp]\{e_2\} & \arrow[l, "\,T"'] \dots
\end{tikzcd}
\]
be the minimal free resolution of $\Fp$ as an $\Fp[\Cp]$-module.
Consider the $\Cp$-equivariant quasi-isomorphism $\psi_\UM \colon W(p) \to \UM(p)$ constructed in \cite{medina2020maysteenrod}, and denote by $\Psi^{\cobar}$ its composition with the morphism $\UM \to \End^{\cobar \schains(X)}$ defining our $E_\infty$-coalgebra structure on $\cobar \schains(X)$.

For $p = 2$, the Steenrod square $P_s$ is represented by the map sending a representative $\mu$ of a class in the $k\th$ homology of $\cobar \schains(X)$ to the following element in its double linear dual:
\[
\mu \mapsto 
\Big( \alpha \mapsto (\alpha \otimes \alpha) \big( \psi^{\cobar}(e_{k+2s})(\mu) \big) \Big).
\]
Similarly, for $p$ an odd prime, we have that the action of the Steenrod operations $\beta^\varepsilon P_{s}$ where $\varepsilon = 0,1$ are represented by
\[
\mu \mapsto
\Big( \alpha \mapsto \alpha^{\otimes p} \big( (-1)^p \, \nu(q) \, \psi^{\cobar}(e_{(2s-q)(p-1)-\varepsilon})(\mu) \big) \Big)
\]
where
\[
q = -k -2s(p-1)+\varepsilon, \qquad
\nu(q) = (-1)^{q(q-1)m/2}(m!)^q, \qquad
m = (p-1)/2.
\]

The map $\psi^{\cobar}$, being a special case of the map $\psi^\cube$ defined in \cite{medina2020maysteenrod}, can be described effectively, and it has been implemented in the specialized computer algebra system \texttt{ComCH} \cite{medina2021computer}.
In similar effective manner, for the mod 2 case, explicit Cartan and Adem coboundaries \cite{medina2020cartan, medina2020adem} can be described for the cochain level Steenrod squares of the cobar construction.

\subsection{Factorization of Adams' map} \label{ss:factorization of adams}

We now use the ideas developed so far to study the based loop space of a pointed topological space $(Z, z)$.

Using the natural isomorphism of algebras
\[
\cobar \schains (\sSing(Z, z)) \cong  \cchains \ccobar (\sSing(Z, z))
\]
of \cref{l:ccobar and cobar}.
Adams' map can be factored as 
\begin{equation} \label{e:factorization of adams}
\theta_Z \colon \cobar \sS(Z, z) \xrightarrow{\cong} 
\cchains \ccobar(\sSing(Z, z)) \xrightarrow{\cchains(\Theta)} 
S^{\cube}(\loops_z Z),
\end{equation}
where second map is given by applying the cubical chains functor $\cchains \colon \Mon_{\cSet} \to \Mon_{\Ch}$ to the map of monoidal cubical sets
\[
\Theta \colon \ccobar(\sSing(Z, z)) \to \cSing(\loops_z Z)
\]
determined through the monoid structure by sending an $n$-simplex $(\sigma \colon \gsimplex^n \to Z)$ to the singular $(n-1)$-cube
\[
P(\sigma) \circ \theta_n \colon \gcube^{n-1} \to \loops_z Z,
\]
where the maps $\theta_n \colon \gcube^{n-1} \to P(\gsimplex^n;0,n)$ are those defined by Adams and discussed in \cref{adamsmaps}.


\subsection{Adams's map as an $E_{\infty}$-bialgebra quasi-isomorphism}

Adams' comparison map $\theta_Z \colon \cobar \sS(Z, z) \to S^{\cube}(\loops_z Z)$ was originally identified to be a morphism of monoids in $\Ch$ which, if applied to simply-connected spaces, induces a homology isomorphism.
Baues showed that $\theta_Z$ is in fact a morphism of monoids in $\coAlg$, i.e., of bialgebras.
Whereas, using that $\sSing(Z, z)$ is a fibrant reduced simplicial set, the second named author and M. Zeinalian \cite{rivera2018cubical} showed that for a possibly non-simply connected pointed space $(Z,y)$ the algebras $\cobar \sS(Z, z)$ and $S^{\cube}(\loops_zZ)$ are quasi-isomorphic.
Later on, the second named author and S. Saneblidze 
\cite{rivera2019path} showed that $\Theta \colon \ccobar(\sSing(Z, z)) \to \cSing(\loops_z Z)$ is a weak homotopy equivalence monoids in $\cSet$.

We now show that $\theta_Z$ is in fact a quasi-isomorphism of $E_{\infty}$-bialgebras.
Considering the factorization \eqref{e:factorization of adams} of Adams' comparison map.
The isomorphism is by construction one of $\UM$-bialgebras, whereas the second map, induced from a weak-equivalence of monoids in $\cSet$, is also a morphism of $\UM$-bialgebras
since $\cchainsUM$ is monoidal (\cref{l:cubical e-infty chains are monoidal}).
We summarize this discussion, which constitutes the second half of \cref{t:1st main thm in the intro}, in the following.

\begin{lemma} \label{l:adams comparison is an e-infty bialgebra map}
	Adams' comparison map $\theta_Z \colon \cobar \sS(Z, z) \to S^{\cube}(\loops_z Z)$ is a quasi-isomorphism of $E_{\infty}$-bialgebras or, more specifically, of monoids in $\coAlg_\UM$ for any pointed space $(Z, z)$.
\end{lemma}
