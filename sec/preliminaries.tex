
\section{Conventions and preliminaries} \label{s:preliminaries}

\subsection{Presheaves}

Given categories $\sB$ and $\sC$ we denote their associated \textit{functor category} by $\Fun[\sB, \sC]$.
Recall that a category is said to be \textit{small} if its objects and morphisms form sets.
We denote the Cartesian category of small categories by $\Cat$.
A category is said to be \textit{cocomplete} if any functor to it from a small category has a colimit.
If $\sA$ is small and $\sC$ cocomplete, then the (left) \textit{Kan extension of $g$ along $f$} exists for any pair of functors $f$ and $g$ in the diagram below, and it is the initial object in $\Fun[\sB, \sC]$ making
\begin{equation*}
\begin{tikzcd}[column sep=normal, row sep=normal]
\sA \arrow[d, "f"'] \arrow[r, "g"] & \sC \\
\sB \arrow[dashed, ur, bend right] & \quad
\end{tikzcd}
\end{equation*}
commute.
A Kan extension along the \textit{Yoneda embedding}, i.e., the functor
\[
\yoneda \colon \sA \to \Fun[\sA^\op, \Set]
\]
induced by the assignment
\[
a \mapsto \big( a^\prime \mapsto \sA(a^\prime, a) \big),
\]
is referred to as a \textit{Yoneda extension}.
We refer to the objects in $\Fun[\sA^\op, \Set]$ as \textit{presheaves on $\sA$} and to those in the image of the Yoneda embedding as \textit{representables}.
We remark that for any presheaf $P$ on $\sA$ and object $a$ of $\sA$
\[
P(a) \, \cong \colim_{\yoneda(a) \to P} \yoneda(a).
\]

\subsection{Chain complexes}

Throughout this article $\k$ denotes a commutative and unital ring and we work over its associated closed symmetric monoidal category of differential (homological) graded $\k$-modules $(\Ch, \otimes, \k)$.
We refer to the objects and morphisms of this category as \textit{chain complexes} and \textit{chain maps} respectively. We denote by $\Hom(C, C^\prime)$ the chain complex of $\k$-linear maps between chain complexes $C$ and $C^\prime$, and refer to the functor $\Hom(-, \k)$ as \textit{linear duality}.
The $i^\th$ \textit{suspension} functor $s^i \colon \Ch \to \Ch$ is defined at the level of graded modules by $(s^{i}M)_n = M_{n-i}$.

\subsection{Coalgebras} \label{ss:coalgebras}

A \textit{coalgebra} consists of an object $C$ in $\Ch$ and morphisms $\Delta \colon C \to C \otimes C$ and $\varepsilon \colon C \to \k$ in this category satisfying the usual coassociativity and counitality relations.
This notion is equivalent to that of a comonoid in $\Ch$.
Denote by $\coAlg$ the category of coalgebras with morphisms being structure preserving chain maps.
The category $\coAlg$ is symmetric monoidal, with braiding induced from $\Ch$ and structure maps of a product $C \otimes C^\prime$ given by
\begin{gather*}
C \otimes C^\prime \xra{\Delta \otimes \Delta^\prime}
(C \otimes C) \otimes (C^\prime \otimes C^\prime) \xra{(23)}
(C \otimes C^\prime) \otimes (C \otimes C^\prime), \\
C \otimes C^\prime \xra{\varepsilon \otimes \varepsilon^\prime}
\k \otimes \k \xra{\cong} \k.
\end{gather*}

\subsection{Monoids}

A \textit{monoid} $(M, \mu, \eta)$ in a monoidal category $(\sC, \otimes, \mathbb{1})$ is an object $M$ together with two morphisms $\mu \colon M \otimes M \to M$, called \textit{multiplication} and $\eta \colon \mathbb{1} \to M$, called \textit{unit}, satisfying the usual associativity and unital relations.
Monoids in $\Ch$ are called \textit{algebras} and those in $\coAlg$ are called \textit{bialgebras}.
We denote the category of monoids in $\sC$ by $\Mon_{\sC}$ and remark that a monoidal functor $\sC \to \sC^\prime$ induces a functor between their categories of monoids $\Mon_\sC \to \Mon_{\sC^\prime}$.

\subsection{The cobar construction} \label{ss:cobar construction}

A \textit{coaugmentation} on a coalgebra $C$ is a morphism $\nu \colon \k \to C$ in $\coAlg$.
A coalgebra is said to be \textit{coaugmented} if it is equipped with a coaugmentation.

We denote by $\coAlg^\ast$ the category of coaugmented coalgebras with morphisms being coaugmentation preserving coalgebra morphisms. In other words, $\coAlg^\ast$ is the slice category $\k \downarrow \mathsf{coMon}_{\Ch},$ where $\mathsf{coMon}_{\Ch}$ is the category of comonoids in $\Ch$ and $\k$ is regarded as an object in $\mathsf{coMon}_{\Ch}$ by considering it as a chain complex concentrated in degree $0$ with coproduct determined by $1_{\k} \mapsto 1_{\k} \otimes 1_{\k}.$

The \textit{cobar construction} is the functor 
\[
\cobar \colon \coAlg^\ast \to \Mon_{\Ch}
\]
defined on objects as follows.
Let $(C, \partial, \Delta, \varepsilon, \nu)$ be the data of a coaugmented coalgebra.
Denote by $\overline{C}$ the cokernel of $\nu \colon \k \to C$ and recall that $s^{-1}$ is the $(-1)^\th$ suspension.
The cobar construction $\cobar C$ of this coaugmented coalgebra is the graded module
\[
T(s^{-1} \overline{C}) = \k \oplus s^{-1}\overline{C} \oplus (s^{-1}\overline{C})^{\otimes 2} \oplus (s^{-1}\overline{C})^{\otimes 3} \oplus \dots
\]
regarded as a monoid in $\Ch$ with $\mu \colon T(s^{-1} \overline{C})^{\otimes 2} \to T(s^{-1} \overline{C})$ given by concatenation, $\eta \colon \k \to T(s^{-1} \overline{C})$ by the obvious inclusion, and differential constructed by extending the linear map
\[
- s^{-1} \circ \partial \circ s^{+1} \ + \ (s^{-1} \otimes s^{-1}) \circ \Delta \circ s^{+1} \colon
s^{-1} \overline{C} \to s^{-1}\overline{C} \oplus (s^{-1}\overline{C} \otimes s^{-1}\overline{C}) \hookrightarrow T(s^{-1}C)
\]
as a derivation.

In this article we are mostly concerned with \textit{connected} coalgebras, namely coaugmented coalgebras that are non-negatively graded and for which the coaugmentation induces an isomorphism $\k \cong C_0$ of $\k$-modules.
We denote by $\coAlg^0$ the full subcategory of $\coAlg^\ast$ consisting of connected coalgebras, and remark that if $C \in \coAlg^0$, then $ \cobar(C)$ is concentrated on non-negative degrees. 

The object $ \cobar(C) \in \Mon_{\Ch}$ may be equipped with a natural augmentation map $\cobar(C) \to \k$ in $\Mon_{\Ch}$ given by the projection map to the first summand term. Therefore, the cobar construction may be regarded as a functor between slice categories
\[
\cobar \colon \k \downarrow \mathsf{coMon}_{\Ch} \to \Mon_{\Ch} \downarrow \k,
\]
however, we will not require the data of augmentations in this article. 

We now review the central examples of coaugmented coalgebras to which Adams applied the cobar construction.

\subsection{Simplicial sets}

The \textit{simplex category} is denoted by $\simplex$ and its objects by $[n]$.
Its morphisms are generated by the usual (simplicial) \textit{coface} and \textit{codegeneracy maps}
\[
\delta_i \colon [n-1] \to [n], \qquad \sigma_i \colon [n+1] \to [n]
\]
for $j \in \{0, \dots, n\}$.
The category of \textit{simplicial sets} $\Fun[\simplex^\op, \Set]$ is denoted by $\sSet$ and the \textit{standard $n$-simplex} $\yoneda [n] = \simplex(-, [n])$ by $\simplex^n$.
For any simplicial set $X$ we write, as usual, $X_n$ instead of $X \big( [n] \big)$, and identify the elements of $\simplex^n_m$ with increasing tuples $[v_0, \dots, v_m]$ where $v_i \in \{0, \dots, n\}$.

If $X$ is such that $X_0$ is a singleton set we say that $X$ is \textit{reduced}.
We denote the full subcategory of reduced simplicial sets by $\sSet^0$.

\subsection{Simplicial chains} \label{ss:simplicial sets}

For non-negative integers $m$ and $n$, let $\simplex_{\deg} \big( [m], [n] \big)$ be the subset of \textit{degenerate morphisms} in $\simplex \big( [m], [n] \big)$, i.e., those of the form $\sigma_i \circ \tau$ with $\tau$ any morphism in $\simplex \big( [m], [n+1] \big)$.
The functor of \textit{simplicial chains} $\schains \colon \sSet \to \Ch$ is the Yoneda extension of the functor $\simplex \to \Ch$ defined next.
To an object $[n]$ it assigns the chain complex having in degree $m$ the module
\[
\frac{\k\{\simplex \big( [m], [n] \big) \}}{\k\{\simplex_{\deg} \big( [m], [n] \big) \}},
\]
and differential
\[
\partial(\tau) = \sum (-1)^i \tau \circ \delta_i.
\]
To a morphism $\tau \colon [n] \to [n^\prime]$ it assigns the chain map
\[
\begin{tikzcd}[row sep=-3pt, column sep=normal,
/tikz/column 1/.append style={anchor=base east},
/tikz/column 2/.append style={anchor=base west}]
\schains(\simplex^n)_m \arrow[r] & \schains(\simplex^{n^\prime})_m \\
\big( [m] \to [n] \big) \arrow[r, mapsto] & \big( [m] \to [n] \xra{\tau} [n^\prime] \big).
\end{tikzcd}
\]
When no confusion arises from doing so we write $\chains$ instead of $\schains$ and refer to it simply as the functor of \textit{chains}.

\subsection{Alexander--Whitney coalgebra} \label{ss:aw coalgebra}

We review a classical lift to coalgebras of the functor of chains:
\[
\begin{tikzcd}
& \coAlg \arrow[d] \\
\sSet \arrow[r, "\schains"] \arrow[ur, "\schainsAs", out=70, in=180] & \Ch.
\end{tikzcd}
\]

Using a Yoneda extension, it suffices to equip the chains on standard simplices with a natural coalgebra structure.
For any $n \in \N$, define $\epsilon \colon \chains(\simplex^n) \to \k$ by
\[
\epsilon \big( [v_0, \dots, v_q] \big) = \begin{cases} 1 & \text{ if } q = 0, \\ 0 & \text{ if } q>0, \end{cases}
\]
and $\Delta \colon \chains(\simplex^n) \to \chains(\simplex^n)^{\otimes2}$ by
\[
\Delta \big( [v_0, \dots, v_q] \big) = \sum_{i=0}^q \ [v_0, \dots, v_i] \otimes [v_i, \dots, v_q].
\]

If $X$ is a reduced simplicial set then $\schainsAs(X)$ becomes a connected coalgebra with the coaugmentation $\nu \colon \k \to \schains(X)$ induced by sending $1$ to the basis element represented by the unique $0$-simplex of $X$.
Hence, the functor $\schainsAs$ restricts to a functor
\[
\schainsAs \colon \sSet^0 \to \coAlg^0.
\]

\subsection{Simplicial singular complex}

Consider the topological $n$-simplex
\[
\gsimplex^{n} = \{(x_0, \dots, x_n) \in [0,1]^{n+1} \mid \ \textstyle{\sum}_i x_i = 1\}.
\]
The assignment $[n] \to \gsimplex^n$ defines a functor $\simplex \to \Top$ whose Yoneda extension is known as (simplicial) \textit{geometric realization}.
It has a right adjoint $\sSing \colon \Top \to \sSet$ given by
\[
Z \to \Big( [n] \mapsto \Top(\gsimplex^n, Z) \Big)
\]
and referred to as the \textit{simplicial singular complex} of the topological space $Z$.
The chain complex $\chains(\cSing Z)$ is referred to as the \textit{simplicial singular chains} of $Z$.

We modify this construction for a pointed topological space $(Z, z)$ by considering only maps of $\gsimplex^n \to Z$ sending all vertices to $z$.
This produces a reduced simplicial set $\sSing(Z, z)$.
We will refer to $\schains(\sSing(Z, z))$ as the \textit{simplicial singular chains on $(Z, z)$} and for simplicity denote it by $\sS(Z, z)$.

\subsection{Cubical sets} \label{ss:cubical sets}

The objects of the \textit{cube category} $\cube$ are the sets $2^n = \{0, 1\}^n$ with $2^0 = \{0\}$ for $n \in \N$, and its morphisms are generated by the \textit{coface, codegeneracy} and \textit{coconnection} maps defined by
\begin{align*}
\delta_i^\varepsilon & =
\mathrm{id}_{2^{i-1}} \times \delta^\varepsilon \times \mathrm{id}_{2^{n-1-i}} \colon 2^{n-1} \to 2^n, \\
\sigma_i & =
\mathrm{id}_{2^{i-1}} \times \, \sigma \times \mathrm{id}_{2^{n-i}} \quad \colon 2^{n} \to 2^{n-1}, \\
\gamma_i & =
\mathrm{id}_{2^{i-1}} \times \, \gamma \times \mathrm{id}_{2^{n-i}} \quad \colon 2^{n+1} \to 2^{n},
\end{align*}
where $\varepsilon \in \{0,1\}$ and
\[
\begin{tikzcd}
1 \arrow[r, out=45, in=135, "\delta^0"] \arrow[r, out=-45, in=-135, "\delta^1"'] & 2 \arrow[l, "\sigma"'] &[-10pt] \arrow[l, "\gamma"'] 2 \times 2
\end{tikzcd}
\]
are defined by
\begin{gather*}
\delta^0(0) = 0, \qquad
\delta^1(0) = 1, \qquad
\sigma(0) = \sigma(1) = 0, \qquad \\
\gamma^1(0,0) = 0, \qquad
\gamma^1(0,1) = 1, \qquad
\gamma^1(1,0) = 1, \qquad
\gamma^1(1,1) = 1.
\end{gather*}
We remark that the cube category without connections will not be considered in this work.

The category of \textit{cubical sets} $\Fun[\cube^\op, \Set]$ is denoted by $\cSet$ and
the \textit{standard $n$-cube} $\yoneda(2^n) = \cube(-, 2^n)$ by $\cube^n$.
For any cubical set $Y$ we write, as usual, $Y_n$ instead of $Y(2^n)$.

\subsection{Cubical chains}

For non-negative integers $m$ and $n$, let $\cube_{\deg}(2^m, 2^n)$ be the subset of \textit{degenerate morphisms} in $\cube(2^m, 2^n)$, i.e., those of the form $\sigma_i \circ \tau$ or $\gamma_i \circ \tau$ with $\tau$ any morphism in $\cube(2^m, 2^{n+1})$.
The functor of (cubical) \textit{chains} $\cchains \colon \cSet \to \Ch$ is the Yoneda extension of the functor $\cube \to \Ch$ defined next.
To an object $2^n$ it assigning the chain complex having in degree $m$ the module
\[
\frac{\k\{\cube(2^m, 2^n)\}}{\k\{\cube_{\deg}(2^m, 2^n)\}}
\]
and differential defined by
\[
\partial (\id_{2^n}) = \sum_{i=1}^{n} \ (-1)^i \
\big(\delta_i^1 - \delta_i^0 \big).
\]
To a morphism $\tau \colon 2^n \to 2^{n^\prime}$ it assigns the chain map
\[
\begin{tikzcd}[row sep=-3pt, column sep=normal,
/tikz/column 1/.append style={anchor=base east},
/tikz/column 2/.append style={anchor=base west}]
\cchains(\cube^n)_m \arrow[r] & \cchains(\cube^{n^\prime})_m \\
\big( 2^m \to 2^n \big) \arrow[r, mapsto] & \big( 2^m \to 2^n \stackrel{\tau}{\to} 2^{n^\prime} \big).
\end{tikzcd}
\]

When no confusion arises from doing so we write $\chains$ instead of $\cchains$.

We remark that $\chains(\cube^n)$ is canonically isomorphic to $\chains(\cube^1)^{\otimes n}$ and that $\chains(\cube^1)$ is canonically isomorphic to the cellular chains on the topological interval with its usual CW structure.
We use these isomorphisms to denote the elements of $\chains(\cube^n)$ as sums of terms of the form $x_1 \otimes \dots \otimes x_n$ with $x_i \in \big\{[0], [0,1], [1] \big\}$.

\subsection{Serre coalgebra} \label{ss:serre coalgebra}

We review a classical lift to the category of coalgebras of the functor of cubical chains:
\[
\begin{tikzcd}
& \coAlg \arrow[d] \\
\cSet \arrow[r, "\cchains"] \arrow[ur, "\cchainsAs", out=70, in=180] & \Ch.
\end{tikzcd}
\]
Using a Yoneda extension, it suffices to equip the chains on standard cubes with a natural coalgebra structure.
For any $n \in \N$, define $\epsilon \colon \chains(\cube^n) \to \k$ by
\[
\epsilon \left( x_1 \otimes \dots \otimes x_d \right) = \epsilon(x_1) \dotsm \epsilon(x_n),
\]
where
\[
\epsilon([0]) = \epsilon([1]) = 1, \qquad \epsilon([0, 1]) = 0,
\]
and $\Delta \colon \chains(\cube^n) \to \chains(\cube^n)^{\otimes 2}$ by
\[
\Delta (x_1 \otimes \dots \otimes x_n) =
\sum \pm \left( x_1^{(1)} \otimes \dots \otimes x_n^{(1)} \right) \otimes
\left( x_1^{(2)} \otimes \dots \otimes x_n^{(2)} \right),
\]
where the sign is determined using the Koszul convention, and we are using Sweedler's notation
\[
\Delta(x_i) = \sum x_i^{(1)} \otimes x_i^{(2)}
\]
for the chain map $\Delta \colon \chains(\cube^1) \to \chains(\cube^1)^{\otimes 2}$ defined by
\[
\Delta([0]) = [0] \otimes [0], \quad \Delta([1]) = [1] \otimes [1], \quad \Delta([0,1]) = [0] \otimes [0,1] + [0,1] \otimes [1].
\]

We remark that, using the canonical isomorphism $\chains(\cube^n) \cong \chains(\cube^1)^{\otimes n}$, the coproduct $\Delta$ can be described as the composition
\[
\begin{tikzcd}
\chains(\cube^1)^{\otimes n} \arrow[r, "\Delta^{\otimes n}"] & \left( \chains(\cube^1)^{\otimes 2} \right)^{\otimes n} \arrow[r, "sh"] & \left( \chains(\cube^1)^{\otimes n} \right)^{\otimes 2}
\end{tikzcd}
\]
where $sh$ is the shuffle map that places tensor factors in odd position first.

\subsection{Cubical singular complex}

Consider the topological $n$-cube
\[
\gcube^{n} = \{(x_1, \dots, x_n) \mid x_i \in [0,1]\}.
\]
The assignment $2^n \to \gcube^n$ defines a functor $\cube \to \Top$ whose Yoneda extension is known as \textit{geometric realization}.
It has a right adjoint $\cSing \colon \Top \to \cSet$ given by
\[
Z \to \Big(2^n \mapsto \Top(\gcube^n, Z)\Big)
\]
and referred to as the \textit{cubical singular complex} of the topological space $Z$.
We will refer to $\chains(\cSing Z)$ as the \textit{cubical singular chains} of $Z$ and for simplicity denote it by $\cS(Z)$.