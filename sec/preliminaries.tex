% !TEX root = ../cobar1.tex

\section{Conventions and preliminaries}\label{s:preliminaries}

\subsection{Presheaves}

Given categories $\sB$ and $\sC$ with $\sB$ small we denote their associated \textit{functor category} by $\Fun(\sB, \sC)$.
A category is said to be \textit{cocomplete} if any functor to it from a small category has a colimit.
If $\sA$ is small and $\sC$ cocomplete, then the (left) \textit{Kan extension} of $g$ along $f$ exists for any pair of functors $f$ and $g$ in the diagram below, and it is the initial object in $\Fun(\sB, \sC)$ making
\begin{equation*}
	\begin{tikzcd}[column sep=normal, row sep=normal]
		\sA \arrow[d, "f"'] \arrow[r, "g"] & \sC \\
		\sB \arrow[dashed, ur, bend right] & \quad
	\end{tikzcd}
\end{equation*}
commute.
A Kan extension along the \textit{Yoneda embedding}, i.e., the functor
\[
\yoneda \colon \sA \to \Fun(\sA^\op, \Set)
\]
induced by the assignment
\[
a \mapsto \big(a^\prime \mapsto \sA(a^\prime, a)\big),
\]
is referred to as a \textit{Yoneda extension}.
Objects in the image of the Yoneda embedding are said to be \textit{representable}.
Any presheaf $P$ is isomorphic to a colimit of representables presheaves $P \, \cong \colim_{\yoneda(a) \to P} \yoneda(a)$.

If $\sA$ is a monoidal category, then the Day convolution product makes $\Fun(\sA^\op, \Set)$ into a monoidal category.

\subsection{Chain complexes}

Throughout this article $\k$ denotes a commutative and unital ring and we work over its associated closed symmetric monoidal category of differential (homologically) graded $\k$-modules $(\Ch, \ot, \k)$.
We refer to the objects and morphisms of this category as \textit{chain complexes} and \textit{chain maps} respectively.
We denote by $\Hom(C, C^\prime)$ the chain complex of $\k$-linear maps between chain complexes $C$ and $C^\prime$, and refer to the functor $\Hom(-, \k)$ as \textit{linear duality}.
The $i^\th$ \textit{suspension} functor $s^i \colon \Ch \to \Ch$ is defined at the level of graded modules by $(s^{i}M)_n = M_{n-i}$.

\subsection{Monoids}

A \textit{monoid} $(M, \mu, \eta)$ in a monoidal category $(\sC, \ot, \mathbb{1})$ is an object $M$ together with two morphisms $\mu \colon M \ot M \to M$ and $\eta \colon \mathbb{1} \to M$ satisfying the usual associativity and unital relations.
We denote the category of monoids in $\sC$ by $\Mon_{\sC}$ and remark that a monoidal functor $\sC \to \sC^\prime$ induces a functor between their categories of monoids $\Mon_\sC \to \Mon_{\sC^\prime}$.

For example, a monoid in $(\Ch, \ot, \k)$ is a differential graded $\k$-algebra. A morphism $f: A \to A'$ in $\Mon_\Ch$ inducing an isomorphism on homology will be called a \textit{monoidal quasi-isomorphism.}

\subsection{Coalgebras}\label{ss:coalgebras}

A \textit{coalgebra} consists of a chain complex $C$ and chain maps $\Delta \colon C \to C \ot C$ and $\varepsilon \colon C \to \k$ satisfying the usual coassociativity and counitality relations.
This notion is equivalent to that of a comonoid in $\Ch$.
Denote by $\coAlg$ the category of coalgebras with morphisms being structure preserving chain maps.
The category $\coAlg$ is symmetric monoidal, with braiding induced from $\Ch$ and structure maps of a product $C \ot C^\prime$ given by
\begin{gather*}
	C \ot C^\prime \xra{\Delta \ot \Delta^\prime}
	(C \ot C) \ot (C^\prime \ot C^\prime) \xra{(23)}
	(C \ot C^\prime) \ot (C \ot C^\prime), \\
	C \ot C^\prime \xra{\varepsilon \ot \varepsilon^\prime}
	\k \ot \k \xra{\cong} \k.
\end{gather*}

A \textit{coaugmentation} on a coalgebra $C$ is a coalgebra map $\nu \colon \k \to C$.
A coalgebra is said to be \textit{coaugmented} if it is equipped with a coaugmentation.
We denote by $\coAlg^\ast$ the category of coaugmented coalgebras with morphisms being coaugmentation preserving coalgebra maps.
A coaugmented coalgebra is a \textit{connected coalgebra} if it is $0$ in negative degrees and the coaugmentation induces an isomorphism $\k \cong C_0$ of $\k$-modules.
We denote by $\coAlg^0$ the full subcategory of $\coAlg^\ast$ defined by these.

\subsection{Simplicial theory}\label{ss:simplicial}

The \textit{simplex category} is denoted by $\simplex$ and its objects by $[n]$. The morphisms in $\simplex$ are generated by coface maps, denoted by $\partial^{j}_n \colon [n-1] \to [n]$ for $0\leq j \leq n$, and codegeneracy maps, denoted by $\xi^{j}_n \colon [n+1] \to [n]$ for $0 \leq j \leq n+1$. These satisfy the usual cosimplicial identities. For simplicity, we omit the subscript in the notation when there is no risk of confusion and simply denote these maps by $\partial^{j} \colon [n-1] \to [n]$ and $\xi^{j} \colon [n+1] \to [n]$.


The category of \textit{simplicial sets} $\Fun(\simplex^\op, \Set)$ is denoted by $\sSet$ and the \textit{standard $n$-simplex} $\yoneda [n] = \simplex(-, [n])$ by $\simplex^n$.
For any simplicial set $X$ we write, as usual, $X_n$ instead of $X[n]$, and identify the elements of $\simplex^n_m$ with increasing tuples $[v_0, \dots, v_m]$ where $v_i \in \{0, \dots, n\}$.

If $X$ is such that $X_0$ is a singleton set we say that $X$ is \textit{reduced}.
We denote the full subcategory of reduced simplicial sets by $\sSet^0$.

We denote the topological $n$-simplex by $\gsimplex^{n}$ and consider the usual adjunction pair defined by the \textit{geometric realization} and \textit{singular complex}
\[
\begin{tikzcd}
	\bars{\ } \colon \sSet \arrow[r, shift left=.5ex] &
	\Top :\! \sSing. \arrow[l, shift left=.5ex]
\end{tikzcd}
\]

The functor of (normalized) \textit{simplicial chains} $\schains \colon \sSet \to \Ch$ is obtained from the geometric realization and the functor of cellular chains.
When no confusion arises from doing so we write $\chains$ instead of $\schains$ and refer to it simply as the functor of \textit{chains}.
We will denote the functor of (simplicial) \textit{singular chains} $\schains \circ \sSing \colon \Top \to \Ch$ by $\sSchains$.

We modify this construction for a pointed topological space $(\fX,x)$ by only considering maps $\gsimplex^n \to \fX$ sending all vertices to $x$.
This produces a reduced simplicial set $\sSing(\fX,x)$ whose normalized chains we denote by $\sSchains(\fX,x)$.

There is a classical lift of the functor of chains to coalgebras:
\[
\begin{tikzcd}
	& \coAlg \arrow[d] \\
	\sSet \arrow[r, "\schains"] \arrow[ur, "\schainsA", out=70, in=180] & \Ch
\end{tikzcd}
\]
defined, via a Yoneda extension, by the following structure on the chains of standard simplices.
For any $n \in \N$, define $\epsilon \colon \chains(\simplex^n) \to \k$ by
\[
\epsilon \big( [v_0, \dots, v_q] \big) = \begin{cases} 1 & \text{ if } q = 0, \\ 0 & \text{ if } q>0, \end{cases}
\]
and $\Delta \colon \chains(\simplex^n) \to \chains(\simplex^n)^{\ot2}$ by
\[
\Delta \big( [v_0, \dots, v_q] \big) = \sum_{i=0}^q \ [v_0, \dots, v_i] \ot [v_i, \dots, v_q].
\]
We refer to this lift as the \textit{Alexander--Whitney coalgebra} structure on simplicial chains.

If $X$ is a reduced simplicial set then $\schainsA(X)$ becomes a connected coalgebra with the coaugmentation $\nu \colon \k \to \schains(X)$ induced by sending $1$ to the basis element represented by the unique $0$-simplex of $X$.
Hence, the functor $\schainsA$ restricts to a functor
\[
\schainsA \colon \sSet^0 \to \coAlg^0.
\]

\subsection{Cubical theory}\label{ss:cubical}

The \textit{cube category} (with a connection) $\cube$ is the subcategory of $\Set$ whose objects are $2^n = \{0, 1\}^n$ for $n \in \N$ with $2^0 = \{0\}$.
We will denote $2^1$ by $2$ and $2^0$ by $1$, which is the unit of the monoidal structure of $\cube$ given by $2^n \times 2^m = 2^{n+m}$.
The morphisms of $\cube$ are monoidally generated by
%the \textit{coface, codegeneracy} and \textit{coconnection} maps
\[
\begin{tikzcd}
	1 \arrow[r, out=45, in=135, "\delta^0"] \arrow[r, out=-45, in=-135, "\delta^1"'] & 2 \arrow[l, "\sigma"'] &[-10pt] \arrow[l, "\gamma"'] 2 \times 2
\end{tikzcd}
\]
defined for $x, y \in \{0,1\}$ by
\begin{gather*}
	\delta^x(0) = x, \qquad
	\sigma(x) = 0, \qquad
	\gamma(x,y) = \max(x,y).
\end{gather*}
We remark that the cube category without connections will not be considered in this work.

The category of \textit{cubical sets} $\Fun(\cube^\op, \Set)$ is denoted by $\cSet$ and
the \textit{standard $n$-cube} $\yoneda 2^n = \cube(-, 2^n)$ by $\cube^n$.
For any cubical set $Y$ we write, as usual, $Y_n$ instead of $Y(2^n)$.
The monoidal structure on $\cube$ induces one on $\cSet$ as described in ???.

We denote the topological $n$-cube by $\gcube^{n}$ and consider the usual adjunction pair defined by the \textit{geometric realization} and \textit{singular complex}
\[
\begin{tikzcd}
	\bars{\ } \colon \cSet \arrow[r, shift left=.5ex] &
	\Top :\! \cSing. \arrow[l, shift left=.5ex]
\end{tikzcd}
\]

The functor of (normalized) \textit{cubical chains} $\cchains \colon \cSet \to \Ch$ is Yoneda extension of the functor defined by $\chains(\cube^n) = \chains(\cube^1)^{\ot n}$.
When no confusion arises from doing so we write $\chains$ instead of $\cchains$ and refer to it simply as the functor of \textit{chains}.
We remark that $\cchains$ is naturally equivalent to the composition of the geometric realization and cellular chains functors.
We will denote the functor of (cubical) \textit{singular chains} $\cchains \circ \cSing \colon \Top \to \Ch$ by $\cSchains$.

There is a classical lift $\cchainsA$ of the functor of cubical chains to the category of coalgebras
\[
\begin{tikzcd}
	& \coAlg \arrow[d] \\
	\cSet \arrow[r, "\cchains"] \arrow[ur, "\cchainsA", out=70, in=180] & \Ch
\end{tikzcd}
\]
defined, using a Yoneda extension, by the tensor product structure (\cref{ss:coalgebras}) defined on $\chains(\cube^n) = \chains(\cube^1)^{\ot n}$ via the identification of $\chains(\cube^1)$ and $\chains(\simplex^1)$.
We refer to this lift as the \textit{Serre coalgebra} structure on cubical chains.
We will prove in \cref{ss:cube_einfty revisited} a generalization of the following reformulation of this definition.

\begin{proposition}\label{p:serre coalgebra}
	The functor $\cchainsA \colon \cSet \to \coAlg$ is the unique monoidal functor defined by the coalgebra structure on $\chains(\cube^1)$.
\end{proposition}