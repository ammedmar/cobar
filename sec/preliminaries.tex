% !TEX root = ../cobar1.tex

\section{Conventions and preliminaries}\label{s:preliminaries}

\subsection{Coalgebras}\label{ss:coalgebras}

Throughout this article $\k$ denotes a commutative and unital ring and we work over its associated symmetric monoidal category of (homologically) graded chain complexes of $\k$-modules $(\Ch, \ot, \k)$.

A \textit{coalgebra} consists of a chain complex $C$ and chain maps $\Delta \colon C \to C \ot C$ and $\varepsilon \colon C \to \k$ satisfying the usual coassociativity and counitality relations.
Denote by $\coAlg$ the category of coalgebras with morphisms being structure preserving chain maps.
The category $\coAlg$ is symmetric monoidal, with braiding induced from $\Ch$ and structure maps of a product $C \ot C^\prime$ given by
\begin{gather*}
	C \ot C^\prime \xra{\Delta \ot \Delta^\prime}
	(C \ot C) \ot (C^\prime \ot C^\prime) \xra{(23)}
	(C \ot C^\prime) \ot (C \ot C^\prime), \\
	C \ot C^\prime \xra{\varepsilon \ot \varepsilon^\prime}
	\k \ot \k \xra{\cong} \k.
\end{gather*}

A \textit{coaugmentation} on a coalgebra $C$ is a coalgebra map $\nu \colon \k \to C$.
A coalgebra is said to be \textit{coaugmented} if it is equipped with a coaugmentation.
We denote by $\coAlg^\ast$ the category of coaugmented coalgebras with morphisms being coaugmentation preserving coalgebra maps.
A coaugmented coalgebra is a \textit{connected coalgebra} if it is $0$ in negative degrees and the coaugmentation induces an isomorphism $\k \cong C_0$ of $\k$-modules.
We denote by $\coAlg^0$ the full subcategory of $\coAlg^\ast$ defined by these.

\subsection{Monoids}

A \textit{monoidal object} in a monoidal category $(\sC, \ot, \mathbb{1})$ is an object $M$ together with morphisms $\mu \colon M \ot M \to M$ and $\eta \colon \mathbb{1} \to M$ satisfying the usual associativity and unital relations.
The category of these together with structure preserving morphisms, referred to as \textit{monoidal morphisms}, is denoted $\Mon_\sC$.
We remark that a lax monoidal functor $\sC \to \sC^\prime$ induces a functor between their categories of monoids $\Mon_\sC \to \Mon_{\sC^\prime}$.
For example, the forgetful functor from $\coAlg$ to $\Ch$ is monoidal and so, it induces a forgetful functor from monoidal coalgebras to monoidal chain complexes, which are more commonly known as bialgebras and algebras respectively, but this terminology is not well suited for our purposes.

\subsection{Simplicial theory}\label{ss:simplicial}

The \textit{simplex category} is denoted by $\simplex$ and its objects by $[n]$.
The morphisms in $\simplex$ are generated by coface maps, denoted by $\partial^{j}_n \colon [n-1] \to [n]$ for $0\leq j \leq n$, and codegeneracy maps, denoted by $\xi^{j}_n \colon [n+1] \to [n]$ for $0 \leq j \leq n+1$.
These satisfy the usual cosimplicial identities.
For simplicity, we omit the subscript in the notation when there is no risk of confusion and simply denote these maps by $\partial^{j} \colon [n-1] \to [n]$ and $\xi^{j} \colon [n+1] \to [n]$.

The category of \textit{simplicial sets} $\Fun(\simplex^\op, \Set)$ is denoted by $\sSet$ and the \textit{standard $n$-simplex} $\simplex(-, [n])$ by $\simplex^n$.
For any simplicial set $X$ we write, as usual, $X_n$ instead of $X[n]$, and identify the elements of $\simplex^n_m$ with increasing tuples $[v_0, \dots, v_m]$ where $v_i \in \{0, \dots, n\}$.

If $X$ is such that $X_0$ is a singleton set we say that $X$ is \textit{reduced}.
We denote the full subcategory of reduced simplicial sets by $\sSet^0$.

We consider the \textit{topological $n$-simplex} $\gsimplex^n$ embedded in $\mathbb{R}^{n+1}$ as
\[ \gsimplex^n = \{ (x_0, \dots, x_n) \in \mathbb{R}^{n+1} | x_0^2 + \dots + x_n^2=1\} \]
so that the $i$-th vertex is given by the standard basis vector $v_i=(0,\ldots,1, \ldots,0) \in \mathbb{R}^{n+1}$ with $1$ in the $i$-th entry and $0$ in all other entries.
Consider the usual adjunction pair formed by the \textit{geometric realization} and \textit{singular complex}
\[
\begin{tikzcd}
	\bars{\ } \colon \sSet \arrow[r, shift left=.5ex] &
	\Top :\! \sSing, \arrow[l, shift left=.5ex]
\end{tikzcd}
\]
determined by the spaces $\gsimplex^n$ and usual coface inclusions $\delta^i \colon \gsimplex^{n-1} \to \gsimplex^n$ and codegeneracy projections $\sigma^j \colon \gsimplex^n \to \gsimplex^{n-1}$.

The functor of (normalized) \textit{simplicial chains} $\schains \colon \sSet \to \Ch$ is defined on any simplicial set $X$ by first considering the chain complex $(\k[X], \partial)$, given on degree $n$ by the free $\k$-module generated by the $n$-simplices of $X$ with differential given by the alternating sum of face maps, and then modding out by the sub-chain complex of degenerate elements.
We remark that this functor is naturally equivalent to the composition of the geometric realization functor and that of cellular chains with respect to the standard CW structure.
When no confusion arises from doing so we write $\chains$ instead of $\schains$ and refer to it simply as the functor of \textit{chains}.
We will denote the functor of (simplicial) \textit{singular chains} $\schains \circ \sSing \colon \Top \to \Ch$ by $\sSchains$, where $\sSing \colon \Top \to \sSet$ denotes the singular complex functor. We modify this construction for a pointed topological space $(\fX,x)$ by only considering maps $\gsimplex^n \to \fX$ sending all vertices to $x$.
This produces a reduced simplicial set $\sSing(\fX,x)$ whose normalized chains we denote by $\sSchains(\fX,x)$.

We now recall a classical chain approximation of the diagonal map on the chains of a simplicial set, i.e. a lift of the functor of chains to coalgebras:
\[
\begin{tikzcd}
	& \coAlg \arrow[d] \\
	\sSet \arrow[r, "\schains"] \arrow[ur, "\schainsA", out=70, in=180] & \Ch.
\end{tikzcd}
\]
It suffices to define this functor $\schainsA$ on standard simplices. For any $n \in \N$, define $\Delta \colon \chains(\simplex^n) \to \chains(\simplex^n)^{\ot2}$ by
\[
\Delta \big( [v_0, \dots, v_q] \big) = \sum_{i=0}^q \ [v_0, \dots, v_i] \ot [v_i, \dots, v_q]
\]
and $\epsilon \colon \chains(\simplex^n) \to \k$ by
\[
\epsilon \big( [v_0, \dots, v_q] \big) = \begin{cases} 1 & \text{ if } q = 0, \\ 0 & \text{ if } q>0. \end{cases}
\]
We referred to this lift as the \textit{Alexander--Whitney coalgebra} structure on simplicial chains.

If $X$ is a reduced simplicial set then $\schainsA(X)$ becomes a connected coalgebra with the coaugmentation $\nu \colon \k \to \schains(X)$ induced by sending $1$ to the basis element represented by the unique $0$-simplex of $X$.
Hence, the functor $\schainsA$ restricts to a functor
\[
\schainsA \colon \sSet^0 \to \coAlg^0.
\]

\subsection{Cubical theory}\label{ss:cubical}

We include an abridged presentation of cubical sets parallel to the our treatment of simplicial sets in the previous section.
For more details we refer the reader to any of \cite{grandis2003cubical, medina2022cube_einfty, medina2023flowing}.

The \textit{cube category} (with a connection) $\cube$ is the subcategory of $\Set$ whose objects are $2^n = \{0, 1\}^n$ for $n \in \N$ with $2^0 = \{0\}$.
We will denote $2^1$ by $2$ and $2^0$ by $1$, which is the unit of the monoidal structure of $\cube$ given by $2^n \times 2^m = 2^{n+m}$.
The morphisms of $\cube$ are monoidally generated by
\begin{equation}\label{eq:cube generators}
	\begin{tikzcd}
		1 \arrow[r, out=45, in=135, "\delta_0"] \arrow[r, out=-45, in=-135, "\delta_1"'] & 2 \arrow[l, "\sigma"'] &[-10pt] \arrow[l, "\gamma"'] 2 \times 2
	\end{tikzcd}
\end{equation}
defined by
\begin{gather*}
	\delta_0(0) = 0, \qquad
	\delta_1(0) = 1, \qquad
	\sigma(p) = 0, \qquad
	\gamma(p,q) = \max(p,q),
\end{gather*}
for $p,q \in \{0,1\}$.
We remark that the cube category without connections will not be considered in this work.

The category of \textit{cubical sets} $\Fun(\cube^\op, \Set)$ is denoted by $\cSet$ and
the \textit{standard $n$-cube} $\cube(-, 2^n)$ by $\cube^n$.
For any cubical set $Y$ we write, as usual, $Y_n$ instead of $Y(2^n)$.
The monoidal structure on $\cube$ induces one on $\cSet$.
More explicitly, for two cubical sets we have
\[
(Y \times Y')_n = \bigsqcup_{i+j=n} Y_i \times Y'_j.
\]

Consider the monoidal functor $\cube \to \Top$ defined by assigning $2$ to the usual interval $\gcube^1$ and \eqref{eq:cube generators} to the continuous maps
\[
\begin{tikzcd}
	\gcube^0 \arrow[r, out=45, in=135, "\delta_0"] \arrow[r, out=-45, in=-135, "\delta_1"'] & \gcube^1 \arrow[l, "\sigma"'] &[-10pt] \arrow[l, "\gamma"'] \gcube^2,
\end{tikzcd}
\]
where $\delta_i$ and $\sigma$ are the canonical cubical co-face and co-degeneracy maps, respectively, and $\gamma(s,t)=\text{max}(s,t)$.

From this functor we obtain the usual adjunction pair formed by the \textit{geometric realization} and \textit{singular complex}
\[
\begin{tikzcd}
	\bars{\ } \colon \cSet \arrow[r, shift left=.5ex] &
	\Top :\! \cSing. \arrow[l, shift left=.5ex]
\end{tikzcd}
\]
Notice that $\cSing$ is lax monoidal
\[
(\gcube^n \to \fX) \times (\gcube^m \to \fX') \mapsto (\gcube^{n+m} \to \fX \times \fX').
\]

The functor of (normalized) \textit{cubical chains} $\cchains \colon \cSet \to \Ch$ is the unique monoidal functor defined by assigning to $\cube^1$ the usual cellular chains of the interval $\gcube^1$.
Explicitly, it is defined on any cubical set $Y$ by first considering the chain complex $(\k[Y], \delta)$, given on degree $n$ by the free $\k$-module generated by the $n$-cubes of $Y$ with differential $\delta$ given on any $n$-cube $y \in Y_n$ by
\[
\delta(y) = \sum_{i=1}^n (-1)^i
\big(
Y(\text{id}_2^{i-1} \times \delta_1 \times \text{id}_2^{n-i})(y) -
Y(\text{id}_2^{i-1} \times \delta_0 \times \text{id}_2^{n-i})(y)
\big),
\]
and then modding out by the sub-complex of degenerate cubes.
When no confusion arises from doing so we write $\chains$ instead of $\cchains$ and refer to it simply as the functor of \textit{chains}.
We remark that $\cchains$ is naturally equivalent to the composition of the geometric realization and cellular chains functors with respect to the canonical CW structure.
We will denote the functor of (cubical) \textit{singular chains} $\cchains \circ \cSing \colon \Top \to \Ch$ by $\cSchains$.

Since $\chains(\cube^1)$ is a coalgebra and the category of coalgebras is monoidal, there is a unique monoidal functor $\cchainsA$ lifting the functor of chains
\[
\begin{tikzcd}
	& \coAlg \arrow[d] \\
	\cSet \arrow[r, "\cchains"] \arrow[ur, "\cchainsA", out=70, in=180] & \Ch.
\end{tikzcd}
\]
We refer to this lift as the \textit{Serre coalgebra} structure on cubical chains, and to the coproduct $\Delta$ as the Serre diagonal.