%The original cobar construction of Adams produces a monoid in $\Ch$ from a connected coalgebra. Baues enriched this construction to produce a monoid in $\coAlg$.
%He did this by using a version of the isomorphism $\cchains \ccobar \cong \cobar \schainsA$ of \cref{t:ccobar and cobar} and the Serre coalgebra on cubical chains.
%Since $\ccobar$ produces monoids in $\cSet$ and $\cchainsA$ is monoidal, the functor $\cchainsA \ccobar$ produces bialgebras, as claimed.


%\subsection{Finer homotopy invariants at the chain level}
%
%As mentioned in the introduction, Adams' small models for based loop spaces provide advantages for the explicit analysis of their homotopical invariants.
%Our construction allows for the study of the coaction of the dual Steenrod algebra on the mod~$p$ homology
%\[
%\rH \xra{\cong}
%\rH^{\vee\vee} \!\xra{\rS^\vee}
%(\cA \ot \rH^{\vee})^\vee \xla{\cong}
%\cA^\vee \!\ot \rH^{\vee\vee} \xla{\cong}
%\cA^\vee \!\ot \rH
%\]
%of these spaces \cite{steenrod1962cohomology} at the chain level.
%Explicit descriptions of the map $\rS$ are presented at the chain level in \cite{medina2021may_st} and \cite{medina2022dennis}, which were implemented in the computer algebra system \texttt{ComCH} \cite{medina2021comch}.
%For $p=2$, secondary operations are also available at the cochain level following \cite{medina2020cartan} and \cite{medina2021adem}.

%\subsection{Steenrod operations}
%
%Throughout this subsection let us consider $\k$ to be the finite field $\Fp$ and $X$ a reduced simplicial set.
%
%Since $\cobar \schainsA(X)$ is isomorphic to the chains on a cubical set (\cref{t:ccobar and cobar}), its mod~$p$ homology is a priori equipped with a right action of the Steenrod algebra \cite{steenrod1962cohomology, milnor1958steenrod}.
%As an application of the explicit construction of a natural $\UM$-coalgebra structure on $\cobar \schainsA(X)$, we can describe at the chain level this action using the constructions of \cite{medina2021may_st}.
%More explicitly, let $\Cp$ be the cyclic group of order $p$ and
%\[
%\begin{tikzcd} [column sep = .5cm]
%	\cW(p) = \Fp[\Cp]\{e_0\} & \arrow[l, "\,T"'] \Fp[\Cp]\{e_1\} & \arrow[l, "\,N"'] \Fp[\Cp]\{e_2\} & \arrow[l, "\,T"'] \dotsb
%\end{tikzcd}
%\]
%be the minimal free resolution of $\Fp$ as an $\Fp[\Cp]$-module.
%Consider the $\Cp$-equivari-ant quasi-isomorphism $\psi_\UM \colon \cW(p) \to \UM(p)$ constructed in \cite{medina2021may_st} or more explicitly in \cite{medina2022dennis}, and denote by $\psi^{\cobar}$ its composition with the morphism $\UM \to \End^{\cobar \schainsA(X)}$ defining our $E_\infty$-coalgebra structure on $\cobar \schainsA(X)$.
%
%If $p$ is the even prime, the Steenrod square $P_s$ is represented by the map sending a representative $\mu$ of a class in the $k^\th$ homology of $\cobar \schainsA(X)$ to the following element in its double linear dual:
%\[
%\mu \mapsto
%\Big( \alpha \mapsto (\alpha \ot \alpha) \big( \psi^{\cobar}(e_{k+2s})(\mu) \big) \Big).
%\]
%Similarly, if $p$ is an odd prime, the action of the Steenrod operations $\beta^\varepsilon P_{s}$ where $\varepsilon = 0,1$ are represented by
%\[
%\mu \mapsto
%\Big( \alpha \mapsto \alpha^{\ot p} \big( (-1)^p \, \nu(q) \, \psi^{\cobar}(e_{(2s-q)(p-1)-\varepsilon})(\mu) \big) \Big)
%\]
%where
%\[
%q = -k -2s(p-1) + \varepsilon, \qquad
%\nu(q) = (-1)^{q(q-1)m/2}(m!)^q, \qquad
%m = (p-1)/2.
%\]
%
%The map $\psi^{\cobar}$, being a special case of the map $\psi^\cube$ defined in \cite{medina2021may_st}, can be described effectively, and it has been implemented in the specialized computer algebra system \texttt{ComCH} \cite{medina2021comch}.
%In similar effective manner, explicit Cartan and Adem coboundaries \cite{medina2020cartan, medina2021adem} can be described in the $p = 2$ case.


%\subsubsection{Cubical chains}
%
%For non-negative integers $m$ and $n$, let $\cube_{\deg}(2^m, 2^n)$ be the subset of \textit{degenerate morphisms} in $\cube(2^m, 2^n)$, i.e., those of the form $\sigma_i \circ \tau$ or $\gamma_i \circ \tau$ with $\tau$ any morphism in $\cube(2^m, 2^{n+1})$.
%The functor of (cubical) \textit{chains} $\cchains \colon \cSet \to \Ch$ is the Yoneda extension of the functor $\cube \to \Ch$ defined next.
%It assigns to an object $2^n$ the chain complex having in degree $m$ the module
%\[
%\frac{\k\{\cube(2^m, 2^n)\}}{\k\{\cube_{\deg}(2^m, 2^n)\}}
%\]
%and differential induced by
%\[
%\partial (\id_{2^n}) = \sum_{i=1}^{n} \ (-1)^i \
%\big(\delta_i^1 - \delta_i^0 \big).
%\]
%To a morphism $\tau \colon 2^n \to 2^{n^\prime}$ it assigns the chain map
%\[
%\begin{tikzcd}[row sep=-3pt, column sep=normal,
%	/tikz/column 1/.append style={anchor=base east},
%	/tikz/column 2/.append style={anchor=base west}]
%	\cchains(\cube^n)_m \arrow[r] & \cchains(\cube^{n^\prime})_m \\
%	\big( 2^m \to 2^n \big) \arrow[r, mapsto] & \big( 2^m \to 2^n \stackrel{\tau}{\to} 2^{n^\prime} \big).
%\end{tikzcd}
%\]
%
%When no confusion arises we write $\chains$ instead of $\cchains$.


%\subsubsection{Cubical singular complex}
%
%Consider the topological $n$-cube
%\[
%\gcube^{n} = \{(x_1, \dots, x_n) \mid x_i \in [0,1]\}.
%\]
%The assignment $2^n \to \gcube^n$ defines a functor $\cube \to \Top$ whose Yoneda extension is known as \textit{geometric realization}.
%It has a right adjoint $\cSing \colon \Top \to \cSet$ given by
%\[
%Z \to \Big(2^n \mapsto \Top(\gcube^n, Z)\Big)
%\]
%and referred to as the \textit{cubical singular complex} of the topological space $Z$.
%We will refer to $\chains(\cSing Z)$ as the \textit{cubical singular chains} of $Z$ and for simplicity denote it by $\cSchains(Z)$.

%\subsubsection{Serre coalgebra}\label{ss:serre coalgebra}
%
%We review a classical lift of the functor of cubical chains to the category of coalgebras
%\[
%\begin{tikzcd}
%	& \coAlg \arrow[d] \\
%	\cSet \arrow[r, "\cchains"] \arrow[ur, "\cchainsA", out=70, in=180] & \Ch.
%\end{tikzcd}
%\]
%Using a Yoneda extension, it suffices to equip the chains on standard cubes with a natural coalgebra structure.
%For any $n \in \N$, define $\epsilon \colon \chains(\cube^n) \to \k$ by
%\[
%\epsilon \left( x_1 \ot \dots \ot x_d \right) = \epsilon(x_1) \dotsm \epsilon(x_n),
%\]
%where
%\[
%\epsilon([0]) = \epsilon([1]) = 1, \qquad \epsilon([0, 1]) = 0,
%\]
%and $\Delta \colon \chains(\cube^n) \to \chains(\cube^n)^{\ot 2}$ by
%\[
%\Delta (x_1 \ot \dots \ot x_n) =
%\sum \pm \left( x_1^{(1)} \ot \dots \ot x_n^{(1)} \right) \ot
%\left( x_1^{(2)} \ot \dots \ot x_n^{(2)} \right),
%\]
%where the sign is determined using the Koszul convention, and we are using Sweedler's notation
%\[
%\Delta(x_i) = \sum x_i^{(1)} \ot x_i^{(2)}
%\]
%for the chain map $\Delta \colon \chains(\cube^1) \to \chains(\cube^1)^{\ot 2}$ defined by
%\[
%\Delta([0]) = [0] \ot [0], \quad \Delta([1]) = [1] \ot [1], \quad \Delta([0,1]) = [0] \ot [0,1] + [0,1] \ot [1].
%\]
%
%Using the canonical isomorphism $\chains(\cube^n) \cong \chains(\cube^1)^{\ot n}$, the coproduct $\Delta$ can be described as the composition
%\[
%\begin{tikzcd}
%	\chains(\cube^1)^{\ot n} \arrow[r, "\Delta^{\ot n}"] & \left( \chains(\cube^1)^{\ot 2} \right)^{\ot n} \arrow[r, "sh"] & \left( \chains(\cube^1)^{\ot n} \right)^{\ot 2},
%\end{tikzcd}
%\]
%where $sh$ is the shuffle map that places tensor factors in odd position first.










%Recall that for any pointed topological space $(\fZ,z)$ there is a naturally associated topological monoid $\loops_z \fZ$ -- its \textit{based loop space} -- whose points are pairs $(\gamma, r)$ where $r \in \R_{\geq 0}$ and $\gamma \colon [0,r] \to \fZ$ is a continuous map with $\gamma(0) = z = \gamma(r)$.
%We consider $\loops_z \fZ$  equipped with the compact-open topology and the continuous multiplication defined by \textit{concatenation} of loops.
\todo{@manuel: this is important because of ...}
%whose identity element is the \textit{constant loop} $(c_r,0)$.
%For any loop $\gamma \colon [0,r] \to \fZ $ its concatenation with the loop $\overline{\gamma} \colon [0,r] \to \fZ$ which runs $\gamma$ in the opposite direction, namely $\overline{\gamma}(t) = \gamma(r-t)$, is homotopic to the constant loop.
%Consequently, we may say that $\loops_z \fZ$ has inverses \textit{up to homotopy}, and we have that the set of path components $\pi_0(\loops_z \fZ)$ has an induced group structure naturally isomorphic to the fundamental group $\pi_1(\fZ,z)$.
%Furthermore, $\loops_z \fZ$ is weak homotopy equivalent to a topological group where the group identities hold strictly \cite{milnor1956bundles}, \cite{berger1995loops}.
%The pointed space $(\fZ,z)$ may then be recovered, up to weak homotopy equivalence, from the weak homotopy type of the topological monoid $\loops_z \fZ$ through the \textit{classifying space}, or \textit{delooping}, construction.

%One of the principal goals of algebraic topology is to encode the theory of topological spaces up to some specified notion of equivalence in terms of combinatorial or algebraic objects, allowing for the effective analysis of topological properties.
%We will focus on spaces up to weak homotopy equivalence, where a first example of an algebraization procedure is provided by the functor of simplicial or cubical singular chains
%\[
%\Schains \colon \Top \to \Ch,
%\]
%which associates a chain complex over a fixed commutative ring to any topological space.

%In this work we are interested in approximating the homotopy type of this topological monoid $\loops_z \fZ$ by monoids in chain complexes with additional structure.

%, a purely algebraic functor
%\[
%\cobar \colon \coAlg^\ast \to \Mon_{\Ch},
%\]

%Diagrammatically, this can be represented as lifts of the form
%\begin{equation}
%	\begin{tikzcd}[column sep=normal, row sep=small]
	%		& \coAlg_{E_\infty} \arrow[d] \\
	%		& \coAlg \arrow[d] \\
	%		\Top \arrow[uur, "\Schains_{E_\infty}", out=90, in=180] \arrow[r, "\Schains"]
	%		\arrow[ur, "\SchainsA", out=60, in=180, near end]
	%		\arrow[r, "\Schains"]
	%		& \Ch.
	%	\end{tikzcd}
%\end{equation}
%The study of $E_\infty$-structures has a long history, where (co)homology operations \cite{steenrod1962cohomology, may1970general}, the recognition of infinite loop spaces \cite{boardman1973homotopy, may1972geometry}, and the complete algebraic representation of the $p$-adic homotopy category \cite{mandell2001padic} are key milestones.

%For based loop spaces we will consider two algebraic models which we now describe for a pointed space $(Z, z)$.
%The first is given by the cubical singular chains $\cSchains(\loops_z Z)$ which, as we recall below, is a monoid in $\Ch$.
%The second, introduced by F.~Adams in \cite{adams1956cobar}, is defined by applying his \textit{cobar construction}
%\[
%\cobar \colon \coAlg^\ast \to \Mon_{\Ch},
%\]
%a functor from coaugemented (dg) coassociative coalgebras to monoids in $\Ch$, to the coalgebra of pointed simplicial singular chains $\sSchainsA(\fZ, z)$.
%In the same reference, Adams constructed a comparison map between these models, a natural chain map of monoids in $\Ch$
%\begin{equation} \label{e:adams map}
%	\theta_Z \colon \cobar \sSchainsA(\fZ, z) \to \cSchains(\loops_z Z),
%\end{equation}
%which he proved to be a quasi-isomorphism if $Z$ is simply connected, a result extended to path-connected spaces in \cite{rivera2018cubical} by considering finer homotopical algebra, and in \cite{hess2010cobar} by a localization procedure.
%
%In this article we enhance both of these models by explicitly constructing an $E_\infty$-bialgebra structure on them, i.e., an $E_\infty$-coalgebra structure compatible with its monoid structure, and show that Adams' comparison map is a quasi-isomorphism of $E_\infty$-bialgebras.

%Our starting point is groundbreaking work by H.~Baues proving similar statements to ours.
%that allowed him to accomplish analogous statements using monoids in coalgebras instead of in $E_\infty$-coalgebras.
%His insights are best explained in terms of \textit{cubical sets} (with connections).
%, i.e., objects in the category $\cSet$ of presheaves over the strict monoidal category generated by the diagram
%\[
%\begin{tikzcd}
%	1 \arrow[r, out=45, in=135, "\delta^0"] \arrow[r, out=-45, in=-135, "\delta^1"'] & 2 \arrow[l, "\sigma"'] &[-10pt] \arrow[l, "\gamma"'] 2 \times 2
%\end{tikzcd}
%\]
%restricted by certain relations \cite{brown1981cubes}.
%With the Day convolution monoidal structure on $\cSet$, the cubical singular complex functor $\cSing \colon \cSet \to \Top$ and the functor of (normalized) chains $\cchains \colon \cSet \to \Ch$ are both monoidal, which explains the monoid structure on
%\[
%\cSchains(\loops_z Z) \defeq \cchains \cSing(\loops_z Z).
%\]
%Furthermore, cubical chains are equipped with a natural coalgebra structure coming from Serre's chain approximation to the diagonal.
%This structure provides a lift of $\cchains$ to the category of coalgebras via a monoidal functor
%\[
%\cchainsA \colon \cSet \to \coAlg.
%\]
%In particular, this construction makes $\cSchainsA(\loops_z Z) = \cchainsA\cSing(\loops_z Z)$ into a bialgebra, i.e., a monoid in $\coAlg$.



%Furthermore, Adams' comparison map
%\[
%\theta_Z \colon \cobar \sSchainsA(\fZ, z) \to \cSchains(\loops_z Z)
%\]
%is a quasi-isomorphism of $E_{\infty}$-bialgebras for any pointed space $(\fZ, z)$.
%
%Regarding Adams' comparison map, Baues showed that it factors as a composition
%\begin{equation} \label{e:baues factorization of adams map}
%	\begin{tikzcd}[column sep = small]
	%		\theta_Z \colon \cobar \sSchainsA(\fZ, z) \arrow[r] &
	%		\cchainsA\big(\!\ccobar \sSing(\fZ, z)\big) \arrow[r] &
	%		\cSchainsA(\loops_z Z),
	%	\end{tikzcd}
%\end{equation}
%where the first chain map is an isomorphism and the second is induced from an inclusion
%\[
%\ccobar \sSing(\fZ, z) \to \cSing(\loops_z Z)
%\]
%of monoids in $\cSet$, which is a weak equivalence following \cite{rivera2019path}.
%
%The first map in \eqref{e:baues factorization of adams map} is defined for a general reduced simplicial set $X$ and it is always an isomorphism.
%This permitted Baues to transfer the Serre coalgebra structure on $\cchainsA \ccobar(X)$ to $\cobar \schainsA(X)$ and show it compatible with the monoid structure.
%In other words, Baues' lifted Adams' cobar construction $\cobar \schainsA$ so that it fits in the following diagram commuting up to a natural isomorphism:
%\[
%\begin{tikzcd}
%	\Mon_{\cSet} \arrow[r, "\cchainsA \ "] & \Mon_{\coAlg} \arrow[d] \\
%	\sSet^0 \arrow[r, "\cobar \schainsA"] \arrow[u, "\ccobar"] & \Mon_{\Ch}.
%\end{tikzcd}
%\]
%
%The first result presented in this article is the construction of a lift of $\cchainsA \ccobar \colon \sSet^0 \to \Mon_{\coAlg}$ to the category of monoids in $\coAlg_\UM$, where $\coAlg_\UM$ is the category of coalgebras over the operad $\UM$ introduced via a finitely presented prop in \cite{medina2020prop1}.
%This is a model of the $E_\infty$-operad that explicitly acts on both simplicial and cubical chains.
%For this lift to be meaningful, we first need to define a monoidal structure on $\coAlg_\UM$.
%We do so by introducing a Hopf structure on $\UM$, that is to say, by making it into an operad over $\coAlg$, a structure that has other potential applications \cite{livernet2008hopf}.
%Once we have shown that $\coAlg_\UM$ is a monoidal category, we prove the main technical result of this work; that the lift
%\[
%\cchainsUM \colon \cSet \to \coAlg_\UM
%\]
%of $\cchainsA$ defined in \cite{medina2022cube_einfty} is monoidal.
%
%This has consequences for both algebraic models for the based loop space considered so far.
%Let $(\fZ, z)$ be a pointed space and $X$ a reduced simplicial set, for example $X = \sSing(\fZ, z)$.
%Since $\cSing(\loops_z Z)$ and $\ccobar X$ are monoids in $\cSet$ we have that both $\cSchainsUM(\loops_z Z)$ and $\cchainsUM \ccobar(X) \cong \cobar \schainsA(X)$ are monoids in $\coAlg_\UM$.
%Furthermore, if $X = \sSing(\fZ, z)$, Adams' comparison map $\theta_Z$ preserves this additional structure.
%We condense this discussion in our first main result.
%
%\begin{theorem*}
%	The functor $\cchainsUM$ is monoidal and its associated functor on monoids fits in the following diagram commuting up to a natural isomorphism:
%	\[
%	\begin{tikzcd} [row sep=small]
	%		& \Mon_{\coAlg_\UM} \arrow[d] \\
	%		\Mon_{\cSet} \arrow[ru, "\cchainsUM", out=70, in=180, near start] \arrow[r, "\cchainsA"]
	%		& \Mon_{\coAlg} \arrow[d] \\
	%		\sSet^0 \arrow[r, "\cobar \schainsA"] \arrow[u, "\ccobar"]
	%		& \Mon_{\Ch}.
	%	\end{tikzcd}
%	\]
%	Furthermore, Adams' comparison map
%	\[
%	\theta_Z \colon \cobar \sSchainsA(\fZ, z) \to \cSchains(\loops_z Z)
%	\]
%	is a quasi-isomorphism of $E_{\infty}$-bialgebras for any pointed space $(\fZ, z)$.
%\end{theorem*}
%
%\begin{theorem*}
%	There is a lift of the
%	\[
%	\begin{tikzcd} [row sep=small]
	%		& \Mon_{\coAlg_\UM} \arrow[d] \\
	%		\Mon_{\cSet} \arrow[ru, "\cchainsUM", out=70, in=180, near start] \arrow[r, "\cchainsA"]
	%		& \Mon_{\coAlg} \arrow[d] \\
	%		\sSet^0 \arrow[r, "\cobar \schainsA"] \arrow[u, "\ccobar"]
	%		& \Mon_{\Ch}.
	%	\end{tikzcd}
%	\]
%	Furthermore, Adams' comparison map
%	\[
%	\theta_Z \colon \cobar \sSchainsA(\fZ, z) \to \cSchains(\loops_z Z)
%	\]
%	is a quasi-isomorphism of $E_{\infty}$-bialgebras for any pointed space $(\fZ, z)$.
%\end{theorem*}

%\subsection*{Relation to previous work}
%
%Using more abstract methods, in \cite{hess2006adamshilton} Hess, Parent, Scott, and Tonks constructed a coproduct on $\cobar \schainsA(X)$ for any $1$-reduced simplicial set $X$, which coincides with Baues' coproduct.
%A priori, their methods imply that this coproduct corresponds, up to homotopy, to the Alexander--Whitney coproduct on $\schains(\kan(X))$, a result that Franz showed to hold strictly in \cite{franz2020szczarba}.
%Our approach allows us to go beyond the coalgebra structure and compare directly the cobar construction and the chains on the Kan loop group as $E_{\infty}$-coalgebras.
%
%The works of V.A. Smirnov \cite{smirnov1990iterated}, J.R. Smith \cite{smith1994cobar, smith2000operads}, T. Kadeishvili and S. Saneblidze \cite{kadeishvili1998iterating} predict the existence of an $E_\infty$-coalgebra structure on the cobar construction on the chains of a reduced simplicial set $\cobar \schainsA(X)$.
%In \cite{fresse2010bar}, B. Fresse used a model category structure on reduced operads \cite{berger2003modelcategory, hinich1997homologicalalgebra} to iterate the bar construction on algebras over cofibrant $E_\infty$-operads.
%We mention that the reduced version of the $E_\infty$-operad $\UM$ is cofibrant, although we do not use this observation here.
%In the work of Fresse, finiteness assumptions and restrictions on the fundamental group are required to model the based loop space.
%
%The results we have surveyed establish the existence of higher structures on the bar and cobar constructions associated to reduced simplicial sets.
%We now review part of the history of the problem of describing such structures effectively, and our contribution to it.
%Kadeishvili described explicitly Steenrod cup-$i$ coproducts on $\cobar \schainsA(X)$ compatible with its monoid structure \cite{kadeishvili1999coproducts, kadeishvili2003cupi}.
%Kadeishvili, as Baues, used cubical methods to define these operations and to compare them, in the $1$-connected setting, to cup-$i$ coproducts extending the Serre coalgebra structure on the cubical singular chains on the based loop space.
%Kadeishvili's cup-$i$ coproducts can be obtained from the arity 2 part of our action of $\UM$ on the cobar construction.
%Taking Kadeishvili's work as a starting point, Fresse provided the bar construction of an algebra over the surjection operad with the structure of a comonoid in the category of algebras over the Barratt--Eccles operad \cite{fresse2003hopf}.
%Fresse then proves that, for any simplicial set $X$ with suitable restrictions on its fundamental group and finiteness assumptions on its cohomology, the bar construction of the surjection algebra of simplicial cochains on $X$ is quasi-isomorphic to the singular cochains on the based loop space of $|X|$ as an $E_{\infty}$-algebra.
%
%Our results in the first part of the present article are similar to those obtained by Fresse in \cite{fresse2003hopf} and described above.
%However, we make use of coalgebras instead of algebras, which allows us to relate our constructions to the based loop space without imposing restrictions on the underlying homotopy type.
%Furthermore, by grounding our approach on the cubical perspective at the heart of Adams' and Baues' seminal papers, we are also able to relate the cobar construction and the singular chains on the based loop space as $E_{\infty}$-bialgebras, i.e. as monoids in the category of $E_{\infty}$-coalgebras, and not just as $E_{\infty}$-coalgebras.
%
%\subsection*{Outline}
%
%In \cref{s:preliminaries}, we recall well known algebraic and categorical definitions and constructions that will be used throughout this article.
%We review, in \cref{s:operads and props}, the foundations of the theory of operads and props with the goal of recalling the $E_{\infty}$-operad $\UM$ and its actions on simplicial and cubical chains.
%We construct a Hopf operad structure on $\UM$ in \cref{s:monoidal}, and show that the cubical chains functor from cubical sets to $\UM$-coalgebras is monoidal.
%We prove \cref{t:1st main thm in the intro} in \cref{s:theorem1} using Baues' cubical cobar construction \eqref{e:cubical cobar construction} and the $\UM$-structure on cubical chains.