
%\subsection{Necklaces} For any ordered sequence of positive integers $(n_1,\dots,n_k)$ we denote by $\Delta^{n_1} \vee...\vee \Delta^{n_k} \in \sSet$ the simplicial set obtained by identifying the last vertex of $\Delta^{n_i}$ with the first vertex of $\Delta^{n_{i+1}}.$ A simplicial set of the form $T=\Delta^{n_1} \vee...\vee \Delta^{n_k}$, for $n_i>0$ if $k>1$ and $n_i \geq 0$ if $k=1$, is called a \textit{necklace}. For any necklace $T= \Delta^{n_1} \vee...\vee \Delta^{n_k}$ we call each $\Delta^{n_i}$ the \textit{$i$-th bead} of $T$. The set $V_T=T_0$ will be called the \textit{vertices} of the necklace $T$. We denote by $J_T$ to be the subset of $V_T$ consisting of those vertices which are the first or final vertex of some bead; the elements of $J_T$ are called \textit{joints}. For any necklace $T$, the ordering of the vertices of each bead together with the ordering of the beads of $T$ induces a total ordering on $T_0.$ We denote the first and last vertices of $T$ by $\alpha_T$ and $\omega_T$, respectively. The \textit{dimension} of a necklace $T=\Delta^{n_1} \vee...\vee \Delta^{n_k}$ is defined to be $\text{dim}(T)=n_1 + ... +n_k-k.$

%Necklaces form a category $\Nec$ with morphisms being maps of simplicial sets $f \colon T \to T'$ that preserve the first and last vertices, i.e. satisfying $f(\alpha_T)=\alpha_{T'}$ and $f(\omega_T)=\omega_{T'}.$ A \textit{neckilcal set} is a functor $K \colon \Nec^{op} \to \Set.$ Necklical sets form a category, denoted by $\nSet$, with morphisms being natural transformations. Denote by $$Y \colon \Nec \to \nSet$$ the Yoneda embedding, namely the functor given by $\mathcal{Y}(T)(T')= \text{Hom}_{\Nec}(T',T)$ for any $T,T' \in \Nec.$

%The category of necklical sets $\nSet$ is a (non-symmetric) monoidal category when equipped with monoidal structure $\otimes\colon \Nec \times \Nec \to \Nec$ given by $$K \otimes K' = \colim_{\mathcal{Y}(T) \to K, Y(T')\to K'} Y(T \vee T').$$

%We recall Proposition 3.1 of \cite{Rivera and Zeinalian, cubical rigidification}, which describes a set of generators for the morphisms in $\Nec.$

%\begin{proposition}\label{Necmorphs} Any non-identity morphism in $\Nec$ is a composition of morphisms of the following type
%\begin{item}
%\item (i)  $f \colon T \to T'$ is an injective morphism of necklaces and $ \text{dim}(T')- \text{dim}(T)=1$
%\\
%\item (ii) $f \colon \Delta^{n_1} \vee ... \vee \Delta^{n_k} \to \Delta^{m_1} \vee ... \vee \Delta^{m_k}$ is a morphism of necklaces of the form $f=f_1 \vee ... \vee f_k$ such that for exactly one $p \in \{1,\dots,k\}$, $m_p=n_p-1$ and $f_p= s^j \colon \Delta^{n_p} \to \Delta^{m_p}= \Delta^{n_p-1}$ is a codegeneracy morphism for some $j \in \{0,\dots,n_p-1\}$, and for all $i \neq p$, $n_i=m_i$ and $f_i=id_{\Delta^{n_i}}$. 
%\\
%\item (iii) $f \colon \Delta^{n_1} \vee ...\vee \Delta^{n_{p-1}} \vee \Delta^1 \vee \Delta^{n_{p+1}} \vee... \vee  \Delta^{n_k} \to \Delta^{n_1} \vee ...\vee \Delta^{n_{p-1}} \vee \Delta^{n_{p+1}} \vee... \vee  \Delta^{n_k}$ is a morphism of necklaces such that $f$ collapses the $p$-th bead $\Delta^1$  in the domain to the last vertex of the $(p-1)$-th bead in the target and the restriction of $f$ to all the other beads is injective. 
%\end{item}
%\end{proposition}

%\begin{remark}\label{cofaces} Let $T= \Delta^{n_1} \vee ... \vee \Delta^{n_k}$ and $T'= \Delta^{m_1}\vee ... \vee \Delta^{m_l}$ be two necklaces such that $\text{dim}(T')-\text{dim}(T)=1$. Morphisms $f \colon T \to T'$ of type $(i)$ as in the above proposition can be of one of the following forms \colon
%\begin{item}
%\item $(ia)$ We have $k=l$ and there exists $p \in \{1,\dots,k\}$ such that $n_p= m_p - 1$ and $n_i = m_i$ for all $i \neq p$. The morphism $f \colon T\to T'$ is of the form $$f=id_{\Delta^{n_1}} \vee ... \vee id_{\Delta^{n_{p-1}}} \vee \partial^j \vee id_{\Delta^{n_{p+1}}} \vee id_{\Delta^{n_k}},$$ where $\partial^j \colon \Delta^{m_p-1} \to \Delta^{m_p}$ is the $j$-th coface map for some $j \in \{1,\dots,m_p-1\}.$ 

%\item $(ib)$ We have $k=l+1$ and there exists $p \in \{1,,\dots,k\}$ such that $n_p + n_{p+1} = m_p$ and $n_i=m_i$ for all $i \neq p, p+1$. The morphism $f \colon T \to T'$ is of the form
%$$f=id_{\Delta^{n_1}} \vee ... \vee id_{\Delta^{n_{p-1}}} \vee f_p \vee id_{\Delta^{n_{p+2}}} \vee id_{\Delta^{n_k}},$$
%where $f_p \colon \Delta^{n_p} \vee \Delta^{n_{p+1}} \to \Delta^{n_p + n_{p+1}}=\Delta^{m_p}$ is an injective map of necklaces. 
%\end{item}
%\end{remark} 

\subsection{Necklaces} We follow section 4.3 of \cite{galvez2020hopf} to define the category of necklaces. First let $\mathbf{\Delta}_{*,*}$ be the subcategory of $\mathbf{\Delta}$ consisting of objects $\{ [1], [2],\dots\}$ and morphisms being functors $f \colon [n] \to [m]$ satisfying $f(0)=0$ and $f(n)=m$. The category $\mathbf{\Delta}_{*,*}$ is a strict monoidal category when equipped with the monoidal structure $[n] \otimes [m]= [n+m]$ given by identifying objects $n \in [n]$ and $0 \in [m]$.

Given any category $\mathcal{C}$ one can generate a free strict monoidal category $\mathcal{B}(\mathcal{C})$ with monoid structure denoted by $\vee$. The  objects of $\mathcal{B}(\mathcal{C})$ are ordered sequences $C_1 \vee ... \vee C_k$ of objects $C_i \in ob(\mathcal{C})$ together with morphisms given by ordered sequences $$(f_1 \vee ... \vee f_k) \colon C_1 \vee ... \vee C_k \to C'_1 \vee ... \vee C'_k$$ of morphisms $f_i \colon C_i \to C'_i$ in $\mathcal{C}.$ The monoidal structure $\vee$ is then given by concatenation of sequences and the empty sequence is the identity object. If $\mathcal{C}$ is equipped with a monoidal structure $\otimes \colon \mathcal{C} \times \mathcal{C} \to C$ we may define a new category $\mathcal{B}(C, \otimes)$ by formally adding morphisms $\Delta_{C,C'} \colon C \vee C' \to C \otimes C'$, for all objects $C, C' \in \mathcal{C},$ to $\mathcal{B}(\mathcal{C}).$
 
Define the \textit{category of necklaces} $\Nec$ to be the category  $\mathcal{B}(\mathbf{\Delta}_{*,*}, \otimes)$, where $[n] \otimes [m]=[n+m]$, as defined above. More explicitly, the objects of $\Nec$ are ordered sequences $[n_1] \vee ... \vee[n_k]$ and there are four types of generating morphisms in $\Nec$
\begin{itemize}
\item $\partial^j \colon [n-1] \to  [n] $
for $j=1,\dots,n-1$
\item $\Delta_{[j],[n-j]} \colon  [j]\vee [n-j] \to [n]$ for $j=1,\dots,n-1$
\item $ s^j \colon [n+1] \to [n]$ for $j=0,\dots,n$ and $n>0$, and 
\item $s^0 \colon [1] \to [0]$
\end{itemize}

The objects of $\Nec$ are called \textit{necklaces}. Given any necklace $T=[n_1] \vee ... \vee[n_k] \in \Nec$ define the \textit{dimension} of $T$ by $\text{dim}(T)=n_1+ ...+n_k-k$. A \textit{necklical set} is a functor $K \colon \Nec^{op} \to \Set$. Necklical sets form a category with natural transformations as morphisms, which we denote by $\nSet.$ Any necklace $T=[n_1] \vee ... \vee[n_k] \in \Nec$ gives rise to a simplicial set $\mathcal{S}(T)= \Delta^{n_1} \vee... \vee \Delta^{n_k} \in \sSet$, where the wedge symbol now means we identify the last vertex of $\Delta^{n_i}$ with the first vertex of $\Delta^{n_{i+1}}$ for $i=1,\dots,k-1$. Any necklace $T \in \Nec$ also gives rise to a necklical set $\mathcal{Y}(T)=\Hom_{\Nec}(\text{ \_ }, T) \colon \Nec^{op} \to \Set$. These constructions give rise to functors 
$$\mathcal{S} \colon \Nec \to \sSet$$
and
$$\mathcal{Y} \colon \Nec \to \nSet,$$ respectively. 

The category of necklical sets $\nSet$ is a (non-symmetric) monoidal category when equipped with monoidal structure $\otimes\colon \nSet \times \nSet \to \nSet$ given by $$K \otimes K' = \colim_{\mathcal{Y}(T) \to K, \mathcal{Y}(T')\to K'} \mathcal{Y}(T \vee T').$$

We now construct a monoidal functor
$$\mathcal{P} \colon \Nec \to \cube.$$

If $T= \Delta^0$ we set $\mathcal{P}(\Delta^0)=2^0$ and on any other $T \in \Nec$ we define $\mathcal{P}( T )= 2^{\text{dim}(T)}$. In order to define $\mathcal{P}$ on morphisms, it is sufficient to consider the following cases.
\begin{itemize}
\item For any $\partial^j \colon [n] \to [n+1]$ such that $0< j<{n+1}$, define $\mathcal{P}(\partial^j) \colon 2^{n-1}\to 2^{n}$ to be the cubical coface functor $\mathcal{P}(f)= \delta_0^{j}.$ 

\item For any $\Delta_{[j], [n+1-j]} \colon [j] \vee [n+1-j] \to [n+1]$ such that $0<j<n+1$, define $\mathcal{P}(\Delta_{[j], [n+1-j]}) \colon 2^{n-1}\to 2^{n}$ to be the cubical coface functor $\mathcal{P}(f)=\delta_1^{j}$.

\item We now consider morphisms of the form $s^j \colon [n+1] \to [n]$ for $n>0$. If $j=0$ or $j=n$, then $\mathcal{P}(f) \colon 2^n \to 2^{n-1}$ is defined to be the cubical codegeneracy functor $\mathcal{P}(s^j)= \varepsilon^{j}.$ If $0<j<n$, then we define $\mathcal{P}(s^j) \colon 2^n \to 2^{n-1}$ to be the cubical coconnection functor $\gamma^{j}.$

\item For $s^0 \colon [1] \to [0]$ we define $\mathcal{P}(s^0) \colon 2^0 \to 2^0$ to be the identity functor.

\end{itemize}
\begin{remark}
The functor $\mathcal{P}$ is neither faithful or full. However, for any necklace $T' \in \Nec$ with $\text{dim}(T')=n+1$ and any cubical coface functor $\delta_{\epsilon}^j \colon 2^n \to 2^{n+1}$ for $0 \leq j \leq n+1$, there exists a unique pair $(T, f \colon T \hookrightarrow T')$, where $T \in \Nec$ with $\text{dim}(T)=n$ and $f \colon T \hookrightarrow T'$ is an injective morphism in $\Nec$, such that $\mathcal{P}(f)=\delta_{\epsilon}^j $.
\end{remark}

The functor $\mathcal{P} \colon \Nec \to \cube$ induces an adjunction between $\cSet$ and $\nSet$
with right adjoint
$$\mathcal{P}^* \colon \cSet \to \nSet$$
and left adjoint
$$\mathcal{P}_{!}  \colon \nSet \to \cSet_c.$$
Given a cubical set with connections $C \colon \square_c^{op} \to \Set$, we have $$\mathcal{P}^*(C)= C \circ \mathcal{P}^{op}.$$ Given a necklical set $K \colon \Nec^{op} \to \Set$, we have $$\mathcal{P}_{!}(K)= \colim_{(\mathcal{Y}(T) \to K) \in \mathcal{Y} \downarrow K} \mathcal{P}(T) \cong \colim_{(\mathcal{Y}(T) \to K) \in \mathcal{Y} \downarrow K} \cube^{\text{dim}(T)}.$$ 
The functor $\mathcal{P}_{!}: \nSet \to \cSet$ is clearly a monoidal functor. 

\subsection{A monoidal necklical set model for the based loop space}
We define a functor 
$$\gcobar \colon \sSet^0 \to \Mon_{\nSet}.$$
For any $0$-reduced simplicial set $X$, the underlying necklical set $\gcobar(X) \colon \Nec^{op} \to \Set$ is given by
$$\gcobar(X) = \colim_{(f \colon \mathcal{S}(T) \to X) \in  \mathcal{S} \downarrow X} \mathcal{Y}(T).$$
The monoidal structure $$\gcobar(X) \otimes \gcobar(X) \to \gcobar(X)$$
is induced by the monoidal structure $\vee: \Nec \times \Nec \to \Nec$ together with the monoidal structure $\otimes: \nSet \times \nSet \to \nSet$ on necklicals sets. More precisely, for any $S, S' \in \Nec$, the product $$\gcobar(X)(S) \otimes \gcobar(X)(S') \to \gcobar(X)(S \vee S')$$ is given by $$[f\colon \mathcal{S}(T) \to X, a] \otimes [f'\colon \mathcal{S}(T') \to X, a'] \mapsto [f \vee g\colon \mathcal{S}(T\vee T') \to X, a \otimes  a'],$$
where $(f\colon \mathcal{S}(T) \to X), (f'\colon \mathcal{S}(T') \to X) \in \mathcal{S} \downarrow X$, $a\in \mathcal{Y}(T)(S)$, and $a'\in \mathcal{Y}(T')(S')$.
We may formally invert all the $1$-simplices of $X$ inside $\gcobar(X)$ to obtain a localized version of $\gcobar$ as follows. Define 
$$\widehat{\gcobar} \colon \sSet^0 \to \Mon_{\nSet}$$
by letting
$$\widehat{\gcobar}(X)= \mathcal{L}_{X_1}\gcobar(X),$$
namely $\widehat{\gcobar}(X)$ is the monoidal necklical set obtained by adding formal inverses for all $f\colon \Delta^1 \to X$ subject to the usual relations. This results in inverting all $(f: \mathcal{S}(T) \to X) \in \mathcal{S} \downarrow X$ such that $\text{dim}(T)=0.$

\begin{remark}
We give a few remarks regarding the functor $\gcobar \colon \sSet^0 \to \Mon_{\nSet}$.
\\
$i)$ We may define a triangulation functor $\mathcal{T} \colon \cSet \to \sSet$ by $$\mathcal{T}(C) = \colim_{\square^n_c \to C} (\Delta^1)^{\times n}.$$
The composition of functors
$$\mathcal{T} \circ \mathcal{P}_! \circ  \gcobar\colon \sSet^0 \to \Mon_{\sSet}$$ coincides with the \textit{rigidification functor} introduced by Cordier and studied by Lurie, Dugger and Spivak in the theory of $\infty$-categories \cite{Rivera-Zeinalian}. The rigidification functor is the left adjoint of the homotopy coherent nerve functor. 
\\
$(ii)$ For any $X \in \sSet^0$, we also obtain a simplicial monoid $\mathcal{T}\mathcal{P}_! \widehat{\gcobar}(X) \in \Mon_{\sSet}$ and the monoid $\pi_0( \mathcal{T}\mathcal{P}_! \widehat{\gcobar}(X))$ has the property of being a group.
\\
$(iii)$ In the case of arbitrary (not necessarily $0$-reduced) simplicial sets, we may extend  $\gcobar$ to a functor
$$\sSet \to \mathsf{Cat}_{\nSet}$$ from the category of simpicial sets to the category of small categories enriched over the monoidal category of necklical sets. Similarly, we may extend $\widehat{\gcobar}$ to obtain combinatorial model for the path category of an arbitrary simplicial set.
\end{remark}

The main feature of the combinatorial model given by $\gcobar(X)$ is that it coincides with the cobar construction after applying the normalized cubical chains functor. 

\begin{proposition}
There is a natural isomorphisms of functors 
$$\cchains \circ \mathcal{P}_! \circ \gcobar \cong \mathbf{\Omega} \circ \chains: \sSet^0 \to \Mon_{\Ch}.$$
\end{proposition}

\begin{proof} 
Denote by $\iota_n \in (\square^n_c)_n$ the top dimensional non-degenerate element of the standard $n$-cube with connections $\square^n_c$. Note that for any $X \in \sSet^0$, we may represent any non-degenerate $n$-cube $\alpha \in (\mathcal{P}_!(\gcobar(X)))_n$ as a pair $\alpha=[\sigma: \mathcal{Y}(T) \to X, \iota_n]$ for some $T=[n_1] \vee ... \vee [n_k] \in \Nec$ with $\text{dim}(T)=n_1+ ...+n_k-k=n.$

To define an algebra map
$$\varphi_X: \cchains(\mathcal{P}_!(\gcobar(X))) \xrightarrow{\cong} \mathbf{\Omega}(\chains(X))$$
it suffices to define it on any generator of the form $\alpha=[\sigma \colon \Delta^{n+1} \to X, \iota_{n}]$, i.e. when $T$ is of the form $T=[n+1]$, for some $n\geq0$. If $n=0$ let $\varphi_X(\alpha)= [\overline{\sigma}]+ 1_R$, where $[\overline{\sigma}] \in s^{-1} \overline{ \chains(X)} \subset \mathbf{\Omega}(\chains(X))$ denotes the (length $1$) generator in the cobar construction of $\chains(X)$ determined by $\sigma \in X_{n+1}$ and $1_R$ denotes the unit of the underlying ring $R$. If $n>0$, we let $\varphi_X(\alpha)=[\overline{\sigma}]$. A straightforward computation yields that this gives rise to a well defined isomorphism of algebras, which is compatible with the differentials, and natural with respect to maps of simplicial sets.  
\end{proof}
In \cite{Hess and Tonks}, a localized version of the cobar construction of the chains on a $0$-reduced simplicial set was introduced. This construction is not quite functorial on $\coAlg$ since it depends on a choice of basis of the degree $1$ $R$-module of the underlying coalgebra. However, one may define a functor $$\widehat{\mathbf{\Omega}}: \sSet^0 \to \Mon_{\Ch}$$
defined by formally inverting the set of cycles $\{ [\overline{\sigma}]+1_R | \sigma \in X_1 \} \subset \mathbf{\Omega}(\chains(X))_0$ in the associative algebra $\mathbf{\Omega}(\chains(X))$. Hess and Tonks showed that there is a natural chain homotopy equivalence of associative algebras $$\widehat{\mathbf{\Omega}}(X)\simeq \chains(GX),$$ where $GX$ denotes the Kan loop group of $X$. As an immediate consequence of our previous result, we obtain the following localized version of the isomorphism.

\begin{corollary}
There is a natural isomorphism of functors
$$\cchains \circ \mathcal{P}_! \circ \widehat{\gcobar} \cong \widehat{\mathbf{\Omega}} :\sSet^0 \to \Mon_{\Ch},$$ where $\widehat{\mathbf{\Omega}}$ denotes the extended cobar construction of \cite{Hess and Tonks}. 
\end{corollary}




