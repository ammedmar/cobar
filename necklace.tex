
\subsection{Necklaces} For any ordered sequence of positive integers $(n_1,...,n_k)$ we denote by $\Delta^{n_1} \vee...\vee \Delta^{n_k} \in \sSet$ the simplicial set obtained by identifying the last vertex of $\Delta^{n_i}$ with the first vertex of $\Delta^{n_{i+1}}.$ A simplicial set of the form $T=\Delta^{n_1} \vee...\vee \Delta^{n_k}$, for $n_i>0$ if $k>0$ and $n_i \geq 0$ if $k=0$, is called a \textit{necklace}. For any necklace $T= \Delta^{n_1} \vee...\vee \Delta^{n_k}$ we call each $\Delta^{n_i}$ the \textit{$i$-th bead} of $T$. The set $V_T=T_0$ will be called the \textit{vertices} of the necklace $T$. We denote by $J_T$ to be the subset of $V_T$ consisting of those vertices which are the first or final vertex of some bead; the elements of $J_T$ are called \textit{joints}. For any necklace $T$, the ordering of the vertices of each bead together with the ordering of the beads of $T$ induces a total ordering on $T_0.$ We denote the first and last vertices of $T$ by $\alpha_T$ and $\omega_T$, respectively. The dimension of a necklace $T=\Delta^{n_1} \vee...\vee \Delta^{n_k}$ is defined to be $\text{dim}(T)=n_1 + ... +n_k-k.$

Necklaces form a category $\Nec$ with morphisms being maps of simplicial sets $f: T \to T'$ that preserve the first and last vertices, i.e. satisfying $f(\alpha_T)=\alpha_{T'}$ and $f(\omega_T)=\omega_{T'}.$ A \textit{neckilcal set} is a functor $K: \Nec^{op} \to \Set.$ Necklical sets form a category, denoted by $\nSet$, with morphisms being natural transformations. Denote by $$Y: \Nec \to \nSet$$ the Yoneda embedding, namely the functor given by $Y(T)(T')= \text{Hom}_{\Nec}(T',T)$ for any $T,T' \in \Nec.$

The category of necklical sets $\nSet$ is a (non-symmetric) monoidal category when equipped with product $\otimes$ given by $$K \otimes K' = \colim_{Y(T) \to K, Y(T')\to K'} Y(T \vee T').$$

Necklaces have been studied in Baues, Dugger and Spivak, Rivera and Zeinalian,  among others. We recall Proposition 3.1 of \cite{Rivera and Zeinalian, cubical rigidification}, which describes a set of generators for the morphisms in $\Nec.$

\begin{proposition}\label{Necmorphs} Any non-identity morphism in $\Nec$ is a composition of morphisms of the following type
\begin{item}
\item (i)  $f: T \to T'$ is an injective morphism of necklaces and $ \text{dim}(T')- \text{dim}(T)=1$
\\
\item (ii) $f: \Delta^{n_1} \vee ... \vee \Delta^{n_k} \to \Delta^{m_1} \vee ... \vee \Delta^{m_k}$ is a morphism of necklaces of the form $f=f_1 \vee ... \vee f_k$ such that for exactly one $p \in \{1,...,k\}$, $m_p=n_p-1$ and $f_p= s^j: \Delta^{n_p} \to \Delta^{m_p}= \Delta^{n_p-1}$ is a codegeneracy morphism for some $j \in \{0,...,n_p-1\}$, and for all $i \neq p$, $n_i=m_i$ and $f_i=id_{\Delta^{n_i}}$. 
\\
\item (iii) $f: \Delta^{n_1} \vee ...\vee \Delta^{n_{p-1}} \vee \Delta^1 \vee \Delta^{n_{p+1}} \vee... \vee  \Delta^{n_k} \to \Delta^{n_1} \vee ...\vee \Delta^{n_{p-1}} \vee \Delta^{n_{p+1}} \vee... \vee  \Delta^{n_k}$ is a morphism of necklaces such that $f$ collapses the $p$-th bead $\Delta^1$  in the domain to the last vertex of the $(p-1)$-th bead in the target and the restriction of $f$ to all the other beads is injective. 
\end{item}
\end{proposition}

\begin{remark}\label{cofaces} Let $T= \Delta^{n_1} \vee ... \vee \Delta^{n_k}$ and $T'= \Delta^{m_1}\vee ... \vee \Delta^{m_l}$ be two necklaces such that $\text{dim}(T')-\text{dim}(T)=1$. Morphisms $f: T \to T'$ of type $(i)$ as in the above proposition can be of one of the following forms:
\begin{item}
\item $(ia)$ We have $k=l$ and there exists $p \in \{1,...,k\}$ such that $n_p= m_p - 1$ and $n_i = m_i$ for all $i \neq p$. The morphism $f: T\to T'$ is of the form $$f=id_{\Delta^{n_1}} \vee ... \vee id_{\Delta^{n_{p-1}}} \vee \partial^j \vee id_{\Delta^{n_{p+1}}} \vee id_{\Delta^{n_k}},$$ where $\partial^j: \Delta^{m_p-1} \to \Delta^{m_p}$ is the $j$-th coface map for some $j \in \{1,...,m_p-1\}.$ 

\item $(ib)$ We have $k=l+1$ and there exists $p \in \{1,,...,k\}$ such that $n_p + n_{p+1} = m_p$ and $n_i=m_i$ for all $i \neq p, p+1$. The morphism $f: T \to T'$ is of the form
$$f=id_{\Delta^{n_1}} \vee ... \vee id_{\Delta^{n_{p-1}}} \vee f_p \vee id_{\Delta^{n_{p+2}}} \vee id_{\Delta^{n_k}},$$
where $f_p: \Delta^{n_p} \vee \Delta^{n_{p+1}} \to \Delta^{n_p + n_{p+1}}=\Delta^{m_p}$ is an injective map of necklaces. 
\end{item}
\end{remark} 

We give an equivalent description of the category $\Nec$ in terms of a construction described in section 4.3 of \cite{Galvez-Carillo, Kaufmann, Tonks}. Let $\mathbf{\Delta}_{*,*}$ be the subcategory of $\mathbf{\Delta}$ with the same set of objects $\{ [0], [1], [2],...\}$ and morphisms being functors $f: [n] \to [m]$ satisfying $f(0)=0$ and $f(n)=m$. The category $\mathbf{\Delta}_{*,*}$ is a strict monoidal category when equipped with the monoidal structure $[n] \otimes [m]= [n+m]$ given by identifying objects $n \in [n]$ and $0 \in [m]$.

Given any monoidal category $(\mathcal{C}, \otimes)$ one can generate a free monoidal category $\mathcal{F}(\mathcal{C}, \otimes)$ with objects being ordered sequences $C_1 \vee ... \vee C_k$ of objects $C_i \in ob(\mathcal{C})$ together with morphisms given by ordered sequences $(f_1 \vee ... \vee f_k): C_1 \vee ... \vee C_k \to C'_1 \vee ... \vee C'_k$ of morphisms $f_i: C_i \to C'_i$ in $\mathcal{C}.$ The monoidal structure is then given by concatenation of sequences. 

By identifying necklaces $\Delta^{n_1} \vee ... \vee \Delta^{n_k}$ with sequences $[n_1] \vee ... \vee [n_k]$,
we may obtain $\Nec$ by formally adding morphisms to $\mathcal{F}( \mathbf{\Delta}_{*,*}, \otimes)$ as described in the following proposition. 

\begin{proposition}\label{freecat}
The category $\Nec$ is isomorphic to the strict monoidal category generated by $\mathcal{F}(\mathbf{\Delta}_{*,*}, \otimes)$ together with morphisms $[n] \vee [m] \to [n] \otimes [m]=[n+m]$.
\end{proposition}
\begin{proof}
This follows directly from Proposition \ref{Necmorphs} and Remark \ref{cofaces}.
\end{proof}


We now construct a functor
$$\mathcal{P}: \Nec \to \square_c.$$
Define $\mathcal{P}$ on objects by $\mathcal{P}( T )= [1]^{\text{dim{T}}}$ for any necklace $T \in \Nec$. If $T= \Delta^0$ we set $\mathcal{P}(\Delta^0)=[0].$ By Proposition \ref{freecat}, in order to define $\mathcal{P}$ on morphisms, it is sufficient to consider the following cases.
\begin{itemize}
\item For any $\partial^j: [n] \to [n+1]$ such that $0< j<{n+1}$, define $\mathcal{P}(\partial^j): [1]^{n-1}\to [1]^{n}$ to be the cubical coface functor $\mathcal{P}(f)= \delta_0^{j}.$ 

\item For any $[j] \vee [n+1-j] \to [n+1]$ such that $0<j<n+1$, define $\mathcal{P}(f): [1]^{n-1}\to [1]^{n}$ to be the cubical coface functor $\mathcal{P}(f)=\delta_1^{j}$.

\item We now consider morphisms of the form $s^j: [n+1] \to [n]$ for $n>0$. If $j=0$ or $j=n$, then $\mathcal{P}(f): [1]^n \to [1]^{n-1}$ is defined to be the cubical codegeneracy functor $\mathcal{P}(f)= \varepsilon^{j}.$ If $0<j<n$, then we define $\mathcal{P}(f): [1]^n \to [1]^{n-1}$ to be the cubical coconnection functor $\gamma^{j}.$

\item For $s^0: \Delta^1 \to \Delta^0$ we define $\mathcal{P}(s^0): [0] \to [0]$ to be the identity functor $id_{[0]}.$

\end{itemize}
\begin{remark}
The functor $\mathcal{P}$ is neither faithful or full. However, for any necklace $T' \in \Nec$ with $\text{dim}(T')=n+1$ and any cubical coface functor $\delta_{\epsilon}^j: [1]^n \to [1]^{n+1}$ for $0 \leq j \leq n+1$, there exists a unique pair $(T, f: T \hookrightarrow T')$, where $T \in \Nec$ with $\text{dim}(T)=n$ and $f: T \hookrightarrow T'$ is an injective morphism in $\Nec$, such that $\mathcal{P}(f)=\delta_{\epsilon}^j $.
\end{remark}

The functor $\mathcal{P}: \Nec \to \square_c$ induces an adjunction between $\cSet$ and $\nSet$
with right adjoint
$$\mathcal{P}^*: \cSet \to \nSet$$
and left adjoint
$$\mathcal{P}_{!} : \nSet \to \cSet.$$
Given a cubical set with connections $C: \square_c^{op} \to \Set$, we have $$\mathcal{P}^*(C)= C \circ \mathcal{P}^{op}.$$ Given a necklical set $K: \Nec^{op} \to \Set$, we have $$\mathcal{P}_{!}(K)= \colim_{Y(T) \to K} \mathcal{P}(T) \cong \colim_{Y(T) \to K} \square_c^{\text{dim}(T)}.$$