
\subsection{Cubes}

The \textit{cube category} $\square$ is the free strict monoidal category with a \textit{bipointed object}
\begin{equation*}
\begin{tikzcd}
1 \arrow[r, bend left, "\delta^0"] \arrow[r, bend right, "\delta^1"'] & 2 \arrow[r, "\sigma"] & 1
\end{tikzcd}
\end{equation*}
such that $\sigma \circ \delta^0 = \sigma \circ \delta^1 = \mathrm{id}$. Explicitly, it contains an object $2^n$ for each non-negative integer $n$ and its morphisms are generated by the \textit{coface} and \textit{codegeneracy maps} defined by
\begin{align*}
\delta_i^\varepsilon & = \mathrm{id}_{2^{i-1}} \times \delta^\varepsilon \times \mathrm{id}_{2^{n-1-i}} \colon 2^{n-1} \to 2^n, \\
\sigma_i & = \mathrm{id}_{2^{i-1}} \times \, \sigma \times \mathrm{id}_{2^{n-i}} \colon 2^{n} \to 2^{n-1}.
\end{align*}

A \textit{cubical set} $X$ is a contravariant functor from the cube category to the category of sets and a cubical map is a natural transformation between two cubical sets. As is customary, we use the notation
\begin{equation*}
X\big( 2^n \big) = X_n \qquad X(\delta^\varepsilon_i) = d^\varepsilon_i \qquad X(\sigma_i) = s_i,
\end{equation*}
and refer to elements in the image of any $s_i$ as \textit{degenerate}.

For each $n \in \mathbb{N}$, the cubical set $\square^n$ is defined by
\begin{equation*}
\square^n_k  = \Hom_{\square} \big( 2^k, 2^n \big), \qquad 
d^\varepsilon_i(x) = x \circ \delta^\varepsilon_i, \qquad 
s_i(x) = x \circ \sigma_i,
\end{equation*}
notice that iteratively
\begin{equation*}
\square^n = \overbrace{\square^1 \times \cdots \times \square^1}^{n \text{ times }}.
\end{equation*}
We represent the non-degenerate elements of $\square^n$ as sequences $x_1 \cdots\, x_n$ with each $x_i \in \{[0], [1], [0,1]\}$. Any cubical set can be expressed as a colimit of these
\begin{equation*}
X \cong \colim_{\square^n \to X} \square^n.
\end{equation*}

The functor $N_\bullet$ of \textit{normalized chains} is defined as follows. The chain complex $N_\bullet(\square^1)$ is isomorphic to the cellular chain complex of the interval
\begin{equation*}
\begin{tikzcd} [column sep = small, row sep = 0.1pt]
\Z\{[0], [1]\}  & \arrow[l] \Z\{[0,1]\} \\
{[1] - [0]} & \arrow[l, |->] \left[0,1\right].
\end{tikzcd}
\end{equation*}
Set
\begin{equation} \label{eq: chains on I^n}
N_\bullet(\square^n) = N_\bullet(\square^1)^{\otimes n}
\end{equation}
and define
\begin{equation*}
N_\bullet(X) = \colim_{\square^n \to X} N_\bullet(\square^n).
\end{equation*}

The functor of \textit{normalized cochains} $N^\bullet$ is defined by composing $N_\bullet$ with the linear duality functor $\Hom(-, \Z)$.

The \textit{Serre diagonal} $\Delta \colon N_\bullet(X) \to N_\bullet(X) \otimes N_\bullet(X)$ is defined as follows. Let $\Delta \colon \chains(\square^1) \to \chains(\square^1)^{\otimes 2}$ be defined on basis elements by	
\begin{gather*}	
\Delta([0]) = [0] \otimes [0], \qquad 
\Delta([1]) = [1] \otimes [1], \qquad
\Delta([0, 1]) = [0] \otimes [0, 1] + [0, 1] \otimes [1].	
\end{gather*}	
Then, let
$\Delta \colon \chains(\square^n) \to \chains(\square^n)^{\otimes 2}$
be the composite
\begin{equation*}
\begin{tikzcd}
\chains(\square^1)^{\otimes n} \arrow[r, "\Delta^{\otimes n}"] & \left( \chains(\square^1)^{\otimes 2} \right)^{\otimes n} \arrow[r, "sh"] & \left( \chains(\square^1)^{\otimes n} \right)^{\otimes 2},
\end{tikzcd}
\end{equation*}
where $sh$ is the shuffle map that reorders tensor factors so that those in odd positions occur first. More explicitly, using Sweedler's notation, we have that if $x_i^{(1)}$ and $x_i^{(2)}$ are defined through the identity
\begin{equation*}	
\Delta(x_i) = \sum x_i^{(1)} \otimes x_i^{(2)},
\end{equation*}	
then,
\begin{equation} \label{E: Delta}	
\Delta (x_1 \otimes \cdots \otimes x_n) = 	
\sum \pm \left( x_1^{(1)} \otimes \cdots \otimes x_n^{(1)} \right) \otimes 	
\left( x_1^{(2)} \otimes \cdots \otimes x_n^{(2)} \right),
\end{equation}	
where the sign is determined by the Koszul convention.

There is an action of $\S_d$ on $\chains(\square^d)$ given by permuting the factors of $\chains(\square^1)^{\otimes d}$.
We now show that the Serre diagonal is equivariant with respect to this action in the following sense.

\begin{proposition} \label{p:serre diagonal invariant}
	Let $T$ be the braiding on the symmetric monoidal category $(\Ch, \otimes, \Z)$. The following diagram commutes
	\begin{equation*}
	\begin{tikzcd}
	\chains(\square^1)^{\otimes 2} \arrow[r, "\Delta"] \arrow[d, "T"] &
	\chains(\square^1)^{\otimes 2} \otimes \chains(\square^1)^{\otimes 2} \arrow[d, "T \otimes T"] \\
	\chains(\square^1)^{\otimes 2} \arrow[r, "\Delta"] &
	\chains(\square^1)^{\otimes 2} \otimes \chains(\square^1)^{\otimes 2}.
	\end{tikzcd}
	\end{equation*}
\end{proposition}

\begin{proof}
	This can be established by a straightforward verification. More conceptually, the reason behind this proposition is that in $\S_4$ one has $(12)(34)(23) = (23)(13)(24)$.
\end{proof}