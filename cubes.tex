
\subsection{Cubes}

For any positive integer $n$, write $[1]^n$ for the cartesian product of $n$ copies of the category $[1]=\{0 \to 1\}$. We adopt the convention that $[1]^0=[0]$. 

The \textit{cube category with connections} $\square_c$ is the category with objects given by the set $\{[1]^0, [1]^1, [1]^2,...\}$ as objects and whose morphisms are generated by the following three kinds of functors:
\\
\textit{cubical co-face functors} $\delta^{\epsilon}_{j}: [1]^n \to [1]^{n+1}$, where $j=1,...,n+1$, and $\epsilon \in \{0,1\}$, defined by
\begin{eqnarray*}
\delta^{j}_{\epsilon}(t_1,...,t_n)=(t_1,...,t_{j-1},\epsilon,t_j,...,t_n),
\end{eqnarray*}
\textit{cubical co-degeneracy functors} $\varepsilon_{j}: [1]^n \to [1]^{n-1}$, where $j=1,...,n$, defined by
\begin{eqnarray*}
\varepsilon^{j}(t_1,...,t_n)=(t_1,...,t_{j-1},t_{j+1},...,t_n), \text{ and }
\end{eqnarray*}
\textit{cubical co-connection functors} $\gamma_{j}: [1]^n \to [1]^{n-1}$, where $j=1,...,n-1$, $n\geq 2$, defined by
\begin{eqnarray*}
\gamma^{j}(t_1,...,t_n)=(t_1,...,t_{j-1},\text{max}(t_j,t_{j+1}),t_{j+2},...,t_n).
\end{eqnarray*}
A \textit{cubical set with connections} is a functor $C: \square_c^{op} \to \Set.$ Given any cubical set with connections $C$, we will write $C_n= C( [1]^n ), \delta^{\epsilon}_j := C( \delta^{j}_{\epsilon}): C_n \to C_{n-1}, \varepsilon_j=C(\varepsilon^j): C_{n-1} \to C_n$, and $\gamma_j=C(\gamma^j): C_{n-1} \to C_n$, and call $\delta^{\epsilon}_j, \varepsilon_j,$ and $\gamma_j$ the face, degeneracy, and connection maps of $C$, respectively. Cubical sets with connections form a category, denoted by $\cSet$, with morphisms being natural transformations. We denote by $\square_c^n \in \cSet$ the cubical set with connections corepresented by $[1]^n,$ i.e $\square_c^n:= \Hom_{\square_c}(\text{ \_ ,}[1]^n)$. 

The category of cubical sets with connections $\cSet$ is a (non-symmetric) monoidal category when equipped with product $\otimes$ given by $$C \otimes C' = \colim_{\square^n_c \to C, \square^m_c \to C'} \square^{n+m}_c.$$

\begin{remark} The \textit{cube category} $\square$ is the subcategory of $\square_c$ with the same objects and morphisms generated by cubical co-face and co-degeneracy functors. Cubical sets are usually defined as functors $\square^{op} \to \Set$. However, for this article we will need to include connections maps into our data, so this more standard notion will not be considered. 
\end{remark}




%The \textit{cube category} $\square$ is the free strict monoidal category with a \textit{bipointed object}
%\begin{equation*}
%\begin{tikzcd}
%1 \arrow[r, bend left, "\delta^0"] \arrow[r, bend right, "\delta^1"'] & 2 \arrow[r, "\sigma"] & 1
%\end{tikzcd}
%\end{equation*}
%such that $\sigma \circ \delta^0 = \sigma \circ \delta^1 = \mathrm{id}$. Explicitly, it contains an object $2^n$ for each non-negative integer $n$ and its morphisms are generated by the \textit{coface} and \textit{codegeneracy maps} defined by
%\begin{align*}
%\delta_i^\varepsilon & = \mathrm{id}_{2^{i-1}} \times \delta^\varepsilon \times \mathrm{id}_{2^{n-1-i}} \colon 2^{n-1} \to 2^n, \\
%\sigma_i & = \mathrm{id}_{2^{i-1}} \times \, \sigma \times \mathrm{id}_{2^{n-i}} \colon 2^{n} \to 2^{n-1}.
%\end{align*}

%A \textit{cubical set} $X$ is a contravariant functor from the cube category to the category of sets and a cubical map is a natural transformation between two cubical sets. As is customary, we use the notation
%\begin{equation*}
%X\big( 2^n \big) = X_n \qquad X(\delta^\varepsilon_i) = d^\varepsilon_i \qquad X(\sigma_i) = s_i,
%\end{equation*}
%and refer to elements in the image of any $s_i$ as \textit{degenerate}.

%For each $n \in \mathbb{N}$, the cubical set $\square^n$ is defined by
%\begin{equation*}
%\square^n_k  = \Hom_{\square} \big( 2^k, 2^n \big), \qquad 
%d^\varepsilon_i(x) = x \circ \delta^\varepsilon_i, \qquad 
%s_i(x) = x \circ \sigma_i,
%\end{equation*}
%notice that iteratively
%\begin{equation*}
%\square^n = \overbrace{\square^1 \times \cdots \times \square^1}^{n \text{ times }}.
%\end{equation*}
%We represent the non-degenerate elements of $\square^n$ as sequences $x_1 \cdots\, x_n$ with each $x_i \in \{[0], [1], [0,1]\}$. Any cubical set can be expressed as a colimit of these
%\begin{equation*}
%X \cong \colim_{\square^n \to X} \square^n.
%\end{equation*}

