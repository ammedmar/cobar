
\section{Algebraic and categorical preliminaries}

\subsection{Kan extensions}

Given categories $\mathsf{B}$ and $\C$ we denote their associated \textit{functor category} by $\Fun(\mathsf{B}, \C)$.
Recall that a category is said to be \textit{small} if its objects and morphisms form sets.
We denote the category of small categories by $\Cat$.
A category is said to be \textit{cocomplete} if any functor to it from a small category has a colimit.
If $\mathsf{A}$ is small and $\mathsf{C}$ cocomplete, then the \textit{(left) Kan extension of $g$ along $f$} exists for any pair of functors $f$ and $g$ in the diagram below, and it is the initial object in $\Fun(\mathsf{B}, \mathsf{C})$ making
\begin{equation*}
\begin{tikzcd}[column sep=normal, row sep=normal]
\mathsf{A} \arrow[d, "f"'] \arrow[r, "g"] & \mathsf{C} \\ 
\mathsf{B} \arrow[dashed, ur, bend right] & \quad 
\end{tikzcd}
\end{equation*}
commute.

\subsection{Chain complexes}

Fix a commutative and unital ring $\k$ and denoted by $\Ch$ the category of differential graded modules (of $\k$-modules).
Its objects and morphisms are referred to as \textit{chain complexes} and \textit{chain maps} respectively.
For chain complexes $C$ and $C^\prime$ we regard, as usual, their product $C \otimes C^\prime$ and set of linear maps $\Hom(C, C^\prime)$ as a chain complexes, making $\Ch$ into a closed symmetric monoidal category.
The $i\th$ \textit{suspension} functor $s^i \colon \Ch \to \Ch$ is defined at the level of graded modules by $(s^{i}M)_n = M_{n-i}$.

\subsection{Coalgebras} \label{ss:coalgebras}

A \textit{coalgebra} consists of an object $C$ in $\Ch$ and morphisms $\Delta \colon C \to C \otimes C$ and $\varepsilon \colon C \to R$ in this category satisfying the usual coassociativity and counitality relations. 
This notion is equivalent to that of a comonoid in $\Ch$.
Denote by $\coAlg$ the category of coalgebras with morphisms being structure preserving chain maps.
The category $\coAlg$ is monoidal, with the structure maps of a product $C \otimes C^\prime$ given by:
\begin{equation} \label{e:coalgebras are symmetric monoidal1}
C \otimes C^\prime \xra{\Delta \otimes \Delta^\prime}
(C \otimes C) \otimes (C^\prime \otimes C^\prime) \xra{(23)}
(C \otimes C^\prime) \otimes (C \otimes C^\prime),
\end{equation}
\begin{equation} \label{e:coalgebras are symmetric monoidal2}
C \otimes C^\prime \xra{\varepsilon \otimes \varepsilon^\prime}
R \otimes R \xra{\cong}
R.
\end{equation}

\subsection{Monoids}

Categorically...

We will denote the category of monoids in a monoidal category $\C$ by $\Mon_{\C}$.

Monoids in $\Ch$ are called \textit{algebras} and those in $\coAlg$, \textit{bialgebras}.

\subsection{The cobar construction}

A \textit{coaugmentation} on a coalgebra $C$ is a morphism $\nu \colon R \to C$ in $\coAlg$.
We denote by $\coAlg^*$ the category of coaugmented coalgebras with morphisms being coaugmentation preserving coalgebra morphisms.

The \textit{cobar construction} is the functor $\cobar \colon \coAlg^* \to \Mon_{\Ch}$ defined on objects as follows.
Let $(C, \partial, \Delta, \varepsilon, \nu)$ be the data of a coaugmented coalgebra.
Denote by $\overline{C}$ the cokernel of $\nu \colon R \to C$ and recall that $s^{-1}$ is the $(-1)\th$ suspension.
The cobar construction $\cobar C$ of this coaugmented coalgebra is the graded module
\begin{equation*}
T(s^{-1} \overline{C}) = \k \oplus s^{-1}\overline{C} \oplus (s^{-1}\overline{C})^{\otimes 2} \oplus (s^{-1}\overline{C})^{\otimes 3} \oplus \cdots
\end{equation*}
regarded as a monoid in $\Ch$ with $\mu \colon T(s^{-1} \overline{C})^{\otimes 2} \to T(s^{-1} \overline{C})$ given by concatenation, $\eta \colon \k \to T(s^{-1} \overline{C})$ by the obvious inclusion, and differential constructed by extending the linear map
\begin{equation*}
- s^{-1} \circ \partial \circ s^{+1} \ + \ (s^{-1} \otimes s^{-1}) \circ \Delta \circ s^{+1} \colon
s^{-1} \overline{C} \to s^{-1}\overline{C} \oplus (s^{-1}\overline{C} \otimes s^{-1}\overline{C}) \hookrightarrow T(s^{-1}C)
\end{equation*}
as a derivation.

We now review the central examples of augmented coalgebras on which Adams used this construction.

In this article we are mostly concerned with \textit{connected} coalgebras, namely coaugmented coalgebras which are non-negatively graded and for which the coaugmentation induces an isomorphism of (not graded or) coalgebras $R \cong C_0$.

\anibal{Say this better, coalgebras are dg.}

We denote by $\coAlg^0$ the full subcategory of $\coAlg^*$ consisting of connected coalgebras, and remark that if $C \in \coAlg^0$, then $ \cobar(C)$ is concentrated on non-negative degrees.

%The cobar construction was originally applied by Adams to the coalgebra of normalized chains on a $1$-reduced simplicial set to obtain a model for the chains on the based loop space \cite{Adams}.
%The cobar construction as a model for the based loop space was furthered studied in \cite{Baues} and more recently, in the non-simply connected setting in \cite{Hess--Tonks}, \cite{rivera-zeinalian-cubical}.

\subsection{Simplicial sets}

For any non-negative integer $n$ denote by $[n]$ the category generated by the poset $\{0 \leq 1 \leq \dots \leq n\}$.
The \textit{simplex category} $\simplex$ is the category with set of objects $\big\{ [n] \big\}$ morphisms given by functors.
These are generated by the usual (simplicial) \textit{coface} and \textit{codegeneracy maps}
\begin{equation*}
\delta_i \colon [n-1] \to [n], \qquad \sigma_i \colon [n+1] \to [n]
\end{equation*}
for $j \in \{0, \dots, n\}$.

The category of \textit{simplicial sets} is the functor category $\sSet = \Fun(\simplex^\op, \Set)$.
The \textit{standard $n$-simplex} is the simplicial set $\simplex^n = \simplex(-, [n])$, and the \textit{Yoneda embedding} $\Y \colon \simplex \to \sSet$ is the functor induced by $[n] \mapsto \cube^n$.
For any simplicial set $X$ we write, as usual, $X_n$ instead of $X([n])$, and remark that
\begin{equation*}
X_n \cong \colim_{\simplex^n \to X} \simplex^n.
\end{equation*}
If $X$ is such that $X_0$ is a singleton we say it is \textit{reduced}, and we denote the full subcategory of reduced simplicial sets by $\sSet^0$.

\subsection{Simplicial chains} \label{ss:simplicial sets}

For non-negative integers $m$ and $n$, let $\simplex_{\deg} \big( [m], [n] \big)$ be the subset of \textit{degenerate morphisms} in $\simplex \big( [m], [n] \big)$, i.e., those of the form $\sigma_i \circ \tau$ with $\tau$ any morphism in $\simplex \big( [m], [n+1] \big)$.
The functor of (simplicial) \textit{chains} $\schains \colon \sSet \to \Ch$ is the Kan extension along the Yoneda embedding of the functor $\triangle \to \Ch$ defined next.
To an object $[n]$ it assigns the chain complex having in degree $m$ the module
\begin{equation*}
\schains([n])_m = \frac{\k\{\simplex \big( [m], [n] \big) \}}{\k\{\simplex_{\deg} \big( [m], [n] \big) \}},
\end{equation*}
and differential
\begin{equation*}
\partial(\tau) = \sum (-1)^i \tau \circ \delta_i.
\end{equation*}
To a morphism $\tau \colon [n] \to [n^\prime]$ it assigns the chain map
\begin{equation*}
\begin{tikzcd}[row sep=-3pt, column sep=normal,
/tikz/column 1/.append style={anchor=base east},
/tikz/column 2/.append style={anchor=base west}]
\schains(\simplex^n)_m \arrow[r] &  \schains(\simplex^{n^\prime})_m \\
\big( [m] \to [n] \big) \arrow[r, mapsto] & \big( [m] \to [n] \xra{\tau} [n^\prime] \big).
\end{tikzcd}
\end{equation*}

When no confusion arises from doing so we write $\chains$ instead of $\schains$.

We denote, as usual, basis element in $\chains(\simplex^n)_m$ by increasing tuples $[v_0, \dots, v_m]$ with $v_i \in \{0, \dots, n\}$.

\subsection{Alexander--Whitney coalgebra} \label{ss:aw coalgebra}

We review a classical lift to coalgebras of the functor of simplicial chains:
\begin{equation*}
\begin{tikzcd}
& \coAlg \arrow[d] \\
\sSet \arrow[r, "\schains"] \arrow[ur, "\schainsAS", out=70, in=180] & \Ch.
\end{tikzcd}
\end{equation*}

Using a Kan extension along the Yoneda embedding, it suffices to equip the chains on standard simplices with a natural coalgebra structure.
For any $n \in \N$, define $\epsilon  \colon \chains(\simplex^n) \to \Z$ by
\begin{equation*}
\epsilon \big( [v_0, \dots, v_q] \big) = \begin{cases} 1 & \text{ if } q = 0, \\ 0 & \text{ if } q>0, \end{cases}
\end{equation*}
and $\Delta \colon \chains(\simplex^n) \to \chains(\simplex^n)^{\otimes2})$ by
\begin{equation*}
\Delta \big( [v_0, \dots, v_q] \big) = \sum_{i=0}^q [v_0, \dots, v_i] \otimes [v_i, \dots, v_q].
\end{equation*}

\subsection{Simplicial singular complex}

Consider the topological $n$-simplex
\begin{equation*}
\gsimplex^{n} = \{(x_0, \dots, x_n) \in [0,1]^{n+1} \mid \ \textstyle{\sum}_i x_i = 1\}.
\end{equation*}
The assignment $[n] \to \gsimplex^n$ defines a functor $\simplex \to \Top$ whose Kan extension is known as (simplicial) \textit{geometric realization}.
It has a right adjoint $\sSing \colon \Top \to \sSet$ given by
\begin{equation*}
Y \to \Big( 2^n \to \Top(\gsimplex^n, Y) \Big)
\end{equation*}
and referred to as the (simplicial) \textit{singular complex} of the topological space $Y$.
The chain complex $\chains(\cSing Y)$ is referred to as the \textit{cubical singular chains} of $Y$.

We modify this construction for a pointed topological space $(Y, y)$ by considering only maps of $\gsimplex^n \to Y$ sending all vertices to $y$.
This produces a reduced simplicial set $\sSing(Y, y)$.

\subsection{Cubical sets} \label{ss:cubical sets}

Following \cite{brown1981cubes}, the \textit{cube category with connections} $\cube$ is the extension of the usual subcategory of $\Cat$ with objects $2^n = \{0 \leq 1\}^n$ whose morphisms are generated by \textit{coface, codegeneracy} and \textit{coconnection} maps defined by
\begin{align*}
\delta_i^\varepsilon & = \mathrm{id}_{2^{i-1}} \times \delta^\varepsilon \times \mathrm{id}_{2^{n-1-i}} \colon 2^{n-1} \to 2^n, \\
\sigma_i & = \mathrm{id}_{2^{i-1}} \times \, \sigma \times \mathrm{id}_{2^{n-i}} \quad \colon 2^{n} \to 2^{n-1}, \\
... &
\end{align*}
where $\varepsilon \in \{0,1\}$ and
\begin{equation*}
\begin{tikzcd}
1 \arrow[r, out=45, in=135, "\delta^0"] \arrow[r, out=-45, in=-135, "\delta^1"'] & 2 \arrow[l, "\sigma"'] &[-10pt] \arrow[l, "\gamma"'] 2 \times 2
\end{tikzcd}
\end{equation*}
are defined by
\begin{equation*}
\delta^0(0) = 0, \qquad \delta^1(0) = 1, \qquad \sigma(0) = \sigma(1) = 0.
\end{equation*}
\anibal{complete with connections}

The category of \textit{cubical sets with connections} is the functor category $\cSet = \Fun(\cube^\op, \Set)$.
The \textit{standard $n$-cube} is the cubical set $\cube^n = \cube(-, 2^n)$, and the \textit{Yoneda embedding} $\cube \to \cSet$ is the functor induced by $2^n \mapsto \cube^n$.

For any cubical set with connections $X$ we have
\begin{equation*}
X_n \cong \colim_{\cube^n \to X} \cube^n,
\end{equation*}
and we denote $X(\delta_i^\varepsilon)$ and $X(\sigma_i)$ by $d_i^\varepsilon$ and $s_i$.

\subsection{Cubical chains}

For non-negative integers $m$ and $n$, let $\cube_{\deg}(2^m, 2^n)$ be the subset of \textit{degenerate morphisms} in $\cube(2^m, 2^n)$, i.e., those of the form $\sigma_i \circ \tau$ with $\tau$ any morphism in $\cube(2^m, 2^{n+1})$.
The functor of (cubical) \textit{chains} $\cchains \colon \cSet \to \Ch$ is the Kan extension along the Yoneda embedding of the functor $\cube \to \Ch$ defined next.
To an object $2^n$ it assigning the chain complex having in degree $m$ the module
\begin{equation*}
\frac{\k\{\cube(2^m, 2^n)\}}{\k\{\cube_{\deg}(2^m, 2^n)\}}
\end{equation*}
and differential defined by
\begin{equation*}
\partial (\id_{2^n}) = \sum_{i=1}^{n} \ (-1)^i \
\big(\delta_i^1 - \delta_i^0 \big).
\end{equation*}
To a morphism $\tau \colon 2^n \to 2^{n^\prime}$ it assigns the chain map
\begin{equation*}
\begin{tikzcd}[row sep=-3pt, column sep=normal,
/tikz/column 1/.append style={anchor=base east},
/tikz/column 2/.append style={anchor=base west}]
\cchains(\cube^n)_m \arrow[r] &  \cchains(\cube^{n^\prime})_m \\
\big( 2^m \to 2^n \big) \arrow[r, mapsto] & \big( 2^m \to 2^n \stackrel{\tau}{\to} 2^{n^\prime} \big).
\end{tikzcd}
\end{equation*}

When no confusion arises from doing so we write $\chains$ instead of $\cchains$.

We remark that $\chains(\cube^n)$ is isomorphic to the tensor product of $n$ copies of the cellular chains on the topological interval with its usual CW structure.
We use this isomorphism to denote the elements of $\chains(\cube^n)$ as sums of terms of the form $x_1 \otimes \cdots \otimes x_n$ with $x_i \in \big\{[0], [0,1], [1] \big\}$.

\subsection{Serre coalgebra} \label{ss:serre coalgebra}

We review a classical lift to coalgebras of the functor of cubical chains:
\begin{equation*}
\begin{tikzcd}
& \coAlg \arrow[d] \\
\cSet \arrow[r, "\cchains"] \arrow[ur, "\cchainsAS", out=70, in=180] & \Ch.
\end{tikzcd}
\end{equation*}

Using a Kan extension along the Yoneda embedding, it suffices to equip the chains on standard cubes with a natural coalgebra structure.
For any $n \in \N$, define $\epsilon \colon \chains(\square^n) \to \k$ by
\begin{equation*}
\epsilon \left( x_1 \otimes \cdots \otimes x_d \right) = \epsilon(x_1) \cdots \, \epsilon(x_n),
\end{equation*}
where
\begin{equation*}
\epsilon([0]) = \epsilon([1]) = 1, \qquad \epsilon([0, 1]) = 0,
\end{equation*} 
and $\Delta \colon \chains(\square^n) \to \chains(\square^n)^{\otimes 2}$ by
\begin{equation*}	
\Delta (x_1 \otimes \cdots \otimes x_n) = 	
\sum \pm \left( x_1^{(1)} \otimes \cdots \otimes x_n^{(1)} \right) \otimes 	
\left( x_1^{(2)} \otimes \cdots \otimes x_n^{(2)} \right),	
\end{equation*}	
where the sign is determined using the Koszul convention, and we are using Sweedler's notation
\begin{equation*}	
\Delta(x_i) = \sum x_i^{(1)} \otimes x_i^{(2)}
\end{equation*}
for the chain map $\Delta \colon \chains(\square^1) \to \chains(\square^1)^{\otimes 2}$ defined by
\begin{equation*}
\Delta([0]) = [0] \otimes [0], \quad \Delta([1]) = [1] \otimes [1], \quad \Delta([0,1]) = [0] \otimes [0,1] + [0,1] \otimes [1].
\end{equation*}

We remark that, using the canonical isomorphism $\chains(\square^n) \cong \chains(\square^1)^{\otimes n}$, the coproduct $\Delta$ can be described as the composition
\begin{equation*}
\begin{tikzcd}
\chains(\square^1)^{\otimes n} \arrow[r, "\Delta^{\otimes n}"] & \left( \chains(\square^1)^{\otimes 2}  \right)^{\otimes n} \arrow[r, "sh"] & \left( \chains(\square^1)^{\otimes n} \right)^{\otimes 2}
\end{tikzcd}
\end{equation*}
where $sh$ is the shuffle map that places tensor factors in odd position first. \vspace*{5pt} \\

\subsection{Cubical singular complex}

Consider the topological $n$-cube
\begin{equation*}
\gcube^{n} = \{(x_1, \dots, x_n) \mid x_i \in [0,1]\}.
\end{equation*}
The assignment $2^n \to \gcube^n$ defines a functor $\cube \to \Top$ whose Kan extension is known as \textit{geometric realization}.
It has a right adjoint $\cSing \colon \Top \to \cSet$ given by
\begin{equation*}
Z \to \Big(2^n \to \Top(\gcube^n, Z)\Big)
\end{equation*}
and referred to as the \textit{cubical singular complex} of the topological space $Z$.
The chain complex $\chains(\cSing Z)$ is referred to as the \textit{cubical singular chains} of $Z$.

\subsection{Triangulation and its right adjoint}

TBW


%Explicitly,
%\begin{equation*}
%\cchains(X) = \bigoplus_{n \geq 0} \cchains(\cube^n) \otimes \k[X_n] \ \Big/ \sim
%\end{equation*}
%where $(2^m \to 2^n) \circ \delta_i^\varepsilon \otimes x \sim (2^m \to 2^n) \otimes d_i^\varepsilon(x)$.