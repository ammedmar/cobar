\section{Introduction}

The philosophy is to construct an explicit $E_{\infty}$-bialgebra structure on the cobar model for the based loop space such that
\\
1) it is given by the action of a ``small" cofibrant model for the $E_{\infty}$-operad, recently introduced by A. Medina-Medina
\\
2) the topological applications include general spaces with arbitrary fundamental group and no finite type assumptions using the recent observations of M. Rivera and M. Zeinalian 

\ \\

Adams ...
\begin{equation*}
\begin{tikzcd}
& \Mon_{\Ch} \\
\sSet^1 \arrow[ru] \arrow[r, "\chains"] & \coAlg \arrow[u, "\cobar"]
\end{tikzcd}
\end{equation*}

Modeling the loop space. Category of $1$-reduced simplicial sets.

Baues lifted this construction through the forgetful functor $\coAlg \to \Ch$ to produce a monoid in the category of coalgebras using a monoid in the category of cubical sets.

\begin{equation*}
\begin{tikzcd}
\Mon_{\cSet} \arrow[r, "\chains"] & \Mon_{\coAlg_R} \\
\sSet^1 \arrow[r, "\chains"] \arrow[u, "\gcobar"] & \coAlg_R \arrow[u, "\cobar"]
\end{tikzcd}
\end{equation*}

The geometric structure responsible for Baues diagonal on $\Omega \chains$ is the Serre diagonal.
Whereas the algebraic structure on the simplicial chains $\chains$ giving rise to it is referred to as a homotopy $G$-coalgebra, which corresponds to the $E_2$-structure on $\chains$, naturally an $E_\infty$-coalgebra.
Later, Kadeishvili extended this construction to produce cup-$i$ coproducts on the cobar construction $\Omega \chains$.
Geometrically introducing cup-$i$ products for cubical sets extending the Serre diagonal \cite{Kadeishvili99coproducts} ---see also \cite{Pilarczyk2016cubical}--- and then identifying the algebraic structure on the simplicial cochains giving rise to it \cite{Kadeishvili03cup-i}.
On the algebraic side, Fresse \cite{Fresse03construction} used the surjection coalgebra structure on $\chains$ to construct on the cobar construction $\Omega \chains$ the  structure of a Barratt-Eccles coalgebra, operating therefore at the $E_\infty$-level, but leaving the geometry of the situation unexplored.
Our results are most similar to Fresse's.
As it will become clear, two key differences are that he uses distinct operads for chains and their cobar construction and does not relate his construction to cubical cochains, a hallmark of Baues's insight.
A more abstract point of view for the iteration of the cobar and bar functors was given by Smith \cite{Smith94cobar} and Fresse \cite{Fresse10complex} respectively. 
In Fresse's more general approach, the model category of operads is the operating framework, and he proves that the bar construction can be iterated for $E_\infty$-algebras that are cofibrant.
Although we do not pursue this here, we mention that the reduced $E_\infty$-operad introduced in \cite{Medina20prop1} via a finitely presented prop, a key element in our constructions, is cofibrant.
In this note we consider a coassociative version of this operad which makes both, simplicial and cubical sets into $E_\infty$-coalgebras.
We endow this operad, denoted $U(\M)$, with the structure of a Hopf operad, that is one where the tensor product of coalgebras is again a coalgebra, so we can define monoids, and show that the cobar construction of the simplicial chains is a monoid on the category of $U(\M)$-coalgebras.
The artificial associativity relation on $U(\M)$ is imposed so we have a forgetful map $\coAlg_{U(\M)} \to \coAlg_R$ and Baues construction could be compared to ours. We show that our monoid in the category of $U(\M)$-coalgebras lift Baues's one define in the category of coassociative coalgebras.
\begin{equation*}
\begin{tikzcd}
\Mon_{\cSet} \arrow[r, "\chains"] & \Mon_{\coAlg_{U(\M)}} \\
\sSet^1 \arrow[r, "\chains"] \arrow[u, "\gcobar"] & \coAlg_{U(\M)} \arrow[u, "\cobar"]
\end{tikzcd}
\end{equation*}

The restriction to 1-reduced simplicial sets is not necessary, we ...


%Applications: if the natural chain map between the normalized chains on a cubical set and its triangulation preserves the $U(M)$-coalgebra structure (check?) we may deduce that our model is equivalent as a $E_{\infty}$-bialgebra to the normalized chains on the classical Kan loop group functor (this would use recent work of Rivera, Zeinalian, and Minichello).
