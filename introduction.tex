
\section{Introduction}

In 1956, F. Adams introduced the cobar construction to model algebraically the passage from a simply connected pointed space to its topological monoid of based loops \cite{adams1956cobar}.
Fixing a commutative ring, the algebraic cobar construction may be described as a functor $\cobar \colon \coAlg^* \to \Mon_{\Ch}$ from the category of coaugemented coalgebras to the category of algebras, which we regard in this article as monoids over the symmetric monoidal category $\Ch$ of chain complexes.
Given a pointed space $(Y, y)$, Adams constructed a \textit{comparison map}, a natural chain map of algebras
\begin{equation} \label{e:adams map}
\theta_Y \colon \cobar(\sS(Y, y)) \to S^{\square}(\Omega_y Y),
\end{equation}
where $\sS(Y,y)$ is a pointed version of the coalgebra of (simplicial) singular chains in $Y$ and $S^{\square}(\Omega_yY)$ is the algebra of cubical singular chains on the topological monoid of Moore loops in $Y$ based at~$y$.
More precisely, $\sS(Y,y)$ is obtained by applying the functor of (normalized) chains $\schains$ to the pointed simplicial singular complex $\sSing(Y, y)$ of continuous maps $\gsimplex^n \to Y$ sending all vertices to $y$.
Adams proved that $\theta_Y$ is a quasi-isomorphism if $Y$ is simply connected, a result extended to path-connected spaces in \cite{rivera2018cubical} by considering finer homotopical algebra, and in \cite{hess2010cobar} by a localization procedure.

In this article we study the interplay between the geometric, combinatorial, and algebraic structures arising from the cobar construction and its relationship to the based loop space.
Our starting point is an insight of H.~Baues which is best explained in terms of \textit{cubical sets with connections}, i.e., presheaves over the 
strict monoidal category generated by the diagram
\begin{equation*}
\begin{tikzcd}
1 \arrow[r, out=45, in=135, "\delta^0"] \arrow[r, out=-45, in=-135, "\delta^1"'] & 2 \arrow[l, "\sigma"'] &[-10pt] \arrow[l, "\gamma"'] 2 \times 2
\end{tikzcd}
\end{equation*}
restricted by certain relations \cite{brown1981cubes, grandis2003cubical}.

The functor of cubical singular chains $S^\square$ appearing in the target of Adams' comparison map \eqref{e:adams map} is equal to the composition $\cchains \circ \cSing$, where $\cSing$ is constructed by considering all continuous maps from geometric cubes.
The monoidal structure on the site defining the category of cubical sets with connections $\cSet$ provides it, via Day convolution, with a monoidal structure, which, with the usual monoidal structures on $\Ch$ and $\Top$, makes both $\cSing$ and $\cchains$ into monoidal functors.
Consequently, $S^\square(\Omega_y Y)$ is a monoid in $\Ch$ since $\Omega_y Y$ is a monoid in $\Top$ for any pointed space $(Y, y)$.
Cubical chains are equipped with a natural coalgebra structure coming from Serre's chain approximation to the diagonal.
This structure lifts $\cchains$ to the category of coalgebras $\cchains_\coAlg \colon \cSet \to \coAlg$ as a monoidal functor, which, in particular, makes $S^\square(\Omega_y Y)$ into a monoid in $\coAlg$ for any pointed space $(Y, y)$.

Denoting by $\sSet^0$ the category of reduced simplicial sets, Baues' introduced the \textit{cubical cobar construction}
\begin{equation*}
\gcobar \colon \sSet^0 \to \Mon_{\cSet},
\end{equation*}
a functor he used in \cite{baues1998hopf} to factor Adams' comparison map as a composition
\begin{equation} \label{e:baues factorization of adams map}
\begin{tikzcd}[column sep = small]
\theta_Y \colon \cobar(\sS(Y, y)) \arrow[r] &
\cchains(\gcobar \sSing(Y, y)) \arrow[r] &
S^{\square}(\Omega_y Y),
\end{tikzcd}
\end{equation}
where the first chain map is an isomorphism and the second is induced from an inclusion of cubical sets with connections $\gcobar \sSing(Y, y) \to \cSing(\Omega_y Y)$.
Additionally, Baues proved that when $Y$ is simply connected this inclusion is a homotopy equivalence, a result extended to the non-simply connected setting in \cite{rivera2019path}.

The first map in \eqref{e:baues factorization of adams map} is defined for a general reduced simplicial set $X$ and is always an isomorphism.
Using this map, Baues transferred the Serre coproduct on $\cchains_\coAlg(\gcobar X)$ to $\cobar(\schains(X))$ and proved that, together with its formal monoid structure, this coproduct makes $\cobar(\schains(X))$ into a bialgebra, that is to say, a monoid in $\coAlg$.
In other words, Baues lifted Adams' cobar construction $\cobar \circ \schains$ so that it fits in the following diagram commuting up to a natural isomorphism:
\begin{equation*}
\begin{tikzcd}
\Mon_{\cSet} \arrow[r, "\cchains_\coAlg \ "] & \Mon_{\coAlg} \arrow[d] \\
\sSet^0 \arrow[r, "\cobar \circ \schains"] \arrow[u, "\gcobar"] & \Mon_{\Ch}.
\end{tikzcd}
\end{equation*}
Furthermore, when $(Y, y)$ is a pointed space, Adams' comparison map $\theta_Y \colon \cobar(\sS(Y, y)) \to S^{\square}(\Omega_y Y)$ is a quasi-isomorphism of bialgebras, i.e., of monoids in $\coAlg$.

We extend Baues' work by lifting $\cobar \circ \schains$ to the category of $E_\infty$ bialgebras, that is to say, of monoids in $E_\infty$ coalgebras; a notion that should not be confused with that of bialgebras over an $E_\infty$ prop.
More precisely, our lift is to the category of monoids in $\coAlg_{U(\M)}$, where $\coAlg_{U(\M)}$ is the category of coalgebas over the combinatorial model $U(\M)$ of the $E_\infty$ operad introduced via a finitely presented prop in \cite{medina2020prop1}.
For this statement to be meaningful, we first need to define a monoidal structure on $\coAlg_{U(\M)}$.
We do so by introducing a Hopf structure on $U(\M)$, that is to say, by making it into an operad over the category of coalgebras, a structure that has other potential applications \cite{livernet2008hopf}.
Once we have shown that $\coAlg_{U(\M)}$ is a monoidal category, we prove the main technical result of this work.
It states that the functor $\cchains_{\coAlg_{U(\M)}} \colon \cSet \to \coAlg_{U(\M)}$ defined in \cite{medina2021cubical} as a lift of $\cchains_\coAlg$ is monoidal.

Generalizing Baues' first step in the construction of his lift of Adams' cobar construction, we have that $\cchains_{\coAlg_{U(\M)}}(\gcobar X)$ is a monoid in $\coAlg_{U(\M)}$.
This is because $\gcobar X$ is a monoid in $\cSet$ and $\cchains_{\coAlg_{U(\M)}}$ is shown to be a monoidal functor.
Similar to Baues' second step, we use for any reduced simplicial set $X$ his isomorphism $\cobar(\schains(X)) \to \cchains(\gcobar X)$ to transfer the constructed $E_\infty$ bialgebra structure on $\cchains(\gcobar X)$ to $\cobar(\schains(X))$.
In particular we have, generalizing Baues' result, that for any pointed space $(Y, y)$ Adams' comparison map $\theta_Y \colon \cobar(\sS(Y, y)) \to S^{\square}(\Omega_y Y)$ is a quasi-isomorphism of $E_\infty$ bialgebras.
We condense this discussion in our main result.

\begin{theorem} \label{t:1st main thm in the intro}
    The functor $\cchains_{\coAlg_{U(\M)}}$ is monoidal and its associated functor on monoids fits in the following diagram commuting up to a natural isomorphism:
    \begin{equation*}
    \begin{tikzcd}
    & \Mon_{\coAlg_{U(\M)}} \arrow[d] \\
    \Mon_{\cSet} \arrow[ru, "\cchains_{\coAlg_{U(\M)}}", out=70, in=180] \arrow[r, "\cchains_\coAlg"]
    & \Mon_{\coAlg} \arrow[d] \\
    \sSet^0 \arrow[r, "\cobar \circ \schains"] \arrow[u, "\gcobar"]
    & \Mon_{\Ch}.
    \end{tikzcd}
    \end{equation*}
    Furthermore, when $(Y, y)$ is a pointed space, Adams' comparison map $\theta_Y \colon \cobar(\sS(Y, y)) \to S^{\square}(\Omega_y Y)$ is a quasi-isomorphism of $E_\infty$ bialgebras or, more specifically, of monoids in $\coAlg_{U(\M)}$.
\end{theorem} 

%In \cite{mcclure2003multivariable, berger2004combinatorial}, the functor of (normalized) chains $\schains$ was lifted to that of $E_\infty$ coalgebras, a result generalized in \cite{medina2020prop1} by constructing a lift $\schains_{\coAlg_{U(\M)}} \colon \sSet \to \coAlg_{U(\M)}$.
%We may then naturally ask about the relationship between the $E_\infty$ bialgebra structure on $\sS(\Omega_y Y)$ and the one we construct on $\Omega(\sS(Y, y))$.

For any reduced simplicial set $X$, an alternative construction of an $E_\infty$ bialgebra model is given by the chains on Kan's simplicial loop group $G(X)$.
More explicitly, Kan constructed a functor $G \colon \sSet^0 \to \mathsf{sGrp}$ to the category of simplicial groups such that, for any $X$, the geometric realization of $G(X)$ is homotopy equivalent, as a topological monoid, to the based loop space on the geometric realization of $X$ \cite{berger1995loops}.

TBW

%For any reduced simplicial set $X$, an alternative construction of an $E_\infty$ bialgebra model is given by the chains on Kan's simplicial loop group $G(X)$.
%More explicitly, Kan constructed a functor $G \colon \sSet^0 \to \mathsf{sGrp}$ to the category of simplicial groups such that, for any $X$, the geometric realization of $G(X)$ is homotopy equivalent, as a topological monoid, to the based loop space on the geometric realization of $X$ \cite{berger1995loops}.
%In \cite{mcclure2003multivariate, berger2004combinatorial}, the functor of (normalized) chains $\schains$ was lifted to that of $E_\infty$ coalgebras, a result generalized in \cite{medina2020prop1} by constructing a lift $\schains_{\coAlg_{U(\M)}} \colon \sSet \to \coAlg_{U(\M)}$.
\manuel{I took it from here}
The proof that Adams’ map $\theta_Y \colon \cobar(S^\triangle(Y, y)) \to S^{\square}(\Omega_y Y)$ is a quasi-isomorphism uses the fact that $S^\triangle(Y, y)$ is obtained by applying the chains functor to the fibrant simplicial set $\text{Sing}(Y,y)$. In general, for an arbitrary reduced simplicial set $X \in \sSet^0$ with $X_0=\{b\}$,  it is not true that the algebras
$\cobar(\schains(X))$ and $S^{\square}(\Omega_b |X|)$ are quasi-isomorphic. One way of correcting this failure is by first Kan replacing $X$. Another way, which involves a smaller amount of data, is by localizing the cobar construction at the $1$-simplices of $X$ as proposed in \cite{hess2010cobar}.

The based loop space of an arbitrary reduced simplicial set can be modeled through the classical \textit{Kan loop group functor}. This is a functor 
$$G\colon \sSet^0 \to \mathsf{sGrp}$$ from reduced simplicial sets to the category of simplicial groups with the property that the geometric realization $|G(X)|$ is homotopy equivalent to the based loop space $\Omega_b|X|$ as a topological monoid \cite{berger1995loops}. Furthermore, the Kan loop group together with its right adjoint $$\overline{W}: \mathsf{sGrp} \to \sSet^0,$$ called the classifying space functor, yield a Quillen equivalence of model categories when $\sSet^0$ is equipped with the Kan-Quillen model structure and $\mathsf{sGrp}$ with an appropriate model structure having as weak equivalences maps of simplicial groups whose underlying maps of simplicial sets are weak homotopy equivalences. 

Hess and Tonks considered a localized version of $\mathcal{A} = \cobar \circ \schains \colon \sSet^0 \to \Mon_\Ch$, which they called the \textit{extended cobar construction}, and compared it explicitly with the composition $$\schains \circ G \colon \sSet^0 \to \Mon_\Ch.$$ More precisely, they construct a functor
$$\widehat{\mathcal{A}} \colon \sSet^0 \to \Mon_\Ch$$
given by the localization $$\widehat{\mathcal{A}}(X)= \mathcal{L}_{A_X} \mathcal{A}(X),$$
where $A_X$ is a set of $0$-degree elements in $\mathcal{A}(X)$ determined by the set of $1$-simplices of $X$, together with a natural transformation
$$\phi \colon \widehat{\mathcal{A}} \Longrightarrow \schains \circ G$$  whose components are all quasi-isomorphisms of algebras.

Following Baues’ construction, Franz described a lift of $\widehat{\mathcal{A}}$ to a functor
$$\widehat{\mathcal{A}}_{\coAlg} \colon \sSet^0 \to \Mon_{\coAlg}$$ and showed that Hess and Tonks' natural transformation preserves the coalgebra structure, namely, it induces a natural transformation
$$\phi_{\coAlg} \colon \widehat{\mathcal{A}}_{\coAlg} \Longrightarrow \schains \circ G$$
whose components are all quasi-isomorphisms of bialgebras.

Using similar methods to those that lead to Theorem 1, we construct a further lift
$$\widehat{\mathcal{A}}_{U(\M)} \colon \sSet^0 \to \Mon_{\coAlg_{U(\M)}}$$
of the extended cobar construction. Denoting by $U \colon \Mon_{\coAlg_{U(\M)}} \to \coAlg_{U(\M)}$ the forgetful functor, we show that Hess and Tonks' comparison map gives rise to a natural transformation

$$\phi_{U(\M)} \colon U \circ \widehat{\mathcal{A}}_{U(\M)} \Longrightarrow \schains \circ G$$
whose components are all quasi-isomorphisms of $E_{\infty}$-coalgebras.  We apply the forgetful functor in this discussion since the algebra structure of $\schains(G(X))$ is not compatible with the $E_{\infty}$-coalgebra structure (in contrast to the cubical case). Hence, we may think of $\mathcal{A}_{U(\M)}(X)$ as an algebraic model for the based loop space in which the $E_{\infty}$-coalgebra and associative algebra structures are strictly compatible. 


We summarize this discussion in the following statement.
\begin{theorem} The extended cobar construction $\widehat{\mathcal{A}} \colon \sSet^0 \to \Mon_\Ch$ lifts to a functor $\widehat{\mathcal{A}}_{U(\M)} \colon \sSet^0 \to \Mon_{\coAlg_{U(\M)}}$ such that for any $X \in \sSet^0$, the natural transformation $\phi \colon \widehat{\mathcal{A}} \Longrightarrow \schains \circ G$ induces a quasi-isomorphism of $E_{\infty}$-coalgebras
$$  U(\widehat{\mathcal{A}}_{U(\M)}(X)) \xrightarrow{\simeq} \schains(G(X)).$$
\end{theorem}


%We now describe a comparison between this $E_\infty$ bialgebra model and the one constructed here.
%To do so in full generality we need a localized version of the cobar construction.
%Following K. Hess and A. Tonks \cite{hess2010cobar}, we formally invert all $1$-simplices appearing in Adams' cobar construction $\A = \cobar \circ \schains \colon \sSet^0 \to \Mon_{\Ch}$ and obtain a new functor $\widehat{\A} \colon \sSet^0 \to \Mon_{\Ch}$ together with a natural quasi-isomorphism 


%\begin{equation*}
%\widehat{\A} \colon \sSet^0 \to \Mon_{\Ch}
%\qquad
%\widehat{\A}( X) = \A(X)[A_X^{-1}],
%\end{equation*}
%where $A_X=\{ \sigma +1_R \mid \sigma \in X_1\}\subseteq \A(X)_0$.
%For any $X \in \sSet^0$ the algebra $\widehat{\A}(X)$ is naturally quasi-isomorphic to the algebra of chains on Kan's simplicial loop group of $X$.

% An advantage of working with the coalgebra of chains and the cobar construction, as opposed to the dual algebra of cochains and the bar construction, is that certain classical results may be extended to the non-simply connected case.

% For example, if $X \in \sSet^0$ is a Kan complex then the algebra $\cobar(\schains(X))$ is quasi-isomorphic to the singular chains on the based loop space of $|X|$ \cite{Rivera-Zeinalian}.
% In fact, the $E_{\infty}$-coalgebra structure on $\schains (X)$ completely determines the fundamental group of $X$ \cite{Rivera- Zeinalian Fundamenta Math}.

%We can use our construction of a monoidal lift of the functor of cubical chains to $E_\infty$ coalgebras to endow their model with such a structure.
%we do so by using a localized version $\widehat \gcobar$ of the cubical cobar functor~$\gcobar$.

%\begin{theorem} \label{t:2nd main thm in the intro}
   % Our functor (dashed arrow) fits in the following diagram which is commutative up to natural isomorphism
   % \begin{equation*}
   % \begin{tikzcd}
   % & \Mon_{\coAlg_{U(\M)}} \arrow[d] \\
    %\Mon_{\cSet} \arrow[ru, dashed, out=70, in=180] & \Mon_{\coAlg} \arrow[d] \\
    %\sSet^0 \arrow[r, "\widehat{\A}"] \arrow[u, "\widehat \gcobar"] & \Mon_{\Ch}
    %\end{tikzcd}
    %\end{equation*}
   % where $\widehat{\gcobar} \colon \sSet^0 \to \Mon_{\cSet}$ is a localization of $\gcobar$.
%\end{theorem}

%For any $X \in \sSet^0$, the above result describes an $E_{\infty}$-Hopf algebra structure on $\widehat{\A}(X)$, which may be thought of as an algebraic model for the based loop space of $|X|$ built directly from the combinatorics of $X$.
%This statement can be made precise by relating to Kan's simplicial loop group construction as follows.

%\begin{theorem} \label{t:3rd main thm in the intro}
%Let $X \in \sSet^0$ and denote by $G(X)$ Kan's simplicial loop group construction.
%There is a natural zig-zag of quasi-isomorphisms of $U(\M)$-coalgebras between $\schains(G(X))$ and $\widehat{\A}(X)$.
%In particular, there is a natural isomorphism of bialgebras $H_0(\widehat{\gcobar}(X)) \cong R[\pi_1(|X|)]$.
%\end{theorem}
%The above statement extends a recent result of M. Franz, who showed that $\schains(G(X))$ and $\widehat{\A}(X)$ are naturally quasi-isomorphic as coassociative coalgebras building upon previous results of Hess and Tonks, see \cite{franz2020szczarba}, \cite{hess2010cobar}.

The main idea in the proof of the above result is to use the following ``space-level" observation: after triangulating $\widehat\gcobar({X})$, a localized version of the monoidal cubical set $\gcobar(X)$, one obtains a simplicial monoid that is naturally weak homotopy equivalent to $G(X).$ 
\\
\\
\manuel{maybe add here a short subsection about relation to previous work.}
\\
\\
For any $0$-reduced simplicial -set $X$, the natural coassociative coproduct on $\cobar (\schains (X))$ obtained from Baues construction may be described directly in terms of the $E_2$-coalgebra structure of the simplicial chains $\schains(X).$ Furthermore, in the $1$-connected setting, Kadeishvili described cup-$i$ products extending the Serre product and then identified the algebraic structure on the simplicial cochains on $X$ giving rise to these operations \cite{kadeishvili1999coproducts, kadeishvili2003cup-i}.
See also \cite{pilarczyk2016cubical} for a different description of cubical cup-$i$ products.

B. Fresse used the surjection coalgebra structure on $\schains(X)$ to construct on the cobar construction $\cobar(\schains(X))$ the  structure of a Barratt-Eccles coalgebra \cite{fresse2003hopf}.
His constructions are purely algebraic leaving the cubical geometry of the situation somewhat unexplored.
Our results are similar to Fresse's but we emphasize the geometric perspective developed by Baues' insight.
In fact, the present article may be taken as an extension of Baues' original construction to the context of $E_{\infty}$ coalgebras.

%We also treat the case of non-simply connected spaces.
%Namely, we describe an explicit $E_{\infty}$ coalgebra structure on the cobar construction of the coalgebra of chains on reduced simplicial in terms of the Hopf operad $U(\M)$ and as well as a localized version that models the based loop space in the non-simply connected case.

More abstract viewpoints for the iteration of the cobar and bar functors were respectively given by J. Smith \cite{smith1994cobar} and B. Fresse \cite{fresse2010props}.
In Fresse's approach, a model category structure on operads and their algebras is the operating framework; he proves that the bar construction can be iterated for $E_\infty$ algebras that are cofibrant.
Although we do not pursue the following observation in this article, we mention that the reduced version of the $E_\infty$ operad $U(\M)$ is cofibrant.

%In this note we consider a coassociative version of this operad which makes both, simplicial and cubical sets into $E_\infty$-coalgebras.
%The strict associativity relation is imposed so we have a forgetful map $\coAlg_{U(\M)} \to \coAlg$ and Baues' construction could be compared to ours.

%We also observe that the cubical cobar construction factors as a composition
%$$\sSet^0 \to \Mon_{\nSet} \to \Mon_{\cSet},
%$$
%where $\nSet$ denotes the monoidal category of \textit{necklical sets}. A necklical set can be thought of as a version of a cubical set in which cubes are now labeled by ordered sequences of simplices called \textit{necklaces}. Injective maps of necklaces realize all cubical faces, however, non-injective maps describe a particular selection of cubical degeneracies and cubical connections. In some sense, the monoidal category of necklical sets is the framework with the minimal combinatorial data needed to make categorical sense of Baues' geometric cobar construction. Different aspects of the theory of necklaces and necklical sets have been studied in \cite{Dugger-Spivak}, \cite{Galvez-Kaufmann-Tonks}, and \cite{rivera-zeinalian-cubical}.


%satisfying $\cobar(\schains(X)) \cong \cchains(\gcobar X)$ for any reduced simplicial set $X$ and $\cchains(\gcobar \sSing(Y, y)) \cong C(\Omega^{\square}_y Y)$ for any pointed space $(Y, y)$, and such that, up to a choice of this isomorphism, the second chain map in Baues' factorization \eqref{e:baues factorization of adams map} is induced from a cubical map $\gcobar \mathsf{Sing}(Y, y) \to \mathsf{Sing}^\square(\Omega_y Y)$.


%There are several advantages to working with the coalgebra of chains and the cobar construction, as opposed to the dual algebra of cochains and the bar construction.
%We will expand on two which are related to algebraic models in the non-simply connected setting.