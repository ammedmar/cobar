
\section{Introduction}

In 1956 Adams introduced the cobar construction to model algebraically the passage from a simply connected pointed topological space to its topological monoid of based loops \cite{Adams}. The cobar construction may be described as a functor $\mathbf{\Omega} \colon \coAlg^* \to \Mon_{\Ch}$ from the category of (dg) coaugemented coassociative coalgebras over a commutative ring $R$ to the category of (dg) unital associative algebras over $R$, which we regard in this article as monoids over the monoidal category $(\Ch, \otimes) $ of chain complexes over $R$. Given a pointed topological space $(Y,y)$, Adams constructed a natural chain map of associative algebras
$$\theta_Y \colon \mathbf{\Omega}(S(Y,y)) \xrightarrow{\simeq} S^{\square}(\Omega_yY),$$ where $S(Y,y)$ is a pointed version of the coassociative coalgebra of singular chains in $Y$ and $S^{\square}(\Omega_yY)$ is the associative algebra of cubical singular chains on the topological monoid of (Moore) loops in $Y$ based at $y \in Y$. Adams proved that $\theta_Y$ is a quasi-isomorphism if $Y$ is simply connected. In 2016 the second author together with Zeinalian, proved that $\theta_Y$ is still a quasi-isomorphism for any path-connected space $Y$, but the algebraic homotopy theory of coalgebras in this more general case must be interpreted differently.

In this article we study the interplay between the geometry, combinatorics, and algebraic structures arising from the cobar construction and its relationship to the based loop space. The starting point for our perspective is a combinatorial construction, originally due to Baues, that associates to any pointed space $(Y,y)$ a topological monoid $\Omega^{\square}_yY$ constructed by gluing cubical cells. One of Baues' key insights was that the map $\theta_Y$ factors as a composition
$$\cobar(S(Y,y)) \xrightarrow{\cong} C(\Omega^{\square}_yY) \to S^{\square}(\Omega_yY),$$
where $\Omega^{\square}_yY$ is a topological monoid constructed as a CW-complex with cubical cells and $C(\Omega^{\square}_yY)$ denotes the corresponding algebra of cellular chains. The first map in this composition is an isomorphism of dg associative algebras and the second map is induced by a continuous inclusion of spaces $\Omega^{\square}_yY \to \Omega_yY$. Baues' proved that when $Y$ is simply connected, this inclusion is a homotopy equivalence and in the non-simply connected case an extension of this result was shown by the second author and Saneblidze \cite{Rivera-Saneblidze}. 

Baues used the \textit{Serre diagonal} coproduct on $C(\Omega^{\square}_yY)$ to equip $\cobar(S(Y,y))$ with a coassociative coalgebra structure making it into a dg bialgebra. The Serre diagonal is a coassociative coproduct which serves as a natural cellular approximation for the diagonal map when the cells are cubes. With respect to this construction, Adams' map $\theta_Y$ becomes a quasi-isomorphism of \textit{bialgebras}. 

In this article we make use of a combinatorial version of Baues' construction described in terms of \textit{cubical sets with connections}. The category of cubical sets with connections, denoted by $\sSet_c$, is a version of the category of cubical sets in which the objects are equipped with a set of extra degeneracies called \textit{connections} \cite{Brown-Higgins}. Cubical sets with connections form a monoidal category when equipped with the \textit{Day convolution} product $\otimes$. This framework provides a suitable categorical setting to describe Baues' construction.

More precisely, we consider Baues' construction as a functor $$\gcobar \colon \sSet^0 \to \Mon_{\cSet_c},$$
where $\sSet^0$ denotes the category of $0$-reduced simplicial sets and $\Mon_{\cSet_c}$ is the category of monoids in $(\cSet_c, \otimes)$, fitting into a square of functors
\begin{equation*}
\begin{tikzcd}
\Mon_{\cSet_c} \arrow[r, "\widetilde{\cchains}"] & \Mon_{\coAlg} \arrow[d] \\
\sSet^0 \arrow[r, "\cobar \circ \chains"] \arrow[u, "\gcobar"] & \Mon_{\Ch}
\end{tikzcd}
\end{equation*}
which commutes up to natural isomorphism. In the above diagram, the right vertical arrow is a forgetful functor and $\widetilde{\cchains} \colon \Mon_{\cSet_c} \to \Mon_{\coAlg}$ is induced by the monoidal functor $\cchains \colon \cSet_c \to \coAlg$ of normalized cubical chains equipped with the Serre diagonal as coassociative coproduct. The functor $\gcobar$ is a version of Baues' \textit{geometric cobar construction} and the composition $\widetilde{N} \circ \gcobar$ equips the cobar construction on the (Alexander-Whitney) coalgebra of normalized simplicial chains of a $0$-reduced simplicial set with a natural coassociative coalgebra structure. 

For any $0$-reduced simplicial set $X$, the natural coassociative coproduct on $\cobar (\chains (X))$ obtained from the above construction may be described directly from the $E_2$-coalgebra structure of the simplicial chains $\chains(X).$ Furthermore, in  the $1$-connected setting, Kadeishvili described cup-$i$ products extending the Serre diagonal and then identified the algebraic structure on the simplicial cochains on $X$ giving rise to these operations \cite{Kadeishvili99coproducts},  \cite{Pilarczyk2016cubical}.

Fresse \cite{Fresse03construction} used the surjection coalgebra structure on $\chains(X)$ to construct on the cobar construction $\cobar(\chains(X))$ the  structure of a Barratt-Eccles coalgebra, working therefore at the $E_\infty$-level. His constructions are purely algebraic leaving the cubical geometry of the situation somewhat unexplored.

Our results are most similar to Fresse's.
As it will become clear, two key differences are that he uses distinct operads acting on the normalized chains and the cobar construction and does not relate his construction to cubical chains, a hallmark of Baues's insight.

A more abstract point of view for the iteration of the cobar and bar functors was given by Smith \cite{Smith94cobar} and Fresse \cite{Fresse10complex} respectively. 
In Fresse's more general approach, the model category of operads is the operating framework, and he proves that the bar construction can be iterated for $E_\infty$-algebras that are cofibrant. Although we do not pursue this here, we mention that the reduced $E_\infty$-operad introduced in \cite{Medina20prop1} via a finitely presented prop, a key element in our constructions, is cofibrant.
In this note we consider a coassociative version of this operad which makes both, simplicial and cubical sets into $E_\infty$-coalgebras.
We endow this operad, denoted $U(\M)$, with the structure of a Hopf operad, that is one where the tensor product of coalgebras is again a coalgebra, so we can define monoids, and show that the cobar construction of the simplicial chains is a monoid on the category of $U(\M)$-coalgebras.
The artificial associativity relation on $U(\M)$ is imposed so we have a forgetful map $\coAlg_{U(\M)} \to \coAlg$ and Baues construction could be compared to ours. Our main result is the following.

\begin{theorem}
There exists a monoidal functor $\mathcal{N} \colon \cSet_c \to \coAlg_{U(\M)}$ such that

\begin{enumerate}

\item $\mathcal{N}$ extends the monoidal functor of normalized cubical chains $\cchains: \cSet_c \to \coAlg$, and 
\item the induced functor $\widetilde{\mathcal{N}} \colon \Mon_{\cSet_c} \to \Mon_{\coAlg_{U(\M)}}$ between associated categories of monoids makes the square
\begin{equation*}
\begin{tikzcd}
\Mon_{\cSet_c} \arrow[r, "\widetilde{\mathcal{N}}"] & \Mon_{\coAlg_{U(\M)}} \arrow[d]\\
\sSet^0 \arrow[r, "\widetilde{N} \circ \gcobar"] \arrow[u, "\gcobar"] & \Mon_{\coAlg}
\end{tikzcd}
\end{equation*}
commutative up to natural isomorphism. 
\end{enumerate}
\end{theorem} 

We also observe that the geometric cobar construction $\gcobar \colon \sSet^0 \to \Mon_{\cSet_c}$ factors as a composition
$$\sSet^0 \to \Mon_{\nSet} \to \Mon_{\cSet_c},$$
where $\nSet$ denotes the monoidal category of \textit{necklical sets}. A necklical set is a version of a cubical set in which cubes are now labeled by ordered sequences of simplices called \textit{necklaces}. Injective maps of necklaces realize all cubical faces, however, non-injective maps describe a particular selection of cubical degeneracies and cubical connections. In some sense, the monoidal category of necklical sets is the categorical framework with the minimal combinatorial data needed to make sense of Baues' geometric cobar construction. Different aspects of the theory of necklaces and necklical sets have been studied in \cite{Dugger, Spivak}, \cite{Kaufmann, Galvez, Tonks}, and \cite{Rivera-Zeinalian}.


The bottom horizontal functor in the above square is essentially Baues' construction. Hence, the above theorem provides a monoidal $E_{\infty}$-coalgebra structure on $\cobar(\chains (X))$, naturally associated to any $X \in \sSet^0$, extending Baues' coassociative coalgebra structure. Furthermore, this $E_{\infty}$-coalgebra structure is obtained by applying the normalized cubical chains functor to a cubical model, just like Baues did to obtain the coassociative coalgebra structure on the cobar construction. 

An advantage of working with the coalgebra of chains as opposed to the dual algebra of cochains and the bar construction is that classical results may be extended to the non-simply connected case. For example, if $X \in \sSet^0$ is a Kan complex then the algebra $\cobar(\chains(X))$ is quasi-isomorphic to the singular chains on the based loop space of $|X|$ \cite{Rivera-Zeinalian}. In fact, the algebraic structure $\chains (X)$ completely determines the fundamental group of $X$ \cite{Rivera- Zeinalian Fundamenta Math.}  For a general $X \in \sSet^0$, Hess and Tonks described a functor $\widehat{\cobar}: \sSet^0 \to \Mon_{\Ch}$ which extends the composition $\cobar \circ \chains$ by formally inverting the $1$-simplices of the underlying simplicial set. The algebra $\widehat{\cobar}(X)$ is then naturally quasi-isomorphic to the algebra of singular chains on the based loop space of $|X|$. Our main theorem above has the following application. 


\begin{theorem}
There exists a functor $\widehat{\gcobar} \colon \sSet^0 \to \Mon_{\cSet_c}$ agreeing with $\gcobar$ when restricted to $\sSet^1$ and factoring the extended cobar construction as $\widehat{\cobar} \cong N \circ \widehat{\gcobar}$. Furthermore, for any $X \in sSet^0$, the monoidal $U(\M)$-coalgebra structure on $\widetilde{\mathcal{N}}(\widehat{\gcobar}(X))$ endows the associative algebra $\widehat{\cobar}(X)$ with a natural monoidal $U(\M)$-coalgebra structure whose underlying bialgebra has the property of being a Hopf algebra with zeroth homology naturally isomorphic to the fundamental group ring of $X$.  
\end{theorem}

We say a monoid in the category of $\mathcal{U(\M)}$-coalgebras is an $E_{\infty}$-\textit{Hopf algebra} if its underlying bialgebra has the property of being a Hopf algebra. For any $0$-reduced simplicial set $X$, the above result is provides an $E_{\infty}$-Hopf algebra structure extending Baues' coassociative coalgebra structure on the cobar construction on $\chains(X)$ and modelling the based loop space of $|X|$. This $E_{\infty}$-Hopf algebra structure is built directly from the combinatorics of $X$. 




