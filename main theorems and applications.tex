
\section{Main theorem and applications}

Adams ...
\begin{equation*}
\begin{tikzcd}
& \Mon_{\Ch} \\
\sSet^1 \arrow[ru] \arrow[r, "\chains"] & \coAlg \arrow[u, "\cobar"]
\end{tikzcd}
\end{equation*}

Modeling the loop space. Category of $1$-reduced simplicial sets.

Baues lifted this construction through the forgetful functor $\coAlg \to \Ch$ to the category of coalgebras using a cubical monoid.

\begin{equation*}
\begin{tikzcd}
\Mon_{\cSet} \arrow[r, "\chains"] & \Mon_{\coAlg_R} \\
\sSet^0 \arrow[r, "\chains"] \arrow[u, "\gcobar"] & \coAlg_R \arrow[u, "\cobar"]
\end{tikzcd}
\end{equation*}

The algebraic structure on the simplicial chains responsible for the Baues diagonal on $\Omega \chains$ is referred to as a homotopy $G$-coalgebra, which corresponds to the $E_2$-structure on simplicial chains, naturally an $E_\infty$-coalgebra.
Later, Kadeishvili \cite{Kadeishvili03cup-i} extended this construction to produce cup-$i$ coproducts on the cobar construction. 
The simplicial chains carry the structure of a coalgebra over the surjection operad, and Fresse \cite{Fresse03construction} constructed on the cobar construction of a surjection coalgebra the structure of a Barratt-Eccles coalgebra.
Our result is most similar to Fresse's.
As it will become clear, two key differences are that he uses distinct operads for chains and their cobar construction and does not relate his construction to cubical cochains, a hallmark of Baues's insight.


%Applications: if the natural chain map between the normalized chains on a cubical set and its triangulation preserves the $U(M)$-coalgebra structure (check?) we may deduce that our model is equivalent as a $E_{\infty}$-bialgebra to the normalized chains on the classical Kan loop group functor (this would use recent work of Rivera, Zeinalian, and Minichello).


Us

\begin{equation*}
\begin{tikzcd}
\Mon_{\cSet} \arrow[r, "\chains"] & \Mon_{\coAlg_{U(\M)}} \\
\sSet^0 \arrow[r, "\chains"] \arrow[u, "\gcobar"] & \coAlg_{U(\M)} \arrow[u, "\cobar"]
\end{tikzcd}
\end{equation*}