
\subsection{Simplices}

For any non-negative integer $n$ denote by $[n]$ the category generated by the poset (or finite ordinal) $\{0 \to 1 \to \dots \to n\}$.
The \textit{simplex category} $\simplex$ is the category with objects $[0], [1], \dots$ and morphisms given by all functors $f \colon [n] \to [m]$.
The morphisms in $\simplex$ are generated by the usual
\textit{simplicial co-face} and \textit{co-degeneracy maps}
\begin{equation*}
\delta_i \colon [n-1] \to [n], \qquad \sigma_i \colon [n+1] \to [n]
\end{equation*}
for $j \in \{0, \dots, n\}$.
We denote by $\simplex_{\deg}$ the subcategory with the same objects as $\simplex$ and morphisms of the form $\sigma_i \circ \tau$ for any morphisms $\tau$ of $\simplex$.

The category of \textit{simplicial sets} is the functor category $\sSet = \Fun(\simplex^\op, \Set)$.
The \textit{standard $n$-simplex} is the simplicial set $\simplex^n = \simplex(-, [n])$, and the \textit{Yoneda embedding} $\Y \colon \simplex \to \sSet$ is the functor induced by $[n] \mapsto \cube^n$.
For any simplicial set $X$ we have
\begin{equation*}
X_n \cong \colim_{\simplex^n \to X} \simplex^n.
\end{equation*}

Consider the functor $\simplex \to \Ch$ defined by
\begin{equation*}
\chains([n])_m = \frac{R\{\simplex([m], [n])\}}{R\{\simplex_{\deg}([m], [n])\}},
\qquad \qquad
\partial(\tau) = \sum (-1)^i \tau \circ \delta_i.
\end{equation*}

We review from \cite{Medina20prop1} a natural $\mathcal M$-bialgebra structure on the normalized chains of standard simplices $\chains(\triangle^n)$.
The $\M$-bialgebra is specified by three linear maps, the images of the generators
\begin{equation*}
\counit, \quad \coproduct, \quad \product,
\end{equation*}
satisfying the relations in the presentation of $\mathcal M$. For $n \in \mathbb{N}$, define: \vspace*{5pt} \\
(1) The counit $\epsilon \in \Hom(\chains(\triangle^n), \Z)$ by
\begin{equation*}
\epsilon \big( [v_0, \dots, v_q] \big) = \begin{cases} 1 & \text{ if } q = 0, \\ 0 & \text{ if } q>0. \end{cases}
\end{equation*}
(2) The coproduct $\Delta \in \Hom(\chains(\triangle^n), \chains(\triangle^n)^{\otimes2})$ by
\begin{equation*}
\Delta \big( [v_0, \dots, v_q] \big) = \sum_{i=0}^q [v_0, \dots, v_i] \otimes [v_i, \dots, v_q].
\end{equation*}
(3) The product $\ast \in \Hom(\chains(\triangle^n)^{\otimes 2}, \chains(\triangle^n))$ by
\begin{equation*}
\left[v_0, \dots, v_p \right] \ast \left[v_{p+1}, \dots, v_q\right] = \begin{cases} (-1)^{p+|\pi|} \left[v_{\pi(0)}, \dots, v_{\pi(q)}\right] & \text{ if } v_i \neq v_j \text{ for } i \neq j, \\
0 & \text{ if not}, \end{cases}
\end{equation*}
where $\pi$ is the permutation that orders the totally ordered set of vertices, and $(-1)^{|\pi|}$ its sign. The coproduct $\Delta$ is known as the \textit{Alexander-Whitney diagonal}.

\begin{proposition}[\cite{Medina20prop1}] \label{p:simplicial chain bialgebra}
	For every $n \in \mathbb{N}$, the assignment
	\begin{equation*}
	\counit \mapsto \epsilon, \quad \coproduct \mapsto \Delta, \quad \product \mapsto \ast,
	\end{equation*}
	defines a natural $\mathcal M$-bialgebra structure  on $\chains(\triangle^n)$.
\end{proposition}

\begin{proof}
	See Theorem 4.2 in \cite{Medina20prop1}.
\end{proof}

The category of bialgebras over a prop is in general not cocomplete, but those of algebras and coalgebras over operads are.

Using the forgetful functor $U$ we can considering the previous construction as a functor $\simplex \to \coAlg_{U(\M)}$ and use a Kan extension along the Yoneda embedding to define a lift of the functor of chains regarded as a coalgebra with the Alexander-Whitney diagonal:
\begin{equation*}
\begin{tikzcd}[column sep=small, row sep=small]
& \coAlg_{U(\M)} \arrow[d] \\
& \coAlg \arrow[d] \\
\sSet \arrow[r, "N"] \arrow[dashed, uur, bend left]& \Ch.
\end{tikzcd}
\end{equation*}

As explained in \cite{Medina20prop1}, this $E_\infty$-coalgebra on simplicial chains generalizes that defined by McClure-Smith \cite{mcclure03cochain} and Berger-Fresse \cite{berger04combinatorial}.