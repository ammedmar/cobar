
\subsection{Simplices} 
For any non-negative integner $n$ denote by $[n]$ the category generated by the poset (or finite ordinal) $\{0 \to 1 \to ... \to n-1 \to n\}$. The \textit{simplex category} $\mathbf{\Delta}$ is the category with objects given by the set $\{ [0], [1], [2], ...\}$ and morphisms given by all functors $f: [n] \to [m]$. The morphisms in $\mathbf{\Delta}$ are generated by the usual
\textit{simplicial co-face maps} $\partial^j: [n-1] \to [n]$ for $j=0,...,n$ and \textit{simplicial co-degeneracy maps} $s^j: [n+1] \to [n]$ for $j=0,...,n.$ These maps satisfy the simplicial identities.

A \textit{simplicial set} is a functor $S: \mathbf{\Delta}^{op} \to \Set.$ Given a simplicial set $S$, we will write $S_n= S( [n] )$, $\partial_j = S( \partial^j): S_n \to S_{n-1}$, and $s_j= S(s^j): S_n \to S_{n+1}$, and call $\partial_j$ and $s_j$ the face and degeneracy maps of $S$, respectively. Simplicial sets form a category, denoted by $\sSet$, with morphisms being natural transformations.  We denote by $\Delta^n \in \sSet$ the simplicial set corepresented by $[n]$, i.e. $\Delta^n:= \Hom_{\mathbf{\Delta}}(\text{ \_ ,}  [n])$. 
