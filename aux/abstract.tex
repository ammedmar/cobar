\begin{abstract}
	We extend to its ultimate conclusion a line of work started by F. Adams and continued by J.H. Baues, by showing that the classical map comparing the cobar construction on the singular chains of a pointed space and the singular cubical chains on its based loop space preserves an explicitly defined monoidal $E_\infty$-coalgebra structure, which, after M. Mandell, encodes under mild assumptions the homotopy type of the base loop space.
\end{abstract}



%We construct an explicit monoidal $E_\infty$-coalgebra structure on the cobar construction of chains on a reduced simplicial set and, more generally, on the chains of a monoidal cubical set.
%	This involves using a suitable model for the E-infinity operad and a cubical interpretation of the cobar construction.
%Furthermore, we prove that a classical map constructed by F. Adams comparing his cobar construction on the singular chains of a pointed space and the singular cubical chains on its based loop space is a quasi-isomorphism that preserves this structure.
%This work extends to its ultimate conclusion a theorem of H.J. Baues stating that Adams' map preserves a monoidal coalgebra structure.


%
%the singular chains of a pointed space and the singular cubical chains on its based loop space is a quasi-isomorphism that preserves this structure.
%
%
%constructing an explicit monoidal $E_\infty$-coalgebra structure on the cobar construction of chains on a reduced simplicial set, and proving that
%
%
%We construct an explicit monoidal $E_\infty$-coalgebra structure on the cobar construction of chains on a reduced simplicial set and, more generally, on the chains of a monoidal cubical set.
%Furthermore, we prove that a classical map constructed by F. Adams comparing his cobar construction on the singular chains of a pointed space and the singular cubical chains on its based loop space is a quasi-isomorphism that preserves this structure.



%theorem of H.J. Baues stating that Adams' classical monoidal map, comparing his cobar construction on the singular chains of a pointed space and the singular cubical chains on its based loop space, preserves a monoidal coalgebra structure.

% coming from a chain approximation to the diagonal.
%We explicitly construct an $E_\infty$-extension of these monoidal coalgebras making them commutative up to coherent homotopies, and show that Adams' map preserve said structure.
%The importance of $E_\infty$-structures is
%
%It was shown by Mandell that, under mild assumptions, an homotopy commutativity etension of this coalgebra
%We construct an explicit monoidal $E_\infty$-coalgebra structure on the cobar construction of chains on a reduced simplicial set and, more generally, on the chains of a monoidal cubical set, and prove that Adams' map preserves this structure.
