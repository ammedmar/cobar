
\section{Hopf structures}

By definition, the functor of chains $\cchains \colon \cube \to \Ch$ is monoidal, i.e., the diagram
\begin{equation*}
\begin{tikzcd}
2^p \times 2^q \arrow[r, "\cong"] \arrow[d, "\cchains \otimes \cchains"']& 2^{p+q} \arrow[d, "\cchains"] \\
\cchains(\cube^p) \otimes \cchains(\cube^q) \arrow[r, "\cong"] & \cchains(\cube^{p+q})
\end{tikzcd}
\end{equation*}
commutes and $\cchains(2^0) = R$.
The lift of  $N \colon \cube \to \Ch$ to the symmetric monoidal category $\coAlg$ is also monoidal since the following diagrams associated to the Serre diagonal and augmentation map commute:
\begin{equation*}
\begin{tikzcd}
\cchains(\cube^p) \otimes \cchains(\cube^{q}) \arrow[r, "\cong"] \arrow[d, "(23) \circ (\Delta \otimes \Delta)"'] & \cchains(\cube^{p+q}) \arrow[d, "\Delta"]\\
\cchains(\cube^p)^{\otimes 2} \otimes \cchains(\cube^{q})^{\otimes 2} \arrow[r, "\cong"]& \cchains(\cube^{p+q})^{\otimes 2}
\end{tikzcd}
\qquad \qquad
\begin{tikzcd}
\cchains(\cube^p) \otimes \cchains(\cube^{q}) \arrow[r, "\cong"] \arrow[d, "(\varepsilon \otimes \varepsilon)"'] & \cchains(\cube^{p+q}) \arrow[d, "\varepsilon"] \\
R \otimes R \arrow[r, "\cong"] & R.
\end{tikzcd}
\end{equation*}

In this section we will show that the lift \eqref{e:cubical lift to bialgebras} is also monoidal.
In order to make sense of this statement we first need to describe a monoidal structure on $\M$-bialgebras.

To do so we now make $\M$ into a Hopf prop, that is, a prop over the category $\coAlg$.
Explicitly, we will construct chain maps $\Delta_\M(m,n) \colon \M(m,n) \to \M(m,n) \otimes \M(m,n)$ and $\varepsilon_\M(m,n) \colon \M(m,n) \to R$ for every $(m,n)$ compatible with the prop structure of $\M$ and making each $\M(m,n)$ into a counital coassociative coalgebra.
Then, the structure map of the monoidal product of $\M$-bialgebras is given for any $(m,n)$ by 
\begin{equation*}
\begin{tikzcd} [column sep = small, row sep=large]
\M(m,n) \otimes (C \otimes C^\prime)^{\otimes m} \arrow[r, "\Delta_\M \otimes \id"] & \M(m,n) \otimes \M(m,n) \otimes (C \otimes C^\prime)^{\otimes m} \arrow[dl, out=-150, in=30, "Sh"'] &[-25pt] \\
\big(\M(m,n) \otimes C^{\otimes m}\big) \otimes \big(\M(m,n) \otimes C^{\prime\, \otimes m}\big) \arrow[r] & 
C^{\otimes n} \otimes C^{\prime\, \otimes n} \arrow[r, "Sh"] &[-25pt]
(C \otimes C^\prime)^{\otimes n}.
\end{tikzcd}
\end{equation*}

We will use the following observation about the Serre diagonal to define the Hopf structure on $\M$.
There is an action of $\S_n$ on $\cchains(\square^n)$ given by permuting the factors of $\cchains(\square^1)^{\otimes n}$.
The Serre diagonal is equivariant with respect to this action in the following sense.

\begin{lemma} \label{l:serre diagonal invariant}
	Let $T$ be the braiding of $\Ch$.
	The following diagram commutes
	\begin{equation*}
	\begin{tikzcd}
	\cchains(\square^1)^{\otimes 2} \arrow[r, "\Delta"] \arrow[d, "T"] &
	\cchains(\square^1)^{\otimes 2} \otimes \cchains(\square^1)^{\otimes 2} \arrow[d, "T \otimes T"] \\
	\cchains(\square^1)^{\otimes 2} \arrow[r, "\Delta"] &
	\cchains(\square^1)^{\otimes 2} \otimes \cchains(\square^1)^{\otimes 2}.
	\end{tikzcd}
	\end{equation*}
\end{lemma}

\begin{proof}
	This can be proven by a straightforward verification. Conceptually, it holds because $(12)(34)(23) = (23)(13)(24)$ in $\S_4$.
\end{proof}

As can be seen from \eqref{e:free prop}, the $(m,n)$-part of the free prop is defined only up to a choice of total order on the set of vertices of the $(m,n)$-graph involved.

\begin{definition}
	Let $\Gamma \in \G(m,n)$ be a representative of an element in $\M(m,n)$ of degree $d$. A \textit{characteristic map} for $\Gamma$ is the chain map \vspace*{-5pt}
	\begin{equation} \label{e:order chain map}
	\begin{tikzcd}[row sep=tiny, column sep=small]
	\iota_\Gamma \colon \cchains(\square^d) \arrow[r] & \M(m,n) \\
	\qquad {[0,1]}^{\otimes d} \arrow[r, |->] & \Gamma.
	\end{tikzcd}
	\end{equation}
	induced by a choice of total order of the vertices of $\Gamma$.
\end{definition}

Since, by Lemma~\ref{l:serre diagonal invariant}, the Serre diagonal is equivariant, the following is well-defined.
\begin{definition}
	For any $(m,n)$ the \textit{coproduct} $\Delta_\M(m,n) \colon \M(m,n) \to \M(m,n)^{\otimes 2}$ is defined on a basis element $\Gamma$ by
	\begin{equation} \label{e:diagonal of M}
	\Delta_{\M}(\Gamma) = \iota_\Gamma^{\otimes 2} \circ \Delta \left([0,1]^{\otimes d}\right)
	\end{equation}
	where $\iota_\Gamma$ is a characteristic map of $\Gamma$ and $\Delta$ is the Serre diagonal.
	The \textit{counit} $\varepsilon_\M(m,n) \colon \M(m,n) \to R$ is the chain map whose value on basis elements of degree $0$ is $1$. 
\end{definition}

\begin{theorem}
	The collections of maps $\Delta_\M = \{\Delta_\M(m,n)\}_{m,n \geq 0}$ and $\varepsilon_\M = \{\varepsilon_\M(m,n)\}_{m,n\geq0}$ make $\M$ into a Hopf prop, i.e., a prop over the category $\coAlg$.
\end{theorem}

\begin{proof}
	The collection of maps $\Delta_\M$ are compatible with the composition structure on $\M$ since given characteristic maps $\iota_\Gamma$ and $\iota_{\Gamma^\prime}$,  $\iota_\Gamma \otimes \iota_{\Gamma^\prime}$ is a characteristic map of any (vertical or horizontal) composition of $\Gamma$ and $\Gamma^\prime$.
	It respects the relations defining $\M$ since
	\begin{center}
		\begin{tikzcd}
		\leftcounitcoproduct \arrow[r, "\Delta_\M"] \arrow[d, <->] &[-0pt] \leftcounitcoproduct \otimes \leftcounitcoproduct \arrow[d, <->] \\
		\identity \arrow[r, "\Delta_\M"] & \ \, \identity \otimes \identity \, ,
		\end{tikzcd}
		\qquad
		\begin{tikzcd}
		\rightcounitcoproduct \arrow[r, "\Delta_\M"] \arrow[d, <->] &[-0pt] \rightcounitcoproduct \otimes \rightcounitcoproduct \arrow[d, <->] \\
		\identity \arrow[r, "\Delta_\M"] & \ \, \identity \otimes \identity \, ,
		\end{tikzcd}
		\qquad
		\begin{tikzcd}
		\leftcomb \arrow[r, "\Delta_\M"] \arrow[d, <->] &[-0pt] \leftcomb \, \otimes \leftcomb \arrow[d, <->] \\
		\rightcomb \arrow[r, "\Delta_\M"] & \, \rightcomb \, \otimes \rightcomb,
		\end{tikzcd}
		\qquad
		\begin{tikzcd}
		\productcounit \arrow[r, "\Delta_\M"] \arrow[d, <->] &[-0pt] \counit\ \counit \, \otimes \productcounit + \productcounit \otimes \, \counit\ \counit \arrow[d, <->] \\
		0 \arrow[r, "\Delta_\M"] & \ 0.
		\end{tikzcd}
	\end{center}
	It is a chain map since
	\begin{center}
		\begin{tikzcd}
		\counit \arrow[r, "\Delta_\M"] \arrow[d, "\partial"'] &[-0pt] \counit \otimes \counit \arrow[d, "\partial"'] \\
		0 \arrow[r, "\Delta_\M"] & \ \, 0 \, ,
		\end{tikzcd}
		\qquad
		\begin{tikzcd}
		\coproduct \arrow[r, "\Delta_\M"] \arrow[d, "\partial"'] &[-0pt] \coproduct \otimes \coproduct \arrow[d, "\partial"'] \\
		0 \arrow[r, "\Delta_\M"] & \ \, 0 \, ,
		\end{tikzcd}
		\qquad
		\begin{tikzcd}
		\product \arrow[r, "\Delta_\M"] \arrow[d, "\partial"'] &[-0pt] \leftboundary \ \otimes \product + \product \otimes\ \rightboundary \arrow[d, "\partial"'] \\
		\rightboundary \,-\, \leftboundary \arrow[r, "\Delta_\M"] &
		\, \rightboundary \otimes \rightboundary \,-\, \leftboundary \otimes \leftboundary\,.
		\end{tikzcd}
	\end{center}
	The compatibility of the collection of maps $\varepsilon_\M$ is checked similarly.
\end{proof}
The Hopf structure of $\M$ makes the category $\biAlg_{\M}$ into a monoidal category. We now show the normalized chains functor $\cube \to \biAlg_{\M}$ is monoidal.
\begin{theorem} \label{chainsismonoidal}
	The functor $\cube \to \biAlg_{\M}$ of Proposition~\ref{thm: cubical chain bialgebra} is monoidal.
\end{theorem}

\begin{proof}
	We have already establish that the Serre diagonal and the augmentation map are compatible with the monoidal structure of $\cube$. In diagrammatic terms,
	\begin{equation*}
	\begin{tikzcd}
	\coproduct \otimes \cchains(\cube^{p+q}) \arrow[r, "\Delta_\M \otimes Sh"] \arrow[d] &[20pt]
	\left( \ \coproduct \otimes \coproduct \ \right) \otimes
	\cchains(\cube^{p}) \otimes \cchains(\cube^{q}) \arrow[d] \\
	\cchains(\cube^{p+q})^{\otimes 2} \arrow[r, "Sh"] &
	\cchains(\cube^{p})^{\otimes 2} \otimes \cchains(\cube^{q})^{\otimes 2}
	\end{tikzcd}
	\end{equation*}
	and
	\begin{equation*}
	\begin{tikzcd}
	\counit \ \otimes \cchains(\cube^{p+q}) \arrow[r, "\Delta_\M \otimes Sh"] \arrow[d] &[20pt]
	\left( \ \counit \ \otimes \ \counit \ \right) \otimes
	\cchains(\cube^{p}) \otimes \cchains(\cube^{q}) \arrow[d] \\
	R \arrow[r, "\cong"] &
	R \otimes R
	\end{tikzcd}
	\end{equation*}
	commute.
	We will now show that the following diagram commutes:
	\begin{equation*}
	\begin{tikzcd}
	\product \otimes \cchains(\cube^{p+q})^{\otimes 2} \arrow[d] \arrow[r, "\Delta_\M \otimes Sh"] &[20pt]
	\left(\, \ \leftboundary \ \otimes \product + \product \otimes \ \rightboundary \ \, \right) \otimes \cchains(\cube^{p})^{\otimes 2} \otimes \cchains(\cube^{q})^{\otimes 2} \arrow[d]\\
	\cchains(\cube^{p+q}) \arrow[r, "Sh"] &
	\, \cchains(\cube^{p}) \otimes \cchains(\cube^{q}).
	\end{tikzcd}
	\end{equation*}
	Since this is immediate for $p=0$ or $q=0$ and $\cube$ is generated as a monoidal category by $2^1$, we only need to verify the commutativity of this diagram for $p=q=1$.
	Consider $(x_1 \otimes y_1) \otimes (x_2 \otimes y_2) \in \cchains(\cube^1)^{\otimes 2} \otimes \cchains(\cube^1)^{\otimes 2}$.
	We have
	\begin{equation*}
	\begin{split}
	\big((\id \otimes \varepsilon) \otimes \ast \ + \ \ast \otimes (\varepsilon \otimes \Delta)\big) \ (-1)^{\bars{y_1} \bars{x_2}} \ (x_1 \otimes x_2) \otimes (y_1 \otimes y_2) \ & = \\
	(-1)^{\bars{y_1} \bars{x_2} + \bars{x_1} + \bars{x_2} + \bars{y_1}} \ x_1 \cdot \varepsilon(x_2) \otimes y_1 \ast y_2 \ & + \ 
	(-1)^{\bars{y_1} \bars{x_2} + \bars{x_1}} \ x_1 \ast x_2 \otimes \varepsilon(y_1) \ast y_2.
	\end{split}
	\end{equation*}
	The first summand on the left hand side is non-zero only if $\bars{x_2} = \bars{y_1} = \bars{y_2} = 0$, whereas the second is non-zero only if $\bars{x_1} = \bars{x_2} = \bars{y_1} = 0$, so the above is also equal to
	\begin{equation} \label{e:join is monoidal 1}
	(-1)^{\bars{x_1}} \ x_1 \cdot \varepsilon(x_2) \otimes y_1 \ast y_2 \ + \ 
	x_1 \ast x_2 \otimes \varepsilon(y_1) \ast y_2.
	\end{equation}
	On the other hand,
	\begin{align*}
	(x_1 \otimes y_1) \ast (x_2 \otimes y_2) \ & =\ 
	(-1)^{\bars{x_1} + \bars{y_1}} \ (x_1 \ast x_2) \otimes \varepsilon(y_1) \cdot y_2 \ +\
	(-1)^{\bars{x_1} + \bars{y_1}} \ x_1 \cdot \varepsilon(x_2) \otimes (y_1 \ast y_2) \\ \ & =\ 
	(x_1 \ast x_2) \otimes \varepsilon(y_1) \cdot y_2 \ +\
	(-1)^{\bars{x_1}} \ x_1 \cdot \varepsilon(x_2) \otimes (y_1 \ast y_2),
	\end{align*}
	which is equal to \eqref{e:join is monoidal 1} as claimed.
\end{proof}

We have the following immediate consequence.

\begin{theorem} \label{t:lift chains on cSet to UM coAlg is monoidal}
    The functor $\cchains_{U(\M)} \colon \cSet \to \coAlg_{U(\M)}$ is monoidal.
\end{theorem}